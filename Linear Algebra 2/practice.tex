\documentclass[11pt,a4paper]{article}
\usepackage{amssymb,amsfonts,amsmath,calc,tikz,geometry}
\usepackage{color}   %May be necessary if you want to color links
\usepackage{hyperref}
\usepackage{amsthm}
\hypersetup{
    colorlinks=false, %set true if you want colored links
    linktoc=all,   %set to all if you want both sections and subsections linked
    linkcolor=black,  %choose some color if you want links to stand out
}
%\setlength{\textwidth}{450pt}
%\setlength{\oddsidemargin}{7pt}
\geometry{margin=1in}
%%%%%%%%%%%%%%%%%%%%%%%%%%%%%%%%%%%%%%%%%%%%%%%%%%%%%%%%%%%%%%%%%%%%%%%%%%%%%%%
\theoremstyle{plain}
\newtheorem{theorem}{Theorem}[section]
\newtheorem{lemma}[theorem]{Lemma}
\newtheorem{proposition}[theorem]{Proposition}
\newtheorem{corollary}[theorem]{Corollary}
\newtheorem{definition}{Definition}[section]
\newtheorem{remark}{Remark}[section]
\DeclareMathOperator{\lcm}{lcm}
\DeclareMathOperator{\idealin}{\triangleleft}
\DeclareMathOperator{\im}{im}
\DeclareMathOperator{\Aut}{Aut}
\DeclareMathOperator{\End}{End}
\DeclareMathOperator{\Inn}{Inn}
\DeclareMathOperator{\Out}{Out}
\DeclareMathOperator{\Mat}{Mat}
\DeclareMathOperator{\std}{std}
\newcommand{\Id}{\text{Id}}
\newcommand{\op}{\mathrm{op}}
\newcommand{\Ker}{\text{Ker}}
\renewcommand{\Im}{\text{Im}}
\newcommand{\Sp}{\text{Sp}}
\newcommand{\tr}{\text{tr}}
\newcommand{\N}{\mathbb{N}}
\newcommand{\Z}{\mathbb{Z}}
\newcommand{\Q}{\mathbb{Q}}
\newcommand{\R}{\mathbb{R}}
\newcommand{\C}{\mathbb{C}}
\newcommand{\F}{\mathbb{F}}
\newcommand{\Omicron}{O}
\newcommand{\bigslant}[2]
{{\raisebox{.2em}{$#1$}\left/\raisebox{-.2em}{$#2$}\right.}}
\newcommand{\ip}[2]{\left\langle #1, #2 \right\rangle}
%%%%%%%%%%%%%%%%%%%%%%%%%%%%%%%%%%%%%%%%%%%%%%%%%%%%%%%%%%%%%%%%%%%%%%%%%%%%%%%
\title{\textbf{Practice}}
\author{Yeheli Fomberg}
\date{326269651}
\usepackage{amsmath}
\begin{document}
	\maketitle
	\newpage
	
	\section{In the following sections show that $V = U \oplus W$ and find
	the projection on $U$ parallel to $W$}
	\subsection{$V = \R[x]$ with 
	\[ W = \text{Sp}\{x^2+x+1\},\quad U = \{p(x)\in V \colon p(0) = 0\} \]
	}
	
	First we will show that $U + W = V$. Let $p\in V$ be a general polynomial:
	\[ p(x) = a_nx^n + \cdots + a_0 \in V\]
	Now choose $w = a_0(x^2+x+1) \in W$ and $u = (p-w) \in U$. We know that
	$(p-w)\in U$ because: 
	\[ (p-w)(0) = p(0) - w(0) = a_0 - a_0 = 0 \]
	Now we see that: 
	\[ u + w = (p-w) + w = p \] 
	That proves that $U + W = P$. Now we will show that $U\bigcap W = \{0\}$
	which will prove that $U\oplus W = V$, as we have shown in the lecture.
	\begin{align*}
		U\cap W &= \{p(x)\in V \colon p(x)\in W \land p(0) = 0\} \\
		&= \{ax^2+ax+a \colon a\in\R \land p(0) = 0\} \\
		&= \{ax^2+ax+a \colon a\in\R \land a = 0\} \\
		&= \{0\}
	\end{align*}
	Now we will find the projection on $U$ parallel to $W$. We have shown that
	the only way to get any specific $p\in V$ is by adding the specific:
	\[
		u_p + w_p = (p-a_0(x^2+x+1)) + a_0(x^2+x+1)
	\]
	So the parallel projection will be $P \colon V\to V$:
	\begin{align*}
		P(p(x)) &= P(a_nx^n + \cdots + a_0) = u_p = (p-a_0(x^2+x+1)) \\
		&= a_nx^n + \cdots + a_3x^3 + (a_2-a_0)x^2 + (a_1-a_0)x
	\end{align*}
	
	\newpage
	
	\subsection{$V = \R^4$ with 
	\[ W = \text{Sp}\{e_1+e_4,e_2+e_4\},\quad U = \{e_1,e_2+e_3\} \] 
	\noindent
	where $E = (e_1,...,e_4)$ is the standard basis.}
	Consider the following matrix with the vectors from $U$ and $W$:
	\[ \begin{pmatrix}
	\text{---} & e_1+e_4 & \text{---}\\
	\text{---} & e_2+e_4 & \text{---}\\
	\text{---} & e_1 & \text{---}\\
	\text{---} & e_2+e_3 & \text{---}\\
	\end{pmatrix} \]
	By applying elementary row operations we get:
	\[ \begin{pmatrix}
	1 & 0 & 0 & 0\\
	0 & 1 & 0 & 0\\
	0 & 0 & 1 & 0\\
	0 & 0 & 0 & 1\\
	\end{pmatrix} \]
	Which means as we know from linear algebra $1$ that $U + W = V$ and from
	Grassman's identity(?) we know that:
	\begin{align*}
		\underbrace{\dim(W+U)}_4 &= \underbrace{\dim(W)}_2
		+ \underbrace{\dim(U)}_2 - \dim(U\cap W) \\
		&\Rightarrow \dim(U\cap W) = 0 \\
		&\Rightarrow U\cap W = \{0\}
	\end{align*}
	Which implies that $U\oplus W = V$. Now we will find the projection on 
	$U$ parallel to $W$. For this we will need to find the unique decomposition
	of any $v\in V$ to vectors $u\in U$ and $w\in W$. Where for 
	$a,b,c,d,x_1,x_2,x_3,x_4\in\F$:
	\begin{align*}
		w &= a(e_1+e_4) + b(e_2+e_4) = 
		\begin{pmatrix}
	a\\
	b\\
	0\\
	a+b\\
		\end{pmatrix} \\
		u &= c(e_1) + d(e_2+e_3) = 
		\begin{pmatrix}
	c\\
	d\\
	d\\
	0\\
		\end{pmatrix} \\
	\end{align*}
	\[
		u+w = 
		\begin{pmatrix}
	\vert & \vert & \vert & \vert\\
	e_1+e_4 & e_2+e_4 & e_1 & e_2+e_3\\
	\vert & \vert & \vert & \vert\\
		\end{pmatrix}
		\begin{pmatrix}
	a\\
	b\\
	c\\
	d\\
		\end{pmatrix} =
		\begin{pmatrix}
	a+c\\
	b+d\\
	d\\
	a+b\\
		\end{pmatrix}
		= v =
		\begin{pmatrix}
	x_1\\
	x_2\\
	x_3\\
	x_4\\
		\end{pmatrix}	
	\]
	So we get that $d=x_3$
	\[
		\begin{pmatrix}
	a+c\\
	b+x_3\\
	x_3\\
	a+b\\
		\end{pmatrix}
		=
		\begin{pmatrix}
	x_1\\
	x_2\\
	x_3\\
	x_4\\
		\end{pmatrix}	
	\]
	Now $b=x_2-x_3$
	\[
		\begin{pmatrix}
	a+c\\
	x_2\\
	x_3\\
	a+x_2-x_3\\
		\end{pmatrix}
		=
		\begin{pmatrix}
	x_1\\
	x_2\\
	x_3\\
	x_4\\
		\end{pmatrix}	
	\]
	So $a = x_4-x_2+x_3$ and we get:
	\[
		\begin{pmatrix}
	x_4-x_2+x_3+c\\
	x_2\\
	x_3\\
	x_4\\
		\end{pmatrix}
		=
		\begin{pmatrix}
	x_1\\
	x_2\\
	x_3\\
	x_4\\
		\end{pmatrix}	
	\]
	So $c = x_1 - x_4 + x_2 - x_3$. Finally we get that for any $v\in V$ such
	that:
	\[
		v =
		\begin{pmatrix}
	x_1\\
	x_2\\
	x_3\\
	x_4\\
		\end{pmatrix}	
	\]
	We get:
	\begin{align*}
		v &= w + u = a(e_1+e_4) + b(e_2+e_4) + c(e_1) + d(e_2+e_3) \\
		  &= (x_4-x_2+x_3)(e_1+e_4) + (x_2-x_3)(e_2+e_4) + 
		     (x_1 - x_4 + x_2 - x_3)(e_1) + (x_3)(e_2+e_3)
	\end{align*}
	Which means the projection on $U$ parallel to $W$ is $P\colon V\to V$
	\[
		\forall\underbrace{
		\begin{pmatrix}
	x_1\\
	x_2\\
	x_3\\
	x_4\\
		\end{pmatrix}}_v
		\in V \colon P(v) = (x_1 - x_4 + x_2 - x_3)(e_1) + (x_3)(e_2+e_3) = 
		\begin{pmatrix}
	x_1 - x_4 + x_2 - x_3\\
	x_3\\
	x_3\\
	0\\
		\end{pmatrix} = u
	\]

	
	\newpage
	
	\section{Prove/Disprove}
	\subsection{The sum of projections is a projection}
	This is false. Let $P_1 = P_2 = \mathrm{Id}_n$ be our projections from $\R^n$ to
	$\R^n$. It is clear these are projections since:
	\[ \mathrm{Id}_{n}^{2} = \mathrm{Id}_n \]
	But the transformation $P = P_1 + P_2$ is not a projection since:
	\[
		P^2 = (P_1 + P_2)^2 = (2\mathrm{Id}_n)^2 = 4\mathrm{Id}_n \neq 2\mathrm{Id}_n = P
	\]
	
	\subsection{The composition of projections is a projection}
	This claim is false. Consider the following projections over $\R^2$:
	\[
		P_1(x,y) = (x+y,0) \quad and \quad P_2(x,y) = (x,x)
	\]
	It's easy to verify that these are indeed projections:
	\begin{align*}
		P_1^2(x,y) &= P_1(x+y,0) = (x+y,0) = P_1(x,x) \\
		P_2^2(x,y) &= P_2(x,x) = (x,x) = P_2(x,y)
	\end{align*}
	Yet if we consider the vector $(2,1)$ we get:
	\begin{align*}
		(P_1\circ P_2)(2,1) &= P_1(2,2) = (4,0) \\
		(P_1\circ P_2)^2(2,1) &= (P_1\circ P_2)(4,0) = P_1(4,4) = (8,0)
	\end{align*}
	So:
	\[
		(P_1\circ P_2) \neq (P_1\circ P_2)^2
	\]
	Which means it's not a projection.
	
	
	
	\newpage
	
	\section{Let $V$ be a finite-dimensional vector space, and let
	$P_1,...,P_n\in \text{End}(V)$ be parallel projections. Denote $\forall i\colon R_i = \text{Im}P_i$}
	\subsection{Show that $\text{tr}P_i=\dim R_i$}
	Since $P_i$ is a parallel projection we know that 
	$V = \Im P_i \oplus \Ker P_i$
	Which means that $\Im P_i\cap \Ker P_i = \{0\}$. We know by a theorem
	we learned in class that exist:
	\[ B_r = \{b_1,...,b_k\} \]
	a basis for $\Im P_i = R_i$. And:
	\[ B_k = \{r_{b+1},...,b_n\} \]
	a basis for $\Ker P_i$ such that the ordered union:
	\[ B = B_r \cup B_k = \{b_1,...,b_k,b_{k+1},...,b_n\} \]
	forms a basis for $V$. That means that the matrix representation of $P_i$
	by the basis $B$ is:
	\[ 
	\begin{pmatrix}
	\vert & \vert & \vert\\
	[P_i(b_1)]_B & ... & [P_i(b_n)]_B\\
	\vert & \vert & \vert\\
	\end{pmatrix}_{n\times n}
	= 
	\begin{pmatrix}
	I_k & 0\\
	0 & 0\\
	\end{pmatrix}_{n\times n}
	\]
	So $\tr ([P_i]_B) = k$. And since the trace of a transformation is just
	the trace of its representing matrix, as shown to be a well defined trait
	of transformations  in linear algebra $1$ we conclude that:
	\[
		\tr (P_i) = \tr ([P_i]_B) = k = \dim\Im P_i = \dim R_i
	\]

	\newpage
	
	\subsection{Let $P_1+\cdots+P_n = \text{Id}$, show that 
	$V = \bigoplus R_i$ and infer that $\forall i\neq j \colon P_iP_j = 0$}
	\underline{$V = \bigoplus R_i$} - From $3.1$ we know that:
	\[
		\dim V = \tr (\mathrm{Id}) = \tr (P_1+\cdots+P_n) = 
		\tr (P_1) + \cdots + \tr(P_n) = \dim R_1 + \cdots + \dim R_n
	\]
	Now we will show that $R_1+\cdots R_n = V$. Let $v\in V$:
	\[
		v = \mathrm{Id}(v) = (P_1+\cdots+P_n)(v) = P_1(v)+\cdots+P_n(v)
	\]
	Since $\forall i \colon P_i(v)\in R_i$ we get that for any $v\in V$ exist
	$P_1(v)\in R_1,...,P_n(v)\in R_n$ such that $v = P_1(v)+\cdots+P_n(v)$.
	So now we know that
	\begin{align*}
		V &= R_1 + \cdots + R_n \\
		\dim V &= \dim R_1 + \cdots + \dim R_n
	\end{align*}
	Denote $B_{R_i}$ the ordered basis for $R_i$ for any $i$, we get:
	\begin{align*}
		V &= \Sp \left\{ \bigcup_i{B_{R_i}} \right\}
		&\Rightarrow \dim V \le \left\vert \bigcup_i{B_{R_i}} \right\vert \\
		\dim V &= \sum_i{\vert B_{R_i} \vert} \ge 
		\left \vert \bigcup_i{B_{R_i}} \right \vert
		&\Rightarrow \left \vert \bigcup_i{B_{R_i}} \right \vert \le \dim V \\
		&\Rightarrow \left \vert \bigcup_i{B_{R_i}} \right \vert = \dim V
	\end{align*}
	So from:
	\[
		\Sp \left\{ \bigcup_i{B_{R_i}} \right \}   = V
		\land 
		\left \vert \bigcup_i{B_{R_i}} \right \vert = \dim V
	\]
	We get that the ordered union of the ordered bases $B_{R_i}$ form a basis
	of $V$ which is equivalent as we've shown in class to saying that 
	$V = \bigoplus R_i$ \\
	\underline{$\forall i\neq j \colon P_iP_j = 0$} - Let $i \neq j$. Now
	suppose that $P_iP_j \neq 0$. that means that exists a $0 \neq v\in V$
	such that $P_iP_j(v) \neq 0$, which means that $P_j(v) \notin \Ker P_i$.
	Since $P_i$ is a projection we know that $\Im P_i \oplus \Ker P_i = V$
	which means that $P_j(v) \in R_i$, but also by definition 
	$P_j(v) \in R_j$, so:
	\begin{align*}
		\underbrace{0}_{R_1} + \cdots + \underbrace{P_j(v)}_{R_j} + \cdots + 
		\underbrace{0}_{R_n} = P_j(v) \\
		\underbrace{0}_{R_1} + \cdots + \underbrace{P_j(v)}_{R_i} + \cdots + 
		\underbrace{0}_{R_n} = P_j(v) 
	\end{align*}
	 but that's a contradiction to $V = \bigoplus R_i$. So 
	 $\forall i\neq j \colon P_iP_j = 0$
	
	\newpage
	
	\section{Let $V$ be a vector space, $T,S\in \text{End}(V)$, and let $S$ be 
	diagonalizable. Prove that the eigenspaces of $S$ are $T$-invariant if 
	and only if $TS = ST$}
	{$(\Leftarrow)$} \\
	For any eignenvalue $\lambda$ of $S$:
	\begin{align*}
		\Ker(S-\lambda I) &= \{s \in V \vert S(s) = \lambda s\} \\
		\Rightarrow T(\Ker(S-\lambda I)) &= \{T(s) \vert S(s) = \lambda s\} \\
		&= \{s \in V \vert \exists w\colon T(w) = s \land S(w) = \lambda w\}
	\end{align*}
	Since for $s\in T(\Ker(S-\lambda I))$:
	\[
		S(s) = S(T(w)) \underset{TS=ST}{=} T(S(w)) = T(\lambda w) = 
		\lambda T(w) = \lambda s
	\]
	We get that $T(\Ker(S-\lambda I)) \subseteq \Ker(S-\lambda I)$ which means
	that all the eigenspaces of $S$ are $T$-invariant. \\\\
	{$(\Rightarrow)$} \\
	We know that $S$ is diagnolizable so exist a base to $V$
	\[
	 B = (b_1,\dots,b_n)
	\]
	such that $[S]_B$ is a diagnonal matrix. We will show that for any $b\in B$
	that $TS(b) = ST(b)$. Let $b\in B$ be an eigenvector of an eigenspace
	with eigenvalue $\lambda$:
	\[
		TS(b) = T(\lambda b) = \lambda (T(b))
	\]
	Now since $b\in V_\lambda^S$\footnote{$\lambda$-eigenspace of $S$ under $V$
	not sure if this is the correct notation.} is $T$-invariant by the
	assumption:
	\[
		\lambda (T(b)) = S(T(b)) = ST(b)
	\]
	We have shown that for any vector from the base $B$ of $V$
	\[
		TS(b) = ST(b)
	\]
	Since $B$ spans $V$ and $S$, $T$ are linear, we know that for any $v\in V$
	\[
		TS(v) = ST(v)
	\]
	Which is what we wanted to prove.
	
	\newpage
	
	\section{Let $V$ be a vector space over a field $\F$, with $\dim V = n$. 
	Let $T \colon V \to V$ such that any $(n-1)$-dimentional vector subspace of 
	$V$ is $T$-invariant. Prove that $V$ is a scalar transformation.}
	Let $v_1\in V$ be a vector such that $T(v_1) = v_2$ and $v_2$ isn't a
	scalar multiply of $v_1$. That means they are linearly independent which
	implies we can complete $\{v_1,v_2\}$ to a basis of $V$ as such:
	\[
		B = (v_1,v_2,\dots,v_n)
	\] 
	Since $\Sp\{v_1,v_3,\dots,v_n\}$ is a $n-1$-dimentional subspace of $V$,
	it is $T$-invariant, which means that:
	\[
		T(v_1) = v_2 \in \Sp(v_1,v_3,\dots,v_n)
	\]
	But that's a contradiction since if $v_2$ were in $\Sp(v_1,v_3,\dots,v_n)$
	then $B$ wouldn't be linearly independent even thought it's a basis of $V$.
	That means that for any $v\in V$ then $T(v)$ is a scalar multiple of $v$.
	Now consider the standard basis $E=(e_1,\dots,e_n)$ we know that:
	\begin{align*} 
		T(e_1) &= \lambda_1e_1 \\
		T(e_2) &= \lambda_2e_2 \\
		&\dots \\
		T(e_n) &= \lambda_ne_n \\
	\end{align*}
	We also know that $T(e_1+\cdots+e_n) = \mu\sum_{i=1}^{n}{e_i}$ so:
	\[
		T(e_1+e_2+\cdots+e_n) = T(e_1) + \cdots + T(e_n) = 
		\sum_{i=1}^{n}{\lambda_ie_i} = \sum_{i=1}^{n}{\mu e_i}
	\]
	Since $e_1,\dots,e_n$ are linearly independent that means that:
	\[
		\lambda_1=\lambda_2=\dots=\lambda_n=\mu
	\]
	Finally since $E$ is a basis, for any $v\in V$ we get that $T(v) = \mu v$.
	In other words that $T$ is a scalar operator.
	
	\newpage
	
	\section{Let $T,S,Q\in\text{End}(v)$ such that $T = Q^{-1}SQ$. 
	Show that $U\subseteq V$ is $T$-invariant $\iff Q(U)$ is $S$-invariant}
	\underline{$(\Rightarrow)$} \\ Suppose that $U\subseteq V$ is $T$-invariant.
	That means that:
	\[
		T(U)\subseteq U
	\]
	Now:
	\[
		S(Q(U)) = SQ(U)
	\]
	But we know that $T = Q^{-1}SQ \Rightarrow QT = SQ$ so:
	\[
		S(Q(U)) = QT(U) = Q(T(U))
	\]
	We know that $T(U)\subseteq U$ so:
	\begin{align*}
		S(Q(U)) &= Q(T(U)) \subseteq Q(U) \\
		&\Rightarrow S(Q(U)) \subseteq Q(U)
	\end{align*}
	In other words - $Q(U)$ is $S$-invariant. \\\\
	\underline{$(\Leftarrow)$} \\ Suppose that $Q(U)$ is $S$-invariant:
	\[
		(*) \quad S(Q(U)) \subseteq Q(U)
	\]
	Now:
	\[
		T(U) = Q^{-1}SQ(U) = Q^{-1}(S(Q(U))) \underset{*}{\subseteq} 
		Q^{-1}(Q(U)) = U
	\]
	So:
	\[
		T(U) \subseteq U
	\]
	In other words $U$ is $T$-invariant.
	
	\newpage
	
	\section{The one it won't be fun to typeset.}
	\subsection{Find the Jordan normal form, a jordan basis, and the minimal
	polynomial of the following matrix:
	\[ A = 
\begin{pmatrix}
	-1 & -1 & 0 & 0 \\
	2 & 2 & 0 & 0 \\
	4 & 2 & 2 & 1 \\
	-2 & -1 & -1 & 0 \\
\end{pmatrix}\in M_n(\C)
	\]}
	First we're gonna find the characteristic polynomial of this matrix. We
	notice that the matrix is a blockwise triangular matrix:
	\[ A = 
\begin{pmatrix}
	-1 & -1 & 0 & 0\\
	2 & 2 & 0 & 0\\
	4 & 2 & 2 & 1\\
	-2 & -1 & -1 & 0\\
\end{pmatrix}
		=
\begin{pmatrix}
	B_{2\times 2} & 0\\
	* & C_{2\times 2}
\end{pmatrix}
	\]
	So we can solve it like we did in linear algebra $1$:
	\begin{align*}
		p_A(\lambda) &= p_B(\lambda)p_C(\lambda) = 
		((-1-\lambda)(2-\lambda)+2)((2-\lambda)(0-\lambda)+2) \\
		&= (\lambda^2-\lambda)(\lambda^2-2\lambda+1) = 
		   (\lambda(\lambda-1))(\lambda^2-2\lambda+1) = 
		   \lambda(\lambda-1)^3
	\end{align*}
	But we can also notice that the sum of columns of these blocks is $1$ so
	$1$ is an eigenvalue of both of them, and since the sum of the eigenvalues
	of a matrix is equal to its trace we can find the other eigen value.
	We see that $\lambda = 0$ is an eigenvalue of algebraic multiplicity $1$
	and $\lambda = 1$ is an eigenvalue of algebraic multiplicity $3$ so the\
	Jordan normal form will have a Jordan block $J_1(0)$ and some Jordan blocks
	of total size $3$. Now we will find $\null(A-I)$ to find out how many Jordan
	blocks are there:
	\begin{align*}
			A-I &= 
	\begin{pmatrix}
		-2 & -1 & 0 & 0\\
		2 & 1 & 0 & 0\\
		4 & 2 & 1 & 1\\
		-2 & -1 & -1 & -1\\
	\end{pmatrix} \to
	\begin{pmatrix}
		0 & 0 & 0 & 0\\
		2 & 1 & 0 & 0\\
		2 & 1 & 0 & 0\\
		-2 & -1 & -1 & -1\\
	\end{pmatrix} \to
	\begin{pmatrix}
		0 & 0 & 0 & 0\\
		0 & 0 & 0 & 0\\
		2 & 1 & 0 & 0\\
		0 & 0 & -1 & -1\\
	\end{pmatrix} \\ &\to
	\begin{pmatrix}
		2 & 1 & 0 & 0\\
		0 & 0 & -1 & -1\\
		0 & 0 & 0 & 0\\
		0 & 0 & 0 & 0\\
	\end{pmatrix}
	\end{align*}
	So we see that $\null(A-I) = 2$ so there are two Jordan blocks of 
	$\lambda = 1$. That that the Jordan normal form of $A$ must be
	of the form $J_2(1) \oplus J_1(1) \oplus J_1(0)$. So we want to find
	Jordan chains of the form:
	\begin{center}
	\begin{tabular}{c c | c} 
	 $\lambda=1$ &  & $\lambda=2$ \\
	 \hline
	 $v_2$ &  &  \\ 
	 $\downarrow$ &  &  \\
	 $v_1$ & $v_3$ & $v_4$ \\
	\end{tabular}
	\end{center}
	We shall continue with some more calculation to find the generalized
	eigenspaces of $A$.	
	\[
		\ker(A-I) = \ker\left(
		\begin{pmatrix}
		2 & 1 & 0 & 0\\
		0 & 0 & -1 & -1\\
		0 & 0 & 0 & 0\\
		0 & 0 & 0 & 0\\
		\end{pmatrix}\right) = 
		\Sp
		\left\{ \begin{pmatrix} 1\\ -2\\ 0\\ 0\\ \end{pmatrix},
		   \begin{pmatrix} 0\\ 0\\ 1\\ -1\\ \end{pmatrix} \right\}
	\]
	\begin{align*}
		(A-I)^2 &= 
	\begin{pmatrix}
		-2 & -1 & 0 & 0\\
		2 & 1 & 0 & 0\\
		4 & 2 & 1 & 1\\
		-2 & -1 & -1 & -1\\
	\end{pmatrix} \to
	\begin{pmatrix}
		-2 & -1 & 0 & 0\\
		2 & 1 & 0 & 0\\
		2 & 1 & 0 & 0\\
		0 & 0 & 0 & 0\\
	\end{pmatrix} \to
	\begin{pmatrix}
		2 & 1 & 0 & 0\\
		0 & 0 & 0 & 0\\
		0 & 0 & 0 & 0\\
		0 & 0 & 0 & 0\\
	\end{pmatrix} \\
	\ker(A-I)^2 &= \ker\left(
		\begin{pmatrix}
		2 & 1 & 0 & 0\\
		0 & 0 & 0 & 0\\
		0 & 0 & 0 & 0\\
		0 & 0 & 0 & 0\\
		\end{pmatrix}\right) = 
		\Sp
		\left\{ \begin{pmatrix} 1\\ -2\\ 0\\ 0\\ \end{pmatrix},
		   \begin{pmatrix} 0\\ 0\\ 1\\ 0\\ \end{pmatrix},
		   \begin{pmatrix} 0\\ 0\\ 0\\ 1\\ \end{pmatrix} \right\} = 
		   \overset{\sim}{V_1} \\
	\ker(A) &= \ker\left(
		\begin{pmatrix}
			-1 & -1 & 0 & 0\\
			2 & 2 & 0 & 0\\
			4 & 2 & 2 & 1\\
			-2 & -1 & -1 & 0\\
		\end{pmatrix}\right) = \ker\left(
		\begin{pmatrix}
			1 & 1 & 0 & 0\\
			0 & 0 & 0 & 0\\
			2 & 0 & 2 & 1\\
			0 & 1 & -1 & 0\\
		\end{pmatrix}\right) = \ker\left(
		\begin{pmatrix}
			1 & 1 & 0 & 0\\
			0 & 0 & 0 & 0\\
			0 & 0 & 0 & 1\\
			0 & 1 & -1 & 0\\
		\end{pmatrix}\right) \\ &= 
		\Sp\left\{\begin{pmatrix} -1\\ 1\\ 1\\ 0\\ \end{pmatrix}\right\} = 
		\overset{\sim}{V_0}
	\end{align*}
	To test our calculations against the generalized eigenspace decomposition
	theorem we see that indeed:
	\[
	V = \Sp\left\{
		\begin{pmatrix} -1\\ 1\\ 1\\ 0\\ \end{pmatrix}\right\} \oplus
		\Sp\left\{
		\begin{pmatrix} 1\\ -2\\ 0\\ 0\\ \end{pmatrix},
		\begin{pmatrix} 0\\ 0\\ 1\\ 0\\ \end{pmatrix},
		\begin{pmatrix} 0\\ 0\\ 0\\ 1\\ \end{pmatrix}\right\} = 
		\overset{\sim}{V_0} \oplus \overset{\sim}{V_1}
	\]
	To find $v_2$ we would need to find a vector in $\ker(A-I)^2$ that
	is not in $\ker(A-I)$ for example:
	\[
		v_2 = \begin{pmatrix} 0\\ 0\\ 1\\ 1\\ \end{pmatrix}
	\]
	Then:
	\[
		v_1 = (A-I)v_2 = \begin{pmatrix} 0\\ 0\\ 2\\ 0\\ \end{pmatrix}
	\]
	Now fo find $v_3$ we will just find a vector that will complement 
	$\Sp\{v_1,v_2\}$ to $\overset{\sim}{V_1}$ for example:
	\[
		v_3 = \begin{pmatrix} 1\\ -2\\ 0\\ 0\\ \end{pmatrix}
	\]
	And for the last vector we can just choose any vector that is in 
	$\overset{\sim}{V_1}$ for example:
	\[
		v_4 = \begin{pmatrix} -1\\ 1\\ 1\\ 0\\ \end{pmatrix}
	\]
	So we found all of our Jordan chains and also the Jordan basis for $A$:
	\[
		B_J = 
		\left\{
		\begin{pmatrix} 0\\ 0\\ 2\\ 0\\ \end{pmatrix}
		\begin{pmatrix} 0\\ 0\\ 1\\ 1\\ \end{pmatrix}
		\begin{pmatrix} 1\\ -2\\ 0\\ 0\\ \end{pmatrix}
		\begin{pmatrix} -1\\ 1\\ 1\\ 0\\ \end{pmatrix}
		\right\}
	\]
	Now we will find the minimal polynomial. To find the minimal polynomial
	we will see that it is excactly the product of the the polynomials
	of the form $p(x)=(x-\lambda)^r$ for each distinct eigenvalue $\lambda$
	of $A$ and $r$ being the size of the longest Jordan chain of its
	respective $\lambda$, since each vector in $V$ can be represented as a 
	linear combination of the Jordan base, and for any polynome that doesn't 
	include one of these multiples of $(x-\lambda)$ we can take the top of
	the chain of this lambda and see that it will not be a root of the supposed
	polynome. Therefore:
	\[
		m_A(x) = (x-1)^2 (x-2)
	\]

	\newpage
	
	\section{The one with the polynomial operator}
	\subsection{Let $T\colon \R_3[x]\to\R_3[x]$ be the operator 
	\[
	T(ax^3+bx^2+cx+d) = 2ax^3+(2b+3c+d)x^2+(2c+3d)x+2d
	\]
	Does exist a basis to $\R_3[x]$ such that: 
	\[ 
	[T^2-4T+4I]_B = 
	\begin{pmatrix}
	0 & 1 & 0 & 0\\
	0 & 0 & 1 & 0\\
	0 & 0 & 0 & 0\\
	0 & 0 & 0 & 0\\
	\end{pmatrix} \]}
	We notice that:
	\[ [T^2-4T+4I]_B = 
	\begin{pmatrix}
	0 & 1 & 0 & 0\\
	0 & 0 & 1 & 0\\
	0 & 0 & 0 & 0\\
	0 & 0 & 0 & 0\\
	\end{pmatrix}    = J_3(0) \oplus J_1(0) \]
	And since we know that the Jordan normal form of a transformation
	is unqiue up to order, it suffices to show that the Jordan normal form
	of $T^2-4T+4I$ is the same or different than $J_3(0) \oplus J_1(0)$.
	Making some calculations we get that represented by the standard
	basis:
	\[
		[T^2-4T+4I]_E = 
		\begin{pmatrix}
		0 & 0 & 0 & 0\\
		0 & 0 & 0 & 9\\
		0 & 0 & 0 & 0\\
		0 & 0 & 0 & 0\\
		\end{pmatrix} \underbrace{=}_{\text{denotion}} B 
	\]
	Which means that the characteristic polynomial of it is:
	\[
		p_{B}(x) = 
		\begin{vmatrix}
	-\lambda & 0 & 0 & 0\\
	0 & -\lambda & 0 & 9\\
	0 & 0 & -\lambda & 0\\
	0 & 0 & 0 & -\lambda\\
		\end{vmatrix} = 
		-\lambda(-\lambda^3+9*0) =
		\lambda^4
	\]
	So the only eigenvalue of $T^2-4T+4I$ is $0$, of algebraic multiplicity $4$.
	We know by a theorem we have proved in class that there must be at least:
	\[\dim\ker(T^2-4T+4I) = 3\]
	Jordan blocks in $T^2-4T+4I$'s Jordan normal form. This means that it
	can't have the Jordan normal form of $J_3(0) \oplus J_1(0)$, so we have
	shown that there does not exist a basis $B$ to $V$ such that 
	\[[T^2-4T+4I]_B = 
	\begin{pmatrix}
	0 & 1 & 0 & 0\\
	0 & 0 & 1 & 0\\
	0 & 0 & 0 & 0\\
	0 & 0 & 0 & 0\\
	\end{pmatrix}\]

	\newpage

	\section{The one with the ranks}
	\subsection{Let $A\in M_7(\R)$ such that:
	\[
		rk(A-I)^2 > rk(A-I)^3 = rk(A-I)^4
	\] and $rk(A)=3$. Calculate the Jordan normal form of $A$.}
	We know that $rk(A)=\dim\Im(A) = 3$ and since we also know that:
	\[
		\underbrace{\dim\Im(A)}_3 + \dim\ker(A) = \underbrace{\dim \R^7}_7
	\]
	We know that $\dim\ker(A) = 4$ which tells us that there are $4$ Jordan blocks
	in the Jordan normal form of $A$ with eigenvalue $0$. From similar 
	considerations we also see that:
	\[
		\dim\ker(A-I)^3 = 7 - rk(A-I)^3 = 7 - rk(A-I)^4 = \dim\ker(A-I)^4
	\]
	So we know that there are:
	\[ \dim\ker(A-I)^4 - \dim\ker(A-I)^3 = 0 \]
	Jordan blocks with eigenvalue $1$ of size at least $4$. Also:
	\[
		\dim\ker(A-I)^2 = 7 - rk(A-I)^2 < 7 - rk(A-I)^3 = \dim\ker(A-I)^3
	\]
	So there is at least $1$ Jordan block of size $3$ in the Jordan normal form 
	of $A$. Since as we have shown, there must be $4$ Jordan blocks in the Jordan
	normal form with eigenvalue $0$, and the sum of the order of the Jordan blocks
	must be equal to $7$ the only option for the Jordan normal form of $A$ is:
	\[
		J_3(1) \oplus J_1(0) \oplus J_1(0) \oplus J_1(0) \oplus J_1(0)
	\]  

	\newpage

	\section{The one with the inverses}
	\subsection{Let $\F$ be a field and $0\neq\lambda\in\F$. Find the Jordan 
	normal form of $J_n(\lambda)^{-1}$. No need to explicitly compute the
	inverse.}
	We can write the Jordan block $J_n(\lambda)$ as the sum of a scalar and
	a nilpotent matrix like so:
	\[
		J_n(\lambda) = \lambda I + J_n(0)
	\]
	Now we notice that since $\lambda \neq 0$ we can multiply both sided by 
	$\lambda^{-1} I$:
	\[
		\lambda^{-1} I J_n(\lambda) =\lambda^{-1} I(\lambda I + J_n(0)) = 
		I + \lambda^{-1}J_n(0)
	\]
	And that:
	\[
		(I - \lambda^{-1}J_n(0))(I + \lambda^{-1}J_n(0)) = 
		I - \lambda^{-2}J_n^2(0)
	\]
	Now since:
	\[
		(I + \lambda^{-2}J_n^2(0))(I - \lambda^{-2}J_n^2(0)) = 
		I - \lambda^{-4}J_n^4(0)
	\]
	We can keep going like:
	\[
		(I + \lambda^{-4}J_n^4(0))(I - \lambda^{-4}J_n^4(0)) = 
		I - \lambda^{-8}J_n^8(0)
	\]
	So we see know that:
	\[
		\left(\prod_{i=1}^{k}{(I + \lambda^{-2^k}J_n^{2^k}(0))}\right)
		(I - \lambda^{-1}J_n(0))(\lambda^{-1} I)J_n(\lambda) = I - \lambda^{-2^{k+1}}
		J_n^{2^{k+1}}(0)
	\]
	Since $J_n(0)$ is nilpotent of order $n-1$ we can choose $k\in\N$ such that
	$2^{k+1} > n$ and then:
	\[
		\left(\prod_{i=1}^{k}{(I + \lambda^{-2^k}J_n^{2^k}(0))}\right)
		(I - \lambda^{-1}J_n(0))(\lambda^{-1} I)J_n(\lambda) = I - \lambda^{-2^{k+1}}
		J_n^{2^{k+1}}(0) = I
	\]
	From linear algebra $1$ we know that a if $AB=I$ then $BA=I$ which means
	that we found the inverse of $J_n(\lambda)$:
	\[
		J_n(\lambda)^{-1} = 
		\left(\prod_{i=1}^{k}{(I + \lambda^{-2^k}J_n^{2^k}(0))}
		\right)(I - \lambda^{-1}J_n(0))(\lambda^{-1} I)
	\]

	\newpage

	\section{The one with the 9s}
	\subsection{Prove that exists a matrix $A\in M_n(\R)$ that satisfies:
	\[
		A^9+A^{99} = \begin{pmatrix}
		2 & 99 & 999\\
		0 & 2 & -9\\
		0 & 0 & 2\\
	\end{pmatrix}
	\] There's no need to find one explicitly.}
	Since the matrix we get by the calculation is of order $3$ we know that $A$
	is also of order $3$. Consider $A = J_3(\lambda)$, by a theorem we proved
	in class we can see that for the polynome $f(x) = x^9 + x^{99}$:
	\[
		f(A) = 
		\begin{pmatrix}
		f(\lambda) & f'(\lambda) & \frac{1}{2}f''(\lambda)\\
		0 & f(\lambda) & f'(\lambda)\\
		0 & 0 & f(\lambda)\\
		\end{pmatrix}
	\]

	\newpage

	\section{The one with high powers}
	\subsection{Find all the matrices $A\in M_4(\C)$ that satisfy $A^4-2A^2+1=0$
	up to similarity.}

	\newpage

	\section{The one with invariant subspaces}
	\subsection{Compute the invariant subspaces of a jordan block $J_n(\lambda)$.
	Use what we saw in the rehearsal about the invariant subspaces of $J_n(0)$.}

	\newpage

	\subsection{Let $T\in End(V)$ where $V$ is a complex vector space of finite
	dimension. Show that there is a finite amount of $T$-invariant subspaces iff
	$p_T(x) = m_T(X)$}

\newpage
	\section{The one with the Cauchy-Schwartz inequality}
	
	\subsection{Show that for all positive $x_1,\dots,x_n\in\R$:
	\[
	n^2 \le (x_1+\cdots+x_n)\left(\frac{1}{x_1}+\cdots+\frac{1}{x_n}\right)
	\]}
	Let $x_1,\dots,x_n$ be positive real numbers.
	Recall that the Cauchy-Shwartz inequality states that for any $v,u$ in an 
	inner product space, and specifically for $(\R^n,\langle,\rangle_{\std})$
	we get:
	\[
		\vert\langle v,u\rangle\vert^2 \le \langle v,v\rangle\langle u,u\rangle
	\]
	Since $x_1,\dots,x_n$ are positive we can take their roots and then for:
	\[ 
		v=(\sqrt{x_1},\dots,\sqrt{x_n}) \quad\text{and}\quad 
		u=(\frac{1}{\sqrt{x_1}},\dots,\frac{1}{\sqrt{x_n}})
	\]
	We get:
	\[
		\vert\langle v,u\rangle\vert^2 = 
		\vert\langle (\sqrt{x_1},\dots,\sqrt{x_n}),
		(\frac{1}{\sqrt{x_1}},\dots,\frac{1}{\sqrt{x_n}})\rangle\vert^2 =
		\vert n\vert^2 = n^2
	\]
	And:
	\[
		\langle v,v\rangle\langle u,u\rangle = 
		(x_1+\cdots+x_n)(\frac{1}{x_1}+\cdots+\frac{1}{x_n})
	\]
	Now substituting we get:
	\[
		n^2 \le (x_1+\cdots+x_n)\left(\frac{1}{x_1}+\cdots+\frac{1}{x_n}\right)
	\]
	Which is what we wanted to prove.

	\newpage

	\section{The one with the integral}
	Let $V=\R_2[x]$ and let:
	\begin{align*}
		\langle(p(x),q(x))\rangle_1 = 
		\int_{0}^{1}{p(x)q(x)\,dx} \\
		\langle(p(x),q(x))\rangle_2 = 
		\sum_{x\in\{-1,0,1\}}{p(x)q(x)}
	\end{align*}
	Two inner products on $V$, and let:
	\[
		W = \{p(x)\in V \vert p(x) = p(-x)\}
	\]
	
	\subsection{Find a basis for $W$ and complete it to a basis for $V$.}
	We know that $W\neq V$ and $W\neq 0$ so since $x^2,1\in W$ and are linearly
	independant we get that $\dim W=2$ and that
	\[
		B_W = \{x^2, 1\}
	\]
	is a basis for $W$. We can complete it to a basis for $V$ as such:
	\[
		B_V = \{x^2, 1, x\}
	\]
	
	\newpage

	\subsection{Apply the Gram-Schmidt process on $V$ relative to each of the
	inner products, find $W^\bot$ and the orthogonal projection $P_W$ on $W$.}
	According to $\langle ,\rangle_1$ we get:
	\begin{align*}
		u_1' &= v_1 = x^2 \\
		u_2' &= v_2 - \sum_{i=1}^{1}
		{\frac{\langle v_2,u'_i\rangle}{\langle u'_i,u'_i\rangle}u'_i} = 
		v_2 - \frac{\langle v_2,u'_1\rangle}{\langle u'_1,u'_1\rangle} u'_1 = 
		1 - \frac{\langle 1,x^2\rangle}{\langle x^2,x^2\rangle} x^2 = 
		1 - \frac{\frac{1}{3}}{\frac{1}{5}} x^2 = 1 - \frac{5}{3}x^2 \\
		u_3' &= v_3 - \sum_{i=1}^{2}
		{\frac{\langle v_3,u'_i\rangle}{\langle u'_i,u'_i\rangle}u'_i} = 
		x - \frac{\langle x,x^2\rangle}{\langle x^2,x^2\rangle}x^2 - 
		\frac{\langle x,1\rangle}{\langle 1,1\rangle}1 = 
		x - \frac{4}{3}x^2 - \frac{1}{2}
	\end{align*}
	Now to normalize the vectors:
	\begin{align*}
		u_1 &= \frac{u_1'}{\|u_1'\|} = 
		\frac{x^2}{\sqrt{\langle x^2,x^2\rangle}} = 2x^2
		\\
		u_2 &= \frac{u_2'}{\|u_2'\|} = 
		\frac{1-\frac{5}{3}x^2}
		{\sqrt{\langle 1-\frac{5}{3}x^2,1-\frac{5}{3}x^2\rangle}} = 
		\frac{3}{2}-\frac{5}{2}x^2
		\\
		u_3 &= \frac{u_3'}{\|u_3'\|} = 
		\frac{x - \frac{4}{3}x^2 - \frac{1}{2}}
		{\sqrt{\langle x - \frac{4}{3}x^2 - \frac{1}{2},x - \frac{4}{3}x^2 - 
		\frac{1}{2}\rangle}} = 
		\frac{30}{\sqrt{195}}x-\frac{40}{\sqrt{195}}x^2-\frac{15}{\sqrt{195}}
	\end{align*}
	Notice that when we applied Gram-Schmidt we first found an orthonormal basis
	for $W$, and since we know that $V = W \oplus W^\bot$ we get:
	\[ W^\bot = \Sp\{u_3\} = \Sp
	\left\{\frac{30}{\sqrt{195}}x-\frac{40}{\sqrt{195}}
	x^2-\frac{15}{\sqrt{195}} \right\} \]
	And as we know from the lectures for all $v\in V = \R_2[x]$ we get:
	\begin{align*}
		P_W(v) = \sum_{i=1}^{2}{\langle v(x),u_i \rangle u_i} &= 
		\langle v(x),2x^2 \rangle 2x^2 + \langle v(x),\frac{3}{2} - 
		\frac{5}{2}x^2 \rangle \left(\frac{3}{2}-\frac{5}{2}x^2\right) \\ &= 
		\left(\int_{0}^{1}{v(x) 2x^2\,dx}\right) 2x^2 + 
		\left(\int_{0}^{1}{v(x)\left(\frac{3}{2} - \frac{5}{2}x^2\right)\,dx}
		\right) \left(\frac{3}{2}-\frac{5}{2}x^2\right)
	\end{align*}
	
	\newpage
	
	According to $\langle ,\rangle_2$ we get:
	\begin{align*}
		u_1' &= v_1 = x^2 \\
		u_2' &= v_2 - \sum_{i=1}^{1}
		{\frac{\langle v_2,u'_i\rangle}{\langle u'_i,u'_i\rangle}u'_i} = 
		v_2 - \frac{\langle v_2,u'_1\rangle}{\langle u'_1,u'_1\rangle} u'_1 = 
		1 - \frac{\langle 1,x^2\rangle}{\langle x^2,x^2\rangle} x^2 = 
		1 - \frac{2}{2} x^2 = 1 - x^2 \\
		u_3' &= v_3 - \sum_{i=1}^{2}
		{\frac{\langle v_3,u'_i\rangle}{\langle u'_i,u'_i\rangle}u'_i} = 
		x - \frac{\langle x,x^2\rangle}{\langle x^2,x^2\rangle}x^2 - 
		\frac{\langle x,1\rangle}{\langle 1,1\rangle}1 = 
		x
	\end{align*}
	Now to normalize the vectors:
	\begin{align*}
		u_1 &= \frac{u_1'}{\|u_1'\|} = 
		\frac{x^2}{\sqrt{\langle x^2,x^2\rangle}} = 
		\frac{x^2}{\sqrt{2}}
		\\
		u_2 &= \frac{u_2'}{\|u_2'\|} = 
		\frac{1 - x^2}{\sqrt{\langle 1 - x^2,1 - x^2\rangle}} = 
		1-x^2
		\\
		u_3 &= \frac{u_3'}{\|u_3'\|} = 
		\frac{x}{\sqrt{\langle x, x\rangle}} = 
		\frac{x}{\sqrt{2}}
	\end{align*}
	Notice that when we applied Gram-Schmidt we first found an orthonormal basis
	for $W$, and since we know that $V = W \oplus W^\bot$ we get:
	\[ W^\bot = \Sp\{u_3\} = \Sp\left\{\frac{x}{\sqrt{2}}\right\} \]
	And as we know from the lectures for all $v\in V = \R_2[x]$ we get:
	\begin{align*}
		P_W(v) = \sum_{i=1}^{2}{\langle v,u_i \rangle u_i} &= 
		\langle v(x),\frac{x^2}{\sqrt{2}} \rangle \frac{x^2}{\sqrt{2}} + 
		\langle v(x),1-x^2 \rangle \left(1-x^2\right) \\ &= 
		\left(\sum_{x\in\{-1,0,1\}}
		{v(x)\left(\frac{x^2}{\sqrt{2}}\right)}\right)
		\left(\frac{x^2}{\sqrt{2}}\right) + 
		\left(\sum_{x\in\{-1,0,1\}}{v(x)(1-x^2)}\right)\left(1-x^2\right) \\ &= 
		\left(v(1)+v(-1)\right)\left(\frac{x^2}{2}\right) + v(0)
		\left(1-x^2\right)
	\end{align*}
	
	\newpage

	\subsection{Find the distance of $f(x)=x+1$ from $W$ according to each of
	the inner products.}
	We know that the distance of $f(x)=x+1$ from $W$ is the distance between
	$x+1$ and $P_W(x+1)$ which is the point ``closest'' to $x+1$ on $W$. So
	first we shall calculate $P_W(x+1)$ according to each of the inner product
	spaces:
	\begin{align*}
		P_W(x+1) &= \left(2+0\right)\left(\frac{x^2}{2}\right) + 1
		\left(1-x^2\right) = 1 \\
		P_W(x+1) &= \left(\int_{0}^{1}{(x+1)2x^2\,dx}\right) 2x^2 + 
		\left(\int_{0}^{1}{(x+1)\left(\frac{3}{2} - \frac{5}{2}x^2\right)\,dx}
		\right) \left(\frac{3}{2}-\frac{5}{2}x^2\right) \\ &=
		\left(\int_{0}^{1}{2x^3+2x^2\,dx}\right) 2x^2 +
		\left(\int_{0}^{1}{\frac{3}{2} - \frac{5}{2}x^2 + \frac{3}{2}x - 
		\frac{5}{2}x^3\,dx}\right) \left(\frac{3}{2}-\frac{5}{2}x^2\right) \\ &=
		\frac{7}{3}x^2 + \frac{19(3-5x^2)}{48} =
		\frac{112x^2}{48} + \frac{57 - 95x^2}{48} = \frac{17x^2+57}{48}
	\end{align*}
	So now according to $\langle ,\rangle_1$ we get that the distance is:
	\[
		\sqrt{\langle x+1,1\rangle} = \sqrt{\int_{0}^{1}{x+1\,dx}} = 
		\sqrt{\frac{3}{2}} = \frac{\sqrt{6}}{2}
	\]
	And now according to $\langle ,\rangle_2$ we get that the distance is:
	\[
		\sqrt{\langle x+1, \frac{17x^2+57}{48}\rangle} = 
		\sqrt{\sum_{x=-1,0,1}{\frac{17x^2+57(x+1)}{48}\,dx}} = 
		\sqrt{\frac{17+57+131}{48}} = \sqrt{\frac{205}{48}} = 
		\frac{\sqrt{615}}{12}
	\]
	\newpage

	\section{The one with the contraction}
	Let $V$ be a finite dimension inner product space and let $P\in\End(V)$ 
	be a contraction - that is $\forall v\in V(\|Pv\|\le\|v\|)$.
	
	\subsection{Show that $P$ is the orthogonal projection on its own image.}
	We will first show that $V = \im P \oplus \ker P$. Since $P$ is a projection
	we must have $P(v) = P^2(v)$ which implies $P(P(v)-v) = 0$ so 
	$P(v) - v = \epsilon \in \ker P$ and then $v = P(v) + (-\epsilon)$ which
	shows that $V = \im P + \ker P$. Now let $v\in \im P \cap \ker P$. We
	get that for some $u\in V$ that $P(u) = v$ and $P^2(u) = P(v) = 0$ since
	$v\in \ker P$. But since $P^2(u) = P(u)$ we get $v = 0$. This shows
	$V = \im P \oplus \ker P$. We also know that $V = \im P \oplus \im P^\bot$.
	This shows that $\dim\im P^\bot = \dim\ker P$. Now we will show that
	$\im P^\perp \subseteq \ker P$. Let $v\in\im P^\perp$ We know that 
	$P(v)\in\im P$ so:
	\[
		\langle P(v), v\rangle = 0
	\]
	This implies that:
	\[
		0 = \langle P(v), v\rangle = \frac{1}{4}
		\left(
			\|P(v)+v\|^2 - \|P(v)-v\|^2 + i\|P(v)-v\|^2 - i\|P(v)+v\|^2
		\right)
	\]
	This implies that:
	\[
		\|P(v)+v\| - \|P(v)-v\| = 0
	\]
	So using the reverse triangle identities we get:
	\[
		0 \le \|P(v)\| - \|v\| - \|P(v)-v\| \le \|P(v)+v\| - \|P(v)-v\| = 0
	\]
	So:
	\[
		\|P(v)\| - \|v\| = \|P(v)-v\|
	\]
	So from what we know $\|P(v)\| - \|v\|$ is a non-negative number and
	$\|P(v)\| \le \|v\|$ which implies  $\|P(v)\| - \|v\| = 0$ which gives:
	\[
		\|P(v)-v\| = 0 \Rightarrow P(v) - v = 0 \Rightarrow P(v) = v
	\]
	This shows that $v\in\im P$, and since $v\in\im P^\bot$ we know $v=0$.
	But we assumed that $v\notin\ker P$ so this can't be the case, and we get a
	contradiction. Which means that $\im P^\perp \subseteq \ker P$ and we know
	$\dim\im P^\bot = \dim\ker P$ so $\im P^\bot = \ker P$. so $P$ is an 
	orthogonal projection on its own image.
	
	\newpage

	\section{The one with the weird inequality}
	Let $V=\C_3[x]$ with the inner product 
	$\langle p(x),q(x)\rangle = \sum_{x=0}^{x=3}{p(x)\overline{q(x)}}$.
	
	\subsection{Find the minimal positive constant $C$ such that 
	for all $p\in V$:
	\[
	\| p(i) \| \le
	C\sqrt{\sum_{x=0}^{3}{\| p(x) \|^2}}
	\]}
	Notice that the following $\varphi \colon V \to \C$:
	\[
		\varphi(p(x)) = p(i)
	\]
	is a functional since for $\alpha\in\C$ and $p,q\in V$:
	\[
		\varphi(\alpha p + q) = (\alpha p + q)(i) = \alpha p(i) + q(i) = 
		\alpha \varphi(p) + \varphi(q)
	\]
	Using riesz representation theorem we get that exists $w$ such that:
	\[
		\varphi(p) = p(i) = \langle p, w\rangle
	\]
	Denote $w = a+bx+cx^2+dx^3$, 
	We see that for the basis $B = \{1,x,x^2,x^3\}$:
	\begin{align*}
		1 &= \varphi(1) = \langle 1, w\rangle = \overline{w(0)} + 
		\overline{w(1)} + \overline{w(2)} + \overline{w(3)} 
		= {w(0)} + {w(1)} + {w(2)} + {w(3)} = 1 \\
		i &= \varphi(x) = \langle x, w\rangle = 0\overline{w(0)} + 
		1\overline{w(1)} + 2\overline{w(2)} + 3\overline{w(3)} 
		\Rightarrow w(1)+2w(2)+3w(3) = -i \\
		-1 &= \varphi(x^2) = \langle x^2, w\rangle = 0\overline{w(0)} + 
		1\overline{w(1)} + 4\overline{w(2)} + 9\overline{w(3)} = 
		w(1 + 4w(2) + 9w(3)) = -1 \\
		-i &= \varphi(x^3) = \langle x^3, w\rangle = 0\overline{w(0)} + 
		1\overline{w(1)} + 8\overline{w(2)} + 27\overline{w(3)} \Rightarrow
		w(1) + 8w(2) + 27w(3) = i \\
	\end{align*}
	Solving this system of equations gives:
	\[
		(w(0),w(1),w(2),w(3)) = 
		\left(\frac{5}{3}i,\frac{5-5i}{2},-2+i,\frac{1}{2}-
		\frac{1}{6}i\right)
	\]
	And now we can solve for $p(i)$ for any $p\in V$. By Cauchy-Schwartz we get:
	\[
		\|p(i)\| = |\langle p,w\rangle| \le \|p(x)\|\|w(x)\| = 
		\sqrt{\sum_{x=0}^{3}{\| p(x) \|^2}} \sqrt{\sum_{x=0}^{3}{\| w(x) \|^2}}
	\]
	And we see that:
	\[
		\sqrt{\sum_{x=0}^{3}{\| w(x) \|^2}} = \frac{\sqrt{185}}{3}
	\]
	Since we know that the CS inequality can also be an equality we get that
	this is the minimal constant such that the inequality is satisfied and then:
	\[
		\boxed{C = \frac{\sqrt{185}}{3}}
	\]
	
	
	
	\newpage

	\section{The one with the invariance}
	Let $V$ be a finite dimension inner product space and $T\in\End(V)$. 
	
	\subsection{Show that $U\subseteq V$ is $T$-invariant iff $U^\bot$ is
	$T^*$-invariant}
	\underline{$U$ is $T$-invariant $\Rightarrow U^\bot$ is $T^*$-invariant:} \\ 
	Since $U$ is $T$-invariant we know that:
	\[
		T(U) \subseteq U
	\]
	Now suppose that $U^\bot$ is not $T^*$-invariant, that means that exists
	$u\in U^\bot$ such that $T^*(u)\notin U^\bot$, which means that:
	\[
		\langle v, T^*(u)\rangle \neq 0
	\]
	For some $v\in U$. This implies:
	\begin{align*}
		\langle T(v), u \rangle \neq 0
	\end{align*}
	But since $U$ is $T$-invariant we know that $T(v)\in U$, which implies
	that $u\notin U^\bot$ - that means that out assumption must be false
	so $U^\bot$ is $T^*$-invariant. \\
	\underline{$U$ is $T$-invariant $\Leftarrow U^\bot$ is $T^*$-invariant:} \\ 
	Since $U^\bot$ is $T^*$-invariant we know that:
	\[
		T^*(U^\bot) \subseteq U^\bot
	\]
	Now suppose that $U$ is not $T$-invariant, that means that exists
	$u\in U$ such that $T(u)\notin U$, which means that:
	\[
		\langle T(u), v\rangle \neq 0
	\]
	For some $v\in U^\bot$. This implies:
	\begin{align*}
		\langle u, T^*(v) \rangle \neq 0
	\end{align*}
	But since $U^\bot$ is $T^*$-invariant we know that $T^*(v)\in U^\bot$, 
	which implies that $u\notin U$ - that means that out assumption must be
	false so $U$ is $T$-invariant. \\
	
	\newpage

	\section{The one with $T*$}
	In the following sections find $T^*$

	\subsection{Let $(V,\langle , \rangle)$ be a finite dimension inner product
	space. Let $\alpha,\beta\in V$ and define $T=T_{\alpha,\beta}\in\End(V)$ 
	as such:
	\[
	T_{\alpha,\beta}(v) = \langle v, \alpha \rangle \beta
	\]}
	We see that for $T$ defined as such:
	\[
		\langle T(v), u \rangle = 
		\langle \langle v, \alpha \rangle \beta, u \rangle = 
		\langle v, \alpha \rangle \langle \beta, u \rangle = 
		\langle v, \alpha \overline{\langle \beta, u \rangle} \rangle = 
		\langle v, \alpha \langle u, \beta \rangle \rangle
	\]
	And since we know that:
	\[
		\langle T(v), u\rangle = \langle v, T^*(u)\rangle = 
		\langle v, \alpha \langle u, \beta \rangle \rangle
	\]
	We get that:
	\[
		T^*(u) = 
		\langle u, \beta \rangle \alpha
	\]

	\newpage

	\subsection{Let $V = (\Mat_n(\F),\langle,\rangle_{\std})$. 
	Let $Q\in\Mat_n(\F)$ be invertible and define $T=T_Q\in\End(V)$ as such:
	\[
	T_Q(A) = QAQ^{-1}
	\]}
	We see that from properties of trace:
	\begin{align*}
		\langle T(A), B \rangle &= 
		\langle QAQ^{-1}, B \rangle = \tr(QAQ^{-1}B^{t}) = 
		\tr(B^{t}QAQ^{-1}) \\ &= \tr(Q^{-1}B^{t}QA) = \tr(AQ^{-1}B^{t}Q) = 
		\langle A, (Q^{-1}B^{t}Q)^{t} \rangle
	\end{align*}
	And since we know that:
	\[
		\langle T(A), B\rangle = \langle A, T^*(B)\rangle = 
		\langle A, (Q^{-1}B^{t}Q)^{t} \rangle
	\]
	We get that:
	\[
		T^*(B) = 
		(Q^{-1}B^{t}Q)^{t} = Q^{t}B(Q^{-1})^{t}
	\]

	\newpage

	\subsection{Let $Tv=J_n(\lambda)v$ for $V=\F_n$ with 
	$\langle,\rangle_{\std}$}
	We see that for $T$ defined as such:
	\[
		\langle T(v), u \rangle = 
		\langle J_n(\lambda)v, u \rangle = (J_n(\lambda)v)^{t} u = 
		v^{t} J_n(\lambda)^{t} u = \langle v, J_n(\lambda)^{t} u \rangle
	\]
	And since we know that:
	\[
		\langle T(v), u\rangle = \langle v, T^*(u)\rangle = 
		\langle v, J_n(\lambda)^{t} u \rangle
	\]
	We get that:
	\[
		T^*(u) = 
		J_n(\lambda)^{t} u
	\]
	
	\newpage
	\section{The one with the adjoint operator}
	Let $a\in\C, |a| \neq 1$ and let $V$ be a finite dimension inner product
	space, $T\in\End(V)$
	\subsection{Show that if $T=aT^{*}$ then $T=0$}
	We first see that $T$ is normal since:
	\[
		TT* = aT^*T^* = T^*aT^* = T^*T
	\]
	This means that exists an orthonormal basis of eigenvectors of $T$ which
	we shall denote $B=(v_1,\dots,v_n)$ such that:
	\[
		[T]_B = \text{diag}(\lambda_1,\dots,\lambda_n)
	\]
	Where $\lambda_1,\dots,\lambda_n$ denote the corresponding eigenvalues.
	We see that for all $1\le i \le n$ that:
	\[
		T(v_i) = \lambda_i v_i
	\]
	But on the other hand that:
	\[
		T(v_i) = aT^*(v_i)
	\]
	We know from a theorem that if $v_i$ is an eigenvector of $T$ with 
	eigenvalue $\lambda_i$ then it is also an eigenvector of $T^*$ with
	eigenvalue $\overline{\lambda_i}$ so we get:
	\[
		\lambda_i v_i = a \overline{\lambda_i} v_i \Rightarrow
		\lambda_i = a \overline{\lambda_i}
	\]
	And in particular that:
	\[
		\vert \lambda_i \vert = \vert a \overline{\lambda_i} \vert \Rightarrow
		\vert \lambda_i \vert = \vert a \vert \vert \overline{\lambda_i} \vert
	\]
	But since also $\vert \lambda_i \vert = \vert \overline{\lambda_i} \vert$
	we get:
	\[
		\vert \lambda_i \vert (1-\vert a \vert) = 0
	\]
	And since $\vert a \vert \neq 1$ we get that $\lambda_i = 0$ which means
	that:
	\[
		[T]_B = 0
	\]
	So $T = 0$.
	
	\newpage
	
	\subsection{Show that if $T$ is normal then $\ker T = \ker(T - aT^{*})$}
	We can represent these tranformations and get that:
	\begin{align*}
		[T]_B &= \text{diag}(\lambda_1,\dots,\lambda_n) \\
		[T - aT^{*})]_B &= \text{diag}(\lambda_1 - a \overline{\lambda_1},\dots,
		\lambda_n - a \overline{\lambda_n})
	\end{align*}
	We know that the kernel of $v\in \ker(T)$ if and only if $v$ is in the span
	of $v_i$ with eigenvalue $0$, and that $v\in \ker(T - aT^{*})$ if and only
	if $v$ is in the span of $v_i$ with eigenvalue $0$ but we see:
	\begin{align*}
		\lambda_i = 0 &\Rightarrow \lambda_i = \overline{\lambda_i} = 0 
		\Rightarrow \lambda_i - a \overline{\lambda_i} = 0 \\
		\lambda_i - a \overline{\lambda_i} = 0 &\Rightarrow
		\lambda_i = a \overline{\lambda_i} \Rightarrow
		\vert \lambda_i \vert = \vert a \overline{\lambda_i} \vert \Rightarrow
		\vert \lambda_i \vert = \vert a \vert \vert \overline{\lambda_i} \vert
		\Rightarrow \vert \lambda_i \vert (1-\vert a \vert) = 0 \Rightarrow
		\lambda_i = 0
	\end{align*}
	Which shows that:
	\[
		\lambda_i = 0 \iff \lambda_i - a \overline{\lambda_i} = 0
	\]
	Which implies that the span of eignevectors from $B$ with eigenvalue
	$0$ in relation of $T$ will also have eigenvalue $0$ in relation to
	$T - aT^{*}$ so $\ker T = \ker(T - aT^{*})$ as wanted.
	
	
	\newpage
	
	\section{The one with the matrix}
	Given:
	\[
	A = 
		\begin{pmatrix}
		1 & -4 & 2 \\
		-4 & 1 & -2 \\
		2 & -2 & -2 
		\end{pmatrix}
	\]
	\subsection{Find an orthogonal matrix $O$ and a diagonal matrix $D$ such
	that $O^{T}AO=D$}
	We see that $A$ is symmetric so it must also be normal. From the spectral
	theorem for normal transformations we know that exists a basis $B$ to $V$
	such that $B$ is an orthogonal basis in realtion to the standard inner
	product and also comprises of eigenvectors of $A$. To find that $B$ we
	first will find the eigenvalues of $A$.
	\[
	A = 
		\left\vert
		\begin{pmatrix}
		1-\lambda & -4 & 2 \\
		-4 & 1-\lambda & -2 \\
		2 & -2 & -2-\lambda 
		\end{pmatrix}
		\right\vert = -(\lambda+3)(\lambda+3)(\lambda-6) = 0
	\]
	So the eigenvalues are $\lambda_1 = -3$ and $\lambda_2=6$. Now to find
	an orthogonal basis for $\ker(A-3)$ we do:
	\[
		\begin{pmatrix}
		4 & -4 & 2 \\
		-4 & 4 & -2 \\
		2 & -2 & 1 
		\end{pmatrix}
		\begin{pmatrix}
		x \\ y \\ z \\
		\end{pmatrix} = 0
	\]
	And we find that:
	\[
		\ker(A-3) = 
		\left\{
		a\begin{pmatrix} 1\\ 1\\ 0\\ \end{pmatrix} + 
		b\begin{pmatrix} -1\\ 0\\ 2\\ \end{pmatrix}
		\biggr\vert a,b\in\R
		\right\}
	\]
	So we can choose the orthonormal basis to be:
	\[
		B_{1} = 
		\left\{
		\frac{1}{\sqrt{2}} \begin{pmatrix} 1\\ 1\\ 0\\ \end{pmatrix}, 
		\frac{1}{\sqrt{18}} \begin{pmatrix} -1\\ 1\\ 4\\ \end{pmatrix}
		\right\}
	\]
	Now for $\ker(A+6)$ we do:
	\[
		\begin{pmatrix}
		-5 & -4 & 2 \\
		-4 & -5 & -2 \\
		2 & -2 & -8 
		\end{pmatrix}
		\begin{pmatrix}
		x \\ y \\ z \\
		\end{pmatrix} = 0
	\]
	And we find that:
	\[
		\ker(A+6) = 
		\left\{
		a\begin{pmatrix} 2\\ -2\\ 1\\ \end{pmatrix}
		\biggr\vert a\in\R
		\right\}
	\]
	So we can choose the orthonormal basis to be:
	\[
		B_{2} = 
		\left\{
		\frac{1}{3}\begin{pmatrix} 2\\ -2\\ 1\\ \end{pmatrix}
		\right\}
	\]
	We know that vectors of different eigenspaces are always orthogonal
	so we know that:
	\[
		B = B_{1} \cup B_{2} = 
		\left\{
		\frac{1}{\sqrt{2}} \begin{pmatrix} 1\\ 1\\ 0\\ \end{pmatrix}, 
		\frac{1}{\sqrt{18}} \begin{pmatrix} -1\\ 1\\ 4\\ \end{pmatrix}, 
		\frac{1}{3}\begin{pmatrix} 2\\ -2\\ 1\\ \end{pmatrix}
		\right\}
	\]
	And as we know from the unitary diagnolization theorem the orthogonal
	matrix that would diagonalize $A$ is the matrix with these columns so:
	\[
		O = \begin{pmatrix}
		\cfrac{1}{\sqrt{2}} & -\cfrac{1}{\sqrt{18}} & \cfrac{2}{3} \\
		\cfrac{1}{\sqrt{2}} & \cfrac{1}{\sqrt{18}} & -\cfrac{2}{3} \\
		0 & \cfrac{4}{\sqrt{18}} & \cfrac{1}{3}
		\end{pmatrix}
	\]
	And $D$ is just the matrix with the eigenvalues we found on the diagonal:
	\[
		D = \begin{pmatrix}
			-3 & 0 & 0\\
			0 & -3 & 0\\
			0 & 0 & 6\\
			\end{pmatrix}
	\]
	And:
	\[
		O^TAO = D
	\]
	
	\newpage
	
	\section{The one with the prove disprove}
	Let $T$ be an operator over a finite dimension inner product space. 
	Prove or disprove the following:
	\subsection{$T$ is unitary iff $T$ is invertible and exists an orthonormal
	basis $E$ such that $\|Te\|=1$ for all $e\in E$}
	This is false. Define $T\colon \R^2 \to \R^2$:
	\[
		T(1,0) = (1,0) \quad\text{and}\quad 
		T(0,1) = \left(\frac{1}{\sqrt{2}},\frac{1}{\sqrt{2}}\right)
	\]
	We can see that it is invertible, and exists the standard basis $E$ 
	which is orthonormal such that $\|T(e1)\| = \|T(e2)\| = 1$, yet if we 
	consider $T(1,1)$ we see that:
	\[
		\|(1,1)\| = \sqrt{2} \neq 
		\sqrt{2 + \sqrt{2}} = 
		\left\|\left(1+\frac{1}{\sqrt{2}},\frac{1}{\sqrt{2}}\right)\right\| = 
		\|T(1,1)\|
	\]
	So we found a vector $v = (1,1)$ such that:
	\[
		\|v\| \neq \|T(v)\|
	\]
	Which means that $T$ isn't unitary.
	
	
	\newpage
	
	\subsection{$T$ is unitary iff $\|Tv\|=1$ for all $v\in V$ such that 
	$\|v\|=1$}
	$(\Rightarrow)$ \\
	Let $T$ be unitary, then we know that for any $v\in V$ such that $\|v\| = 1$
	that:
	\[
		\|Tv\| = \|v\| = 1
	\]
	$(\Leftarrow)$ \\
	Suppose that $v'$ is an eigenvector of $T$ with eigenvalue $\lambda$.
	We can normalize $v'$ and consider:
	\[
		v = \frac{v'}{\|v'\|}
	\]
	This vetcor is also an eigenvector of $T$ with eigenvalue $\lambda$ so
	\[
		T(v) = \lambda v
	\]
	But since $\|v\| = 1$ we also know that:
	\[
		\|T(v)\| = \|\lambda\| \|v\| = 1 \Rightarrow \|\lambda\| = 1
	\]
	And we know that if for any eigenvalue $\lambda$ of $T$ that 
	$\|\lambda\|=1$  then $T$ is unitary. That means that we have just shown
	that $T$ is unitary.
	
	\newpage
	
	\subsection{$T$ is unitary iff for all orthonormal vectors $v,u$ then 
	$Tv,Tu$ are also orthonormal}
	This is true. From the Gram-Schmidt theorem we know that exists 
	$B=(v_1,\dots,v_n)$ an orthonormal basis for $V$, since any two vectors 
	$u,v\in B$ are orthonormal we get that any $T(u),T(v)\in T(B)$ are also 
	orthonormal. So the set $T(B)$ is also orthonormal. Suppose it werent 
	linearly independent we get that exist $(a_1,\dots,a_n)\neq 0$ such that:
	\[
		\sum_{i}{a_i T(v_i)} = 0
	\]
	Using Parseval's identity we get that:
	\[
		\left\|\sum_{i}{a_i T(v_i)}\right\| = 
		\sqrt{\sum_{i}{\|a_i\|}} = \|0\| = 0
	\]
	But this can only happen if $\forall i(a_i = 0)$ so $T(B)$ is linearly
	indepenedent and we got that $T$ sends the orthonormal basis $B$ to 
	$T(B)$ an orthonormal basis. Let $v=\sum_{i}{a_i v_i}\in V$ we see that 
	using Parseval's identity twice gives:
	\[
		\|T(v)\| = \left\|T\left(\sum_{i}{a_i v_i}\right)\right\| 
		= \left\|\sum_{i}{a_i T(v_i)}\right\| = 
		\sqrt{\sum_{i}{\|a_i\|}} = \|v\|
	\]
	We know that this is equivalent to $T$ being unitary which completes the
	proof.
	
	\newpage
	
	\section{The one with the inequality}
	Let $T$ be a operator over an inner product space $V$ and let 
	$TT^{*}=\alpha T + \beta I$ for some $\alpha,\beta\in\R$.
	\subsection{Show that $\alpha^2+4\beta \ge 0$}
	\underline{case $a=0$}
	\[
		TT^* = \beta I
	\]
	So \[ \beta T^{-1} = T^* \] This implies??? \\ 
	\underline{case $a \neq 0$} \\
	We know that $TT^*$ is self-adjoint, and since $\alpha,\beta\in\R$ 
	we get that:
	\[
		\alpha T + \beta I = (TT^*)
		= (TT^*)^* = (\alpha T + \beta I)^* = \alpha T^* + \beta I
	\]
	Because $a\neq 0$ we get:
	\[
		T = T^*
	\]
	Which means that:
	\begin{align*}
		T^2 &= \alpha T + \beta I \\ \Rightarrow 
		p(T) &= T^2 -  \alpha T - \beta I = 0
	\end{align*}
	This implies???
	
	\newpage
	
	\section{The one with the square root}
	Let $T$ be a self-adjoint operator over a finite inner product space.
	\subsection{Prove that exist non-negative operators $A,B$ such that:
	\[
		T = A-B,\quad \sqrt{TT^{*}}=A+B,\quad AB=BA=0
	\]}
	We know that if $T$ is self-adjoint which implies it is unitary 
	diagonalizable over $\R$, so exist $O\in O(n)$ and $D$ diagonal such that:
	\[
		O^{T}DO = [T]_C
	\]
	For $C$ the basis with the $i$th vector being the $i$th column of $O$.
	Since $T$ is self-adjoint we know that all of eigenvalues are real.
	We can denote them by the entries of the main diagonal of $D$ as such:
	$\lambda_i = D_{ii}$, and now we can define two matrices:
	\[
		(A')_{ij} = 
		\begin{cases}
			D_{ii} & i=j \land D_{ii} \geq 0 \\
			0 & \text{otherwise}
		\end{cases}
	\]
	And:
	\[
		(B')_{ij} = 
		\begin{cases}
			-D_{ii} & i=j \land D_{ii} \leq 0 \\
			0 & \text{otherwise}
		\end{cases}
	\]
	And define the operator $A,B$ as such:
	\[
		A(v) = (O^{T} A' O)(v) \quad\text{and}\quad B(v) = (O^{T} B' O)(v)
	\]
	We see that $A,B$ are self-adjoint since $O^* = O^{T}$ and since all of
	their eigenvalues by construction are non-negative we know that they are
	non-negative operators. We may notice that:
	\[
		A - B = O^{T} A' O - (O^{T} B' O) = O^{T} (A'-B') O = O^{T} D O = T
	\]
	And also that:
	\[
		\sqrt{TT^*} = \sqrt{O^{T} DD^* O} = O^{T} |D| O = O^{T} (A'+B') O = 
		A+B
	\]
	And since diagonal matrices commute under matrix multiplication and also
	$O^{T} = O^{-1}$ we see:
	\[
		AB = BA = A'B' = 0
	\]
	Since $A'$ multiplies all the rows different than $0$ in $B$ and all the rows
	that are zero in a scalar. This completes the proof.
	
	\newpage
	\section{The one with the polynomial}
	Let $T$ be a self-conjugate polynomial over the inner product space $V$, 
	with eigenvalues $\lambda_1,\dots,\lambda_n$.
	\subsection{For any $p(x)\in\F[x]$ show that the singular values of $p(T)$ 
	are $\vert p(\lambda_i)\vert$ up to inner order.}
	Since $p(x)$ is a polynomial we can write:
	\[
		p(x) = a_nx^n+a_{n-1}x^{n-1}+\cdots+a_0
	\]
	And:
	\begin{align*}
		p(T) &= a_nT^n+a_{n-1}T^{n-1}+\cdots+a_0I \\
		p(T)^* &= (a_nT^n)^* + (a_{n-1}T^{n-1})^* + \cdots + (a_0I)^* = 
		\overline{a_n}(T^*)^n + \overline{a_{n-1}}(T^*)^{n-1} + \cdots + 
		\overline{a_0}I
	\end{align*}
	Let $\lambda$ be an eigenvalue asscociated with an eigenvector $v$ of $T$. 
	We see that:
	\begin{align*}
		p(T)(v) &= (a_nT^n+a_{n-1}T^{n-1}+\cdots+a_0I)(v) \\ &= 
		a_nT(v)^n + a_{n-1}T(v)^{n-1} + \cdots + a_0I(v) \\ &= 
		a_n\lambda^{n}v + a_{n-1}\lambda^{n-1}v + \cdots + a_0v \\ &= 
		p(\lambda)(v)
	\end{align*}
	And:
	\begin{align*}
		p(T)^*(v) &= 
		(\overline{a_n}(T^*)^n + \overline{a_{n-1}}(T^*)^{n-1} + \cdots + 
		\overline{a_0}I)(v) \\ &=
		\overline{a_n}(T^*(v))^n + \overline{a_{n-1}}(T^*(v))^{n-1} + \cdots + 
		\overline{a_0}v\\ &=
		\overline{a_n\lambda^n}v + \overline{a_{n-1}\lambda^{n-1}}v + \cdots + 
		\overline{a_0}v\\ &=
		\overline{p(\lambda)}v
	\end{align*}
	So the eigenvalues of $p(T)^*p(T)$ are exactly 
	$\overline{p(\lambda)}p(\lambda)$ which is exactly $\|p(\lambda)\|^2$.
	By SVD we know that the singular values of $p(T)$ are the square roots
	of the eigenvalues of $p(T)^*p(T)$, or in other words, the singular values
	of $p(T)$ are $\|p(\lambda_i)\|$ up to order.
	
	\newpage
	
	\section{The one with the operator norm}
	\subsection{Show that $\|T^*T\|_{\text{op}} =  \|T\|_{\text{op}}^{2}$}
	We know that:
	\[
		\|T\|_{\mathrm{op}} = \sup_{\|x\|=1}{\|Tv\|} 
					= \sup_{\|x\|=1}{\sqrt{\ip{T(v)}{T(v)}}}
					= \sup_{\|x\|=1}{\sqrt{\ip{T^*T(v)}{v}}}
	\]
	From this follows that:
	\[
		\|T\|_{\op}^{2} = \sup_{\|x\|=1}{\ip{T^*T(v)}{v}}
	\]
	We may notice that $\ip{T^\ast Tv}{v}$ is a non-negtive number since it's
	just the norm of $\ip{Tv}{Tv}$ which means using Cauchy-Schwartz we get:
	\[
		\|T\|_{\op}^{2} = \sup_{\|x\|=1}{\ip{T^*T(v)}{v}} \le 
		\|T^\ast T\| \|x\| = \|T^\ast T\|
	\]
	So we got that $\|T\|_{\op}^{2} \le \|T^\ast T\|$. To prove the other
	direction we recall that we saw in the rehearsal that $T$ and $T^\ast$
	have the same singular values and in particular that:
	\[
		\|T\|_{\op} = \|T^\ast\|_{\op}
	\]
	So using this and properties of the norm we get:
	\[
		\|T^\ast T\| \le \|T^\ast\| \|T\| = \|T\|^2
	\]
	From this and the other inequality we get:
	\[
		\boxed{\|T^*T\|_{\text{op}} =  \|T\|_{\text{op}}^{2}}
	\]
	
	\newpage
	
	\section{The one with the reflexive bilinear form}
	Let $f$ be a reflexive bilinear form over a finite dimension $V$.
	\subsection{Show that if $\mathrm{rank}f=r$ then exist 
	$\phi_1,\tau_1,\dots,\phi_r,\tau_r\in V^*$ such that:
	\[
		f(x,y) = 
		\phi_{1}(x)\tau_{1}(y) + \cdots + \phi_{r}(x)\tau_{r}(y)
	\]}
 	We know that $\mathrm{rank} f = r$ so if we denote $A = [f]_B$ 
 	we get that $\dim\ker(A) = n - r$. Denote the basis for the
 	kernel at $B_k = \{e_{n-r+1}, \dots, e_n\}$ and complete it
 	to a basis for $V$ as such $B = \{e_{1}, \dots, e_n\}$ 
 	Now for each $u,v\in\mathrm{span}\{e_{1}, \dots, e_r\}$ we can 
 	denote:
	\begin{align*}
		u = \sum_{i=1}^{n}{\alpha_i e_i} \\
		v = \sum_{i=1}^{n}{\beta_i e_i}
	\end{align*}
	And now for any $u,v$ we see:
	\[
		f(u,v) = f\left(\sum_{i=1}^{n}{\alpha_i e_i}, v\right) = 
		\sum_{i=1}^{n}{\alpha_i f(e_i,v)} = 
		\sum_{i=1}^{r}{\alpha_i f(e_i,v)}
	\]
	The last equality is true since we know that:
	\[
		f(v,e_i) = [v]_B A [e_i]_B = [v]_B 0 = 0
	\]
	And since $f$ is reflexive we get $f(e_i,v) = 0$ as well.
	Let $\phi_i\in V^*$ where $1 \le r \le n$ be defined as:
	\[
		\phi_i\left(\sum_{j}{\alpha_j e_j}\right) = \alpha_i
	\]
	And:
	\[
		\tau_i(v) = f(e_i,v)
	\]
	These are trivially linear functionals. From the above calculations we see that:
	\[
		f(u,v) = \sum_{i=1}^{r}{\phi_i(u)\tau_i(v)}
	\]
	Which is what we wanted to prove.
	
	
	
	
	\newpage
	
	\section{The one where we show some things are unique}
	Let $V$ be a finite dimension inner product space over $\R$, $f$ be a 
	bilinear form over $V$.
	\subsection{Show that exists a unique $T\in\End(V)$ such that:
	\[
		f(u,v) = \ip{u}{T(v)}, \quad \forall u,v\in V
	\]}
	We know by Gram-Schmidt that $V$ has an orthonormal basis $B$ which 
	implies:
	\[
		\ip{v}{u} = \ip{[v]_B}{[u]_B}_{\std}
	\]
	So we need to show that exists a unique $T\in\End(T)$ such that:
	\[
		f(u,v) = \ip{[u]_B}{[T(v)]_B}_{\std}, \quad \forall u,v\in V
	\]
	Let:
	\[
		[T(v)]_B = [f]_B [v]_B \in \End(T)
	\]
	We see that:
	\[
		f(u,v) = [u]_B^* [f]_B [v]_B = [u]_B^* [T(v)]_B
		= \ip{[u]_B}{[T(v)]_B}_{\std}
	\]
	This shows that exists a $T$ as wanted, we will now show it's unique.
	Let $S\neq T$ such that:
	\[
		\ip{u}{T(v)} = \ip{u}{S(v)}
	\]
	From this follows that:
	\begin{align*}
		\ip{u}{T(v)} - \ip{u}{S(v)} &= 0 \\ \Rightarrow
		\ip{u}{T(v)-S(v)} &= 0 \\ \Rightarrow
		\ip{u}{(T-S)(v)} &= 0
	\end{align*}
	Since $T \neq S$ exists $v'$ such that $(T-S)(v')\neq 0$ and for all 
	$u\in V$ and specifivally for $T(v')$ we get:
	\[
		\ip{T(v')}{(T-S)(v)} = \ip{T(v')}{T(v')} = \|T(v')\|^2 = 0
	\]
	But since $T(v') \neq 0$ this can't be. This implies that $T$ is indeed
	unique.
	
	\newpage
	\section{The one with the inner product}
	Let $A\in \mathrm{Mat}_n(\R)$ be symmetric and also satisfy:
	\[
		(A^2-5A+7I)^3 = I
	\]
	\subsection{Show that:
	\[
		f(x,y) = x^{T} A y
	\]
	is an inner product on $\R^2$}
	To show that this is an inner product on $\R^2$ we need to show that $f$
	is positive-definite. Since $A$ is symmetric and real it is self conjugate.
	By a theorem from class we know that if it is self conjugate and all of 
	its eigenvalues are positive then $A$ is positive definite and then
	$f$ is an inner product. Let $\lambda$ be an eigen value of $A$ with
	a corresponding eigenvector $v_\lambda$ such that:
	\[
		Av_{\lambda} = \lambda v_{\lambda}
	\]
	Since $A$ satisfies the above equality we see that:
	\[
		v_{\lambda} = Iv_{\lambda} = (A^2-5A+7I)^3 v_{\lambda} = 
		(A^2-5A+7I)^2 
		(\lambda^2 v_{\lambda} - 5\lambda v_{\lambda} + 7 v_{\lambda}) = 
		(A^2-5A+7I)^2 (\lambda^2 - 5\lambda + 7)v_{\lambda}
	\]
	Consider the real polynimial $g(x) = x^2-5x+7$. We see that its discriminant
	is $\sqrt{25 - 28}$ which means it doens't have any roots. Since the
	coefficient of $x^2$ is positive that means that $g(x) > 0$ for any
	real $x$ and specifically that $g(\lambda) > 0$ which gives:
	\[
		(A^2-5A+7I)^3 v_{\lambda} = 
		(A^2-5A+7I)^2 g(\lambda) v_{\lambda} = 
		(A^2-5A+7I)   g(\lambda)g(\lambda) v_{\lambda} = 
					  g(\lambda)g(\lambda)g(\lambda) v_{\lambda}
	\]
	This implies that $1 = g(\lambda)^3$. The only real solution to that
	equation is $g(\lambda) = 1$, considering the equation $g(x) = 1$
	we see:
	\[
		g(x) = 1 \Rightarrow x^2-5x+7 - 1 = 0 \Rightarrow (x-2)(x-3) = 0
	\]
	So $\lambda = 2$ or $\lambda = 3$. This implies that all the eigenvalues
	of $A$ are positive and as we said that implies that $f$ is an inner
	product and completes the proof.
	
	\newpage
	
	\section{The one with equivalence}
	\subsection{How many bilinear forms are there over $\R^2$ for which 
	exists $0\neq x\in\R^2$ such that $f(x,x) > 0$ up to isomorphism?}
	Let $B$ be a bilinear form and $E$ be a basis for $\R^2$. We know
	that each bilinear form defines a quadratic form $q$. We also
	know that any quadratic form can be represented by a symmetric matrix $S_q$. 
	Since $S_q$ is symmteric we can use Sylvester's law of inertia and get that 
	each $S_q$ is uniquely congruent to a matrix of the form:
	\[
		I_{n_+} \oplus -I_{n_-} \oplus O_{n_0}
	\]
	We now need to consider all the options that are not negative semi-definite
	so there would be an $x \neq 0$ such that $f(x,x) > 0$.
	Since we are talking about a $2 \times 2$ matrix here there are only $5$
	such options:
	\begin{align*}
		\begin{pmatrix}
			1 & 0 \\
			0 & 1
		\end{pmatrix}
		\quad
		\begin{pmatrix}
			1 & 0 \\
			0 & 0
		\end{pmatrix}
		\quad
		\begin{pmatrix}
			0 & 0 \\
			0 & 1
		\end{pmatrix}
		\quad
		\begin{pmatrix}
			1 & 0 \\
			0 & -1
		\end{pmatrix}
		\quad
		\begin{pmatrix}
			-1 & 0 \\
			0 & 1
		\end{pmatrix}
	\end{align*}
	So there are exactly $5$ bilinear forms over $\R^2$ for which exists $x\neq0$
	such that $f(x,x) > 0$ up to isomorphism.
	
	
	\newpage
	
	\section{The one with the two}
	Let $f$ be a symmetric bilinear form over a real finite-dimension vector 
	space $V$.
	\subsection{Prove that if $W\subseteq V$ is a subspace such that $f\vert_W$ 
	is positive definite, then $\dim W \le n_{+}(f)$}
	Denote $\dim (W) = k$ and
	let $B_W = (v_1,\dots,v_{k})$ be a basis for $W$, and
	$B_v = (v_1,\dots,v_{k},\dots v_n)$ be a basis for $V$.
	We know that $f$ is a symmteric bilinear form, which implies that $[f]_B$ is 
	symmetric. So by Sylvester's law of inertia we get that exists a diagonal 
	matrix $D$ and an invertible matrix $S$ such that $[f]_B$ is congruent to $D$ 
	and:
	\[
		S^{T}[f]_BS = D
	\]
	We also know by the orthogonal diagonalization theorem for real symmetric
	matrices that exists $O\in O(n)$ such that:
	\[
		O^{T}[f]_BO = D'
	\]
	Where $D$ is diagonal with the eigenvalues of $[f]_B$ on its diagonal.
	Since we know that $f\vert_W$ is positive definite that means that all
	of its eigenvalues are positive and moreover that $D'_{11},\dots,D'_{kk}$
	are the eigenvalues of $W$ and thus positive.
	Since the positive values on the diagonal corresponds to $n_+(D')$
	we get that $n_+(D') \geq \dim W$ and since Sylvester's character
	and the rank don't change between congruent matrices
	\footnote{Notice that $D$ and $D'$ are congruent because congruency
	is an equivalence relation} 
	we get that
	$n_+(f) \geq \dim W$ too, which is exactly what we wanted to prove.
	
	\newpage
	
	\subsection{Let $B = (b_1,\dots,b_n)$ be a Sylvester basis such that:
	\[
		[f]_B = I_{n_{+}} \oplus (-I_{n_{-}}) \oplus O_{n_{0}}
	\]
	Does it necessarily follow that 
	$W\subseteq \mathrm{sp}\{b_1,\dots,b_{n_{+}}\}$}
	No. Let $V = \R^2$ and $E = \{e_1,2e_2\}$ be a basis to $\R^2$ such that
	$e_1,e_2$ are the vectors from the standard basis and:
	\[
		[f]_B = 
		\begin{pmatrix}
			1 & 0 \\
			0 & -1 \\
		\end{pmatrix}
	\]
	We see that $E$ is a Sylvester's basis but if we consider:
	\[
		W = \mathrm{sp}\{(1,1)\}
	\]
	Then $W$ is indeed a linear subspace of $V$ and if we let $w=(a,a)\in W$
	we see that:
	\[
		\ip{[f]_B[w]_B}{[w]_B} = 
		\begin{pmatrix}
		2a & a
		\end{pmatrix}
		\begin{pmatrix}
			1 & 0 \\
			0 & -1 \\
		\end{pmatrix}
		\begin{pmatrix}
		2a \\ a
		\end{pmatrix} = 3a^2
	\]
	And of course $f\vert_B$ is also symmetric so by a theorem it is 
	positive definite, yet as we can easily see $W \nsubseteq \mathrm{sp}\{e_1\}$
	
	\newpage
	
	\section{The one where we prove... or disprove?}
	Let $A$ be a symmetric real matrix of order $n\times n$ over $V$.
	\subsection{$A$ is non-negative iff $\Delta_i(A) \geq 0$ for all 
	$i = 1,\dots,n$. Consider both directions}
	\underline{$(\Leftarrow)$} \\
	This is false because we can look at the matrix over $\R$:
	\[
	A = 
		\begin{pmatrix}
			0 & 0 \\
			0 & -1
		\end{pmatrix}
	\]
	We see that:
	\[
		\Delta_1(A) = 0 \quad\text{and}\quad \Delta_2(A) = \det(A) = 0
	\]
	But still we see that is is symmetric and it has a negative eigenvalue.
	\\
	\underline{$(\Rightarrow)$} \\
	Assume that $A$ is non-negative. This clearly implies that any principle 
	minor corresponding to $\Delta_i(A)$ is also non-negative, which means that
	all of its eigenvalues are non-negative. Since the determinant of any
	principle minor is the product of its eigenvalues we get that for all
	$i = 1,\dots,n$ that $\Delta_i(A) \geq 0$ which is what we wanted to prove.
	
	\newpage
	
	\iffalse
	\section{The one with the extra difficulty}
	\subsection{Let $f$, $g$ be inner products on $V$ a finite-dimension vector 
	space. Prove that exists a basis that is orthonormal by $f$ and orthogonal 
	by $g$.}
	\fi
	
	\newpage
	
	\section{Past Tests}
	\textbf{Find the Jordan normal form for the following matrix:
	\[
	A := \begin{pmatrix}
	0 & -1 & -1 & 0\\
	1 & -2 & -1 & -1\\
	0 & 0 & -1 & 1\\
	0 & 0 & 0 & -1\\
	\end{pmatrix} \in M_4(\R)
	\]
	And find an invertible $P$ such that  $P^{-1}AP$ is in Jordan normal form}\\
	First we will start by finding the eigenvalues of $A$ by using the
	charecteristic polynomial:
	\[
		\det(A - \lambda I) = (\lambda + 1)^4
	\]
	So the eigenvalues are $\lambda = -1$ with an algebraic multiplicity of $4$
	and geometric multiplicity of $2$. This implies that there are $2$ Jordan
	blocks with eigenvalues of $-1$ and their sum is $4$ which implies that:
	\[
		\mathrm{JordanNormalForm}(A) = J_2(-1) \oplus J_2(-1)
	\]
	To find $P$ we need to find the generalized eigenspaces of $A$ in relation
	to $\lambda = -1$. We see that:
	\begin{align*}
		\ker(A+I) &= \ker
		\left(\begin{matrix}
		1 & -1 & -1 & 0 \\
		1 & -1 & -1 & -1 \\
		0 & 0 & 0 & 1 \\
		0 & 0 & 0 & 0
		\end{matrix}\right) = 
		\Sp\left\{
		\begin{pmatrix}
			1\\
			1\\
			0\\
			0\\
		\end{pmatrix},
		\begin{pmatrix}
			0\\
			1\\
			-1\\
			0\\
		\end{pmatrix}
		\right\} \\
		\ker(A+I)^2 &= \ker
		\left(\begin{matrix}
		0 & 0 & 0 & 0 \\
		0 & 0 & 0 & 0 \\
		0 & 0 & 0 & 0 \\
		0 & 0 & 0 & 0
		\end{matrix}\right) = \R^4
	\end{align*}
	Now to find the Jordan basis we just need to choose $v_2, v_4$ to
	be linearly independent vectors in $\ker(A+I)^2$ that are not in
	$\ker(A+I)$. We can choose:
	\[
		v_2 = \begin{pmatrix}1 \\ 0\\ 0\\ 0\\ \end{pmatrix} \quad
		v_4 = \begin{pmatrix}0 \\ 0\\ 0\\ 1\\ \end{pmatrix}
	\]
	And now the Jordan base will be:
	\[
		B_J = 
		\left\{
		\begin{pmatrix}0 \\ -1\\ 1\\ 0\\ \end{pmatrix},
		\begin{pmatrix}1 \\ 0\\ 0\\ 0\\ \end{pmatrix},
		\begin{pmatrix}1 \\ 1\\ 0\\ 0\\ \end{pmatrix},
		\begin{pmatrix}0 \\ 0\\ 0\\ 1\\ \end{pmatrix}
		\right\}
	\]
	And since these are the columns of $P$ we know that:
	\[
		P = 
		\left(\begin{matrix}
		0 & -1 & 1 & 0 \\
		1 & 0 & 0 & 0 \\
		1 & 1 & 0 & 0 \\
		0 & 0 & 0 & 1
		\end{matrix}\right)
	\]
	\newpage
	\noindent
	\textbf{Prove that the matrices $A^{-1}$ and $A^3$ are similar.} \\
	We see that:
	\begin{align*}
		A^{-1} &= PJ^{-1}P^{-1} \\
		A^{3}  &= PJ^{-3}P^{-1}
	\end{align*}
	Whis means it is sufficient to prove that $A^{-1}$ and $A^3$ are similar.
	Indeed we see that:
	\[
		A^{-1} = \left(\begin{matrix}
		-1 & -1 & 0 & 0 \\
		0 & -1 & 0 & 0 \\
		0 & 0 & -1 & -1 \\
		0 & 0 & 0 & -1
		\end{matrix}\right) \quad
		A^{3} = \left(\begin{matrix}
		-1 & 3 & 0 & 0 \\
		0 & -1 & 0 & 0 \\
		0 & 0 & -1 & 3 \\
		0 & 0 & 0 & -1
		\end{matrix}\right)
	\]
	Both these matrices have the same eigenvalues with the same algebraic
	and geometric multiplicity so their Jordan normal form is identical
	which means they are similar.
	
	\newpage
	\noindent
	\textbf{Let $V$ be an inner product space over $\C$ and let 
	$T \colon V \to V$ a linear transformation such that:
	\[
		(7I - T)T^* = 10I
	\]
	Show that $T$ is self-adjoint.} \\
	We get that:
	\[
		T^* = \frac{TT^* + 10I}{7}
	\]
	And now:
	\[
		T = \frac{T^*T + 10I}{7}
	\]
	Since we know that $TT^*$ is always self adjoint we get that $T = T^*$ so
	$T$ is self-adjoint. \\
	\textbf{Let $\lambda$ be an eigenvalue of $T$. Show that 
	$2 \le \lambda \le 5$} \\
	We know that all the eigenvalues of $T$ are real since it is a normal 
	operator and by the spectral theorem for normal operators exists a basis
	$B$ of orthonormal eigenvectors of $T$ to $V$. By that basis we get:
	\[
		[T]_B = \mathrm{diag}\{\lambda_1,\dots,\lambda_n\}
	\]
	We can also notice that since the basis is made up by orthonormal vectors:
	\[
		\mathrm{diag}\{\overline{\lambda_1},\dots,\overline{\lambda_n}\} =
		[T]^*_B = [T^*]_B = [T]_B = 
		\mathrm{diag}\{\lambda_1,\dots,\lambda_n\}
	\]
	Which means that for any $1 \le i \le n$ we get 
	$\lambda_i = \overline{\lambda_i}$ or in other words that $\lambda_i\in\R$.
	Suppose $\lambda$ was an eigenvalue of $T$for $v\in V$. We get that:
	\begin{align*}
		(7I - T)T^*(v) &= 10I(v) \\
		\lambda (7I - T)v &= 10v \\
		\lambda (7v - \lambda v) &= 10v \\
		7\lambda v - \lambda^2 v - 10v &= 0 \\
		7\lambda - \lambda^2  - 10 &= 0 \\
		(\lambda - 2)(\lambda - 5) &= 0
	\end{align*}
	This means that $2 \le \lambda \le 5$ as wanted. \\
	\textbf{Let $v \in V$ such that $\|v\| = 1$. Prove that 
	$2 \le \|T(v)\| \le 5$.} \\
	It might be hard considering the norm only, but luckily we can use the inner
	product and see that:
	\[
		\|T(v)\|^2 = \ip{T(v)}{T(v)} = 
		\ip{\sum_{i}{\ip{v}{e_i}Te_i}}{\sum_{i}{\ip{v}{e_i}Te_i}} = 
		\sum_{i}{\|\ip{v}{e_i}\|^2\|\lambda_i\|^2} \le 
		25 \|v\| = 25
	\]
	We get that $\|T(v)\| \le 5$ and similarly we can ge that $2 \le \|T(v)\|$
	and finish the proof.
	
	\newpage
	\noindent
	\textbf{Let $A_1$, $A_2$ be two real symmetrical and invertible matrics of 
	order $2$. Show that if they are not congruent exist $D_1, D_2$ diagonal
	and $P$ invetible matrix such that:
	\begin{align*}
		P^TA_1P = D_1 \\
		P^TA_2P = D_2
	\end{align*}} \\
	Since $A_1$ and $A_2$ are real matrices of order $2$ invetible and 
	symmetrical, by Sylvester's law of inertia at least one of them
	must be congruent to $\pm I$. Which means that exists $P$ invertible 
	such that:
	\[
		P^T A_1 P = \pm I \quad \text{and} \quad 
		P^T A_2 P = B
	\]
	We know that $B$ is symmetrical and invertible so exists an orthogonal
	matrix $Q$ that diagonlaizes it, and then we get:
	\begin{align*}
		(PQ)^t A_1 (PQ) &= Q^T \pm I Q = \pm I = D_1 \\
		(PQ)^t A_2 (PQ) &= Q^T B Q = D_2
	\end{align*}
	Which is what we wanted to prove.
	
	\newpage
	\noindent
	\textbf{Define the following vector space over $\R$
	\[
		V = \{A \in M_2(\C) \vert A = A^*\}
	\]
	And also define $q \colon V \to \R$ as 
	$
		A \mapsto 2 \det(A)
	$.
	Show that $q$ is a quadratic form and find its signature.} \\
	We can notice that:
	\[
		V \left\{
		\begin{pmatrix}
			a & b + ci\\
			b - ci & a\\
		\end{pmatrix}
		 \biggr\vert a,b,c\in\R
		 \right\}
	\]
	Which means that:
	\[
		q(A) = 2(a^2 - b^2 - c^2)
	\]
	Notice that even though it may seem as if we are mapping a $2 \times 2$
	matrix to a real number, we are basically sending the vector $(a,b,c)$
	from $\R^3$ to $2(a^2 - b^2 - c^2)$ and so the symmetrical bilinear
	form that defines $q$ according to the basis:
	\[
		B = 
		\left\{
			\begin{pmatrix}
				1 & 0\\
				0 & 1\\
			\end{pmatrix},
			\begin{pmatrix}
				0 & 1\\
				1 & 0\\
			\end{pmatrix},
			\begin{pmatrix}
				0 & -i\\
				i & 0\\
			\end{pmatrix}
		\right\}
	\]
	is the matrix:
	\[
	[f]_B = \begin{pmatrix}
		2 & 0 & 0\\
		0 & -2 & 0\\
		0 & 0 & -2\\
	\end{pmatrix}
	\]
	This shows that $q$ is a quadratic form and we can see that the signature
	of it is just $(1,2)$ with rank $3$.
	
	\newpage
	\noindent
	\textbf{Let $V$ be a vector space over $\C$ and let $D,T\in\End(v)$ such
	that $TD = DT$ and $D$ is diagonalizable. Prove that the eigenspaces of $D$
	are $T$-invariant.} \\
	Let $\lambda$ be an eigenvalue of $D$. The corresponding eigenspace is:
	\[
		V_\lambda = \{v \in V \mid D(v) = \lambda v \}
	\]
	To show this space is $T$-invariant by definition we can consider 
	$v\in V_\lambda$ and indeed:
	\[
		D(T(v)) = T(D(v)) = T(\lambda v) = \lambda T(v)
	\]
	Which implies that $T(v)\in V_\lambda$ as wanted. \\
	\textbf{Show that exists a basis $B$ of $V$ such that $[D]_B$ is diagonal
	and $[T]_B$ is in Jordan normal form.} \\
	In the previous exercise we have shown that the eigenspaces of $D$ are
	$T$-invariant, and since we know they also span $V$ we get that:
	\[
		V = V_{\lambda_1} \oplus \cdots \oplus V_{\lambda_k}
	\]
	Is a direct product decomposition of $T$. From Jordan's theorem
	for every $V_{\lambda_i}$ exists a basis $B_i$ that changes $T$ to a normal
	form, so we get that the ordered union $B = \bigcup B_i$ is a basis
	such that $[T]_B$ is in Jordan normal form and since it is made of
	eigenvectors of $D$ it of course diagonalizes $D$ as well.
	
	\newpage
	\noindent
	\textbf{For $\R^3$ coupled with the standard inner product we denote the
	norm and metric it induces as $|\cdot|$ and $d(\cdot,\cdot)$ 
	correspondingly. Let:
	\[
	A = \begin{pmatrix}
			4 & 3 & 0\\
			2 & -1 & 0\\
			0 & 0 & 2\\
		\end{pmatrix}
	\]
	Find the value of $\max\{d(Av,v) \mid v\in\R^3 \land \|v\| = 1\}$} \\
	This is a clearly a question about SVD decomposition. We see that:
	\[
		\max\{d(Av,v) \mid v\in\R^3 \land \|v\| = 1\} =
		\max\{\|(A-I)(v)\| \mid v\in\R^3 \land \|v\| = 1\}
	\]
	We denote:
	\[
		B = \begin{pmatrix}
			3 & 3 & 0\\
			2 & -2 & 0\\
			0 & 0 & 1\\
		\end{pmatrix}
	\]
	To find it's singular values we calculate:
	\[
		B^TB = 
		\left(\begin{matrix}
			13 & 5 & 0 \\
			5 & 13 & 0 \\
			0 & 0 & 1
		\end{matrix}\right)
	\]
	This is a block matrix so we can easily find that its eigenvalues are
	\[
		(\lambda_1, \lambda_2, \lambda_3) = (18,8,1)
	\]
	Thus the singular values of $B$ are:
	\[
		(\sigma_1, \sigma_2, \sigma_3) = (\sqrt{18},\sqrt{8},\sqrt{1})
	\]
	So finally:
	\[
		\max\{d(Av,v) \mid v\in\R^3 \land \|v\| = 1\} = \sqrt{18}
	\]
	\textbf{Find a unit vector $v\in V$ that gives that maximum} \\ 
	This is just the unit eigenvector corresponding to the eigenvalue $18$ of
	$B^TB$. We can easily find one:
	\[
		\tilde{v} = \begin{pmatrix} 1 \\ 1 \\ 0 \end{pmatrix}
	\]
	And now normalizing it we get:
	\[
		v = \frac{1}{\sqrt{2}}\begin{pmatrix} 1 \\ 1 \\ 0 \end{pmatrix}
	\]
	
	\newpage
	\noindent
	\textbf{Let $A\in U_n$. Prove that $|\tr(A)| = n$ if and only if $A$ is
	a scalar matrix} \\
	We know that $A$ is unitary so it is also normal and from the spectral
	theorem for normal matrices it is diagonalizable. Since it is unitary
	we also know that all of its eigenvalues are on the unit circle which means
	that:
	\[
		|\tr(A)| = \left|\sum_i{\lambda_i}\right| \le \sum_i{|\lambda_i|}
		= n
	\]
	We know that the equality holds if and only if all the vectors are
	in the same direction. Thus $|\tr(A)| = n$ if and only if $A$ is a
	scalar matrix.
	\\ \noindent
	\textbf{Let $B\in O_n$. Prove that $|\tr(A)| = n$ if and only if $A$ is
	$\pm I$} \\
	Since $A\in O_n$ it is also in $U_n$ which implies that $|\tr(A)| = n$ 
	if and only if $A$ is a scalar matrix. We can see that:
	\[
		n = |\tr(A)| = |\tr(cI)| = |c*n| = |n||c|
	\]
	This implies that $|\tr(A)| = n$ if and only if $A$ is $\pm I$.
	
	\newpage
	\noindent
	\textbf{Let $V$ be a finite dimension vector space over $\R$ and let
	$\varphi,\phi \colon V \to \R$ be linear functionals. Define:
	\[
		q(v) = \varphi(v)\phi(v)
	\]
	Prove that $q(v)$ is a quadratic form on $V$.} \\
	We know that:
	\[
		f(v,u) = \varphi(v)\phi(u)
	\]
	Is a bilinear form, and since the quadratic form it induces is $q$ we get
	that $q$ is a bilinear form by definition. Usually when working with
	bilinear and quadratic forms it's convenient to find the symmetric bilinear
	form that induces $q$ and it is:
	\[
		g(v,u) = \frac 12 \left(\varphi(v)\phi(u) + \varphi(u)\phi(v)\right)
	\]
	\textbf{Show that $n_+ \le 1$ and also that $n_- \le 1$} \\
	Denote $\dim(V) = n$. We know that $\varphi,\phi$ are linear functionals
	and thus, 
	\begin{align*}
		\mathrm{null} \varphi \geq n - 1 \\
		\mathrm{null} \phi \geq n - 1
	\end{align*}
	The intuition is that we want to eventually get to a matrix that has a 
	block of $0_{n-2}$ so we can consider $K = \ker \varphi \cup \ker \phi$. 
	We know that $\dim K \geq n - 2$. Complete it with some space $W$ to $V$ 
	as such:
	\[
		V = W \oplus K
	\]
	We can now make a basis for $V$ as the union of the basis for $W$ and the
	basis for $K$ denoted $B= B_W \cup B_K$. The representing matrix of $g$
	with this basis will be:
	\[
		B = [g|_{W \times W}]_{B_W} \oplus 0_{n - 2}
	\]
	This tells us that $n_- + n_+ \le 2$. Suppose $n_+ = 2$ this would mean
	that $\dim W = 2$ and that for any $0 \neq w\in V - K$ that $q(w) > 0$
	but that can't be the case since $\dim \ker \phi \geq n-1$ so we can
	choose a vector that is not in $W$ such that $q(w) = 0$ in contradiction.
	Thus $n_-,n_+ \le 1$ as wanted. \\
	\textbf{Find examples for functionals that gives all $4$ options $(0,0),
	(1,0),(0,1),(1,1)$} \\
	We can choose the functionals for $(n_+,n_-)$ such that:
	\begin{align*}
		\varphi(a,b) &= n_+a + n_-b \\
		\phi(a,b) &= n_+a - n_-b
	\end{align*}
	
	
	
	
\end{document}