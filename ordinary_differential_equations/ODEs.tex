\documentclass{article}
\usepackage{amssymb,amsfonts,amsmath,calc,tikz,geometry}

\usepackage{color}   %May be necessary if you want to color links
\usepackage{hyperref}
\hypersetup{
    colorlinks=false, %set true if you want colored links
    linktoc=all,     %set to all if you want both sections and subsections linked
    linkcolor=black,  %choose some color if you want links to stand out
}
\geometry{margin=1in}
\newcommand{\N}{\mathbb{N}}
\newcommand{\Z}{\mathbb{Z}}
\newcommand{\Q}{\mathbb{Q}}
\newcommand{\R}{\mathbb{R}}
\newcommand{\Omicron}{O}

\title{\textbf{Ordinary Differential}}
\author{Yeheli Fomberg}
\date{}

\usepackage{amsmath}
\begin{document}
	\maketitle
	\newpage
	\tableofcontents
	\newpage
	\section{Introduction}
  This is the best configuration.	
	ODE is short for Oridnary differential equation.	An ODE of order $n$ is defined as such
	\[
		F(x,y,y',\ldots,y^{(n)})=0
	\]
	In most cases we would rather write the equation as such:
	\[
		y^{(n)}=f(x,y,y',\ldots,y^{(n-1)})
	\]
	\subsection{Homogeneity and Linearity of ODEs}
	A linear ODE is of the following is an ODE of the following format:
	\[
		F(x,y,y',\ldots,y^{(n)})=\sum_{i=0}^{i=n}{a_i(x)y^{(i)}}=b(x)
	\]
	$\forall i(a_i(x)\text{ is a differentiable function})$
	\\ A linear ODE is called homogeneous if $b(x)=0$	
	
	\newpage
	\section{Linear ODEs of first order and IVPs}
	Recall the form of this type of ODE is:
	\[
		y'=p(x)y+q(x)
	\]
	Solving $y'=q(x)$ could give us infinitly many solutions because of the integration constant. That's why we usually have these kind of problem coupled with an initial condition - $y(x_0)=y_0$. Given an ODE and an inital condition we get an IVP ir an \textbf{I}nitial \textbf{V}alue \textbf{P}roblem.
	\subsection{Existence and Uniqueness Theorem}
	Given an IVP
	\begin{equation}
    \begin{cases}
    y'=f(x,y) \\
    y(x_0)=y_0
    \end{cases}
	\end{equation}
	Such that $f(x,y)$ and $f'_y(x,y)$ are continuous functions over $D\in\R^2$ and $(x_0,y_0)\in D$ then there exists an $\varepsilon>0$ such that there exists only one unique solution to the problem in $[x_0-\varepsilon,x_0+\varepsilon]$ \\
	\\
	Accordingly there couldn't be two intersecting solutions to such an equation.
	\subsection{General Solution to First Order Homogeneous Linear ODEs}
	
	\[
		y'+p(x)y=0
	\]
	\[
		\frac{y'}{y}=-p(x) \text{ , $y\ne0$\footnote{That's the trivial solution}}
	\]
	\[
		\int\frac{y'(x)}{y(x)}dx=\int-p(x)dx
	\]
	\[\ldots\]
	\[
		y(x)=k*e^{-\int p(x)dx}, \forall k\in\R
	\]
	\newpage
	\subsection{General Solution to First Order Non-Homogeneous Linear ODEs}
	After normalization we get:
	\[
		y' + p(x)y = q(x)
	\]
	Let $U$ be a any function. We'll call $U$ the \textbf{integration factor}.
	\[
		Uy' + Up(x)y = Uq(x)
	\]
	Now let's force $U'=Uq(x)$. We already know how to solve these kind of problems. Let's pronouce one such solution $u$. Now:
	\[
		uy' + u'y = Uq(x)
	\]
	\[
		(uy)' = z(x)
	\]
	\[
		uy = \int z(x) = a(x) + C
	\]
	\[
		y = \frac{a(x)}{u(x)}+\frac{C}{u(x)}
	\]
	*Notice that the first argument is one solution to this IVP and the second is the general solution to the according Homogeneous ODE. 
	
	\newpage
	\section{More ODEs of First Order}
	\subsection{Separable First Order ODEs}
	A seperable first order ODE is of the next form:
	\[
		y'=h(x)g(y)
	\]
	\[
		\frac{y'}{g(y)}=h(x)
	\]
	\[
		\frac 1{g(y)} \frac{dy}{dx}=h(x)
	\]
	\[
		\int{\frac {dy}{g(y)}} = \int h(x)dx
	\]
	\[
		G(y(x)) = H(x) + C
	\]
	That's an implicit solution to the ODE. Sometimes we can write it explicitly and sometimes we can't. If there exists a scalar $y_0$ such that $g(y_0)=0$ then $y(x)=y_0$ is a solution - called the \textbf{singular solution}. Think why.
	\subsection{Homogeneous ODEs}
	Different then linear homogeneous ODEs. These are equations of the form:
	$$y' = F(\frac yx)$$
	It's called that since a homogeneous function of order $m$ is a function such that $f(tx,ty) = t^mf(x,y)$ and $F(\frac yx)$ is a homogeneous function of order 0.\\
	Define $v(x) = \frac {y(x)}x \Rightarrow y'(x) = v'(x)x + v(x)$ plugging that in we get
	\[
		v'x + v = F(\frac yx)
	\]
	\[	
		\frac {v'}{F(v)-v} = \frac 1x
	\]
	\[
		\int\frac {dv}{F(v)-v} =\int\frac 1x dx
	\]
	\[
		G(v(x)) = ln|x|+C
	\]
	This solution is implicit, we must plug $v(x) = \frac {y(x)}x$ back in. Notice that if there exists a $v_0$ such that $F(v_0)=v_0$ then $v(x)=v_0$ is a singular solution to the separable equation. Thus $y(x)=v_0x$ is a singular solution to the ODE.
	\newpage
	\subsection{Switching $x$ and $y$}
	Consider the following ODE:
	\[
	y'=\frac y{x+y^3}
	\]
	That's not an ODE we have encountered so far. In analysis we talked about inverse functions and we saw that $\frac{dy}{dx} = \frac{1}{\frac{dx}{dy}}$ and so we can instead solve
	\[
		x' = \frac{x+y^3}{y}
	\]
as we would for a linear nnon-homogeneous ODE of first order.
\newpage
\section{Exact ODEs}
Looking at equations of this form:
\[
	P(x,y)+Q(x,y)y'=0
\]
Or, using Leibniz notation
\[
	(x,y)dx+Q(x,y) = 0
\]
We'll call the equation \textbf{exact} if there exists $F$ such that
\begin{equation*}
    \begin{cases}
    F_x'=P(x,y) \\
    F_y'=Q(x,y)
    \end{cases}
\end{equation*}
And its solution is given implicitly with the equation $F(x,y)=c$. If we derive both sides we get
\[
	F_x'(x)'+F_y'(y(x))'=0
\]
\[
	P(x)+Q(x)y'=0
\]
\emph{Theorem} - Let $Q(x,y),P(x,y)$ be partially continuously differentiable functions on a simple connected domain $D$.
$$\exists F: F_x'=P\land F_y'=Q \iff P_y'=Q_x'$$
\subsection{Almost exact ODEs}
If $P_y'\ne P_x'$ we can mulitply everything by an integration factor $u(x,y)$
\[
	u(x,y)P(x,y)dx+u(x,y)Q(x,y) = 0
\]
And we want
\[
	u_y'P+uP_y' = u_x'+uQ_x'
\]
\[
	u_y'P+uP_y' - u_x'Q - uQ_x' = 0
\]
Which is a Partially Differentiable Equation. We can solve these under certain circumstances. If $u=u(x) \Rightarrow u(P_y' - Q_x') = u_x'Q \Rightarrow$
\[
	\frac{u_x'}{u}=\frac{P_y'-Q_x'}{Q}
\]
So the left fraction is dependent on $x$ alone. A similar fraction can be generated for $y$.
\newpage
\section{Geometrical Aspects of ODEs}
Let $f(x,y)$ be the function that gives the slope of a linear function that intersects with $(0,0),(x,y)$. Now considering the ODE $y'=f(x,y)$ and recalling that a deravative of a function gives us the slope near the point of derivation we can define a graph such that each point has a vector pointing at the direction of the slope given by $y'$. That graph is known as This ODE's directional field.
\\\\
\textbf{Isoclines} are defined as the points that solve $f(x,y)=C$\\
\textbf{Nullclines} are defined as the points that solve $f(x,y)=0$
\\\\
\emph{Definition} - $y=\alpha(x)$ is called a 
\\
\emph{Theorem} -  Let


\end{document}
