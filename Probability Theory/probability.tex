\documentclass[11pt,a4paper]{article}
\usepackage{amssymb,amsfonts,amsmath,calc,tikz,pgfplots,geometry}
\usepackage{color}   %May be necessary if you want to color links
\usepackage{hyperref}
\usepackage{amsthm}
\usepackage{fancyhdr}
\pagestyle{fancy}
\usetikzlibrary{positioning}
\geometry{margin=1in}
\pgfplotsset{compat=1.18}
\setlength{\headheight}{14.6pt}
\addtolength{\topmargin}{-1.6pt}
\hypersetup{
    colorlinks=false, %set true if you want colored links
    linktoc=all,   %set to all if you want both sections and subsections linked
    linkcolor=black,  %choose some color if you want links to stand out
}
%%%%%%%%%%%%%%%%%%%%%%%%%%%%%%%%%%%%%%%%%%%%%%%%%%%%%%%%%%%%%%%%%%%%%%%%%%%%%%%
\theoremstyle{definition}
\newtheorem{definition}{Definition}[section]
\newtheorem{remark}{Remark}[section]
\newtheorem{example}{Example}[section]
\theoremstyle{plain}
\newtheorem{theorem}{Theorem}[section]
\newtheorem{proposition}[theorem]{Proposition}
\newtheorem{lemma}[theorem]{Lemma}
\newtheorem{corollary}[theorem]{Corollary}
\newtheorem{paradox}{Paradox}[section]

\DeclareMathOperator{\lcm}{lcm}
\DeclareMathOperator{\idealin}{\triangleleft}
\DeclareMathOperator{\im}{im}
\DeclareMathOperator{\Aut}{Aut}
\DeclareMathOperator{\End}{End}
\DeclareMathOperator{\Inn}{Inn}
\DeclareMathOperator{\Out}{Out}
\DeclareMathOperator{\Mat}{Mat}
\DeclareMathOperator{\std}{std}
\DeclareMathOperator{\Int}{Int}
\DeclareMathOperator{\diam}{diam}

\newcommand{\N}{\mathbb{N}}
\newcommand{\st}{\text{ s.t. }}
\newcommand{\Z}{\mathbb{Z}}
\newcommand{\Q}{\mathbb{Q}}
\newcommand{\R}{\mathbb{R}}
\newcommand{\C}{\mathbb{C}}
\newcommand{\F}{\mathbb{F}}
\newcommand{\Omicron}{O}
\newcommand{\ip}[2]{\langle #1, #2 \rangle}
\newcommand{\set}[2]{ \left\{ #1 \mid #2 \right\} }
\newcommand{\abs}[1]{\left\lvert #1\right\rvert}
\newcommand{\norm}[1]{\left\lVert #1\right\rVert}
\renewcommand{\tt}[1]{\textnormal{\textbf{(#1).}}} %tt=theorem title
\newcommand{\bigslant}[2]
{{\raisebox{.2em}{$#1$}\left/\raisebox{-.2em}{$#2$}\right.}}
%%%%%%%%%%%%%%%%%%%%%%%%%%%%%%%%%%%%%%%%%%%%%%%%%%%%%%%%%%%%%%%%%%%%%%%%%%%%%%%
\title{\textbf{}}
\author{}
\date{}
\begin{document}
	\maketitle
	\newpage
  \section{Probability Spaces}
  Before diving in into the definition of a probability space, the main object
  of this course, we must note that this course is an introductory course in 
  probability theory, which means we don't have the tools from measure theory
  to formalize probability. Thus, some proofs will be omitted, and we will
  also need to formalize discrete and continuous probability theory seperately.

  First, let us introduce a paradox.
  \begin{paradox}
    \tt{Bertrand's Paradox}
    Consider an equilateral triangle inscribed in a circle. 
    Suppose a chord of the circle is chosen at random. 
    What is the probability that the chord is longer than a side of the 
    triangle? 
  \end{paradox}

  We can ponder about this paradox for a while, but Bertrand himself came up
  with three solutions, each with a different answer. The main difference in
  his methods lies in the way in which we choose the chords.

  \begin{definition}
    The sample space of an experiment, is a set $\Omega$ which contains all
    the possible outcomes of the experiment.
  \end{definition}

  A good thing to note, is that we can choose different sample spaces for the
  same experiment. For example, if the experiment consists of rolling two
  dice, and we want to check for the sum of the results, we can set either
  $\Omega = \{1,2,3,4,5,6\}^2$, for the result of each dice, or 
  $\Omega = \{1,2,\dots,11,12\}$ for the sum of the results of the dice. 

  \begin{definition}[Probability theory, intuitive definition]
    A discrete probability space is a pair $(\Omega, \mathbf P)$, where
    $\Omega$ is a countable sample set, and $\mathbf P \colon \Omega \to 
    [0,1]$ is a function such that 
    $\sum_{\omega \in \Omega}{\mathbf P(\omega)} = 1$. 
    Intuitively, we say that $\mathbf P(\omega)$ represents the probability
    that $\omega$ will happen.
  \end{definition}
  \begin{definition}
    A subset of the sample space $A \subseteq \Omega$ is called an event.
    We also define:
    \[
      \mathbf P(A) := \sum_{\omega \in \Omega}{\mathbf P(\omega)}
    \]
  \end{definition}
  
  Here are a few properties of probability functions we can immediately 
  verify:
  \begin{enumerate}
    \item $\mathbf P(\Omega) = 1$
    \item $\mathbf P(\emptyset) = 0$
    \item For $A \subset \Omega$ we have $\mathbf P(A^c) = 1 - \mathbf P(A)$
    \item If $\{A_n\}_{n=1}^{N}$ are disjoint sets then 
      \[
        \mathbf P\left(\cup_{n=1}^{n}{A_n}\right) = 
        \sum_{n=1}^{N}{\mathbf P(A_n)}.
      \]
    \item If $\{A_n\}_{n=1}^{\infty}$ is a sequence of pairwise disjoint sets 
      then
      \[
        \mathbf P\left(\cup_{n=1}^{\infty}{A_n}\right) = 
        \sum_{n=1}^{\infty}{\mathbf P(A_n)}.
      \]
  \end{enumerate}

  In a finite probability space we say that the probability function is
  continuous if for every $\omega \in \Omega$ we have 
  $\mathbf P(\omega) = \frac{1}{|\Omega|}$.

  We now proceed to consider an experiment in which we choose a direction in
  $\R^2$ at random, on $S^1$ and write it. The sample space is:
  \[
    \Omega = S^1 = \set{e^{i \theta}}{\theta \in [0,2 \pi)}.
  \]
  A natural question to ask, is if we can define a uniform probability function
  in the sense that for any arc $[a,b] \subset S^1$ we have 
  $\mathbf P([a,b]) = b - a$. The answer is that with the definition we have
  worked with so far, we can't. We see that $\mathbf P(\{a\}) = 0$ for 
  any $a \in S^1$, and thus we have that
  \[
    \mathbf P(\Omega) = \sum_{\omega \in \Omega}{\mathbf P(\omega)} = 0.
  \]
  To solve this problem, we may try to define a new function $\mathbf P \colon 
  2^\Omega \to [0,1]$ that will directly assign each event its probability,
  but unfortunately for us, such a function, that satisfies the desired
  properties of a probability function, does not exist. The proof for this
  is in the course ``real valued function'', and will not be discussed here.
  However, we can give a proof, under the assumption of the following lemma.

  \begin{lemma}
    Exists a set $E \subset S^1$ such that for any rational number 
    $q \in (0, 2 \pi) \cap \Q$ we have $e^{i q}E \cap E = \emptyset$.
  \end{lemma}

  Indeed we see that

  \[
    1 = \mathbf P(\Omega) = 
    \mathbf P\left(\bigcup_{q \in [0,2\pi] \cap \Q}{e^{i q} E}\right) = 
    \sum_{q \in [0,2\pi) \cap \Q}{\mathbf P(e^{i q} E)} = 
    \sum_{q \in [0,2\pi) \cap \Q}{\mathbf P(E)}
  \]

  And now we have a contradiction because if we set $\mathbf P(E) = a$ then
  we get
  \[ 1 = \sum_{q \in [0,2\pi) \cap \Q}{a} \]
  and this equation has no solution.

  The classical solution to this problem, is to only define the probability
  function only on certain subsets of the sample space. Suppose we denote
  this new domain as $\mathcal F \in 2^\Omega$. In order for the desired
  properties to hold we must also accept that $\mathcal F$ holds certain
  conditions.

  \begin{definition}[$\sigma$-algebra]
    Let $\Omega$ be a set. We say that $\mathcal \subset 2^\Omega$ is a
    $\sigma$-algebra of sets (sometimes also called a $\sigma$-field), it
    satisfies the following properties:
    \begin{enumerate}
      \item $\Omega \in \mathcal F$.
      \item If $A \in \mathcal F$ then $A^c \in \mathcal F$.
      \item If $(A_n)_{n=1}^{\infty} \subset \mathcal F$, then
        $\cup_{n=1}^{\infty}{A_n} \in \mathcal F$.
    \end{enumerate}
  \end{definition}

  We can now formally define a probability space.

  \begin{definition}[Probability Space]
    A probability space is a triplet $(\Omega, \mathcal F, \mathbf P)$ such
    that $\Omega$ is a set, $\mathcal F$ is a $\sigma$-algebra of $\Omega$,
    and $\mathbf P \colon \mathcal \to [0,1]$ is a probability function that
    satisfies:
    \begin{enumerate}
      \item $\mathbf P(\Omega) = 1$
      \item If $(A_n)_{n=1}^{\infty} \subset \mathcal F$ are disjoint, then
        $\mathbf P\left(\cup_{n=1}^{\infty}{A_n}\right) = 
        \sum_{n=1}^{\infty}{\mathbf P(A_n)}$.
    \end{enumerate}
  \end{definition}

  In this case we shall call elements of $\mathcal F$ events.

  \begin{proposition}
    Exists a $\sigma$-algebra $\mathfrak B$ of $\Omega = S^1$, and a unique
    function $\mathbf P \colon \mathfrak \to [0,1]$ such that 
    $(\Omega, \mathfrak B, \mathbf P)$ is a probability space and $\mathbf P$
    is invariant to spinning on the sphere.
  \end{proposition}

  \begin{definition}[Algebra of Sets]
    A set $\mathcal C \subset 2^\Omega$ is called an algebra of sets if it 
    satisfies the following properties:
    \begin{enumerate}
      \item $\Omega \in \mathcal C$.
      \item If $A \in \mathcal C$, then $A^c \in \mathcal C$.
      \item if $A,B \in \mathcal C$, then $A \cup B \in \mathcal C$.
    \end{enumerate}
  \end{definition}

  We can immediately verify that any algebra $\mathcal C$ is closed under
  finite unions and finite intersections. We also notice that 
  $\emptyset \in \mathcal C$, and that if $A,B \in \mathcal C$, then
  $A \setminus B \in \mathcal C$. We can also notice that any 
  $\sigma$-algebra is closed under countable intersections, and that
  every $\sigma$-algebra is in particular also an algebra.

  \begin{example}
    If $\Omega$ is a set, and $A \subset \Omega$, then both $2^\Omega$ and 
    $\{\emptyset, A, A^c, \Omega\}$ are $\sigma$-algebras.
  \end{example}

  \begin{example}
    Given a set $\Omega$, the smallest $\sigma$-algebra of $\Omega$ is
    $\{\emptyset, \Omega\}$ which is called the trivial $\sigma$-algebra.
  \end{example}
  
  \begin{proposition}
    Let $(\mathcal F_\alpha)_{\alpha \in I}$ be a family of $\sigma$-algebras,
    then $\cap_{\alpha \in I}{\mathcal F_\alpha}$ is a $\sigma$-algebra.
  \end{proposition}
  \begin{proof}
    Obvious.
  \end{proof}

  \begin{definition}[Minimal Sigma Algebra]
    Let $\Omega$ be a set, and let $H \subset 2^\Omega$ be a family of its
    subsets. Then we define the minimal sigma algebra that contains $H$,
    denoted $\sigma(H)$, as the intersection of all the $\sigma$-algebras
    that contains all the elements in $H$. Notice that the intersection is
    never empty because $2^\Omega$ is a $\sigma$-algebra that will always
    contain the elements of $H$.
  \end{definition}

  \begin{example}
    \tt{Borel's $\sigma$-algebra}
    One of the most important minimal $\sigma$-algebras, is Borel's 
    $\sigma$-algebra defined on $\R$. It is defined as such:
    \[
      \mathfrak B = \mathfrak B(\R) := \sigma(\set{(a,b)}{a < b}).
    \]
    That is, the smallest $\sigma$-algebra that contains all the open 
    intervals in $\R$. Similarly, we can define it on the space $\R^d$
    as follows:
    \[
      \mathfrak B_d = \mathfrak B(\R^d) := 
      \sigma\left(\set{\prod_{i=1}^{d}(a_i,b_i)}{a_i < b_i}\right).
    \]
    Note that in general, Borel's $\sigma$-algebra is defined to be
    the smallest $\sigma$-algebra that contains all the open sets in a
    general topological space. It can be showen that this definition is
    equivalent to the definitions we just gave for $\mathfrak B$ and
    $\mathfrak B_d$.
  \end{example}

  \begin{theorem}\label{thm:cath}
    \tt{Carath\'eodory}
    Let $\Omega$ be a set, let $\mathcal G$ be an algebra of sets of $\Omega$.
    If $\widehat{P} c\colon \mathcal G \to [0,1]$ is a function that satisfies
    $f(\Omega) = 1$, and for each sequence of pairwise disjoint sets
    $\{A_n\}_{n=1}^{\infty}$ that
      \[
        \widehat{\mathbf P} \left(\cup_{n=1}^{\infty}{A_n}\right) = 
        \sum_{n=1}^{\infty}{\widehat{\mathbf P}(A_n)},
      \]
    then exists a single extension 
    $\mathbf P \colon \sigma(\mathcal G) \to [0,1]$ to 
    $\widehat{\mathbf P} \colon \mathcal G \to [0,1]$, such that the triplet
    $(\Omega, \sigma(\mathcal G), \mathbf P)$ is a probability space.
  \end{theorem}

  Now, if we consider again our previous problem, and let $\Omega = S^1$,
  in order to find a uniform probabiliy function on it we can define the
  set $\mathcal G$ to be the set of all finite unions of intervals on $S^1$.
  As it is closed under union of pairs, and complements, it is an algebra.
  Now define $\widehat{\mathbf P} \colon \mathcal G \to [0,1]$ as such:
  \[
    \widehat{\mathbf P}\left(\right)
  \]
  We can see that $\widehat{\mathbf P}$ satisfies the conditions in 
  \autoref{thm:cath} and thus exists an extension $\mathbf P$ defined on
  the sigma algebra $\mathcal B = \sigma(\mathcal G)$ which is also called
  the Borel $\sigma$-algebra of $S^1$. We have that 
  $(\Omega, \mathcal B, \mathbf P)$ is a probability space and we call 
  $\mathbf P$ the uniform probability function on $S^1$.

  Now we can more formally consider the properties of probability functions.
  \begin{proposition}
    Let $(\Omega, \mathcal F, \mathbf P)$ be a probability space.
    \begin{enumerate}
      \item $\mathbf P(\emptyset) = 0$.
      \item If $\{A_n\}_{n=1}^{N} \subset \mathcal F$ are disjoint sets then
        $\cup_{n=1}^{n}{A_n} \in \mathcal F$ and
        \[
          \mathbf P\left(\cup_{n=1}^{n}{A_n}\right) = 
          \sum_{n=1}^{N}{\mathbf P(A_n)}.
        \]
      \item For every $A \in \mathcal F$ we have 
        $\mathbf P(A^c) = 1 - \mathbf P(A)$.
      \item If $A, B \in \mathcal F$ and $A \subset B$, then 
        $\mathbf(B \setminus A) = \mathbf P(B) - \mathbf P(A)$ and thus
        $\mathbf P(A) \le \mathbf P(B)$.
      \item If $A,B \in \mathcal F$, then
        \[
          \mathbf P(A \cup B) = 
          \mathbf P(A) + \mathbf P(B) - \mathbf P(A \cap B)
        \]
    \end{enumerate}
  \end{proposition}
  
  \begin{proposition}[Continuity of the Probability Function]
    Let $(\Omega, \mathcal F, \mathbf P)$ be a probability space.
    \begin{enumerate}
      \item If $(A_n)_{n=1}^{\infty} \subset \mathcal F$ is an increasing
        sequence of events, that is $A_1 \subset A_2 \subset A_3, \dots$,
        then
        \[
          \mathbf P\left(\bigcup_{n=1}^{\infty}{A_n})\right) = 
          \lim_{n \to \infty}{\mathbf (A_n)}.
        \]
      \item If $(A_n)_{n=1}^{\infty} \subset \mathcal F$ is a decreasing
        sequence of events, that is $A_1 \supset A_2 \supset A_3, \dots$,
        then
        \[
          \mathbf P\left(\bigcap_{n=1}^{\infty}{A_n})\right) = 
          \lim_{n \to \infty}{\mathbf (A_n)}.
        \]
    \end{enumerate}
  \end{proposition}

  In fact the last proposition is a not more than a case of the following
  proposition.

  \begin{proposition}
    Let $(A_n)_{n=1}^{\infty}$ be a sequence of events in a probability space
    $(\Omega, \mathcal F, \mathbf P)$. If the limit $\lim_{n \to \infty} A_n$
    exists, then $\lim_{n \to \infty} A_n \in \mathcal F$, and
    \[
      \mathbf P(\lim_{n \to \infty}{A_n}) = 
      \lim_{n \to \infty} \mathbf P(A_n)
    \]
  \end{proposition}
  
  Let us prove this theorem for the case $(A_n)_{n=1}^{\infty}$ is increasing.
  Define the following sequence:
  \begin{align*}
    B_1 &= A_1 \\
    B_n &= A_n \setminus A_{n-1}
  \end{align*}
  It is clear that:
  \begin{enumerate}
    \item The sets $(B_n)_{n=1}^{\infty}$ are disjoint.
    \item For every $N \in \N$ we have: 
      \[ \bigcup_{n=1}^{N} B_n = \bigcup_{n=1}^{N} A_n = A_N. \]
    \item $\cup_{n=1}^{\infty} B_n = \cup_{n=1}^{\infty} A_n$.
  \end{enumerate}
  We now have:
  \begin{align*}
    \mathbf P \left(\bigcup_{n=1}^{\infty} A_n\right) &=
    \mathbf P \left(\bigcup_{n=1}^{\infty} B_n\right) =
    \sum_{n=1}^{\infty} \mathbf(B_n) =
    \lim_{N \to \infty} \sum_{n=1}^{N} \mathbf P(B_n) =
    \lim_{N \to \infty} \mathbf P\left(\bigcup_{n=1}^{N} B_n\right) \\ &=
    \lim_{N \to \infty} \mathbf P(A_N).
  \end{align*}

  \newpage

  \section{Conditional Probability}
  

\end{document}
