\documentclass[11pt,a4paper]{article}
\usepackage{amssymb,amsfonts,amsmath,calc,tikz,pgfplots,geometry,mathtools}
\usepackage{color}   %May be necessary if you want to color links
\usepackage{hyperref}
\usepackage{amsthm}
\usepackage{fancyhdr}
\pagestyle{fancy}
\usetikzlibrary{positioning}
\geometry{margin=1in}
\pgfplotsset{compat=1.18}
\setlength{\headheight}{14.6pt}
\addtolength{\topmargin}{-1.6pt}
\hypersetup{
    colorlinks=false, %set true if you want colored links
    linktoc=all,   %set to all if you want both sections and subsections linked
    linkcolor=black,  %choose some color if you want links to stand out
}
%%%%%%%%%%%%%%%%%%%%%%%%%%%%%%%%%%%%%%%%%%%%%%%%%%%%%%%%%%%%%%%%%%%%%%%%%%%%%%%
\theoremstyle{definition}
\newtheorem{definition}{Definition}[section]
\newtheorem{remark}{Remark}[section]
\newtheorem{example}{Example}[section]
\theoremstyle{plain}
\newtheorem{theorem}{Theorem}[section]
\newtheorem{proposition}[theorem]{Proposition}
\newtheorem{lemma}[theorem]{Lemma}
\newtheorem{corollary}[theorem]{Corollary}
\newtheorem{paradox}{Paradox}[section]

\DeclareMathOperator{\lcm}{lcm}
\DeclareMathOperator{\idealin}{\triangleleft}
\DeclareMathOperator{\im}{im}
\DeclareMathOperator{\Aut}{Aut}
\DeclareMathOperator{\End}{End}
\DeclareMathOperator{\Inn}{Inn}
\DeclareMathOperator{\Out}{Out}
\DeclareMathOperator{\Mat}{Mat}
\DeclareMathOperator{\std}{std}
\DeclareMathOperator{\Int}{Int}
\DeclareMathOperator{\diam}{diam}

\newcommand{\N}{\mathbb{N}}
\newcommand{\st}{\text{ s.t. }}
\newcommand{\Z}{\mathbb{Z}}
\newcommand{\Q}{\mathbb{Q}}
\newcommand{\R}{\mathbb{R}}
\newcommand{\C}{\mathbb{C}}
\newcommand{\F}{\mathbb{F}}
\newcommand{\Omicron}{O}
\newcommand{\ip}[2]{\langle #1, #2 \rangle}
\newcommand{\set}[2]{ \left\{ #1 \mid #2 \right\} }
\newcommand{\abs}[1]{\left\lvert #1\right\rvert}
\newcommand{\norm}[1]{\left\lVert #1\right\rVert}
\renewcommand{\tt}[1]{\textnormal{\textbf{(#1).}}} %tt=theorem title
\newcommand{\bigslant}[2]
{{\raisebox{.2em}{$#1$}\left/\raisebox{-.2em}{$#2$}\right.}}
%%%%%%%%%%%%%%%%%%%%%%%%%%%%%%%%%%%%%%%%%%%%%%%%%%%%%%%%%%%%%%%%%%%%%%%%%%%%%%%
\title{\textbf{Introduction to Probability Theory}}
\author{}
\date{}
\begin{document}
	\maketitle
	\newpage
  \section{Probability Spaces}
  Before diving in into the definition of a probability space, the main object
  of this course, we must note that this course is an introductory course in 
  probability theory, which means we don't have the tools from measure theory
  to formalize probability. Thus, some proofs will be omitted, and we will
  also need to formalize discrete and continuous probability theory seperately.

  First, let us introduce a paradox.
  \begin{paradox}
    \tt{Bertrand's Paradox}
    Consider an equilateral triangle inscribed in a circle. 
    Suppose a chord of the circle is chosen at random. 
    What is the probability that the chord is longer than a side of the 
    triangle? 
  \end{paradox}

  We can ponder about this paradox for a while, but Bertrand himself came up
  with three solutions, each with a different answer. The main difference in
  his methods lies in the way in which we choose the chords.

  \begin{definition}
    The sample space of an experiment, is a set $\Omega$ which contains all
    the possible outcomes of the experiment.
  \end{definition}

  A good thing to note, is that we can choose different sample spaces for the
  same experiment. For example, if the experiment consists of rolling two
  dice, and we want to check for the sum of the results, we can set either
  $\Omega = \{1,2,3,4,5,6\}^2$, for the result of each dice, or 
  $\Omega = \{1,2,\dots,11,12\}$ for the sum of the results of the dice. 

  \begin{definition}[Probability space, intuitive definition]
    A discrete probability space is a pair $(\Omega, \mathbf P)$, where
    $\Omega$ is a countable sample set, and $\mathbf P \colon \Omega \to 
    [0,1]$ is a function such that 
    $\sum_{\omega \in \Omega}{\mathbf P(\omega)} = 1$. 
    Intuitively, we say that $\mathbf P(\omega)$ represents the probability
    that $\omega$ will happen.
  \end{definition}
  \begin{definition}
    A subset of the sample space $A \subseteq \Omega$ is called an event.
    We also define:
    \[
      \mathbf P(A) := \sum_{\omega \in \Omega}{\mathbf P(\omega)}
    \]
  \end{definition}
  
  Here are a few properties of probability functions we can immediately 
  verify:
  \begin{enumerate}
    \item $\mathbf P(\Omega) = 1$
    \item $\mathbf P(\emptyset) = 0$
    \item For $A \subset \Omega$ we have $\mathbf P(A^c) = 1 - \mathbf P(A)$
    \item If $\{A_n\}_{n=1}^{N}$ are disjoint sets then 
      \[
        \mathbf P\left(\cup_{n=1}^{n}{A_n}\right) = 
        \sum_{n=1}^{N}{\mathbf P(A_n)}.
      \]
    \item If $\{A_n\}_{n=1}^{\infty}$ is a sequence of pairwise disjoint sets 
      then
      \[
        \mathbf P\left(\cup_{n=1}^{\infty}{A_n}\right) = 
        \sum_{n=1}^{\infty}{\mathbf P(A_n)}.
      \]
  \end{enumerate}

  In a finite probability space we say that the probability function is
  continuous if for every $\omega \in \Omega$ we have 
  $\mathbf P(\omega) = \frac{1}{|\Omega|}$.

  We now proceed to consider an experiment in which we choose a direction in
  $\R^2$ at random, on $S^1$ and write it. The sample space is:
  \[
    \Omega = S^1 = \set{e^{i \theta}}{\theta \in [0,2 \pi)}.
  \]
  A natural question to ask, is if we can define a uniform probability function
  in the sense that for any arc $[a,b] \subset S^1$ we have 
  $\mathbf P([a,b]) = b - a$. The answer is that with the definition we have
  worked with so far, we can't. We see that $\mathbf P(\{a\}) = 0$ for 
  any $a \in S^1$, and thus we have that
  \[
    \mathbf P(\Omega) = \sum_{\omega \in \Omega}{\mathbf P(\omega)} = 0.
  \]
  To solve this problem, we may try to define a new function $\mathbf P \colon 
  2^\Omega \to [0,1]$ that will directly assign each event its probability,
  but unfortunately for us, such a function, that satisfies the desired
  properties of a probability function, does not exist. The proof for this
  is in the course ``real valued function'', and will not be discussed here.
  However, we can give a proof, under the assumption of the following lemma.

  \begin{lemma}
    Exists a set $E \subset S^1$ such that for any rational number 
    $q \in (0, 2 \pi) \cap \Q$ we have $e^{i q}E \cap E = \emptyset$.
  \end{lemma}

  Indeed we see that

  \[
    1 = \mathbf P(\Omega) = 
    \mathbf P\left({\bigcup_{q \in [0,2\pi] \cap \Q}}{e^{i q} E}\right) = 
    \sum_{q \in [0,2\pi) \cap \Q}{\mathbf P(e^{i q} E)} = 
    \sum_{q \in [0,2\pi) \cap \Q}{\mathbf P(E)}
  \]

  And now we have a contradiction because if we set $\mathbf P(E) = a$ then
  we get
  \[ 1 = \sum_{q \in [0,2\pi) \cap \Q}{a} \]
  and this equation has no solution.

  The classical solution to this problem, is to only define the probability
  function only on certain subsets of the sample space. Suppose we denote
  this new domain as $\mathcal F \in 2^\Omega$. In order for the desired
  properties to hold we must also accept that $\mathcal F$ holds certain
  conditions.

  \begin{definition}[$\sigma$-algebra]
    Let $\Omega$ be a set.\ We say that $\mathcal F \subset 2^\Omega$ is a
    $\sigma$-algebra (sometimes called a $\sigma$-field) of sets, if it
    satisfies the following properties:
    \begin{enumerate}
      \item $\Omega \in \mathcal F$.
      \item If $A \in \mathcal F$ then $A^c \in \mathcal F$.
      \item If $(A_n)_{n=1}^{\infty} \subset \mathcal F$, then
        $\cup_{n=1}^{\infty}{A_n} \in \mathcal F$.
    \end{enumerate}
  \end{definition}

  We can now formally define a probability space.

  \begin{definition}[Probability Space]
    A probability space is a triplet $(\Omega, \mathcal F, \mathbf P)$ such
    that $\Omega$ is a set, $\mathcal F$ is a $\sigma$-algebra of $\Omega$,
    and $\mathbf P \colon \mathcal \to [0,1]$ is a probability function that
    satisfies:
    \begin{enumerate}
      \item $\mathbf P(\Omega) = 1$
      \item If $(A_n)_{n=1}^{\infty} \subset \mathcal F$ are disjoint, then
        $\mathbf P\left(\cup_{n=1}^{\infty}{A_n}\right) = 
        \sum_{n=1}^{\infty}{\mathbf P(A_n)}$.
    \end{enumerate}
  \end{definition}

  In this case we shall call elements of $\mathcal F$ events.

  \begin{proposition}
    Exists a $\sigma$-algebra $\mathfrak B$ of $\Omega = S^1$, and a unique
    function $\mathbf P \colon \mathfrak \to [0,1]$ such that 
    $(\Omega, \mathfrak B, \mathbf P)$ is a probability space and $\mathbf P$
    is invariant to spinning on the sphere.
  \end{proposition}

  \begin{definition}[Algebra of Sets]
    A set $\mathcal C \subset 2^\Omega$ is called an algebra of sets if it 
    satisfies the following properties:
    \begin{enumerate}
      \item $\Omega \in \mathcal C$.
      \item If $A \in \mathcal C$, then $A^c \in \mathcal C$.
      \item if $A,B \in \mathcal C$, then $A \cup B \in \mathcal C$.
    \end{enumerate}
  \end{definition}

  We can immediately verify that any algebra $\mathcal C$ is closed under
  finite unions and finite intersections. We also notice that 
  $\emptyset \in \mathcal C$, and that if $A,B \in \mathcal C$, then
  $A \setminus B \in \mathcal C$. We can also notice that any 
  $\sigma$-algebra is closed under countable intersections, and that
  every $\sigma$-algebra is in particular also an algebra.

  \begin{example}
    If $\Omega$ is a set, and $A \subset \Omega$, then both $2^\Omega$ and 
    $\{\emptyset, A, A^c, \Omega\}$ are $\sigma$-algebras.
  \end{example}

  \begin{example}
    Given a set $\Omega$, the smallest $\sigma$-algebra of $\Omega$ is
    $\{\emptyset, \Omega\}$ which is called the trivial $\sigma$-algebra.
  \end{example}
  
  \begin{proposition}
    Let $(\mathcal F_\alpha)_{\alpha \in I}$ be a family of $\sigma$-algebras,
    then $\cap_{\alpha \in I}{\mathcal F_\alpha}$ is a $\sigma$-algebra.
  \end{proposition}
  \begin{proof}
    Obvious.
  \end{proof}

  \begin{definition}[Minimal Sigma Algebra]
    Let $\Omega$ be a set, and let $H \subset 2^\Omega$ be a family of its
    subsets. Then we define the minimal sigma algebra that contains $H$,
    denoted $\sigma(H)$, as the intersection of all the $\sigma$-algebras
    that contains all the elements in $H$. Notice that the intersection is
    never empty because $2^\Omega$ is a $\sigma$-algebra that will always
    contain the elements of $H$.
  \end{definition}

  \begin{example}
    \tt{Borel's $\sigma$-algebra}
    One of the most important minimal $\sigma$-algebras, is Borel's 
    $\sigma$-algebra defined on $\R$. It is defined as such:
    \[
      \mathfrak B = \mathfrak B(\R) := \sigma(\set{(a,b)}{a < b}).
    \]
    That is, the smallest $\sigma$-algebra that contains all the open 
    intervals in $\R$. Similarly, we can define it on the space $\R^d$
    as follows:
    \[
      \mathfrak B_d = \mathfrak B(\R^d) := 
      \sigma\left(\set{\prod_{i=1}^{d}(a_i,b_i)}{a_i < b_i}\right).
    \]
    Note that in general, Borel's $\sigma$-algebra is defined to be
    the smallest $\sigma$-algebra that contains all the open sets in a
    general topological space. It can be showen that this definition is
    equivalent to the definitions we just gave for $\mathfrak B$ and
    $\mathfrak B_d$.
  \end{example}

  \begin{theorem}\label{thm:cath}
    \tt{Carath\'eodory}
    Let $\Omega$ be a set, let $\mathcal G$ be an algebra of sets of $\Omega$.
    If $\widehat{P} \colon \mathcal G \to [0,1]$ is a function that satisfies
    $f(\Omega) = 1$, and for each sequence of pairwise disjoint sets
    $\{A_n\}_{n=1}^{\infty}$ that
      \[
        \widehat{\mathbf P} \left(\bigcup_{n=1}^{\infty}{A_n}\right) = 
        \sum_{n=1}^{\infty}{\widehat{\mathbf P}(A_n)},
      \]
    then exists a single extension 
    $\mathbf P \colon \sigma(\mathcal G) \to [0,1]$ to 
    $\widehat{\mathbf P} \colon \mathcal G \to [0,1]$, such that the triplet
    $(\Omega, \sigma(\mathcal G), \mathbf P)$ is a probability space.
  \end{theorem}

  Now, if we consider again our previous problem, and let $\Omega = S^1$,
  in order to find a uniform probabiliy function on it we can define the
  set $\mathcal G$ to be the set of all finite unions of intervals on $S^1$.
  As it is closed under union of pairs, and complements, it is an algebra.
  Now define $\widehat{\mathbf P} \colon \mathcal G \to [0,1]$ as such:
  \[
    \widehat{\mathbf{P}}\left(\biguplus_{i=1}^{N}\left(a_{i},b_{i}\right)\right)
    = \sum_{i=1}^{N}\frac{b_{i}-a_{i}}{2\pi},
  \]
  We can see that $\widehat{\mathbf P}$ satisfies the conditions in 
  \autoref{thm:cath} and thus exists an extension $\mathbf P$ defined on
  the sigma algebra $\mathcal B = \sigma(\mathcal G)$ which is also called
  the Borel $\sigma$-algebra of $S^1$. We have that 
  $(\Omega, \mathcal B, \mathbf P)$ is a probability space and we call 
  $\mathbf P$ the uniform probability function on $S^1$.

  Now we can more formally consider the properties of probability functions.
  \begin{proposition}
    Let $(\Omega, \mathcal F, \mathbf P)$ be a probability space.
    \begin{enumerate}
      \item $\mathbf P(\emptyset) = 0$.
      \item If $\{A_n\}_{n=1}^{N} \subset \mathcal F$ are disjoint sets then
        $\cup_{n=1}^{n}{A_n} \in \mathcal F$ and
        \[
          \mathbf P\left(\bigcup_{n=1}^{n}{A_n}\right) = 
          \sum_{n=1}^{N}{\mathbf P(A_n)}.
        \]
      \item For every $A \in \mathcal F$ we have 
        $\mathbf P(A^c) = 1 - \mathbf P(A)$.
      \item If $A, B \in \mathcal F$ and $A \subset B$, then 
        $\mathbf(B \setminus A) = \mathbf P(B) - \mathbf P(A)$ and thus
        $\mathbf P(A) \le \mathbf P(B)$.
      \item If $A,B \in \mathcal F$, then
        \[
          \mathbf P(A \cup B) = 
          \mathbf P(A) + \mathbf P(B) - \mathbf P(A \cap B)
        \]
    \end{enumerate}
  \end{proposition}
  
  \begin{proposition}[Continuity of the Probability Function]
    Let $(\Omega, \mathcal F, \mathbf P)$ be a probability space.
    \begin{enumerate}
      \item If $(A_n)_{n=1}^{\infty} \subset \mathcal F$ is an increasing
        sequence of events, that is $A_1 \subset A_2 \subset A_3, \dots$,
        then
        \[
          \mathbf P\left(\bigcup_{n=1}^{\infty}{A_n}\right) = 
          \lim_{n \to \infty}{\mathbf A_n}.
        \]
      \item If $(A_n)_{n=1}^{\infty} \subset \mathcal F$ is a decreasing
        sequence of events, that is $A_1 \supset A_2 \supset A_3, \dots$,
        then
        \[
          \mathbf P\left(\bigcap_{n=1}^{\infty}{A_n}\right) = 
          \lim_{n \to \infty}{\mathbf A_n}.
        \]
    \end{enumerate}
  \end{proposition}

  In fact the last proposition is a not more than a case of the following
  proposition.

  \begin{proposition}
    Let $(A_n)_{n=1}^{\infty}$ be a sequence of events in a probability space
    $(\Omega, \mathcal F, \mathbf P)$. If the limit $\lim_{n \to \infty} A_n$
    exists, then $\lim_{n \to \infty} A_n \in \mathcal F$, and
    \[
      \mathbf P(\lim_{n \to \infty}{A_n}) = 
      \lim_{n \to \infty} \mathbf P(A_n)
    \]
  \end{proposition}
  
  Let us prove this theorem for the case $(A_n)_{n=1}^{\infty}$ is increasing.
  Define the following sequence:
  \begin{align*}
    B_1 &= A_1 \\
    B_n &= A_n \setminus A_{n-1}
  \end{align*}
  It is clear that:
  \begin{enumerate}
    \item The sets $(B_n)_{n=1}^{\infty}$ are disjoint.
    \item For every $N \in \N$ we have: 
      \[ \bigcup_{n=1}^{N} B_n = \bigcup_{n=1}^{N} A_n = A_N. \]
    \item $\cup_{n=1}^{\infty} B_n = \cup_{n=1}^{\infty} A_n$.
  \end{enumerate}
  We now have:
  \begin{align*}
    \mathbf P \left(\bigcup_{n=1}^{\infty} A_n\right) &=
    \mathbf P \left(\bigcup_{n=1}^{\infty} B_n\right) =
    \sum_{n=1}^{\infty} \mathbf(B_n) =
    \lim_{N \to \infty} \sum_{n=1}^{N} \mathbf P(B_n) =
    \lim_{N \to \infty} \mathbf P\left(\bigcup_{n=1}^{N} B_n\right) \\ &=
    \lim_{N \to \infty} \mathbf P(A_N).
  \end{align*}

  \newpage

  \section{Conditional Probability}
  \begin{definition}[Conditional Probability]
    Let $(\Omega, \mathcal F, \mathbf P)$ be a probability space, and let
    $A,B \in \mathcal F$, such that $\mathbf P(B) > 0$. We define the
    probability of $A$ given that $B$ already happened as:
    \[
      \mathbf P(A \mid B) := \frac{\mathbf P(A \cap B)}{\mathbf P(B)}
    \]
  \end{definition}

  The intuition behind this definition should be clear. We calculate the 
  probability of event $A$ ``inside'' event $B$.

  Notice that we can also use conditional probability to calculate the
  the probability of an intersection of two events.

  \begin{proposition}
    Let $(\Omega, \mathcal F, \mathbf P)$ be a probability space, let 
    $B \in \mathcal F$ be an event such that $\mathbf P(B) > 0$. Then,
    the map $A \mapsto \mathbf P(A \mid B)$ is a probability function.
  \end{proposition}
  The proof that the range of the function is $[0,1]$ and that 
  $\mathbf (\Omega \mid B) = 0$ is clear from expanding the definitions,
  so we will only prove sigma additivity.
  \begin{proof}
    Let $(A_n)_{n=1}^{\infty} \subset \mathcal F$ be disjoint sets, then
    $(A_n \cap B)_{n=1}^{\infty} \subset \mathcal F$ are also disjoint sets
    and we have:
      \begin{align*}
        \mathbf{P}\left( \bigcup_{n=1}^{\infty} A_n \mid B \right) 
        & = \frac{\mathbf{P}\left( \left( \bigcup_{n=1}^{\infty} A_n \right) \cap B \right)}{\mathbf{P}(B)} \\
        & = \frac{\mathbf{P}\left( \bigcup_{n=1}^{\infty} (A_n \cap B) \right)}{\mathbf{P}(B)} \\
        & = \sum_{n=1}^{\infty} \frac{\mathbf{P}(A_n \cap B)}{\mathbf{P}(B)} \\
        & = \sum_{n=1}^{\infty} \mathbf{P}(A_n \mid B)
      \end{align*}
  \end{proof}

  \begin{proposition}[Law of Total Probability]
    Let $(\Omega, \mathcal F, \mathbf P)$ be a probability space.
    Let $N \in \N \cup \{\infty\}$, and $(A_n)_{n=1}^{N}$ be disjoint events
    such that $\cup_{n=1}^{N} A_n = \Omega$. Then,
    \[
      \mathbf{P}(B)=\sum_{n=1}^{N}\mathbf{P}(A_{n})\mathbf{P}(B|A_{n}).
    \]
  \end{proposition}
  \begin{proof}
    \begin{align*}
      \mathbf{P}(B) &= 
      \mathbf{P}(B \cap \Omega) \\ &= 
      \mathbf{P}\left( B \cap \bigcup_{n=1}^{N} A_n \right) \\ &= 
      \mathbf{P}\left( \bigcup_{n=1}^{N} (A_n \cap B) \right) \\ &= 
      \sum_{n=1}^{N} \mathbf{P}(A_n \cap B) \\ &= 
      \sum_{n=1}^{N} \mathbf{P}(A_n) \mathbf{P}(B \mid A_n).
      \end{align*}
  \end{proof}

  \begin{example}[P\'olya's urn, simplified]
     Let there be $1$ white and $1$ black ball in an urn. 
     At each step, one ball is drawn uniformly at random from the urn, 
     and its color observed; 
     it is then returned in the urn, 
     and an additional ball of the same color is added to the urn.
     What is the probability that there are $k$ black balls in the urn after
     the $n$-th step?

     First denote:
     \begin{align*}
       A_{n,k} &= \{\text{there are $k$ black balls after the $n$-th step.}\} \\
       p_{n,k} &= \mathbf P(A_{n,k}).
     \end{align*}
     In order for there to be $k$ black balls after the $n$-th step, there must
     either have been $k-1$ or $k$ black balls in the $n-1$-th step. Thus,
     \begin{align*}
       \mathbf{P}(A_{n,k}) &= 
       \mathbf{P}(A_{n,k} \cap (A_{n-1,k-1} \cup A_{n-1,k})) \\ &= 
       \mathbf{P}(A_{n-1,k-1}) \mathbf{P}(A_{n,k} \mid A_{n-1,k-1}) + 
       \mathbf{P}(A_{n-1,k}) \mathbf{P}(A_{n,k} \mid A_{n-1,k}).
     \end{align*}
     This implies that
     \[ p_{n,k}=\frac{k-1}{n+1}p_{n-1,k-1}+\frac{n+1-k}{n+1}p_{n-1,k}. \]
     Coupled with the fact that $p_{0,1} = 1$ we can verify that the only
     solution under these conditions is $p_{n,k} = \frac{1}{n+1}$.
     In general, these problems are very hard to solve.
  \end{example}

  Another useful trick is Bayes' theorem.
  In its simplified version it states that,
  \[
    {\bf P}(A \mid B)={\frac{{\bf P}(B \mid A){\bf P}(A)}{{\bf P}(B)}},
  \]
  and can be solved without much thought. Here's the general theoerm.

  \begin{theorem}\tt{Bayes' Theorem}
    Let $(\Omega, \mathcal F, \mathbf P)$ be a probability space.
    Let $N \in \N \cup \{\infty\}$, and $(A_n)_{n=1}^{N}$ be disjoint events
    such that $\cup_{n=1}^{N} A_n = \Omega$. Then,
    \[
      \mathbf{P}(A_i \mid B) = 
      {\frac{\mathbf{P}(B \mid A_{i})\mathbf{P}(A_{i})}
      {\sum_{n=1}^{N}\mathbf{P}(A_{n})\mathbf{P}(B \mid A_{n})}}.
    \]
  \end{theorem}
  \begin{proof}
    Left as an exercise to the reader.
  \end{proof}

  \begin{example}
    Suppose we have a test for checking whether a person has the terrible 
    the terrible ``cooties''.
    It has a true positive rate of $0.98$, and a false positive rate of
    $0.01$. Assume that $0.1\%$ of the population has the cooties,
    what is the probability that a person who got a positive result
    has the cooties?

    Denote,
    \begin{align*}
      A &= \{\text{the person is healthy}\} \\
      B &= \{\text{the answer is positive}\}.
    \end{align*}
    From Bayes' theorem we have:
    \[
      \mathbf{P}(A \mid B) = 
      {\frac{\mathbf{P}(B \mid A)\mathbf{P}(A)}{\mathbf{P}(B)}} = 
      {\frac{0.01\cdot0.999}{\mathrm{P}(B)}}.
    \]
    From the law of total probability we have
    \begin{align*}
      \mathbf{P}(B) &= 
      \mathbf{P}(A)\mathbf{P}(B \mid A) + 
      \mathbf{P}(A^{c})\mathbf{P}(B \mid A^{c}) \\ &= 
      0.01\cdot0.999+0.98\cdot0.001=0.01097.
    \end{align*}
    And thus,
    \[ \mathbf{P}(A \mid B)={\frac{0.01\cdot0.999}{0.01097}} \approx 0.91 \]
  \end{example}

  % There's another example here that might be useful

  \newpage

  \section{Independance and Repeating Experiments}
  Intuitively, when we say that the event $A$ is independent from $B$,
  we mean something like
  \[
    \mathbf P(A \mid B) = \mathbf P(A).
  \]
  Thus we can use the definition of conditional probability to formally define
  Independence.
  \begin{definition}[Independence of Two Events]
    Let $(\Omega, \mathcal F, \mathbf P)$ be a probability space, let 
    $A,B \in \mathcal F$ be two events. We say that $A$ and $B$ are independent,
    if
    \[
      \mathbf P(A \cap B) = \mathbf P(A) \mathbf P(B)
    \]
  \end{definition}
  Notice that the interpretation that $A$ is independent of $B$ is only viable
  if we know that $\mathbf P(B) > 0$.

  % More examples of Independence.

  \begin{proposition}
    Let $(\Omega, \mathcal F, \mathbf P)$ be a probability space,
    let $A \in \mathcal F$. The following conditions are equivalent:
    \begin{enumerate}
      \item For each $B \in \mathcal F$ the events $A$ and $B$ are independent.
      \item $A$ is independent of itself.
      \item $\mathbf P(A) \in \{0, 1\}$.
    \end{enumerate}
  \end{proposition}
  \begin{proof}
    Clear from the definitions.
  \end{proof}
  
  \begin{definition}[Independence]
    Let $(\Omega, \mathcal F, \mathbf P)$ be a probability space,
    let $(A_n)_{n=1}^{N}$ be a finite sequence of events.
    We say that $(A_n)_{n=1}^{N}$ are independent if for each 
    $\emptyset \neq K \subset \{1,2,\dots,N\}$ we have
    \[
      \mathbf P\left(\bigcap_{n \in K} A_n\right) = 
      \prod_{n \in K} \mathbf P(A_n).
    \]
  \end{definition}

  \begin{definition}[Pairwise Independence]
    Let $(\Omega, \mathcal F, \mathbf P)$ be a probability space,
    let $(A_n)_{n=1}^{N}$ be a finite sequence of events.
    We say that $(A_n)_{n=1}^{N}$ are independent if for each 
    $1 \le i < j \le N$ we have
    \[
      \mathbf P(A_i \cap A_j) = 
      \mathbf P(A_i) \mathbf P(A_j).
    \]
  \end{definition}

  \begin{definition}[Independence of Infinite Events]
    Let $(\Omega, \mathcal F, \mathbf P)$ be a probability space,
    let $(A_n)_{n=1}^{\infty}$ be an infinite sequence of events.
    We say that $(A_n)_{n=1}^{\infty}$ are (pairwise) independent if each
    finite subset of them is (pairwise) independent.
  \end{definition}

  Note that we only require independence for finite subsets and not for
  infinite subsets.

  \begin{proposition}\label{prop:indp}
    Let $(\Omega, \mathcal F, \mathbf P)$ be a probability space,
    let $(A_n)_{n=1}^{\infty}$ be an infinite sequence of  (pairwise) 
    independent events. Then define a new sequence 
    $(\widetilde{A}_n)_{n=1}^{\infty}$ such that $\widetilde{A}_n = A_n$ or
    $\widetilde{A}_n = A_n^c$ for each $n \in \N$. Then each choice of such
    $(\widetilde{A}_n)_{n=1}^{\infty}$ is (pairwise) independent.
  \end{proposition}
  \begin{proof}
    Using induction on the number of index such that we chose
    $\widetilde{A}_n = A_n^c$. 
    To be completed.
  \end{proof}

  Now, we have the tools to define probability spaces for repeating experiments.
  We assume that the experiment is repeated in exactly the same way, and that
  the results of each experiment are independent.

  Let $(\Omega, \mathcal F, \mathbf P)$ be a probability space for a certain
  experiment. If we want to define a probability space for repeating the
  the experiment a finite number of times, which we will denote $N$, it makes
  sense to define it as such:
  \begin{align*}
    &\Omega_N = \Omega^N. \\
    &\mathcal F_N = \sigma\left(\set{A_1 \times A_2 \times \cdots \times A_n}
    {A_1,\dots,A_n \in \mathcal F}\right). \\
    &\mathbf P_N\left(A_1 \times A_2 \times \cdots A_N\right) = 
    \prod_{i=1}^{N} \mathbf P(A_i).
  \end{align*}
  The fact that $\mathbf P_N$ is a probability measure follows from 
  \autoref{thm:cath}. This measure is called the product measure, and 
  $\mathcal F_N$ is called the product $\sigma$-algebra.

  Similarly, when $N=\infty$, we will define the space to be:
  \begin{align*}
    &\Omega_N = \Omega^{\N}. \\
    &\mathcal F_N = \sigma\left(\set{\prod_{i=1}^{\infty}{A_i}}
    {A_i \in \mathcal F, \, \forall i \geq 1 \text{ and only for finitely many
    $i$'s } A_i \neq \Omega}\right). \\
    &\mathbf P_N\left(\prod_{i=1}^{N}{A_i}\right) = 
    \prod_{i=1}^{N} \mathbf P(A_i).
  \end{align*}
  Notice that we defined the probability function only on finite products of
  events. The extenstion to the rest of the sets will be done by 
  \autoref{thm:cath}.

  \begin{example}[Bernoulli Trial]
    A Bernoulli trial, is a random experiment with only two possible outcomes,
    ``success'' and ``failure'', in which the probability of success is the 
    same every time the experiment is conducted. The probability space that
    models these kind of experiments is defined as such:
    \begin{align*}
      \Omega &= \{0, 1\}. \\
      \mathcal F &= \{\emptyset, \{0\}, \{1\}, \Omega\}. \\
      \mathbf P(\omega) &=
      \begin{cases}
        p, &\omega = 1 \\
        1-p, &\omega= 0
      \end{cases}
    \end{align*}
    Now for $N \in \N \cup \{\infty\}$ we have
    \[ \Omega_N = \{0, 1\}^N \]
    which models repeating independent experiment with two results.
    These kind of experiments are also called Bernoulli trials.

    Now set $N \in \N$. We want to calculate the probability that the
    experiment ends in $k$ successes. We set:
    \begin{align*}
      A_k &= \{\text{$k$ successes}\}. \\
      H_i &= \{\omega_i = 1\}, \quad 1 \le i \le N.
    \end{align*}
    We now notice that if $\omega = (\omega_1,\dots,\omega_N) \in A_k$ then
    \[
      \{\omega\} = \bigcap_{i=1}^{N} \widehat{H}_i,
    \]
    where
    \[
      \widetilde{H}_{i} = 
      \begin{cases}
        H_{i}, &\omega_{i} = 1 \\ 
        H_{i}^{c}, &\omega_{i} = 0
      \end{cases}.
    \]
    From \autoref{prop:indp} and since the events are independent:
    \[
      \mathbf P_N\left(\{w\}\right) =
      \mathbf P_N\left(\bigcap_{i=1}^{N} \widetilde{H}_i \right) = 
      \prod_{i=1}^{N} \mathbf P_N(\widetilde{H}_i) =
      p^k (1-p)^{N-k}
    \]
    Finally, we get
    \[
      P_N(A_k) = 
      \abs{A_k} p^k (1-p)^{N-k} = 
      \binom{N}{k} p^k (1-p)^{N-k}
    \]
    Also, because we know that $(A_k)_{k=1}^{N}$ are all disjoint and that
    $\cup_{k=1}^{N} A_k = \Omega$ we get that:
    \[
      \sum_{k=0}^{N}\mathbf{P}_{N}\left(A_{k}\right) = 
      \sum_{k=0}^{N}{\binom{N}{k}}p^{k}\left(1-p\right)^{N-k} = 
      \left(p+\left(1-p\right)\right)^{N} = 1,
    \]
    just as expected.
  \end{example}

  \begin{example}[Random Walks]
    Let there be a cute cat on the $\Z$ number line. 
    We know that each minute, the cat could move one step to the right with
    a probability of $p$, or one step to the left with probability $1-p$.
    We may wonder what are the chances the cat would be on the number $4$
    after $8$ minutes.

    It's not hard to see that this experiment is just like the previous
    experiment, and indeed if we denote $A_k$ the number of step the cat
    made to the right after in the product probability space 
    $(\Omega_8, \mathcal F_8, \mathbf P_8)$, we would get that:
    \[
      \mathbf P_8(A_k) = \binom{8}{k} p^k (1-p)^{8-k}.
    \]
    The position of the cat after $k$ step to the right would be 
    $k - (8 - k) = 2k - 8$. 
    Since we wonder about the probability of it being on the number $4$,
    we should calculate the probability of $A_6$, and we will get that:
    \[
      \mathbf{P}_{8}(A_{6}) = {\binom{8}{6}} p^{6} (1-p)^{8-6}.
    \]
    Notice that the result makes sense because as $P$ increases we get higher
    results.
  \end{example}
  \begin{example}[Infinite coin flips]
    We now consider the probability space 
    $(\Omega_{\infty}, \mathcal F_{\infty}, \mathbf P_{\infty})$
    corresponding to the product space of infinite Bernoulli experiments.
    We want to calculate the probability that the first success was in 
    the $n$-th experiment. Denote:
    \begin{align*}
      H_i &= \{\text{the $i$-th experiment resulted in success}\} \\
      R_n &= \{\text{first success was in the $n$-th experiment}\} = 
      \left(\bigcap_{i=1}^{n-1} H_i^c\right) \cap H_n.
    \end{align*}
    Since all the Bernoulli experiments are independent, from 
    \autoref{prop:indp} we get
    \[
      \mathbf{P}_{\infty}(R_{n}) = 
      \mathbf{P}_{\infty} \left(\left(\bigcap_{i=1}^{n-1} H_i^c\right) \cap H_n\right) = 
      \left(\prod_{i=1}^{n-1} \mathbf{P}_{\infty} \left(H_{i}^{c}\right)\right) \cdot \mathbf{P}_{\infty} \left(H_{n}\right) =
      (1-p)^{n-1}p.
    \]
    We may notice that the events $(A_n)_{n=1}^{\infty}$ are all independent,
    and also that
    \[
      \bigcup_{n=1}^{\infty} R_n = \Omega_\infty \setminus \{(0,0,0,\dots)\}.
    \]
    And as long as $p > 1$, we may assume that 
    $\mathbf P_\infty(0,0,0,\dots) = 0$.
    Using that assumption, we have that
    \begin{align*}
      \mathbf P_{\infty}(\{0,0,0,\dots\}) &= 
      1 - \mathbf P_\infty(\Omega_\infty \setminus \{0,0,0,\dots\}) =
      1 - \mathbf P_\infty(\bigcup_{n=1}^{\infty} R_n) \\ &=
      1 - \sum_{n=1}^{\infty} \mathbf P_\infty(R_n) =
      1 - p \sum_{n=1}^{\infty} (1-p)^{n-1} \\ &=
      \begin{cases}
        0, &p \in (0,1) \\
        1, &p = 0 \\
        0, &p = 1
      \end{cases}.
    \end{align*}
  \end{example}

  \newpage

  \section{Random Variables}





\end{document}
