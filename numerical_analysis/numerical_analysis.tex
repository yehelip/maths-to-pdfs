\documentclass[11pt,a4paper]{article}

\def\nyear {2025}
\def\nterm {Winter}
\def\nlecturer {}
\def\ncourse {Introduction to Numerical Analysis}
\def\ncoursehead {Numerical Analysis}

\makeatletter

% packages
\usepackage{amssymb,amsfonts,amsmath,calc,tikz,pgfplots,geometry,mathtools}
\usepackage{color}   % May be necessary if you want to color links
\usepackage[hidelinks]{hyperref}
\usepackage{forest}
\usepackage{commath} % ew
\usepackage{esdiff}
\usepackage{amsthm}
\usepackage{fancyhdr}
\usepackage{bm}
\usepackage{witharrows}
\usepackage{bookmark}
\usepackage{tikz-cd}
\usepackage{bbm}
\usepackage{textcomp}
\usepackage{gensymb}
\usepackage{cleveref}

% tikz libraries
\usetikzlibrary{positioning}
\usetikzlibrary{matrix}
\usetikzlibrary{arrows}
\usetikzlibrary{arrows.meta}
\usetikzlibrary{decorations.markings}

% Page style setup
\pagestyle{fancy}
\geometry{margin=1in}
\pgfplotsset{compat=1.18}
\setlength{\headheight}{14.6pt}
\addtolength{\topmargin}{-1.6pt}
\hypersetup{
    colorlinks=false,
    linktoc=section,
    linkcolor=black,
}

%% maketitle setup
\ifx \nauthor\undefined
  \def\nauthor{yehelip}
\else
\fi

\ifx \ncoursehead \undefined
\def\ncoursehead{\ncourse}
\fi

\lhead{\emph{\nouppercase{\leftmark}}}
\ifx \nextra \undefined
  \rhead{
    \ifnum\thepage=1
    \else
      \ncoursehead
    \fi}
\else
  \rhead{
    \ifnum\thepage=1
    \else
      \ncoursehead \ (\nextra)
    \fi}
\fi

\let\@real@maketitle\maketitle
\renewcommand{\maketitle}{\@real@maketitle\begin{center}
\begin{minipage}[c]{0.9\textwidth}\centering\footnotesize
These notes are not endorsed by the lecturers.
I have revised them outside lectures to incorporate supplementary explanations,
clarifications, and material for fun.
While I have strived for accuracy, any errors or misinterpretations 
are most likely mine.
\end{minipage}\end{center}}

% theorem environments
\theoremstyle{definition}
\newtheorem{definition}{Definition}[section]
\newtheorem{remark}{Remark}[section]
\newtheorem{example}{Example}[section]
\newtheorem{exercise}{Exercise}[section]
\newtheorem{paradox}{Paradox}[section]
\newtheorem*{solution}{Solution}
\theoremstyle{plain}
\newtheorem{theorem}{Theorem}[section]
\newtheorem{proposition}[theorem]{Proposition}
\newtheorem{lemma}[theorem]{Lemma}
\newtheorem{corollary}[theorem]{Corollary}

% tikz customization
\pgfarrowsdeclarecombine{twolatex'}{twolatex'}{latex'}{latex'}{latex'}{latex'}
\tikzset{->/.style = {decoration={markings,
                                  mark=at position 1
                                  with {\arrow[scale=2]{latex'}}},
                      postaction={decorate}}}
\tikzset{<-/.style = {decoration={markings,
                                  mark=at position 0 with {\arrowreversed[scale=2]{latex'}}},
                      postaction={decorate}}}
\tikzset{<->/.style = {decoration={markings,
                                   mark=at position 0 with {\arrowreversed[scale=2]{latex'}},
                                   mark=at position 1 with {\arrow[scale=2]{latex'}}},
                       postaction={decorate}}}
\tikzset{->-/.style = {decoration={markings,
                                   mark=at position #1 with {\arrow[scale=2]{latex'}}},
                       postaction={decorate}}}
\tikzset{-<-/.style = {decoration={markings,
                                   mark=at position #1 with {\arrowreversed[scale=2]{latex'}}},
                       postaction={decorate}}}
\tikzset{->>/.style = {decoration={markings,
                                  mark=at position 1 with {\arrow[scale=2]{latex'}}},
                      postaction={decorate}}}
\tikzset{<<-/.style = {decoration={markings,
                                  mark=at position 0 with {\arrowreversed[scale=2]{twolatex'}}},
                      postaction={decorate}}}
\tikzset{<<->>/.style = {decoration={markings,
                                   mark=at position 0 with {\arrowreversed[scale=2]{twolatex'}},
                                   mark=at position 1 with {\arrow[scale=2]{twolatex'}}},
                       postaction={decorate}}}
\tikzset{->>-/.style = {decoration={markings,
                                   mark=at position #1 with {\arrow[scale=2]{twolatex'}}},
                       postaction={decorate}}}
\tikzset{-<<-/.style = {decoration={markings,
                                   mark=at position #1 with {\arrowreversed[scale=2]{twolatex'}}},
                       postaction={decorate}}}

\pgfarrowsdeclare{biggertip}{biggertip}{%
  \setlength{\arrowsize}{1pt}
  \addtolength{\arrowsize}{.1\pgflinewidth}
  \pgfarrowsrightextend{0}
  \pgfarrowsleftextend{-5\arrowsize}
}{%
  \setlength{\arrowsize}{1pt}
  \addtolength{\arrowsize}{.1\pgflinewidth}
  \pgfpathmoveto{\pgfpoint{-5\arrowsize}{4\arrowsize}}
  \pgfpathlineto{\pgfpointorigin}
  \pgfpathlineto{\pgfpoint{-5\arrowsize}{-4\arrowsize}}
  \pgfusepathqstroke
}
\tikzset{
	EdgeStyle/.style = {>=biggertip}
}

\tikzset{circ/.style = {fill, circle, inner sep = 0, minimum size = 3}}
\tikzset{scirc/.style = {fill, circle, inner sep = 0, minimum size = 1.5}}
\tikzset{mstate/.style={circle, draw, black, text=black, minimum width=0.7cm}}

\tikzset{eqpic/.style={baseline={([yshift=-.5ex]current bounding box.center)}}}

\definecolor{mblue}{rgb}{0.2, 0.3, 0.8}
\definecolor{morange}{rgb}{1, 0.5, 0}
\definecolor{mgreen}{rgb}{0, 0.4, 0.2}
\definecolor{mred}{rgb}{0.5, 0, 0}

% topology
\newcommand{\Cells}{\text{Cells}}

% algebra
\DeclareMathOperator{\lcm}{lcm}
\DeclareMathOperator{\Out}{Out}
\DeclareMathOperator{\Aut}{Aut}
\DeclareMathOperator{\End}{End}
\DeclareMathOperator{\Inn}{Inn}
\DeclareMathOperator{\Mat}{Mat}
\DeclareMathOperator{\std}{std}
\DeclareMathOperator{\sgn}{sgn}
\DeclareMathOperator{\id}{id}
\DeclareMathOperator{\op}{op}
\DeclareMathOperator{\GL}{GL} % General linear group
\DeclareMathOperator{\SL}{SL} % Special linear group
\newcommand{\idealin}{\triangleleft}
\newcommand{\ip}[2]{\langle #1, #2 \rangle}
\newcommand{\bigslant}[2]
{{\raisebox{.2em}{$#1$}\left/\raisebox{-.2em}{$#2$}\right.}}

% analysis
\newcommand{\dx}{\dif x}
\newcommand{\dt}{\dif t}
\newcommand{\du}{\dif u}
\newcommand{\dv}{\dif v}
\newcommand{\dz}{\dif z}
\newcommand{\ds}{\dif s}
\newcommand{\dtheta}{\dif \theta}
\DeclareMathOperator{\im}{im}
\DeclareMathOperator{\cis}{cis}
\DeclareMathOperator{\Int}{Int}
\DeclareMathOperator{\diam}{diam}
\DeclareMathOperator{\supp}{supp}
\DeclareMathOperator{\Vol}{Vol} % Volume

% logic
\DeclareMathOperator{\MOD}{MOD}
\DeclareMathOperator{\Theory}{Theory}


% nice
\newcommand{\half}{\frac{1}{2}}
\newcommand{\pair}{\del}
\newcommand{\taking}[1]{\xrightarrow{#1}}
\newcommand{\inv}{^{-1}}
\newcommand{\ot}{\leftarrow}
\newcommand{\ninfty}{-\infty}
\newcommand{\floor}[1]{\left\lfloor #1 \right\rfloor}
\newcommand{\ceil}[1]{\left\lceil #1 \right\rceil}

% probability
\newcommand{\Prob}{\mathbf{P}}
\renewcommand{\vec}[1]{\boldsymbol{\mathbf{#1}}}
\DeclareMathOperator{\Bin}{Bin}
\DeclareMathOperator{\Geo}{Geo}
\DeclareMathOperator{\Poi}{Poi}
\DeclareMathOperator{\Exp}{Exp}
\DeclareMathOperator{\Var}{Var} % Variance
\DeclareMathOperator{\Cov}{Cov}

% special letters
\newcommand{\N}{\mathbb{N}}
\newcommand{\Z}{\mathbb{Z}}
\newcommand{\Q}{\mathbb{Q}}
\newcommand{\R}{\mathbb{R}}
\newcommand{\C}{\mathbb{C}}
\newcommand{\F}{\mathbb{F}}
\newcommand{\E}{\mathbb{E}}
\newcommand{\ps}{\mathcal{P}}
\newcommand{\M}{\mathcal{M}}
\renewcommand{\L}{\mathcal{L}}
\newcommand{\Omicron}{O}
\newcommand{\powerset}{\mathcal{P}}

% text
\newcommand{\st}{\text{ s.t. }}
\newcommand{\tand}{\quad \text{and} \quad}
\newcommand{\tor}{\quad \text{or} \quad}
\newcommand{\stand}{\text{ and }}
\newcommand{\stor}{\text{ or }}
\renewcommand{\tt}[1]{\textnormal{\textbf{(#1).}}} %tt=theorem title GET RID OF

% title format
\title{\textbf{\ncourse}}
\author{Based on lectures by \nlecturer \\\small Notes taken by \nauthor}
\date{\nterm\ \nyear}
\makeatother


\begin{document}
\maketitle

% Insert cool image here

\newpage
\tableofcontents
\newpage

\section{Introduction}
This course addresses what of all the math we have learned so far can
we compute with the computer

For example, given the task of computing the determinant of a matrix $A$,
we want to find the most efficient algorithm.
The most efficient algorithm is the one that uses the least amount of
operations.

One way of calculating $\det A$ is using permutations:
\[
  \det A = \sum_{\gamma \in S_n}
  \sgn(\gamma) a_{1 \sigma(1)} \cdot a_{2 \sigma(2)} \cdots a_{n \sigma(n)}.
\]
Each permutation costs us $n$ operations of multiplication, and to calculate
the sum we perform $n! - 1$ summation operations.
Since there are $n!$ permutations in $S_n$, the total amount of operations
of this algorithm is $n \cdot n! + n! - 1$.

An alternative algorithm would be to find the eigendecomposition
$A = U^{T} D U$ where $U$ is an orthogonal matrix, and then calculate
the product of the elements on the main diagonal of $D$ (which are the
eigenvalues of $A$).
Using this algorithm we can compute $\det A$ in $C n^3 + n - 1$ operations
for some costant $C$.
Although this algorithm is much faster than the previous ones, there are
more efficient algorithms still.

\begin{definition}[Big $O$ notation]
  Let $(x_n)_{n \geq 1}$ and $(y_n)_{n \geq 1}$ be sequences of real numbers.
  We say that $x_n = O(y_n)$ if there exist constants $C$ and $N$ such that
  for all $n > N$
  \[
    |x_n| \le C |y_n|
  \]
\end{definition}

Here are some examples
\begin{align*}
  \frac{n + 1}{n^2} &= O\del{\frac 1n} \\
  Cn^3 + n - 1 &= O(n^3) \\
  n \cdot n! + n! - 1 &= O(n \cdot n!)
\end{align*}

Using big $O$ notation we can say that direct determinant calculations
requires $O(n \dots n!)$ calculations, while using eigendecomposition
to calculate it takes $O(n^3)$ operations.

However, the big $O$ notation is still not very satisfying because
we still have
\[
  C n^3 + n - 1 = O(n \cdot n!).
\]
To fix this problem we introduce the $\Theta$ notation.

\begin{definition}[big $\Theta$ notation]
  Let $(x_n)_{n \geq 1}$ and $(y_n)_{n \geq 1}$ be sequences of real numbers.
  We say that $x_n = \Theta(y_n)$ if there exist constants $0 < c < C$ and $N$ 
  such that for all $n > N$
  \[
    c |y_n| \le |x_n| \le C |y_n|.
  \]
\end{definition}

Here are some examples
\begin{align*}
  a_k n^k + \cdots + a_1 n + a_0 &= \Theta(n^k) \\
  Cn^3 + n - 1 &= \Theta(n^3) \\
  n \cdot n! + n! - 1 &= \Theta(n \cdot n!)
\end{align*}

Another important method before we move to the next section,
is Horner's method.
It allows computing a polynomial $p(x) = \sum_{k=0}^{n} a_k x^n$ in
$\Theta(n)$ operations instead of $\Theta(n^2)$ operations.
It states that
\[
  \begin{aligned}
    &a_{0} + a_{1}x + a_{2}x^{2} + a_{3}x^{3} + \cdots + a_{n}x^{n} \\ = {}
    &a_{0} + x\del{a_{1} + x\del{a_{2} + x\del{a_{3} + \cdots + 
    x(a_{n-1} + x a_{n}) \cdots}}}.
  \end{aligned}
\]

\section{Digital Number Representation}
We usually use base-$10$ expansion to represent number.
For example, the number $x = 123.45$ means
\[
  x = 1 \times 10^2 + 2 \times 10^1 + 3 \times 10^0 + 4 \times 10^{-1} +
    5 \times 10^{-2}.
\]
In this case, we say that $x$ has a finite base-$10$ expansion.
All non-negative real numbers have a (possibly infinite) base-$10$ expansion.

\begin{theorem}
  \label{thm:bases}
  Let $x$ be a real number in $[0,1)$, and $b \geq 2$ an integer.
  Then there exists $L \in \Z$ and a sequence $(c_k)_{k=1}^{\infty}$ where
  each $c_k$ is in $\set{0,1,\dots,b-1}$, so that
  \[
    x = \sum_{k=1}^{\infty} c_k b^{-k}.
  \]
\end{theorem}

The motivation for this theorem is to prove for example, that computers,
who work in base $2$, can represent the same numbers we can represent
in the way that is more convenient to us like base $10$.

\begin{lemma}
  \label{lem:bases}
  If $b \geq 2$ is an integer, $L \in \Z$ and $x \in [0,b^{L+1})$.
  Then there exists $c_L \in \set{0,1,\dots,b-1}$ such that
  $x - c_L b^{L} \in [0,b^L)$.
\end{lemma}
\begin{proof}
  We notice that
  \[
    \bigcup_{j=0}^{b-1} \intco{j b^L, (j + 1) b^L} =
    [0, b^{L+1}).
  \]
  So there exists $c_L \in \set{0,1,\dots,b-1}$ so that 
  $x \in [c_L b^L, (c_L + 1) b^L)$.
  Then
  \[
    r_L = x - c_L b^L \in [0,b^L).
  \]
\end{proof}

We can now go back to prove \Cref{thm:bases}

\begin{proof}
  By assumption $x \in [0,b^0)$.
  By \Cref{lem:bases} with $L = -1$ we now define:
  \[
    r_1 := x - c_1 b^{-1} \in [0,b^{-1}).
  \]
  And continue indefinitely
  \[
    r_S := x - \sum_{k=1}^{S} c_k b^{-k} \in [0,b^{-k}).
  \]
  This implies that $r_S \taking{S \to \infty} 0$ which means that
  \[
    x = \sum_{k=1}^{\infty} c_k b^{-k}.
  \]
\end{proof}

We can now conclude from the theorem that for $x > 0$, we can find $L$
large enough so that $x < b^L$ and so $x b^{-L} < 1$.
We can then write
\[
  x = b^L \sum_{k=1}^{\infty} c_k b^{-k}.
\]

It is also possible in base-$b$ for $b \geq 2$ to represent any
non-negative number as:
\[
  x = (-1)^s \times [1.f]_b \times 2^m,
\]
for some $s \in \set{0,1}$, $m \geq 0$ and $f$ an infinite sequence
of digits.

\begin{remark}
  Obsviouly we can also represent $0$ because $0 \in \set{0,1,\dots,b}$.
\end{remark}

In a computer, we represent integers using $32$ bits.
We use one bit to represent the sign of the integer,
and the other as constants to the powers of two.
In this way we can represent every integer 
$n \in [- (2^{-31} - 1), 2^{31} - 1]$ as such:
\[
  n = (-1)^{c_31} \sum_{j=0}^{30}.
\]
Where $(c_0,c_1,\dots,c_{31})$ represent the bits.
If we add two numbers and get a result larger than $2^31 - 1$,
the computer will give us the result $Inf$.

When we consider rational numbers, we write them as
\[
    x = (-1)^s \times [1.f]_2 \times 2^m.
\]
We use a single bit to encode the sign, eight bits to encode $m$ (also
known as the exponent) and the remaining $32$ bits to encode $f$ (also known
as the normalized mantissa). We call $[1.f]_2$ the mantissa.
This method is known as single precision.

In double precision (which is the standard in Matlab) we use $64$ bits
to store each number. A single bit to encode the sign, $11$ bits to encode 
the exponent and the remaining $52$ bits to encode the normalized mantissa.
In general we only focus on single precision.

For now, since

% Add discussion about what's possible to represent and rounding numbers.

Denote $\floor x$ the closest number we can represent using this method.
Then the absolute error is
\[
  \abs{\floor x - x} \le 2^m 2^{-24}
\]
and the relative error is bounded by
\[
  \frac{\abs{\floor x - x}}{|x|} \le
  \frac{2^m 2^{-24}}{[1.f]_2 \times 2^m} \le
  2^{-24}.
\]
This means that the relative error of rounding is pretty low.
In double precision because we have more bits the relative error
is bounded by $2^{-52} \tilde 2 \times 10^{-16}$.
This number is called $eps$ in Matlab.

\begin{definition}[Relative error]
  Let $f(x)$ be some approximation of $x \neq 0$, the relative error
  is defined by
  \[
    \frac{|f(x) - x|}{|x|} < \epsilon
  \]
  if and only if
  \[
    f(x)= x(1 + \delta)
  \]
  for some $\delta \in (-\epsilon, \epsilon)$.
  We can get this equlity by setting $\delta = \frac{f(x) - x}{x}$.
\end{definition}

\begin{proposition}
  Let $x_1,x_2 > 0$ and assume there exists some $f \colon \R \to \R$ such that
  there exist $\delta_1$, $\delta_2$, $\delta_3$ such that
  \[
    f(x_1) = (1 + \delta_1) x_1 \stand
    f(x_2) = (1 + \delta_2) x_2 \stand
    f(f(x_1) \cdot f(x_2)) = (1 + \delta_3)[f(x_1) \cdot f(x_2)]
  \]
  where $|\delta_1|, |\delta_2|, |\delta_3| < \epsilon$.
  Then
  \[
    f(f(x_1) \cdot f(x_2)) =
    (1 + \delta) x_1 x_2
  \]
  For some $|\delta| < 3 \epsilon + 3 \epsilon^2 + \epsilon^3$.
\end{proposition}
\begin{proof}
  \[
    f(f(x_1) \cdot f(x_2)) =
    (1 + \delta_3)[f(x_1) \cdot f(x_2)] =
    (1 + \delta_3)(1 + \delta_2)(1 + \delta_1) x_1 x_2
  \]
  and
  \[
    (1 + \delta_3)(1 + \delta_2)(1 + \delta_1) =
    1 + \delta_1 + \delta_2 + \delta_3 + 
    \delta_1 \delta_2 + \delta_1 \delta_3 + \delta_2 \delta_3 + 
    \delta_1 \delta_2 \delta_3.
  \]
  Denote
  \[
    \delta =
    \delta_1 + \delta_2 + \delta_3 + 
    \delta_1 \delta_2 + \delta_1 \delta_3 + \delta_2 \delta_3 + 
    \delta_1 \delta_2 \delta_3.
  \]
  we get
  \[
    |\delta| \le 3 \epsilon + 3 \epsilon^2 + \epsilon^3
  \]
  which completes the proof.
\end{proof}





\end{document}
