\documentclass[11pt,a4paper]{article}

\def\nyear {2025}
\def\nterm {Winter}
\def\nlecturer {}
\def\ncourse {Analysis 2}

\makeatletter

% packages
\usepackage{amssymb,amsfonts,amsmath,calc,tikz,pgfplots,geometry,mathtools}
\usepackage{color}   % May be necessary if you want to color links
\usepackage[hidelinks]{hyperref}
\usepackage{forest}
\usepackage{commath} % ew
\usepackage{esdiff}
\usepackage{amsthm}
\usepackage{fancyhdr}
\usepackage{bm}
\usepackage{witharrows}
\usepackage{bookmark}
\usepackage{tikz-cd}
\usepackage{bbm}
\usepackage{textcomp}
\usepackage{gensymb}
\usepackage{cleveref}

% tikz libraries
\usetikzlibrary{positioning}
\usetikzlibrary{matrix}
\usetikzlibrary{arrows}
\usetikzlibrary{arrows.meta}
\usetikzlibrary{decorations.markings}

% Page style setup
\pagestyle{fancy}
\geometry{margin=1in}
\pgfplotsset{compat=1.18}
\setlength{\headheight}{14.6pt}
\addtolength{\topmargin}{-1.6pt}
\hypersetup{
    colorlinks=false,
    linktoc=section,
    linkcolor=black,
}

%% maketitle setup
\ifx \nauthor\undefined
  \def\nauthor{yehelip}
\else
\fi

\ifx \ncoursehead \undefined
\def\ncoursehead{\ncourse}
\fi

\lhead{\emph{\nouppercase{\leftmark}}}
\ifx \nextra \undefined
  \rhead{
    \ifnum\thepage=1
    \else
      \ncoursehead
    \fi}
\else
  \rhead{
    \ifnum\thepage=1
    \else
      \ncoursehead \ (\nextra)
    \fi}
\fi

\let\@real@maketitle\maketitle
\renewcommand{\maketitle}{\@real@maketitle\begin{center}
\begin{minipage}[c]{0.9\textwidth}\centering\footnotesize
These notes are not endorsed by the lecturers.
I have revised them outside lectures to incorporate supplementary explanations,
clarifications, and material for fun.
While I have strived for accuracy, any errors or misinterpretations 
are most likely mine.
\end{minipage}\end{center}}

% theorem environments
\theoremstyle{definition}
\newtheorem{definition}{Definition}[section]
\newtheorem{remark}{Remark}[section]
\newtheorem{example}{Example}[section]
\newtheorem{exercise}{Exercise}[section]
\newtheorem{paradox}{Paradox}[section]
\newtheorem*{solution}{Solution}
\theoremstyle{plain}
\newtheorem{theorem}{Theorem}[section]
\newtheorem{proposition}[theorem]{Proposition}
\newtheorem{lemma}[theorem]{Lemma}
\newtheorem{corollary}[theorem]{Corollary}

% tikz customization
\pgfarrowsdeclarecombine{twolatex'}{twolatex'}{latex'}{latex'}{latex'}{latex'}
\tikzset{->/.style = {decoration={markings,
                                  mark=at position 1
                                  with {\arrow[scale=2]{latex'}}},
                      postaction={decorate}}}
\tikzset{<-/.style = {decoration={markings,
                                  mark=at position 0 with {\arrowreversed[scale=2]{latex'}}},
                      postaction={decorate}}}
\tikzset{<->/.style = {decoration={markings,
                                   mark=at position 0 with {\arrowreversed[scale=2]{latex'}},
                                   mark=at position 1 with {\arrow[scale=2]{latex'}}},
                       postaction={decorate}}}
\tikzset{->-/.style = {decoration={markings,
                                   mark=at position #1 with {\arrow[scale=2]{latex'}}},
                       postaction={decorate}}}
\tikzset{-<-/.style = {decoration={markings,
                                   mark=at position #1 with {\arrowreversed[scale=2]{latex'}}},
                       postaction={decorate}}}
\tikzset{->>/.style = {decoration={markings,
                                  mark=at position 1 with {\arrow[scale=2]{latex'}}},
                      postaction={decorate}}}
\tikzset{<<-/.style = {decoration={markings,
                                  mark=at position 0 with {\arrowreversed[scale=2]{twolatex'}}},
                      postaction={decorate}}}
\tikzset{<<->>/.style = {decoration={markings,
                                   mark=at position 0 with {\arrowreversed[scale=2]{twolatex'}},
                                   mark=at position 1 with {\arrow[scale=2]{twolatex'}}},
                       postaction={decorate}}}
\tikzset{->>-/.style = {decoration={markings,
                                   mark=at position #1 with {\arrow[scale=2]{twolatex'}}},
                       postaction={decorate}}}
\tikzset{-<<-/.style = {decoration={markings,
                                   mark=at position #1 with {\arrowreversed[scale=2]{twolatex'}}},
                       postaction={decorate}}}

\pgfarrowsdeclare{biggertip}{biggertip}{%
  \setlength{\arrowsize}{1pt}
  \addtolength{\arrowsize}{.1\pgflinewidth}
  \pgfarrowsrightextend{0}
  \pgfarrowsleftextend{-5\arrowsize}
}{%
  \setlength{\arrowsize}{1pt}
  \addtolength{\arrowsize}{.1\pgflinewidth}
  \pgfpathmoveto{\pgfpoint{-5\arrowsize}{4\arrowsize}}
  \pgfpathlineto{\pgfpointorigin}
  \pgfpathlineto{\pgfpoint{-5\arrowsize}{-4\arrowsize}}
  \pgfusepathqstroke
}
\tikzset{
	EdgeStyle/.style = {>=biggertip}
}

\tikzset{circ/.style = {fill, circle, inner sep = 0, minimum size = 3}}
\tikzset{scirc/.style = {fill, circle, inner sep = 0, minimum size = 1.5}}
\tikzset{mstate/.style={circle, draw, black, text=black, minimum width=0.7cm}}

\tikzset{eqpic/.style={baseline={([yshift=-.5ex]current bounding box.center)}}}

\definecolor{mblue}{rgb}{0.2, 0.3, 0.8}
\definecolor{morange}{rgb}{1, 0.5, 0}
\definecolor{mgreen}{rgb}{0, 0.4, 0.2}
\definecolor{mred}{rgb}{0.5, 0, 0}

% topology
\newcommand{\Cells}{\text{Cells}}

% algebra
\DeclareMathOperator{\lcm}{lcm}
\DeclareMathOperator{\Out}{Out}
\DeclareMathOperator{\Aut}{Aut}
\DeclareMathOperator{\End}{End}
\DeclareMathOperator{\Inn}{Inn}
\DeclareMathOperator{\Mat}{Mat}
\DeclareMathOperator{\std}{std}
\DeclareMathOperator{\sgn}{sgn}
\DeclareMathOperator{\id}{id}
\DeclareMathOperator{\op}{op}
\DeclareMathOperator{\GL}{GL} % General linear group
\DeclareMathOperator{\SL}{SL} % Special linear group
\newcommand{\idealin}{\triangleleft}
\newcommand{\ip}[2]{\langle #1, #2 \rangle}
\newcommand{\bigslant}[2]
{{\raisebox{.2em}{$#1$}\left/\raisebox{-.2em}{$#2$}\right.}}

% analysis
\newcommand{\dx}{\dif x}
\newcommand{\dt}{\dif t}
\newcommand{\du}{\dif u}
\newcommand{\dv}{\dif v}
\newcommand{\dz}{\dif z}
\newcommand{\ds}{\dif s}
\newcommand{\dtheta}{\dif \theta}
\DeclareMathOperator{\im}{im}
\DeclareMathOperator{\cis}{cis}
\DeclareMathOperator{\Int}{Int}
\DeclareMathOperator{\diam}{diam}
\DeclareMathOperator{\supp}{supp}
\DeclareMathOperator{\Vol}{Vol} % Volume

% logic
\DeclareMathOperator{\MOD}{MOD}
\DeclareMathOperator{\Theory}{Theory}


% nice
\newcommand{\half}{\frac{1}{2}}
\newcommand{\pair}{\del}
\newcommand{\taking}[1]{\xrightarrow{#1}}
\newcommand{\inv}{^{-1}}
\newcommand{\ot}{\leftarrow}
\newcommand{\ninfty}{-\infty}
\newcommand{\floor}[1]{\left\lfloor #1 \right\rfloor}
\newcommand{\ceil}[1]{\left\lceil #1 \right\rceil}

% probability
\newcommand{\Prob}{\mathbf{P}}
\renewcommand{\vec}[1]{\boldsymbol{\mathbf{#1}}}
\DeclareMathOperator{\Bin}{Bin}
\DeclareMathOperator{\Geo}{Geo}
\DeclareMathOperator{\Poi}{Poi}
\DeclareMathOperator{\Exp}{Exp}
\DeclareMathOperator{\Var}{Var} % Variance
\DeclareMathOperator{\Cov}{Cov}

% special letters
\newcommand{\N}{\mathbb{N}}
\newcommand{\Z}{\mathbb{Z}}
\newcommand{\Q}{\mathbb{Q}}
\newcommand{\R}{\mathbb{R}}
\newcommand{\C}{\mathbb{C}}
\newcommand{\F}{\mathbb{F}}
\newcommand{\E}{\mathbb{E}}
\newcommand{\ps}{\mathcal{P}}
\newcommand{\M}{\mathcal{M}}
\renewcommand{\L}{\mathcal{L}}
\newcommand{\Omicron}{O}
\newcommand{\powerset}{\mathcal{P}}

% text
\newcommand{\st}{\text{ s.t. }}
\newcommand{\tand}{\quad \text{and} \quad}
\newcommand{\tor}{\quad \text{or} \quad}
\newcommand{\stand}{\text{ and }}
\newcommand{\stor}{\text{ or }}
\renewcommand{\tt}[1]{\textnormal{\textbf{(#1).}}} %tt=theorem title GET RID OF

% title format
\title{\textbf{\ncourse}}
\author{Based on lectures by \nlecturer \\\small Notes taken by \nauthor}
\date{\nterm\ \nyear}
\makeatother


\begin{document}
	\maketitle

  % Insert cool image here

  \newpage
  \tableofcontents
	\newpage

  \section{Proper Integrals}
  \begin{definition}[Riemann integrability]
    A function $f \colon \R \to \R$ is said to be Riemann integrable on the 
    interval $[a,b]$ if there exists $I \in \R$ such that for all 
    $\varepsilon > 0$ there exists a $\delta > 0$ such that 
    for any partition $X = (x_0,x_1,\dots,x_n)$ with $\lambda(X) < \delta$, 
    any sequence $(c_1,c_2,\ldots,c_n)$ such that $c_i \in[x_{i-1},x_i]$ for
    all $1 \le i \le n$ also satisfies:
    \[
      \abs{\sum_{i=1}^{n}{f(c_i)\Delta X_i} - I} < \varepsilon.
	  \]
    We also denote $I = \int_{a}^{b} f(x)\dx$.
  \end{definition}
  \begin{proposition}
	  Suppose $f$ is a Riemann integrable function,
    then $f^2$ is also Riemann integrable.
  \end{proposition}
  \begin{proof}
    We can notice that
    \begin{align*}
      &U(f,P) - D(f,P) < \varepsilon \\
      \Rightarrow\, &U(f,P) < D(f,P) + \varepsilon \\
      \Rightarrow\, &U(f,P)^2<D(f,P)^2 + 2\varepsilon D(f,P) + \varepsilon^2 \\
      \Rightarrow\, &U(f,P)^2 - D(f,P)^2 < \varepsilon (2D(f,P) + \varepsilon)
    \end{align*}	
    Since $f$ is integrable we know $2D(f,P) + \varepsilon$ is bounded.
    Denote the bound $M$.
    Let $\varepsilon > 0$.
    Choose the $\delta$ that matches 
    $\varepsilon_{\delta} = \min(\frac{\varepsilon}{2M+1},1)$ 
    under $f$'s integrability.
    We get:
      \begin{align*}
        &U(f,P) - D(f,P) < \varepsilon_{\delta} \\
        \Rightarrow\, &U(f,P)^2 - D(f,P)^2 <
          \varepsilon_{\delta} (2D(f,P) + \varepsilon_{\delta}) \\
        \Rightarrow\, &U(f,P)^2 - D(f,P)^2 <
          \frac{\varepsilon}{2M+1} (2M + 1) = \varepsilon \\
        \Rightarrow\, &U(f^2,P) - D(f^2,P) < \varepsilon
      \end{align*}	
    This shows that $f^2$ is Darboux integrable, which implies
    it is Riemann integrable, which completes the proof.
  \end{proof}

  \begin{definition}
    If $f$ is continuous, then $f$ is integrable.
  \end{definition}
  \begin{proof}
    Let $f$ be continuous on $[a,b]$.
    By the Cantor--Heine theorem it is uniformally continuous.
    We have
    \[
      U(f,P) - L(f,P) =
      \sum_{i=1}^{n}(M_i-m_i)\Delta X_i <^{*}
      \sum_{i=1}^n\varepsilon\Delta X_i = \varepsilon(b-a)
    \]
    $(*)$ This is becuase by definition we have that for all $\varepsilon > 0$
    there exists $\delta > 0$ such that $d(x,y) < \delta$. 
    This implies that $d(f(x),f(y)) < \varepsilon$
    and that delta is exactly what we wanted, which completes the proof.
  \end{proof}

  \begin{proposition}[Intermediate Value Theorem for Integrals]
    Let $f$ be a continuous funtion on $[a,b]$ then exists $x_0 \in [a,b]$
    such that: 
    \[
      \int_a^b f(x)\dx = f(x_0)(b - a).
    \]
  \end{proposition}
  \begin{proof}
    Since $f$ is continuous it is Riemann integrable. 
    From Weierstrass's theorem $f$ has a minimum and maximum in $[a,b]$ 
    which we will denote $m$ and $M$.
    We know have
    \[
      m (b - a) \le \int_a^b f(x)\dx \le M (b - a).
    \]
    From this follows that
    \[
      m \le \frac{\int_{a}^{b} f(x)\dx}{b - a} \le M.
    \]
    Denote $c := \frac{\int_{a}^{b} f(x)\dx}{b - a}$.
    By the intermediate value theorem we know that exists $x_0 \in (a,b)$
    such that $f(x_0) = c$ and thus:
    \[
      \int_{a}^{b} f(x)\dx = f(x_0)(b - a)
    \]
  \end{proof}

  \newpage

  \subsection{The fundamental theorem of calculus}
  \begin{theorem}\tt{Fundamental theorem of calculus, part one}
    Let $f$ be a continuous real-valued function
    defined on a closed interval $[a, b]$. 
    Let $F$ be the function defined as:
    \[
      F(x) = \int_{a}^{x} f(t)\,dt
    \]
    for all $x \in [a, b]$.
    Then $F$ is uniformly continuous on $[a, b]$ and 
    differentiable on the open interval $(a, b)$, and
    \[
      F'(x) = f(x) \quad \forall x \in (a, b).
    \]
  \end{theorem}
  \begin{theorem}\tt{Fundamental theorem of calculus, part two}
    Under the conditions of part one, if $f$ is Riemann integrable on $[a,b]$.
    Then:
    \[
      \int_{a}^{b} f(x)\dx = F(b) - F(a).
    \]
  \end{theorem}
  \begin{proof}
  \textbf{Part One}
  For any $x_1, x_1+\Delta x\in [a,b]$ we get:
  \[
    F(x_1 + \Delta x) - F(x_1) = 
    \int_{a}^{x_1 + \Delta x}{f(t)\,dt} - \int_{a}^{x_1}{f(t)\,dt} = 
    \int_{x_1}^{x_1 + \Delta x}{f(t)\,dt}
  \]
  According to the mean value theorem for integration 
  we get that for $c \in [x_1, x_1+\Delta x]$:
  \begin{align*}
    &\int_{x_1}^{x_1 + \Delta x}{f(t)\,dt} = f(c)\Delta x \\
    &\lim_{\Delta x\to 0}{\frac{F(x_1+\Delta x) - F(x_1)}{\Delta x}}
    = \lim_{\Delta x\to 0}{f(c)}
    = f(x_1)
  \end{align*}
  And thus from the squeeze theorem and $f$'s continuity 
  we get $F'(x_1) = f(x_1)$.
  \end{proof}
  \begin{proof}
  \textbf{Part Two}
  Let $P = (x_0, x_1, \ldots, x_n)$ be a partition of $[a,b]$ such 
  that $(x_0,x_n) = (a,b)$.
  Then we have:
  \begin{align*}
    F(b) - F(a) &= F(x_n) + \left[F(x_{n-1})-F(x_{n-1})\right] 
    + \cdots + \left[F(x_{1})-F(x_{1})\right] - F(x_0) \\
    &= \left[F(x_n)-F(x_{n-1})\right] + \left[F(x_{n-1})-F(x_{n-2})\right] 
    + \cdots + \left[F(x_1)-F(x_{0})\right] \\
    &= \sum_{i=1}^n{\left[F(x_{i})-F(x_{i-1})\right]}.
  \end{align*}
  Because $F$ is continuous on $[a,b]$ and differentiable on $(a,b)$, we can
  use Lagrange's theorem on $[x_i,x_{i-1}]$.
  Thus, there exists $c_i \in [x_i,x_{i-1}]$ such that
  \[
    \sum_{i=1}^n{\left[F(x_{i})-F(x_{i-1})\right]} 
    = \sum_{i=1}^n{\left[F'(c_i)(x_i-x_{i-1})\right]}.
  \]
  
  According to part one we get that $F'(c_i) = f(c_i)$ and so
  \begin{align*}
    F(b) - F(a) &= \sum_{i=1}^{n}{\left[f(c_i)(\Delta x_i)\right]} \\
    &\iff \\
    \lim_{\|\Delta x_i\|\to 0}{\left(F(b)-F(a)\right)} &= 
    \lim_{\|\Delta x_i\|\to 0}
    {\left(\sum_{i=1}^n{\left[f(c_i)(\Delta x_i)\right]}\right)} \\
    &\iff \\
    F(b)-F(a) &= \int_{a}^{b} f(x)\dx.
  \end{align*}
  \end{proof}
  
  \subsection{Length of a curve}
  Using integrals, we can actually find a formula for the length of a 
  continuous graph.
  Approximating the length of a graph using the pythagorian theorem for 
  partition $X = (x_0,x_1,\dots,x_n)$ we get:
  \[
    \sum_{i=1}^{n}{\sqrt{d(x_{i-1}, x_i)^2 + d(f(x_{i-1}), f(x)_i)^2}}.
  \]
  Assuming $f$ is continuous on $[a,b]$ and differentiable on $(a,b)$ 
  from Lagrange's theorem we get:
  \begin{align*}
    \sum_{i=1}^{n}{\sqrt{d(x_{i-1}, x_i)^2 + d(f(x_{i-1}), f(x)_i)^2}}
    &= \sum_{i=1}^{n}{\sqrt{(x_i - x_{i-1})^2 + (f'(c_i)(x_{i} - x_{i-1}))^2}}\\
    &= \sum_{i=1}^{n}{\sqrt{(x_i - x_{i-1})^2(1 + (f'(c_i))^2)}} \\
    &= \sum_{i=1}^{n}{\sqrt{1 + (f'(c_i))^2} \Delta X_i}.
  \end{align*}
  We can see that this summation is matching the integral
  \[
    \lim_{\norm{\Delta x_i} \to 0}{\sum_{i=1}^{n}
    {\sqrt{1 + (f'(c_i))^2}\Delta X_i}} = 
    \int_a^b{\sqrt{1 + (f'(x))^2}\dx}
  \]
  The result of this integral will give us the length of any continuous graph.

	\newpage
	
  \subsection{The limit comparison test}

  \begin{proposition}[Limit comparison test]
    Let $f$, $g$ be two integrable positive functions on $[a,M]$ for any 
    $M \in \R$. Suppose that
    \[
      \lim_{x\to\infty} \frac{f(x)}{g(x)} = c.
    \]
    Then
    \begin{itemize}
      \item If $c \in (0,\infty)$ then either both series converge or both 
        series diverge.
      \item If $c = 0$ the converganve of $f$ implies the convergance of $g$.
      \item If $c = \infty$ the converganve of $g$ implies the convergance 
        of $f$.
    \end{itemize}
  \end{proposition}
  \begin{proof}
    Assume $c \in (0,\infty)$.
    Let $\varepsilon > 0$ we know that exists $x_0 \in \R$ such that for all 
    $x_0 < x$ we have:
    \[
      g(x)(c - \varepsilon) < f(x) < g(x)(c + \varepsilon).
    \]
    If $g(x)$ converges then $f(x)$ converges by the squeeze theorem.
    Similarly if $g$ diverges we know that
    \[
      g(x)(c - \varepsilon) < f(x),
    \]
    So from a certain point onwards $f$ will meet the requirments of the 
    direct comparison test and will diverge which completes the proof.
  \end{proof}
  
  Here is some practise of the limit comparison test.

  \begin{exercise}
    Find the value of
    \[
      \lim_{n\to\infty}
      \left(\frac 1{n+1} + \frac 1{n+2} + \cdots + \frac {1}{2n}\right).
    \]
  \end{exercise}
  \begin{solution}
    By direct calculation we have:
    \begin{align*}
    \lim_{n\to\infty}
    {\left(\frac {1}{n+1} + \frac {1}{n+2} + \cdots + \frac {1}{2n}\right)} 
    &= \lim_{n\to\infty}
    {\sum_{k=1}^{n}{\frac {1}{n+k}}} 
    \\
    &= \lim_{n\to\infty}
    {\sum_{k=1}^{n}{\frac {1}{1 + \frac {k}{n}}\frac {1}{n}}} 
    \\
    &= \lim_{n\to\infty}
    {\sum_{k=1}^{n} {f(\frac {k}{n})\frac {1}{n}}} 
    \\
    &= \int_{0}^{1}{\frac{1}{1+x}}
    \end{align*}
  \end{solution}
	
  \begin{exercise}
    Check if the following integral converges or diverges:
    \[
      \int_{\frac{1}{2}}^{1} \frac{1}{x \sqrt{1-x}} \dx.
    \]
  \end{exercise}
  \begin{solution}
    This function seems to behave like $\frac{1}{\sqrt{1-x}}$ near $1$
    so let's compare them using the limit comparison test:
    \[
      \lim_{x\to 1^-} \frac{\frac{1}{x\sqrt{1-x}}}{\frac{1}{\sqrt{1-x}}}
      = \lim_{x\to 1^-}{\frac{1}{x}} = 1.
    \]
    By the limit comparison test we get that the integral converges.
  \end{solution}
	
	\newpage

  \section{Improper Integrals}
  \begin{definition}[Improper integral]
    Let $f \colon [a,\infty) \to \R$ be a function that is integrable on
    $[a,M]$ for all $M > a$. Define the improper integral
    \[
      \int_{a}^{\infty} f(x) \dx :=
      \lim_{M \to \infty} \int_{a}^{M} f(x) \dx.
    \]
    If the limit exists we say that the integral converges.
  \end{definition}
  
  \begin{definition}[Absolute convergence]
    An improper integral of a function $f$ is said to converge absolutely 
    if the integral of the absolute value of the integrand is finite---that is,
    if $\int_a^\infty \abs{f(x)}\dx = L$ for some finite $L \in \R$.
  \end{definition}
  \begin{remark}
    An improper integral of a function $f$ that converges, but does not
    converge absolutely, is said to converge conditionally.
  \end{remark}

  \begin{proposition}
    Let $\int_a^\infty f(x)\dx$ be an improper integral that 
    converge absolutely, then it also converges.
    In other words:
    \[
      \int_a^\infty \abs{f(x)}\dx \quad\mathrm{converges}\quad
      \implies
      \int_a^\infty {f(x)}\dx \quad\mathrm{converges}.
    \]
  \end{proposition}
  \begin{proof}
    Suppose $f$ converges absolutely. Consider:
    \begin{align*}
      f^+&=
      \begin{cases}
          f(x),  & f(x) > 0 \\
          0,  & f(x) \le 0
      \end{cases} \\
      f^-&=
      \begin{cases}
          -f(x),  & f(x) < 0 \\
          0,  & f(x) \ge 0
      \end{cases}
    \end{align*}
    We know that $\abs{f} = f^+ + f^-$ convereges and so by the direct 
    comparison test we get that  $\int_{a}^{\infty} f^+$ and 
    $\int_{a}^{\infty} f^-$ converge.
    Since $f = f^+ - f^-$ we also get that $\int_{a}^{\infty} f$ converges 
    as well which completes the proof.
  \end{proof}

  \begin{proposition}[Dirichlet's Test]
    If $a_n$ is a monotonic sequence and $\lim_{n\to\infty} a_n = 0$ and 
    $\sum b_n$ is bounded by $M$ then
    \[
      \sum_{n=1}^{\infty} a_n b_n \quad\mathrm{ converges}.
    \]
  \end{proposition}
  \begin{proof}
		Denote $B_{n} = \sum_{k=1}^{n}{b_{n}}$ and by summation by parts we see that
		\[
			\sum_{k=1}^{n}{a_{n}b_{n}} =
      a_{n}B_{n} - \sum_{k=1}^{n-1}{(a_{k+1}-a_{k})B_k}.
		\]
		Since $a_{n}$ converges to $0$ and $B_{n}$ is bounded 
		$\lim_{n\to\infty}{a_nB_n}$ exists.
		WLOG assume $a_n$ is increasing, we can also see that
		\[
    \sum_{k=1}^{n-1} \abs{(a_{k+1}-a_{k})B_k} \le
    \sum_{k=1}^{n-1} M\abs{(a_{k+1}-a_{k})} \le
			M \sum_{k=1}^{n-1}{(a_{k+1}-a_{k})}
		\] 
		And
		\[
			M \sum_{k=1}^{n-1}{(a_{k+1}-a_{k})} = M(a_{n}-a_{1})
		\]
		Which converges to $-Ma_1$ since $\lim_{n\to\infty}{a_n}=0$.
		That means that this sequence is bounded. Which means that
		$\sum_{k=1}^{n-1}{|(a_{k+1}-a_{k})B_k|}$ is also bounded.
		It is also monotonic which means it converges. And if
		$\sum_{k=1}^{n-1}{(a_{k+1}-a_{k})B_k}$ converges absolutly it also
		converges conditionally which means $\sum_{k=1}^{n-1}{(a_{k+1}-a_{k})B_k}$
		converges. And by additivity of limits we know 
		$\lim_{n\to\infty}\sum_{k=1}^{n}{a_{n}b_{n}}$ exists so 
		$\sum_{n=1}^\infty a_nb_n$ converges.
  \end{proof}
	
  \begin{proposition}[Abel's test]
	Suppose $\sum_{1}^{\infty}{a_n}$ converges, and $b_n$ is monotone and bounded.
    Then $\sum_{1}^{\infty}{a_nb_n}$ also converges.
  \end{proposition}
	\begin{proof}
		We know $b_n$ is monotone and bounded so it has a limit $\lim_{n\to\infty}{b_n}=b$
		This implies $\lim_{n\to\infty}{b_n-b}=0$. 
		Since $b_n-b$ is also monotonic we know by Dirichle's test that 
		$\sum_{1}^{\infty}{a_n(b_n-b)}$ converges. And by homogenity of series we know that
		 $\sum_{n=1}^{\infty}{a_n b}$ converges as well.
		That means $\sum_{n=1}^{\infty}{(a_n b)+a_n(b_n-b)}$ converges.
		So $\sum_{n=1}^{\infty}{(a_n b)+a_n(b_n-b)} = \sum_{n=1}^{\infty}{a_nb_n}$
    converges.
	\end{proof}

  \begin{proposition}[Root test]
    Suppose that we have the series $\sum_{n=1}{\infty} a_n$.
    Define
    \[
      L :=
      \lim_{n \to \infty} \sqrt[n]{\abs{a_n}} =
      \lim_{n \to \infty} \abs{a_n}^{\frac{1}{n}}.
    \]
    Then for,
    \begin{itemize}
      \item $L < 1$ the series is absolutely convergent (thus also convergent);
      \item $L > 1$ the series is divergent;
      \item $L = 1$ the series may be divergent, conditionally convergent,
        or absolutely convergent.
    \end{itemize}
  \end{proposition}
  
  \begin{proposition}[Ratio test]
    Suppose that we have the series $\sum_{n=1}{\infty} a_n$.
    Define
    \[
      L :=
      \lim_{n \to \infty} \abs{\frac{a_{n+1}}{a_{n}}}.
    \]
    Then for,
    \begin{itemize}
      \item $L < 1$ the series is absolutely convergent (thus also convergent);
      \item $L > 1$ the series is divergent;
      \item $L = 1$ the series may be divergent, conditionally convergent,
        or absolutely convergent.
    \end{itemize}
  \end{proposition}

  In case the ratio test doesn't work, there does exist a stronger
  test which we can use.
  
  \begin{proposition}[Rabbe's test]
    Let $a_n$ be a positive sequence, then if
    \[
      \lim_{n\to\infty} \left(n\left(1-\frac{a_n+1}{a_n}\right)\right) = q,
    \]
    then for,
    \begin{itemize}
      \item $q > 1$ the series is convergent;
      \item $q < 1$ the series is divergent;
      \item $q = 1$ the series may be divergent, conditionally convergent,
        or absolutely convergent.
    \end{itemize}
  \end{proposition}
	
	\newpage
	
  \begin{proposition}[Integral test for series]
    Let $f(x)$ be a positive monotone decreasing function on $[1,\infty]$. 
    Define $a_n = f(n)$. Then
    \[
      \sum_{n=1}^{\infty}{a_n} \quad \text{converges } \iff 
      \int_{1}^{\infty}{f(x)dx} \quad\text{converges}.
    \]
  \end{proposition}
  \begin{proof}
    Suppose the series converges then we get:
    \[
      \abs{\sum_{n=1}^{\infty} a_n} < M,
    \]
    but we also know that:
    \[
      0 \le
      \sum_{n=2}^{\infty} a_n \le 
      \int_{1}^{N+1} f(x)dx \le
      \sum_{n=1}^{\infty} a_n.
    \]
    This means that the integral is increasing and bounded so it's converging. 
    Suppose the integral was converging, to prove the series is also converging 
    we could show similarly it's "bounded" by the integral's bound.
  \end{proof}
	
	\newpage
	
  \begin{proposition}[Cauchy condensation test]
    Let $a_n$ be a non-increasing sequence of non-negative number.
    Then
    \[
       \sum_{n=1}^\infty {f(n)} \le 
       \sum_{n=0}^\infty {2^n f(2^n)} \le
      2\sum_{n=1}^\infty {f(n)}.
    \]
  \end{proposition}
  \begin{proof}
    This follows from a simple rearrangement of the series:
    \begin{align*}
      \sum_{n=1}^\infty{f(n)} 
      & = f(1) +  f(2) + f(3)  +  f(4) + f(5) + f(6) + f(7)  +\cdots \\
      & = f(1) + (f(2) + f(3)) + (f(4) + f(5) + f(6) + f(7)) +\cdots \\
       &\le f(1) + (f(2) + f(2)) + (f(4) + f(4) + f(4) + f(4)) +\cdots  \\
      & =  \sum_{n=0}^\infty{2^nf(2^n)} \\
      & = f(1) + (f(2) + f(2)) + (f(4) + f(4) + f(4) + f(4)) +\cdots \\
      & \le  (f(1)+f(1))+(f(2)+f(2))+(f(3)+f(3))+(f(4)+f(4)) \cdots \\
      & = 2\sum_{n=1}^\infty {f(n)}.
    \end{align*}
  \end{proof}
	
	\newpage
	
  \begin{proposition}[Leibniz's Test]
    Let $a_n$ be a monotone decreasing positive sequence such that
    \[
      \lim_{n\to\infty} a_n = 0.
    \]
    Then
    \[
      \sum_{n=1}^{\infty}(-1)^{\infty}{a_n} \quad \text{ converges}.
    \]
  \end{proposition}
  \begin{proof}
    Since $a_n$ is monotonically decreasing then we can say that
    \begin{align*}
      S_{2(m+1)} &= S_{2m} + (a_{2m+1} - a_{2m}) \ge S_{2m} \\
      S_{2(m+1)+1} &= S_{2m + 1} - (a_{2m+2} + a_{2m+3}) \le S_{2m + 1}
    \end{align*}
    Or in other words $S_{2m}$ monotonically increases and 
    $S_{2m+1}$  monotonically decreases. 
    But we also know that
    \[
      S_{2m+1} - S_{2m} =  a_{2m + 1} \ge 0
    \]
    And that means that
    \[
      a_1 - a_2 = S_2 \le S_{2m} \le S_{2m+1} \le S_1 = a_1
    \]
    In other words our monotone sequences are bounded and so they converge. 
    Recall as $m\to\infty$
    \[
      S_{2m + 1} - S_{2m} =  a_{2m + 1} \to 0
    \]
    So by Cantor's lemma $S_{2m+1}$, $S_{2m}$ converge to the same limit $L$.
    Moreover,
    \[
      S_{2m} \le L \le S_{2m + 1}
    \]
    And also
    \[
      \abs{S_k - L} \le a_{k + 1}
    \]
  \end{proof}
	
  \begin{definition}[Permutation]
    A permutation is defined as a bijection from a set to itself.
  \end{definition}

  \begin{definition}[Absolute convergence]
    A series $\sum_{n=1}^{\infty} a_n$ is said to converge absolutely 
    if $\sum_{n=1}^{\infty} \abs{a_n} = L$ for some finite $L \in \R$.
  \end{definition}
  \begin{remark}
    A series that converges, but does not converges absolutely, is said
    to be conditionally convergent.
  \end{remark}
  \begin{remark}
    If a series converges absolutely, it converges.
  \end{remark}

  \begin{theorem}\tt{Riemann series theorem}
    Suppose that $(a_{1}, a_{2}, a_{3}, \ldots)$ is a sequence of real numbers, 
    and that $\sum_{n=1}^{\infty} a_n$ is conditionally convergent. 
    Let $M \in \R \cup \{\infty,-\infty\}$.
    Then there exists a permutation $\sigma$ such that:
    \[
      \sum_{n=1}^{\infty} a_{\sigma(n)} = M.
    \] 
    There also exists a bijection $\pi$ such that
    $\sum_{n=1}^{\infty} a_{\pi(n)}$ diverges.
  \end{theorem}
  \begin{remark}
    If $\sum_{n=1}^{\infty} a_n$ is converges absolutely to 
    $M \in \R \cup \{\infty, -\infty\}$,
    then for any bijection $\sigma$, the series
    $\sum_{n=1}^{\infty} a_{\sigma(n)}$ will also converge to $M$.
  \end{remark}
  \begin{remark}
    If a series is converging then putting parentheses will not change the
    value to which it converges.
  \end{remark}

  \newpage
  
  \section{Function Sequences}
  \begin{definition}[Pointwise convergece]
    A function sequence $\{f_n\}_{n=1}^{\infty}$ is said to converge
    pointwise to a given function $f$, if for all $x \in \R$ we have
    \[
      \lim _{n \to \infty } f_{n}(x) = f(x).
    \]
    The function $f$ is said to be the pointwise limit of
    $\{f_n\}_{n=1}^{\infty}$.
  \end{definition}

  \begin{definition}[Uniform convergece]
    A function sequence $\{f_n\}_{n=1}^{\infty}$ is said to converge
    uniformly to a given function $f$, if for all $\varepsilon > 0$ there
    exists $N \in \N$ such that for all $n > N$ and $x \in \R$ we have
    \[
      \abs{f_{n}(x) - f(x)} < \epsilon.
    \]
    Equivalently, the sequence $\{f_n\}_{n=1}^{\infty}$ uniformly converges
    to $f$ if
    \[
      \lim_{n \to \infty }
      \sup\{\,\abs{f_{n}(x) - f(x)}:x\in \R\} = 0.
    \]
  \end{definition}

  \begin{theorem}\tt{Dini's theorem}
    Let $f_n(x)\to f(x)$ converge pointwise in $D = [a,b]$.
    If for all $x \in D$ the sequence $f_n(x)$ is monotonic,
    and $f$, $\{f_n\}_{n \geq 1}$ are continuous,
    then $f_n(x)$ converges to $f(x)$ uniformally.
  \end{theorem}

  Here are some important examples to why all the conditions in Dini's theorem
  are necessary.

  \begin{example}
    Consider the function sequence $f_n = x^n$ on the interval $(0,1)$.
    The seuqence converges pointwise to the constant function $f(x) = 0$.
    For all $x \in (0,1)$ the sequence $f_n(x)$ is monotonic.
    The function $f$, $\{f_n\}_{n \geq 1}$ are continuous.
    But since the interval is open and not closed, the conditions of
    Dini's theorem are not fulfilled, and indeed the function sequence
    does not converge uniformly on $(0,1)$.
  \end{example}

  \begin{example}
    The function sequence $f_n = \frac{n x}{1 + n^2 x^2}$ satisfies all
    the conditions exepct that it is not pointwise monotonic, and it
    does not converge unifromly on $\R$.
  \end{example}

  \subsection{Properties of Uniformally Converging Function Sequences}
  \begin{proposition}
    Suppose $f_n$ converges to $f$ uniformally, and that $f_n$ is
    continuous for any natural$n$. Then $f$ is continuous.
  \end{proposition}


  \begin{proposition}
    Suppose $f_n$ converges to $f$ uniformally on $[a,b]$,
    and that $f_n$ is integrable for any natural $n$.
    Then $f$ is integrable, and we have
    \[
      \int_{a}^{b} \int f_n \xrightarrow{n \to \infty}
      \int_{a}^{b} \int f.
    \]
  \end{proposition}

  \begin{proposition}
    Suppose $f_n$ is continuous differentiable on $I$ for any natural $n$
    such that:
    \begin{itemize}
      \item  $f_n'$ converges  uniformly on $I$;
      \item There exists $x_0 \in I$ such that $f_n(x_0)$ converges;
    \end{itemize}
    then $f_n$ uniformally converge on $I$ to the function $f$ and
    $f_n'\to f'$.
  \end{proposition}

  \begin{proposition}[Weierstrass $M$-test]
    Let $\sum_{n=1}^{\infty}{f_n(x)}$ be a function series. 
    Suppose exists a sequence $M_n$ such that:
    \begin{itemize}
        \item For any natural $n$ we have $|f_n(x)| \le M_n$;
        \item The series $\sum_{n=1}^{\infty}{M_n}$ converges.
    \end{itemize}
    Then $\sum_{n=1}^\infty f_n(x)$ converges uniformally.
  \end{proposition}
  \begin{proof}
    Since $M_n$ converges we can use Cauchy's criterion for the convegence 
    of a series get that for all $\varepsilon > 0$ there exists $N \in \N$
    such that for all $n > N$ and $p \in \N$ we have
    \[
      \abs{\sum_{k=1}^{n+p} M_k - \sum_{k=1}^{n} M_k} < \varepsilon.
    \]
    Since $0 \le M_n$ we get
    \[
      \sum_{k=n+1}^{n+p} M_k < \varepsilon,
    \]
    and finally, we have
    \[
      \abs{\sum_{k=n+1}^{n+p} f_k(x)}   \le 
      \sum_{k=n+1}^{n+p} \abs{f_k(x)} \le 
      \sum_{k=n+1}^{n+p} M_n < \varepsilon.
    \]
    which completes the proof.
  \end{proof}
		
	\newpage
	
	\section{Power Series Theorems}
	Suppose $\sum_{n=0}^\infty a_nx^n$ converges for some $x_0$, 
	then it absolutely converges for any $x$ such that $|x|<|x_0|$.
	Since the power series converges $\lim_{n\to\infty}{a_nx_0^n}=0$ 
	and so the sequence is bounded 
	and we denote that bound $M$.
	\[
		0 \le |a_nx^n| 
		= |a_nx_0^n|\left|\frac{x^n}{x_0^n}\right| 
		< M\left|\frac{x}{x_0}\right|^n
	\]
	And this sequence's sum is a geometric series so it converges and so does 
	$\sum_{n=1}^\infty|a_nx^n|$. 
	We also know that $|a_nx^n| < |a_nx_0^n|$ for all $n\in\N$ so according to 
	Weierstrass's M test $\sum_{n=1}^\infty|a_nx^n|$ uniformally converges. 
	Let
	\[
    X = \set{x \in \R}{\sum_{n=1}^\infty{a_n x^n}\text{ converges}}
	\]
	We claim that exists $R = \sup X$ - the radius of convergance - 
	and that the series converges if $|x|<|R|$ and diverges fo $|x|>|R|$. 
	For any $x>R$ the series diverges by definition of $R$. 
	If $x < -R$ we know that exists $x_1$ such that $R < x_1 < |x|$ such that 
	the series converges, in contradiction to $R$'s defintion. 
	if $|x|<|R|$ than there exists $x_2$ such that $|x|<|x_2|<|R|$ 
	for which the series converges and then it converges for $x$ as well.
	
  \begin{exercise}
    We know the series converges uniformally for any close intervel 
    properly inside $[-R,R]$.
    If it converges uniformally on $[0,R]$ then it is converging in $R$ as well.
  \end{exercise}
  \begin{exercise}
    Let a function series converge uniformally to $f$. 
    Prove $f$ is continuous on $(a,b)$.
  \end{exercise}
	
	\newpage
	
	\section{Cauchy--Hadamard + D'alembert Theorem}
  \begin{proposition}[Cauchy--Hadamard]
    Let $\sum_{n=0}^{\infty} a_n x^n$ be a series and 
    let $R$ be the radius of convergence of the series.
    Then:
    \[
      R = \frac{1}{\limsup_{n\to\infty}{\sqrt[n]{|a_n|}}}.
    \]
  \end{proposition}

  \begin{proof}
    Let
    \begin{align*}
      &\lim_{n\to\infty}{\sqrt[n]{|a_n|}} = L
      \\ &\Rightarrow
      \sqrt[n]{|a_nx^n|}=\sqrt[n]{|a_n|}|x|\to L|x|
      \\ &\Rightarrow
      \begin{cases}
      |x| < \frac{1}{L} = R & \text{The series converges} \\
      |x| > \frac{1}{L} = R & \text{The series diverges} \\
      \end{cases}
    \end{align*}
    This proves the series converges or diverges absolutely according to 
    the root test. 
    If it converges absolutly this implies it converges in the usual sense. 
    Suppose it diverges - by the root test we know that 
    if the series diverges the partial sums don't converge to $0$ 
    so the series must diverge as well.
  \end{proof}

  \begin{proposition}[D'alembert]
    Let $\sum_{n=0}^{\infty} a_n x^n$ be a series and 
    let $R$ be the radius of convergence of the series.
    Then:
    \[
      R = \frac{1}{\lim_{n\to\infty}{\left|\frac{a_{n+1}}
        {a_n}\right|}}.
    \]
  \end{proposition}

  \begin{proof}
    Let
    \begin{align*}
      &\lim_{n\to\infty}{\left|\frac{a_{n+1}}{a_n}\right|} = L \\
      &\Rightarrow
      \frac{|a_{n+1}x^{n+1}|}{|a_nx^n|} = 
      \frac{|a_{n+1}|}{|a_n|}|x|\to L|x| \\
      &\Rightarrow
      \begin{cases}
      |x|<\frac{1}{L} = R & \text{The series converges} \\
      |x|>\frac{1}{L} = R & \text{The series diverges} \\
      \end{cases}
    \end{align*}
    If the series converges absolutly we can be sure yet again that it converges. 
    If it diverges - than by the quotient test the partial sums 
    diverge and so the series must also diverge, 
    and the series will diverge as we claimed.
  \end{proof}

	\newpage
	
	\section{A Note on the Taylor Series}
	If $f$ is smooth on $(-R,R)$ then $f$ can be the limit of a power series 
  if and only if:
	\begin{align*}
		&\forall x\in(-R,R) \\
		&\lim_{n\to\infty}{R_n(x) = 
		 \lim_{n\to\infty}{\sum_{n=N+1}^\infty}{a_nx^n} = 0} \\
	\end{align*}
	This is because the following are equivalent:
	\begin{align*}
	    &\text{$f$ can be the limit of a power series} \\
		&\lim_{n\to\infty}{S_n(x)=f(x), \forall x\in(-R,R)} \\
		&\lim_{n\to\infty}{R_n(x)} = 
		\lim_{n\to\infty}{f(x)-S_n(x)=0, \quad \forall x\in(-R,R)} \\
	\end{align*}

	\newpage
	
	\section{Continuous Partial Derivatives imply Differentiability}
	\subsection{Semi-Proof}
	We want to find the tangetial plane to $f$ for $(x_0,y_0)$ 
	assuming that the partial derivatives are continuous at that point. 
	Let's denote
	\[
		z_o=f(x_0,y_0)\quad\mathrm{and}\quad A = 
		\frac{\partial f}{\partial x}(x_0,y_0) \quad
		\mathrm{and} \quad B = \frac{\partial f}{\partial y}(x_0,y_0)
	\]
	Now the tangential lines that intersect at $z_0$ and are parallel 
	to the axes (and in turn are perpendicular to one another) 
	are (since the derivatives are continuous)
	\begin{align*} 
		z &=  B(y-y_0)+z_0 \\ 
		z &=  A(x-x_0)+z_0
	\end{align*}
	Their directional vectors are in turn
	\begin{align*} 
		\vec{\beta} &=  (0,1,B) \\ 
		\vec{\alpha} &=  (1,0,A)
	\end{align*}
	And the normal vector to their spanning plane is
	\[
	\vec{N} = \begin{vmatrix}
		\hat{i} & \hat{j} & \hat{k}\\
		0 & 1 & B\\
		1 & 0 & A\\
	\end{vmatrix}
	 = (A,B,-1)
	\]
	And so the plane equation is
  \[
		A(x-x_0) + B(y-y_0) - (z-z_0) = 0
  \]
  Which means that
  \[
		z =
    z_0 +
    \frac{\partial f}{\partial x}(x - x_0) +
    \frac{\partial f}{\partial y}(x_0,y_0)(y - y_0).
  \]
	We have shown that if continuous partial derivatives exists at $(x_0,y_0)$ 
	then $f$ has a tangential plane at $(x_0,y_0)$ which is equivalent to 
	being differentiable at $(x_0,y_0)$
	
	\subsection{Note on Diffferentiability}
	We say that $f$ is differentiable at $(x_0,y_0)$ if exist $A$, $B$ such that
	\[
		f(x_0+h,y_0+k)-f(x_0,y_0) 
		= Ah + Bk + \alpha(h,k)\sqrt{h^2+k^2}
		= Ah + Bj + \alpha(h,k)h + \beta(h,k)k
	\]
	where $\lim_{(h,k)\to(0,0)}{\alpha(h,k)}=0$ and 
  $\lim_{(h,k)\to (0,0)}{\beta(h,k)}=0$.
  This is equivalent to
	\[
		\lim_{(x,y)\to(x_0,y_0)}
    \cfrac{f(x,y) - f(x_0,y_0) - 
     \cfrac{\partial f}{\partial x}(x_0,y_0) - \cfrac{\partial f}
    {\partial y}(x_0,y_0)}{\sqrt{(y - y_0)^2 + (y - y_0)^2}} = 0
	\]

	\newpage
	
	\section{Leibniz integral rule}
  \begin{theorem}\label{thm:leibniz-integral-rule}\tt{Leibniz integral rule}
    Let $f(x,y)$ be continuous on a rectangle $[a,b] \times [c,d]$ 
    and suppose $\frac{\partial f}{\partial y}(x,y)$ exists and is continuous 
    on $[a,b] \times [c,d]$. 
    Define $F(y) = \int_a^b{f(x,y)}\dx$.
    Then $F$ is differentiable on $[c,d]$ and
    \[
      F'(y) =
      \frac{d}{dy}\int_a^b{f(x,y)}\dx =
      \int_{a}^{b}{\frac{\partial f}{\partial y}(x,y)\, dx}
    \]
  \end{theorem}
  
  In order to prove this theorem, we first need to prove a lemma.
  \begin{lemma}
    Let $f(x,y)$ be continuous on $[a,b] \times [c,d]$.
    Then $F(y) = \int_{a}^{b} f(x,y)\dx$ is uniforamlly contiuous on $[c,d]$.
  \end{lemma}
  \begin{proof}
    We know $f$ is contiuous on a compact space so it is uniformally continuous 
    there. This means that for any $\varepsilon > 0$ there exists $\delta > 0$
    such that for any $p_1 = (x_1,y_2)$, $p_2 = (x_2,y_2)$ such that 
    $d(p_1,p_2) < \delta$ we have
    \[
      \abs{f(p_1) - f(p_2)} < \varepsilon.
    \]
    Now consider $y_1, y_2 \in[c,d]$ such that $d(y_1,y_2) < \delta$.
    We know that for all $x \in [a,b]$ that $d((x,y_1),(x,y_2)) > \delta$
    and then we can see that
    \begin{align*}
      |F(y_1)-F(y_2)| 
      = \abs{\int_{a}^{b} f(x,y_1)\dx - \int_a^b{f(x,y_2)\dx}}
      = \abs{\int_{a}^{b} (f(x,y_1) - f(x,y_2))\dx} \\
      \le \int_{a}^{b} \abs{(f(x,y_1) - f(x,y_2))}\dx < \varepsilon(b - a)
    \end{align*}
  \end{proof}

  We can now go back to prove \autoref{thm:leibniz-integral-rule}.
  \begin{proof}
    Now denote $G(y) := \int_a^b{\frac{\partial f}{\partial y}(x,y)}\dx$,
    by the previous lemma $G$ is continuous. Thus,
    \[
      \Delta F =
      F(y + \Delta y) - F(y) =
      \int_a^b{f(x,y + \Delta y)\dx} - \int_a^b{f(x,y)}\dx =
      \int_a^b{(f(x,y + \Delta y) - f(x,y))\dx}.
    \]
    We know by Lagrange's theorem that exists $t \in (0,1)$ such that
    \begin{align*}
      &\frac{f(x,y + \Delta y) - f(x,y)}{\Delta y} =
      \frac{\partial f}{\partial y}(x,y + \Delta y) \\
      &\Rightarrow \int_a^b{(f(x,y + \Delta y) - f(x,y))\dx} =
      \int_a^b{\left(\frac{\partial f}{\partial y}
      (x,y + t\Delta y)\Delta y\right)}\dx \\
      &\Rightarrow \frac{\Delta F}{\Delta y} =
      \int_a^b{\left(\frac{\partial f}{\partial y}(x, y + t\Delta y)\right)}\dx
      =
      \footnote[2]{Since $G$ is continuous as $\Delta y\to 0$}
      G(y + t\Delta y) \to G(y)
    \end{align*}
    which completes the proof.
  \end{proof}

  \newpage

  We can generalize this rule further to get the generalized Leibniz integral
  rule.
  \begin{theorem}\tt{Generalized Leibniz integral theorem}
    Let $f(x,y)$ be a continuously differentiable function on 
    $[a,b] \times [c,d]$.
    Suppose that $\frac{\partial f}{\partial y}(x,y)$ exists and is continuous
    on $[a,b] \times [c,d]$,
    and that $\alpha(y)$, $\beta(y)$ are differentiable on $[c,d]$.
    Define $F(y) := \int_{\alpha(y)}^{\beta(y)}{f(x,y)}\dx$.
    Then $F$ is differentiable on $[c,d]$ and
    \[
      F'(y) =
      \int_{\alpha(y)}^{\beta(y)}{\frac{\partial f}{\partial y}(x,y)\, dx} + 
      f(\beta(y),y)\beta'(y) - f(\alpha(y),y)\alpha'(y).
    \]
  \end{theorem}
  \begin{proof}
    Denote $\Phi(s,t,y) := \int_s^t{f(x,y)\dx}$. Then
    \[
      F(y) = \Phi(\alpha(y),\beta(y),y) = \int_s^t{f(x,y)\dx}.
    \]
    And now
    \[
      F'(y) = \frac{\partial\Phi}{\partial s}\frac{ds}{dy}
      +\frac{\partial\Phi}{\partial t}\frac{dt}{dy}
      +\frac{\partial\Phi}{\partial y}\frac{dy}{dy}.
    \]
    By \autoref{thm:leibniz-integral-rule} and the fundamental theorem
    we get
    \[
      F'(y) = -f(\alpha(y),y) \alpha'(y) + f(\beta(y),y)\beta'(y)
      +\int_{\alpha(y)}^{\beta(y)}{\frac{\partial f}{\partial y}(x,y)\dx},
    \]
    which completes the proof.
  \end{proof}

  \newpage

  \section{Fubini's Theorem}
  \begin{theorem}[Fubini's theorem]
    Let $f(x,y)$ be a continuous function on $[a,b] \times [c,d]$.
    Then
    \[
      \int_a^b\left(\int_c^d{f(x,y)\, dy}\right)\dx
    = \int_c^d\left(\int_a^b{f(x,y)\, dx}\right)\,dy.
    \]
  \end{theorem}
  \begin{proof}
    Denote
    \begin{align*}
      &\varphi(t) := \int_c^t\left(\int_a^b{f(x,y)\, dx}\right)\,dy \\
      &\Psi(t) := \int_a^b\left(\int_c^t{f(x,y)\, dy}\right)\dx.
    \end{align*}
    Since $f(x,y)$ is continuous we get that $F(y) = \int_a^b{f(x,y)\dx}$ is 
    also continuous. Thus, by the fundemetal theorem,
    \[
      \varphi'(t) = \frac{d}{dt} \int_{c}^{t} F(y)\,dy =
      F(t) = \int_{a}^{b} f(x,t)\dx.
    \]
    Denote $G(x,t) := \int_c^t{f(x,y)\,dy}$.
    Then, by the fundemental theorem again we get
    \[
      \frac{\partial G}{\partial t} = f(x,t).
    \]
    And thus by the generalized Leibniz Integral Rule
    \[
      \Psi'(t) = \frac{d}{dt}\int_a^b{G(x,t)\dx} = \int_a^b{f(x,t)\dx}.
    \]
    We concluded that $\varphi$, $\Psi$ have the same derivative.
    That means that they at most differ by a constant $L$.
    However, since $\varphi(c) = \Psi(c) = 0$ we get that $L = 0$ and thus
    \[
      \varphi = \Psi,
    \]
    and more specifically $\varphi(d) = \Psi(d)$.
    This means that
    \[
      \int_a^b\left(\int_c^d{f(x,y)\, dy}\right)\dx
    = \int_c^d\left(\int_a^b{f(x,y)\, dx}\right)\,dy
    \]
    which completes the proof.
  \end{proof}

  \newpage
  
  \section{The Chain Rule}
  \begin{proposition}[The chain rule]
    Let $f(x,y)$ have continuous partial derivatives on domain $D$.
    Let $x(t)$, $y(t)$ be differentiable on the interval $I$ such that
    for all $t \in I$ we have $(x(t),y(t) \in D)$.
    Denote $F(t) := f(x(t),y(t))$.
    Then,
    \[
      F'(t) = 
      \frac{\partial f}{\partial x}\biggr|_{(x(t),y(t))}\frac{dx}{dt} + 
      \frac{\partial f}{\partial y}\biggr|_{(x(t),y(t))}\frac{dy}{dt}
    \]
  \end{proposition}
  \begin{proof}
    \[
      \frac{dF}{dt} =
      \lim_{\Delta t\to 0}{\frac{F(t+\Delta t)-F(t)}{\Delta t}} =
      \lim_{\Delta t\to 0}
      {\frac{f(x(t+\Delta t),y(t+\Delta t))-f(x(t),y(t))}{\Delta t}} 
    \]
    Denote
    $$\begin{cases}
      \Delta x = x(t+\Delta t) - x(t) \\
      \Delta y = y(t+\Delta t) - y(t) 
    \end{cases}$$
    \[
      = \lim_{\Delta t\to 0}
      {\frac{f(x(t) + \Delta x,y(t)+\Delta y)-f(x(t),y(t))}{\Delta t}} 
      = \lim_{\Delta t\to 0}{\frac{\Delta f}{\Delta t}}.
    \]
    Since $f$ is differentiable
    \[
      \Delta f =
      \frac{\partial f}{\partial x}\Delta x + 
      \frac{\partial f}{\partial y}\Delta y + 
      \alpha(\Delta x,\Delta y)\Delta x + \beta(\Delta x,\Delta y)\Delta y
    \]
    where $\alpha,\beta\to 0 $ so:
    \[
      F'(t) =
      \lim_{\Delta t\to 0}{\frac{\Delta f}{\Delta t}} =
      \lim_{\Delta t\to 0}{\frac{\frac{\partial f}{\partial x}\Delta x + 
      \frac{\partial f}{\partial y}\Delta y +
      \alpha(\Delta x,\Delta y)\Delta x + 
      \beta(\Delta x,\Delta y)\Delta y}{\Delta t}}.
    \]
    \[
      = \frac{\partial f}{\partial x}\biggr|_{(x(t),y(t))}\frac{dx}{dt} + 
        \frac{\partial f}{\partial y}\biggr|_{(x(t),y(t))}\frac{dy}{dt}
    \]
    which completes the proof.
  \end{proof}

  \begin{corollary}
    suppose $F(u,v) = f(x(u,v),y(u,v))$ then we see that
    \[
      \frac{\partial f}{\partial u} =
      \frac{\partial f}{\partial x}
      \frac{\partial x}{\partial u} +
      \frac{\partial f}{\partial y}
      \frac{\partial y}{\partial u}
    \tand
      \frac{\partial f}{\partial v} =
      \frac{\partial f}{\partial x}
      \frac{\partial x}{\partial v} +
      \frac{\partial f}{\partial y}
      \frac{\partial y}{\partial v}.
    \]
  \end{corollary}


	\newpage
	\section{Substitution For Multiple Variables}
	Let $f$ be integrable over Domain $D$.
  Let $x(u,v)$ and $y(u,v)$ be in $C_1$\footnote{continuously differentiable},
  and let them define an invertible transformation $\varphi:D\to E$ 
  where $D$ is defined on an $xy$ plane and $E$ on an $uv$ plane.
  Now suppose
	\[
		\mathbb{J} =\frac{\partial(x,y)}{\partial(u,v)} = 
	  \begin{bmatrix}
      \frac{\partial x}{\partial u} & 
      \frac{\partial x}{\partial v} & \\[1ex]
      \frac{\partial y}{\partial u} & 
      \frac{\partial y}{\partial v}
	  \end{bmatrix} \neq 0 \;\;\;\forall (u,v)\in E
	\]
	$(*)$ It can be equal to $0$ in the domain if the measure of the set of
  those points is $0$.

	Then
	\[
		\iint_D f(x,y) \, dx \, dy =
    \iint_E f(x(u,v),y(u,v)) |\mathbb{J}| \, du \, dv.
	\]


	\newpage
  \begin{exercise}[the Gaussian integral]
    Calculate the value of the following integral:
    \[
      \iint_{-\infty}^{\infty} e^{-x^2}.
    \]
  \end{exercise}
  \begin{solution}
    First notice that
    \[
      \left(\int_{-\infty}^\infty{e^{-x^2}}\, dx\right)^2 = 
      \left(\int_{-\infty}^\infty{e^{-x^2}}\, dx\right) 
      \left(\int_{-\infty}^\infty{e^{-y^2}}\, dy\right) = 
      \int_{-\infty}^\infty\int_{-\infty}^\infty{e^{-(x^2+y^2)}}\, dx\, dy.
    \]

    Now consider the integral in polar coordinates.

    \begin{align*}
      \int_{-\infty}^{\infty}
      \int_{-\infty}^{\infty} e^{-(x^2+y^2)} \, dx\, dy &=
      \int_{0}^{\infty} \int_{0}^{2\pi} e^{-r^2}r\, d\theta\, dr = 
      \int_{0}^{\infty} \int_{0}^{2\pi} r e^{-r^2}\, d\theta\, dr \\ &=
      \int_{0}^{\infty} \theta r e^{-r^2} 
      \biggr\vert^{\theta = 2\pi}_{\theta = 0}\, dr =
      2 \pi \int_{0}^{\infty} r e^{-r^2}\, dr \\ &=
      2 \pi \int_{0}^{\infty} r e^{-r^2}\, dr.
    \end{align*}
    Finally we get 
    \begin{align*}
      2 \pi \int_{0}^{\infty} r e^{-r^2}\, dr &= 
      2 \pi \lim_{M\to\infty} -\frac 12{e^{-r^2}}\biggr\vert^M_0 \\ &=
      2 \pi \lim_{M\to\infty} -\frac 12   (e^{-M^2} - e^{-0^2}) =
      2 \pi (- \frac 12(0-1)) = \pi. \\
    \end{align*}
    Going back to the original integral we get
    \[
      \int_{-\infty}^\infty{e^{-x^2}}\, dx =
      \sqrt{\left(\int_{-\infty}^\infty{e^{-x^2}}\, dx\right)^2} =
      \sqrt{\pi}.
    \]
  \end{solution}
  



\end{document}
