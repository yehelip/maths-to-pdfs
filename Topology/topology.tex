\documentclass[11pt,a4paper]{article}
\usepackage{amssymb,amsfonts,amsmath,calc,tikz,pgfplots,geometry}
\usepackage{color}   %May be necessary if you want to color links
\usepackage{hyperref}
\usepackage{amsthm}
\usetikzlibrary{positioning}
\geometry{margin=1in}
\hypersetup{
    colorlinks=false, %set true if you want colored links
    linktoc=all,   %set to all if you want both sections and subsections linked
    linkcolor=black,  %choose some color if you want links to stand out
}
%%%%%%%%%%%%%%%%%%%%%%%%%%%%%%%%%%%%%%%%%%%%%%%%%%%%%%%%%%%%%%%%%%%%%%%%%%%%%%%
\theoremstyle{plain}
\newtheorem{theorem}{Theorem}[section]
\newtheorem{lemma}[theorem]{Lemma}
\newtheorem{proposition}[theorem]{Proposition}
\newtheorem{corollary}[theorem]{Corollary}
\newtheorem{definition}{Definition}[section]
\newtheorem{remark}{Remark}[section]
\newtheorem{example}{Example}[section]
\DeclareMathOperator{\lcm}{lcm}
\DeclareMathOperator{\idealin}{\triangleleft}
\DeclareMathOperator{\im}{im}
\DeclareMathOperator{\Aut}{Aut}
\DeclareMathOperator{\End}{End}
\DeclareMathOperator{\Inn}{Inn}
\DeclareMathOperator{\Out}{Out}
\DeclareMathOperator{\Mat}{Mat}
\DeclareMathOperator{\std}{std}
\newcommand{\N}{\mathbb{N}}
\newcommand{\Z}{\mathbb{Z}}
\newcommand{\Q}{\mathbb{Q}}
\newcommand{\R}{\mathbb{R}}
\newcommand{\C}{\mathbb{C}}
\newcommand{\F}{\mathbb{F}}
\newcommand{\Omicron}{O}
\newcommand{\ip}[2]{\langle #1, #2 \rangle}
\newcommand{\bigslant}[2]
{{\raisebox{.2em}{$#1$}\left/\raisebox{-.2em}{$#2$}\right.}}
%%%%%%%%%%%%%%%%%%%%%%%%%%%%%%%%%%%%%%%%%%%%%%%%%%%%%%%%%%%%%%%%%%%%%%%%%%%%%%%
\title{\textbf{Introduction to Metric and Topological Spaces}}
\author{Yeheli Fomberg}
\date{326269651}
\begin{document}
	\maketitle
	\newpage
	\section{Metric Spaces}
	First we will begin with metric spaces.
	\begin{definition}
	Let $X$ be a non-empty set. A metric on $X$ is a function 
	$d \colon X \times X \to [0,\infty]$ such that for all $x,y,z \in X$,
	\end{definition}
	\begin{enumerate}
	\item $d(x,y) = 0$ if and only if $x = y$;
	\item $d(x,y) = d(y,x)$ (symmetry);
	\item $d(x,z) \le d(x,y) + d(y,z)$ (triangle inequality);
	\end{enumerate}
	\emph{The pair $(X,d)$ is said to be a \textbf{metric space}.}
	\begin{example}
	Let $X$ be a non-empty set. Let $d \colon X \times X \to [0,\infty)$ be
	the function such that for $x,y \in X$,
	\[
		d(x,y) := \begin{cases}
			0, & x=y \\
			1, & x \neq y
		\end{cases}
	\]
	The function $d$ is a metric and it is called \textbf{the discrete metric}
	on $X$.
	\end{example}
	\begin{example}
	Let $X = \R^n$ and define the function:
	\[
		d(x-y) := |x - y|
	\]
	Where $|\cdot | \colon \R \to \R$ is the Eclidean norm function. 
	Then the pair $(X, d)$ forms a metric space.
	\end{example}
	\begin{example}
	Let $(X, N)$ be an arbitrary normed space and define the function:
	\[
		d(x-y) := N(x - y)
	\] 
	Then the pair $(X, d)$ forms a metric space.
	\end{example}
	
	
	
\end{document}