\documentclass[11pt,a4paper]{article}
\usepackage{amssymb,amsfonts,amsmath,calc,tikz,pgfplots,geometry}
\usepackage{color}   %May be necessary if you want to color links
\usepackage{hyperref}
\usepackage{amsthm}
\usetikzlibrary{positioning}
\geometry{margin=1in}
\hypersetup{
    colorlinks=false, %set true if you want colored links
    linktoc=all,   %set to all if you want both sections and subsections linked
    linkcolor=black,  %choose some color if you want links to stand out
}
%%%%%%%%%%%%%%%%%%%%%%%%%%%%%%%%%%%%%%%%%%%%%%%%%%%%%%%%%%%%%%%%%%%%%%%%%%%%%%%
\theoremstyle{plain}
\newtheorem{theorem}{Theorem}[section]
\newtheorem{lemma}[theorem]{Lemma}
\newtheorem{proposition}[theorem]{Proposition}
\newtheorem{corollary}[theorem]{Corollary}
\newtheorem{definition}{Definition}[section]
\newtheorem{remark}{Remark}[section]
\newtheorem{example}{Example}[section]
\DeclareMathOperator{\lcm}{lcm}
\DeclareMathOperator{\idealin}{\triangleleft}
\DeclareMathOperator{\im}{im}
\DeclareMathOperator{\Aut}{Aut}
\DeclareMathOperator{\End}{End}
\DeclareMathOperator{\Inn}{Inn}
\DeclareMathOperator{\Out}{Out}
\DeclareMathOperator{\Mat}{Mat}
\DeclareMathOperator{\std}{std}
\newcommand{\N}{\mathbb{N}}
\newcommand{\Z}{\mathbb{Z}}
\newcommand{\Q}{\mathbb{Q}}
\newcommand{\R}{\mathbb{R}}
\newcommand{\C}{\mathbb{C}}
\newcommand{\F}{\mathbb{F}}
\newcommand{\Omicron}{O}
\newcommand{\ip}[2]{\langle #1, #2 \rangle}
\newcommand{\bigslant}[2]
{{\raisebox{.2em}{$#1$}\left/\raisebox{-.2em}{$#2$}\right.}}
%%%%%%%%%%%%%%%%%%%%%%%%%%%%%%%%%%%%%%%%%%%%%%%%%%%%%%%%%%%%%%%%%%%%%%%%%%%%%%%
\title{\textbf{Introduction to Metric and Topological Spaces}}
\author{Yeheli Fomberg}
\date{326269651}
\begin{document}
	\maketitle
	\newpage
	\section{Metric Spaces}
	First we will begin with metric spaces.
	\begin{definition}
	Let $X$ be a non-empty set. A metric on $X$ is a function 
	$d \colon X \times X \to [0,\infty]$ such that for all $x,y,z \in X$,
	\end{definition}
	\begin{enumerate}
	\item $d(x,y) = 0$ if and only if $x = y$;
	\item $d(x,y) = d(y,x)$ (symmetry);
	\item $d(x,z) \le d(x,y) + d(y,z)$ (triangle inequality);
	\end{enumerate}
	\emph{The pair $(X,d)$ is said to be a \textbf{metric space}.}
	\begin{example}
	Let $X$ be a non-empty set. Let $d \colon X \times X \to [0,\infty)$ be
	the function such that for $x,y \in X$,
	\[
		d(x,y) := \begin{cases}
			0, & x=y \\
			1, & x \neq y
		\end{cases}
	\]
	The function $d$ is a metric and it is called \textbf{the discrete metric}
	on $X$.
	\end{example}
	\begin{example}
	Let $X = \R^n$ and define the function:
	\[
		d(x-y) := |x - y|
	\]
	Where $|\cdot | \colon \R \to \R$ is the Eclidean norm function. 
	Then the pair $(X, d)$ forms a metric space.
	\end{example}
	\begin{example}
	Let $(X, N)$ be an arbitrary normed space and define the function:
	\[
		d(x-y) := N(x - y)
	\] 
	Then the pair $(X, d)$ forms a metric space.
	\end{example}
	\begin{example}
	The pair $(C([0,1]), d)$ such that $C([0,1])$ is the space of all
	continuous functions on $[0,1]$ paired with the metric:
	\[
		d(f,g) = \int_{0}^{1}{|f(x) - g(x)|\,dx}
	\]
	Is also a metric space.
	\end{example}
	\begin{example}
	The pair $(C([0,1]), d)$ paired with the supremum metric:
	\[
		d(f,g) = \sup_{x \in [0,1]}{|f(x) - g(x)|}
	\]
	Is also also metric space.
	\end{example}
	\begin{example}
	Let $\Lambda$ be a nonempty set which will represent an alphabet.
	The set $\Lambda^{\N}$ represents all the sequences over that alphabet.
	The pair $(\Lambda^\N, d)$ with the metric $d$ defined on two sequences
	$\omega = (\omega_n)_{n=1}^{\infty}, \eta = (\eta_n)_{n=1}^{\infty}$
	as:
	\[
		d(\omega, \eta) = \begin{cases}
			2^{-\min\{n \geq 0 \mid \omega_n \neq \eta_n\}} & 
			\omega \neq \eta \\
			0 & \omega = \eta
		\end{cases}
	\]
	\end{example}
	
	\newpage
	
	\section{Compactness}
	Let $X$ be a fixed topological space.
	\begin{definition}
		A class $\mathcal{U} := \{U_i\}_{i \in I}$ of open subsets of a
		$X$ is said to be an \emph{open cover of $X$} if 
		$X = \bigcup_{i \in I} U_i$. A subclass of $\mathcal{U}$ is said
		to be a subcover of $\mathcal{U}$ if it is in itself an open cover
		of $X$.
	\end{definition}
	\begin{definition}
		The space $X$ is said to be compact if every open cover of $X$
		has a finite subcover.
	\end{definition}
	\begin{definition}
		A subset $Y$ of $X$ is said to be compact if for every family of open
		sets $\{U_i\}_{i \in I}$ such that $Y \subset \bigcup_{i \in I} U_i$
		exists a finite index set $I_0 \subset I$ such that 
		$Y \subset \bigcup_{i \in I_0} U_i$.
	\end{definition}
	\begin{remark}
		It follows easily from the definition of the subspace topology that 
		a nonempty subset $Y$ of $X$ is compact if and only if $Y$ is a 
		compact space when equipped with the subspace topology.
	\end{remark}
	\begin{proposition}
		Suppose that $X$ is compact and let $F \subset X$ be closed. Then
		$F$ is compact.
	\end{proposition}
	\begin{proof}
		Let $\{U_i\}_{i \in I}$ be an open cover of $F$. Since $F$ is closed
		we know that $X \setminus F \cup \{U_i\}_{i \in I}$ is an open cover
		of $X$. Since $X$ is compact exists a finite index set $I_0 \subset I$
		such that $X \setminus F \cup \{U_i\}_{i \in I_0}$ is a finite
		open cover of $X$. It is clear that $F \subset \{U_i\}_{i \in I_0}$
		which completes the proof.
	\end{proof}
	\begin{proposition}
		Suppose $X$ is compact, let $Y$ be a topological space, and let
		$f \colon X \to Y$ be continuous. Then $f(X)$ is compact.
	\end{proposition}
	\begin{proof}
		Let $\{U_i\}_{i \in I}$ be an open cover of $f(X)$. Since $f$
		is continuous $\{f^{-1}(U_i)\}_{i \in I}$ is an open cover of $X$.
		Since $X$ is compact exists a finite index set $I_0 \subset I$
		such that $\{f^{-1}(U_i)\}_{i \in I_0}$ is an open cover of $X$.
		We now have:
		\[
			f(X) = f(\cup_{i \in I_0} f^{-1}(U_i)) = 
			\cup_{i \in I_0} f(f^{-1}(U_i)) = 
			\cup_{i \in I_0} U_i
		\]
		Which completes the proof.
	\end{proof}
	Here are some more equivalent forms of compactness that are ofter easier
	to apply.
	\begin{proposition}
		
	\end{proposition}
	
	
	
\end{document}