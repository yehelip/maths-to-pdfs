\documentclass{article}
\usepackage{amssymb,amsfonts,amsmath,calc,tikz,geometry}
\usepackage{array}
\usepackage{color}   %May be necessary if you want to color links
\usepackage{hyperref}
\hypersetup{
    colorlinks=false, %set true if you want colored links
    linktoc=all,     %set to all if you want both sections and subsections linked
    linkcolor=black,  %choose some color if you want links to stand out
}
\geometry{margin=1in}
\newcommand{\N}{\mathbb{N}}
\newcommand{\Z}{\mathbb{Z}}
\newcommand{\Q}{\mathbb{Q}}
\newcommand{\R}{\mathbb{R}}
\newcommand{\Omicron}{O}
\renewcommand{\thefootnote}{\fnsymbol{footnote}}

\title{\textbf{Group Theory}}
\author{Yeheli Fomberg}
\date{}

\usepackage{amsmath}
\begin{document}
	\maketitle
	\newpage
	\section{Groups}
	\subsection{definition}
	Let $A$ be a non-empty set and $*$ a binary operation on $A$. Under the following axioms
	\begin{itemize}
		\item $\forall (z,y,z)\in A^3:(x*y)*z=x*(y*z)$
		\item $\exists e\in A:\forall a\in A:a*e=e*a=a$
		\item $\forall a\in A:\exists a^{-1}:a*a^{-1}=a^{-1}*a=e$
	\end{itemize}
	We shall call $(A,*)$ a group. Groups can also be described in "Cayley tables":
	\[
\setlength\extrarowheight{3pt}
\begin{tabular}{c | c c c}
    (A,*) & e & x & y \\
    \cline{1-4}
    e & e & x & y \\
    x & x & ? & ? \\
    y & y & ? & ? 
\end{tabular}
	\]
	There is only one way to complete this table. Consider the axioms.
	
	\subsection{Isomorphisms of Groups}
	Let $G_1,G_2$ be groups and let $\phi:G_1\to G_2$ be a function such that $\forall x,y\in G_1$
	\[
		\phi(x)*\phi(y) = \phi(x*y)
	\] 
	Table of number of groups up to isomorphisms
	
	\[
\setlength\extrarowheight{3pt}
\begin{tabular}{c | c}
    \text{Order} & \text{Number} \\
    \cline{1-2}
    1 & 1 \\
    2 & 1 \\
    3 & 1 \\
    4 & 2 \\
    5 & 1 \\
    6 & 2 \\
    7 & 1 \\
    8 & 5 \\
    9 & 2
\end{tabular}
	\]
	
	\newpage
	\section{Greatest Common Divisor}
	Let $(A,*)$ be a group and suppose $a,b\in A$. We'll denote $d = gcd(a,b)$ if:
	\begin{itemize}
		\item $d>0$
		\item $d|b \quad\mathrm{and}\quad d|a$
		\item $c|b \quad\mathrm{and}\quad c|a \rightarrow c|d$
	\end{itemize}
	Let $a,b\in \Z\setminus\{0\}$ then $d = gcd(a,b)$ exists and is unique and $\exists n,m:d=ma+nb$
	Consider the following set
	\[
		A = \{ma+nb|m,n\in\Z \land ma+nb>0\}
	\]
	The set isn't empty since $a^2+b^2\in A$ and so vy the well ordering theorem has a first element which we'll pronounce $d$.
	\begin{itemize}
		\item $d>0$ by definition
		\item Without loss of generality suppose $b=qd+r$ and $r\ne 0$.
		\begin{align*}
			b &= q(ma+nb)+r \\
			r &= (-qm)a + (1-qn)b
		\end{align*}
		$r\ne 0 \Rightarrow r\in A$ but $r<d$ which is a contradiciton! 
		\item $c|b \quad\mathrm{and}\quad c|a \rightarrow c$ divides all linear combinations of $a,b\rightarrow c|d$
	\end{itemize}
	NOTE: if $gcd(x,y) = 1$ then we say $a$ and $b$ are coprimes. That's equivalent to saying $\exists m,n\in Z\setminus\{0\}:ma+nb =  1$
	\subsection{Fundamental theorem of arithmetic}
	Every integer greater than 1 can be represented uniquely as a product of prime numbers, up to the order of the factors. We'll prove this by induction.\\
	For $n = 2$ we know that $2=p_1$. Let $p_1*\ldots*p_m=2$. Since $2$ is the smallest prime number we know that our factorization was unique.\\
	For $n > 2$ if $n$ is prime then we finished. If $n=n_1n_2$ we know that $1<n_1,n_2<n$ and so by the induction $n_1=p*_1*\ldots *p_n$ and $n_2=p^*_1,\ldots,p^*_m$ then we know that $n=(p_1*\ldots *p_n)(p^*_1*\ldots *,p^*_m)$ like we wanted.\\
	Suppose $n=p_1*\ldots *p_n=q_1*\ldots *q_m$ We know $p_1|q_1*\ldots *q_m$ so $p_1=q_j$ for some $j$ then we can rearrange the elements such that $p_2*\ldots *p_n = q_2*\ldots *q_m$ and so on to show that the factorization is unique every time.
	\subsection{$\Z^*_n$}
	Prove that $\Z^*_n$ which is the set of all coprimes to $n$ from the set $[n]$ coupled with multiplication under modular arithmetic is a group.
	\newpage
	\section{More About Groups}
	We'll denote the order of a group $G$ - it's size - as $|G|$, and suppose $g\in G$ and $g^n=e$ we'll call $n$ the order of $g$ and denote $O(g)=n$. If that's never the case we'll denote $|G|=\infty \quad\mathrm{and}\quad O(g)=\infty$
	\subsection{Abelian Groups}
	A group $(A,*)$ is abelian if
	\[
		\forall (x,y)\in A^2:xy=yx
	\]
	The Cayley table for an abelian group is symmetric.
	\subsection{The Symmetric Group}
	The symmetric group is denoted as $S(X_n)$ or $S_n$ and is defined on the set $X=\{1,2,\ldots,n\}$ by being the set of all bijections $\sigma:X\to X$ with the operation of function composition.
	\subsection{Practise}
	\subsubsection{If $G$ is of finite order every element of $G$ also has finite order}
	Let $|G|=n$ and let $g\in G$ consider the elements $g,g^2,\ldots,g^{n+1}$ from the pigeonhole principle we know
	\[
		\exists i\neq j:g^i=g^j \Rightarrow g^{i-j}=e
	\]
	And thus $O(g)$ is finite.
	
	\newpage
	\section{Subgroups}
	Let $G$ be a groups and $H\subseteq G$ then $H\neq\emptyset$ is a subgroup if and only if
	\begin{itemize}
		\item $\forall(x,y)\in H^2:xy\in H$
		\item $e\in H$
		\item $x\in H \Rightarrow x^{-1}\in H$
	\end{itemize}
	One condition is not necessary. Think which. If the $G$ is a finite group then two conditions are not necessary. Think why.
	\subsection{Cyclic Groups}
	$G$ is a cyclic groups if $G$ has a generator $x$ such that for some $n\in\N$
	\[	G=\left<x\right>=\{g:g=x^k \land k\in\Z\}
	\]
	If the group is of finite order $n$ every subgroup is of order $k|n$. Prove by contradiction. A group generated from a set $S$ is
	\[	G=\left<S\right>=\bigcap_{S\subseteq H_a}{H_a}
	\]
	Where $H_a$ are all the subgroups that contain $S$. Let $S = \{a,b\}$ then the group will contain all possible products from $a,b$ and their inverses.
	
	\newpage
	\section{Lagrange theorem}
	


\end{document}