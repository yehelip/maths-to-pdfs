\documentclass[11pt,a4paper]{article}
\usepackage{amssymb,amsfonts,amsmath,calc,tikz,pgfplots,geometry}
\usepackage{color}   %May be necessary if you want to color links
\usepackage{hyperref}
\usepackage{amsthm}
\usepackage{fancyhdr}
\pagestyle{fancy}
\usetikzlibrary{positioning}
\geometry{margin=1in}
\pgfplotsset{compat=1.18}
\setlength{\headheight}{14.6pt}
\addtolength{\topmargin}{-1.6pt}
\hypersetup{
    colorlinks=false, %set true if you want colored links
    linktoc=all,   %set to all if you want both sections and subsections linked
    linkcolor=black,  %choose some color if you want links to stand out
}
%%%%%%%%%%%%%%%%%%%%%%%%%%%%%%%%%%%%%%%%%%%%%%%%%%%%%%%%%%%%%%%%%%%%%%%%%%%%%%%
\theoremstyle{definition}
\newtheorem{definition}{Definition}[section]
\newtheorem{remark}{Remark}[section]
\newtheorem{example}{Example}[section]
\theoremstyle{plain}
\newtheorem{theorem}{Theorem}[section]
\newtheorem{proposition}[theorem]{Proposition}
\newtheorem{lemma}[theorem]{Lemma}
\newtheorem{corollary}[theorem]{Corollary}

\DeclareMathOperator{\lcm}{lcm}
\DeclareMathOperator{\idealin}{\triangleleft}
\DeclareMathOperator{\im}{im}
\DeclareMathOperator{\Aut}{Aut}
\DeclareMathOperator{\End}{End}
\DeclareMathOperator{\Inn}{Inn}
\DeclareMathOperator{\Out}{Out}
\DeclareMathOperator{\Mat}{Mat}
\DeclareMathOperator{\std}{std}
\DeclareMathOperator{\Int}{Int}
\DeclareMathOperator{\diam}{diam}

\newcommand{\N}{\mathbb{N}}
\newcommand{\Z}{\mathbb{Z}}
\newcommand{\Q}{\mathbb{Q}}
\newcommand{\R}{\mathbb{R}}
\newcommand{\C}{\mathbb{C}}
\newcommand{\F}{\mathbb{F}}
\newcommand{\Omicron}{O}
\newcommand{\st}{\text{ s.t. }}
\newcommand{\tand}{\quad \text{and} \quad}
\newcommand{\tor}{\quad \text{or} \quad}
\newcommand{\ip}[2]{\langle #1, #2 \rangle}
\newcommand{\set}[2]{ \left\{ #1 \mid #2 \right\} }
\newcommand{\abs}[1]{\left\lvert #1\right\rvert}
\newcommand{\norm}[1]{\left\lVert #1\right\rVert}
\renewcommand{\tt}[1]{\textnormal{\textbf{(#1).}}} %tt=theorem title
\newcommand{\bigslant}[2]
{{\raisebox{.2em}{$#1$}\left/\raisebox{-.2em}{$#2$}\right.}}
%%%%%%%%%%%%%%%%%%%%%%%%%%%%%%%%%%%%%%%%%%%%%%%%%%%%%%%%%%%%%%%%%%%%%%%%%%%%%%%
\title{\textbf{Group Theory}}
\author{}
\date{}
\begin{document}
	\maketitle
	\newpage
  \section{Introduction}
  \begin{definition}[Binary operation]
    A binary operation on a set $S$ is a mapping $f$ from $S \times S$
    to $S$.
  \end{definition}

  \begin{definition}[Group]
    Let $G$ be a non-empty set and $*$ a binary operation on $A$.
    The pair $(G,*)$ is called a group if the following are satisfied:
	  \begin{itemize}
      \item For all $a,b,c \in G$ we have $(a * b) * c = a * (b * c)$;
        (Associativity)
      \item There exists $e \in G$ such that for all $a \in G$ we have
        $a * e = e * a = a$; (Identity element)
      \item For all $a \in G$ there exists $a^{-1} \in G$ such that
        $a * a^{-1} = a^{-1} * a = e$. (Inverse element)
	  \end{itemize}
  \end{definition}

  \begin{definition}[Cayley table]
    A Cayley table is a way to describe a finite group by arranging all
    the possible products of any two elements of the group. For example
    the table
    \begin{center}
    \begin{tabular}{c | c c c}
      $(A,*)$ & $e$ & $x$ & $y$ \\
      \cline{1-4}
      $e$ & $e$ & $x$ & $y$ \\
      $x$ & $x$ & ? & ? \\
      $y$ & $y$ & ? & ? 
    \end{tabular}
    \end{center}
    is the Cayley table of some group such that $A = \{e,x,y\}$.
  \end{definition}
  \begin{remark}
    There is only one way to complete the above table such that it would
    describe a group.
  \end{remark}

  \begin{definition}[Homomorphism of groups]
    Let $(G,*_G)$ and $(H,*_H)$ be groups.
    A homomorphism of groups is a function $\varphi \colon G \to H$ such
    that for any $a, b \in G$ we have
    \[
      \varphi(x *_G y) = \varphi(x) *_H \varphi(y).
    \]
    If there exists a homomorphism between $G$ and $H$, they are called
    homomorphic groups.
  \end{definition}

  \begin{definition}[Isomorphism of groups]
    An isomorphicm of groups is a bijective homomorphism.
    If there exists a homomorphism between two group $G$ and $H$,
    they are called isomorphic groups.
  \end{definition}

  We see that an isomorphism is a function the preserves the structure
  of the group in the sense that applying the function on the product 
  of the elements $x$, $y$ in $G$, is the same as taking the product of 
  the elements $\varphi(x)$, $\varphi(y)$ in $H$.

  We can see that the Cayley tables of isomorphic groups are the same.
  For example, if $G$ and $H$ are isomorphic groups of size $3$,
  with the isomorphism $\varphi \colon G \to H$ we can see that
  \begin{center}
    \begin{tabular}{c | c c c}
      $(G,*_G)$ & $e$ & $x$ & $y$ \\
      \cline{1-4}
      $e$ & $e *_G e$ & $e *_G x$ & $e *_G y$ \\
      $x$ & $x *_G e$ & $x *_G x$ & $x *_G y$ \\
      $y$ & $y *_G e$ & $y *_G x$ & $y *_G y$ 
    \end{tabular} $\approx$
    \begin{tabular}{c | c c c}
      $(H,*_H)$ & $\varphi(e)$ & $\varphi(x)$ & $\varphi(y)$ \\
      \cline{1-4}
      $\varphi(e)$ & $\varphi(e *_G e)$ & $\varphi(e *_G x)$ & $\varphi(e *_G y)$ \\
      $\varphi(x)$ & $\varphi(x *_G e)$ & $\varphi(x *_G x)$ & $\varphi(x *_G y)$ \\
      $\varphi(y)$ & $\varphi(y *_G e)$ & $\varphi(y *_G x)$ & $\varphi(y *_G y)$
    \end{tabular}
  \end{center}
  Then by applying the homomorphism property we get that the original
  table is approximately
  \begin{center} 
    \begin{tabular}{c | c c c}
      $(H,*_H)$ & $\varphi(e)$ & $\varphi(x)$ & $\varphi(y)$ \\
      \cline{1-4}
      $\varphi(e)$ & $\varphi(e) *_H \varphi(e)$ & $\varphi(e) *_H \varphi(x)$ & 
        $\varphi(e) *_H \varphi(y)$ \\
      $\varphi(x)$ & $\varphi(x) *_H \varphi(e)$ & $\varphi(x) *_H \varphi(x)$ &
        $\varphi(x) *_H \varphi(y)$ \\
      $\varphi(y)$ & $\varphi(y) *_H \varphi(e)$ & $\varphi(y) *_H \varphi(x)$ &
        $\varphi(y) *_H \varphi(y)$
    \end{tabular}
  \end{center}

  which is exactly the Cayley group of $H$.
  
  \begin{definition}[Order of a group]
    Let $(G, *)$ be a group. The size $|G|$ is said to be the order
    of the group.
  \end{definition}

  The following table shows the amount of different groups up to isomorhism
  by their order:
  \begin{center}
    \begin{tabular}{c | c}
        Order & Number \\
        \cline{1-2}
        $1$ & $1$ \\
        $2$ & $1$ \\
        $3$ & $1$ \\
        $4$ & $2$ \\
        $5$ & $1$ \\
        $6$ & $2$ \\
        $7$ & $1$ \\
        $8$ & $5$ \\
        $9$ & $2$
    \end{tabular}
  \end{center}

  \begin{definition}[Greatest common divisor]
    The greatest common divisor (GCD) of integers $a$ and $b$,
    at least one of which is nonzero,
    is the greatest positive integer $d$ such that $d$ is a divisor of 
    both $a$ and $b$.
    The greatest common divisor of $a$ and $b$ is denoted $\gcd(a,b)$.
  \end{definition}
  \begin{remark}
    We define $\gcd(0,0) = 0$, but this is mostly not relevant.
  \end{remark}

  \begin{definition}[Coprime]
    Let $a,b \in \Z$.
    We say that $a$ and $b$ are coprime if $\gcd(x,y) = 1$.
  \end{definition}

  \begin{proposition}
    Let $a,b \in \Z$. Then $\gcd(a,b)$ exists and is unique.
    Moreover, there exist $n,m \in \Z$ such that $d = am + nb$.
  \end{proposition}
  \begin{proof}
	Consider the following set
	\[
    A := \set{ma + nb}{m,n \in\Z \tand ma + nb > 0}.
	\]
	The set isn't empty since $a^2 + b^2 \in A$,
  so by the well ordering theorem, it follows that it has a first element 
  which we will call $d$.
  By the construction $d$ is a positive integer.

	\begin{itemize}
		\item Without loss of generality suppose $b=qd+r$ and $r\ne 0$.
		\begin{align*}
			b &= q (ma + nb) + r \\
			r &= (-qm) a + (1 - qn) b
		\end{align*}
		$r \neq 0 \Rightarrow r \in A$ but $r < d$ which is a contradiciton!
		\item $c|b \quad\mathrm{and}\quad c|a \rightarrow c$ divides all linear 
      combinations of $a,b\rightarrow c|d$
	\end{itemize}
  \end{proof}

  \begin{proposition}
    Every integer greater than $1$ can be represented uniquely as a product of
    prime numbers, up to the order of the factors.
  \end{proposition}
  \begin{proof}
    We will prove this by induction on $n$.
  	For the base case $n = 2$ we know that $2 = p_1$.
    Since $2$ is the smallest prime number this product (of one element) is
    unique.

	  Let $n > 2$.
    If $n$ is a prime number then the proof is trivial.
    If is not prime, then $n = n_1 * n_2$ for some $1 < n_1, n_2 < n$. 
    By the induction hypothesis $n_1 = p_1 * \cdots * p_n$ and 
    $n_2 = p'_1, \cdots, p'_m$. 
    Therefore $n = (p_1 * \cdots * p_n) * (p'_1 * \cdots * p'_m)$.

    Suppose $n = p_1*\ldots *p_n=q_1*\ldots *q_m$
    We know $p_1|q_1*\ldots *q_m$ so $p_1=q_j$ for some $j$ 
    then we can rearrange the elements such that 
    $p_2*\ldots *p_n = q_2*\ldots *q_m$ and so on to show that
    the factorization is unique every time.
  \end{proof}
	
  \begin{definition}[The set $\Z^*_n$]
    Let $n$ be a natural number.
    We define
    \[
      \Z^*_n = \set{m \in \Z_n}{\gcd(m,n) = 1}.
    \]
  \end{definition}
  \begin{proposition}
    The pair $(\Z^*_n, *)$ is a group where $*$ denotes modular multiplication.
  \end{proposition}

  \begin{definition}[Order of an element]
    Let $(G,*)$ be a group, let $g \in G$.
    Let $n$ be the smallest positive integer such that $g^n = e$ where
    $e$ is the unit element of $G$. We denote $|g| = n$.
    If there does not exist such $n$, we define $|g| = \infty$.
  \end{definition}
  
  \begin{definition}[Abelian group]
    Let $(G,*)$ be a group. We say that $G$ is abelian if for all $a,b \in G$
    we have $a * b = b * a$.
  \end{definition}
  \begin{remark}
  	The Cayley table for an abelian group is symmetric.
  \end{remark}

  \begin{definition}[The symmetric group]
    Set $X_n := \{1,2,\dots,n\}$. 
	  The symmetric group denoted as $S(X_n)$ or $S_n$, is defined as the set 
    of all bijections $\sigma \colon X_n \to X_n$ coupled with the operation 
    of function composition.
  \end{definition}

  \begin{proposition}
    If $(G, *)$ is a group of finite order, then every element of $G$ also has
    a finite order.
  \end{proposition}
  \begin{proof}
    Denote $|G|=n$, and let $g \in G$.
    Consider the elements $g, g^2, \dots, g^{n+1}$.
    From the pigeonhole principle there exists $1 \le i \neq j \le n+1$
    such that $g^i = g^j$. This implies that $g^{i-j} = e$.
  	Therefore $O(g)$ is finite.
  \end{proof}

  \begin{definition}[Subgroup]
    Let $(G, *)$ be a group. If the set $(H,*_H)$ such that $H \subset G$
    and $*_H = *\vert_H$ is a group, then $H$ is called a subgroup of $G$.
  \end{definition}

  \begin{proposition}
    Let $(G.*)$ be a group and $\emptyset \neq H \subseteq G$.
    Then $H$ is a subgroup if and only if the following conditions are
    satisfied:
    \begin{enumerate}
      \item[(1)] For all $a,b \in H$ we have $x * y \in H$;
      \item[(2)] For all $a \in H$ we have $a^{-1} \in H$;
      \item[(3)] $e \in H$.
    \end{enumerate}
  \end{proposition}
  \begin{remark}
    Condition $(3)$ is not necessary.
    If $G$ is finite condition $(2)$ is also not necessary.
  \end{remark}

  \begin{definition}[Cyclic group]
    Let $(G,*)$ be a group. We say that $G$ is cyclic if there exists
    an element $g \in G$ such that
    \[
      G = \langle x \rangle := \set{g^k}{k \in \Z}.
    \]
  \end{definition}

  If the group is of finite order $n$ every subgroup is of order $k|n$.
  Prove by contradiction.
  A group generated from a set $S$ is
  \[
    G = \left<S\right> := \bigcap_{S \subseteq H_a}{H_a}
  \]
  Where $H_a$ are all the subgroups that contain $S$.
  Let $S = \{a,b\}$ then the group will contain all possible products 
  from $a,b$ and their inverses.

  \begin{theorem}\tt{Lagrange's theorem}
  \end{theorem}
  



\end{document}
