\documentclass[11pt,a4paper]{article}

\def\nyear {2025}
\def\nterm {Winter}
\def\nlecturer {}
\def\ncourse {Topology}

\makeatletter

% packages
\usepackage{amssymb,amsfonts,amsmath,calc,tikz,pgfplots,geometry,mathtools}
\usepackage{color}   % May be necessary if you want to color links
\usepackage[hidelinks]{hyperref}
\usepackage{forest}
\usepackage{commath} % ew
\usepackage{esdiff}
\usepackage{amsthm}
\usepackage{fancyhdr}
\usepackage{bm}
\usepackage{witharrows}
\usepackage{bookmark}
\usepackage{tikz-cd}
\usepackage{bbm}
\usepackage{textcomp}
\usepackage{gensymb}
\usepackage{cleveref}

% tikz libraries
\usetikzlibrary{positioning}
\usetikzlibrary{matrix}
\usetikzlibrary{arrows}
\usetikzlibrary{arrows.meta}
\usetikzlibrary{decorations.markings}

% Page style setup
\pagestyle{fancy}
\geometry{margin=1in}
\pgfplotsset{compat=1.18}
\setlength{\headheight}{14.6pt}
\addtolength{\topmargin}{-1.6pt}
\hypersetup{
    colorlinks=false,
    linktoc=section,
    linkcolor=black,
}

%% maketitle setup
\ifx \nauthor\undefined
  \def\nauthor{yehelip}
\else
\fi

\ifx \ncoursehead \undefined
\def\ncoursehead{\ncourse}
\fi

\lhead{\emph{\nouppercase{\leftmark}}}
\ifx \nextra \undefined
  \rhead{
    \ifnum\thepage=1
    \else
      \ncoursehead
    \fi}
\else
  \rhead{
    \ifnum\thepage=1
    \else
      \ncoursehead \ (\nextra)
    \fi}
\fi

\let\@real@maketitle\maketitle
\renewcommand{\maketitle}{\@real@maketitle\begin{center}
\begin{minipage}[c]{0.9\textwidth}\centering\footnotesize
These notes are not endorsed by the lecturers.
I have revised them outside lectures to incorporate supplementary explanations,
clarifications, and material for fun.
While I have strived for accuracy, any errors or misinterpretations 
are most likely mine.
\end{minipage}\end{center}}

% theorem environments
\theoremstyle{definition}
\newtheorem{definition}{Definition}[section]
\newtheorem{remark}{Remark}[section]
\newtheorem{example}{Example}[section]
\newtheorem{exercise}{Exercise}[section]
\newtheorem{paradox}{Paradox}[section]
\newtheorem*{solution}{Solution}
\theoremstyle{plain}
\newtheorem{theorem}{Theorem}[section]
\newtheorem{proposition}[theorem]{Proposition}
\newtheorem{lemma}[theorem]{Lemma}
\newtheorem{corollary}[theorem]{Corollary}

% tikz customization
\pgfarrowsdeclarecombine{twolatex'}{twolatex'}{latex'}{latex'}{latex'}{latex'}
\tikzset{->/.style = {decoration={markings,
                                  mark=at position 1
                                  with {\arrow[scale=2]{latex'}}},
                      postaction={decorate}}}
\tikzset{<-/.style = {decoration={markings,
                                  mark=at position 0 with {\arrowreversed[scale=2]{latex'}}},
                      postaction={decorate}}}
\tikzset{<->/.style = {decoration={markings,
                                   mark=at position 0 with {\arrowreversed[scale=2]{latex'}},
                                   mark=at position 1 with {\arrow[scale=2]{latex'}}},
                       postaction={decorate}}}
\tikzset{->-/.style = {decoration={markings,
                                   mark=at position #1 with {\arrow[scale=2]{latex'}}},
                       postaction={decorate}}}
\tikzset{-<-/.style = {decoration={markings,
                                   mark=at position #1 with {\arrowreversed[scale=2]{latex'}}},
                       postaction={decorate}}}
\tikzset{->>/.style = {decoration={markings,
                                  mark=at position 1 with {\arrow[scale=2]{latex'}}},
                      postaction={decorate}}}
\tikzset{<<-/.style = {decoration={markings,
                                  mark=at position 0 with {\arrowreversed[scale=2]{twolatex'}}},
                      postaction={decorate}}}
\tikzset{<<->>/.style = {decoration={markings,
                                   mark=at position 0 with {\arrowreversed[scale=2]{twolatex'}},
                                   mark=at position 1 with {\arrow[scale=2]{twolatex'}}},
                       postaction={decorate}}}
\tikzset{->>-/.style = {decoration={markings,
                                   mark=at position #1 with {\arrow[scale=2]{twolatex'}}},
                       postaction={decorate}}}
\tikzset{-<<-/.style = {decoration={markings,
                                   mark=at position #1 with {\arrowreversed[scale=2]{twolatex'}}},
                       postaction={decorate}}}

\pgfarrowsdeclare{biggertip}{biggertip}{%
  \setlength{\arrowsize}{1pt}
  \addtolength{\arrowsize}{.1\pgflinewidth}
  \pgfarrowsrightextend{0}
  \pgfarrowsleftextend{-5\arrowsize}
}{%
  \setlength{\arrowsize}{1pt}
  \addtolength{\arrowsize}{.1\pgflinewidth}
  \pgfpathmoveto{\pgfpoint{-5\arrowsize}{4\arrowsize}}
  \pgfpathlineto{\pgfpointorigin}
  \pgfpathlineto{\pgfpoint{-5\arrowsize}{-4\arrowsize}}
  \pgfusepathqstroke
}
\tikzset{
	EdgeStyle/.style = {>=biggertip}
}

\tikzset{circ/.style = {fill, circle, inner sep = 0, minimum size = 3}}
\tikzset{scirc/.style = {fill, circle, inner sep = 0, minimum size = 1.5}}
\tikzset{mstate/.style={circle, draw, black, text=black, minimum width=0.7cm}}

\tikzset{eqpic/.style={baseline={([yshift=-.5ex]current bounding box.center)}}}

\definecolor{mblue}{rgb}{0.2, 0.3, 0.8}
\definecolor{morange}{rgb}{1, 0.5, 0}
\definecolor{mgreen}{rgb}{0, 0.4, 0.2}
\definecolor{mred}{rgb}{0.5, 0, 0}

% topology
\newcommand{\Cells}{\text{Cells}}

% algebra
\DeclareMathOperator{\lcm}{lcm}
\DeclareMathOperator{\Out}{Out}
\DeclareMathOperator{\Aut}{Aut}
\DeclareMathOperator{\End}{End}
\DeclareMathOperator{\Inn}{Inn}
\DeclareMathOperator{\Mat}{Mat}
\DeclareMathOperator{\std}{std}
\DeclareMathOperator{\sgn}{sgn}
\DeclareMathOperator{\id}{id}
\DeclareMathOperator{\op}{op}
\DeclareMathOperator{\GL}{GL} % General linear group
\DeclareMathOperator{\SL}{SL} % Special linear group
\newcommand{\idealin}{\triangleleft}
\newcommand{\ip}[2]{\langle #1, #2 \rangle}
\newcommand{\bigslant}[2]
{{\raisebox{.2em}{$#1$}\left/\raisebox{-.2em}{$#2$}\right.}}

% analysis
\newcommand{\dx}{\dif x}
\newcommand{\dt}{\dif t}
\newcommand{\du}{\dif u}
\newcommand{\dv}{\dif v}
\newcommand{\dz}{\dif z}
\newcommand{\ds}{\dif s}
\newcommand{\dtheta}{\dif \theta}
\DeclareMathOperator{\im}{im}
\DeclareMathOperator{\cis}{cis}
\DeclareMathOperator{\Int}{Int}
\DeclareMathOperator{\diam}{diam}
\DeclareMathOperator{\supp}{supp}
\DeclareMathOperator{\Vol}{Vol} % Volume

% logic
\DeclareMathOperator{\MOD}{MOD}
\DeclareMathOperator{\Theory}{Theory}


% nice
\newcommand{\half}{\frac{1}{2}}
\newcommand{\pair}{\del}
\newcommand{\taking}[1]{\xrightarrow{#1}}
\newcommand{\inv}{^{-1}}
\newcommand{\ot}{\leftarrow}
\newcommand{\ninfty}{-\infty}
\newcommand{\floor}[1]{\left\lfloor #1 \right\rfloor}
\newcommand{\ceil}[1]{\left\lceil #1 \right\rceil}

% probability
\newcommand{\Prob}{\mathbf{P}}
\renewcommand{\vec}[1]{\boldsymbol{\mathbf{#1}}}
\DeclareMathOperator{\Bin}{Bin}
\DeclareMathOperator{\Geo}{Geo}
\DeclareMathOperator{\Poi}{Poi}
\DeclareMathOperator{\Exp}{Exp}
\DeclareMathOperator{\Var}{Var} % Variance
\DeclareMathOperator{\Cov}{Cov}

% special letters
\newcommand{\N}{\mathbb{N}}
\newcommand{\Z}{\mathbb{Z}}
\newcommand{\Q}{\mathbb{Q}}
\newcommand{\R}{\mathbb{R}}
\newcommand{\C}{\mathbb{C}}
\newcommand{\F}{\mathbb{F}}
\newcommand{\E}{\mathbb{E}}
\newcommand{\ps}{\mathcal{P}}
\newcommand{\M}{\mathcal{M}}
\renewcommand{\L}{\mathcal{L}}
\newcommand{\Omicron}{O}
\newcommand{\powerset}{\mathcal{P}}

% text
\newcommand{\st}{\text{ s.t. }}
\newcommand{\tand}{\quad \text{and} \quad}
\newcommand{\tor}{\quad \text{or} \quad}
\newcommand{\stand}{\text{ and }}
\newcommand{\stor}{\text{ or }}
\renewcommand{\tt}[1]{\textnormal{\textbf{(#1).}}} %tt=theorem title GET RID OF

% title format
\title{\textbf{\ncourse}}
\author{Based on lectures by \nlecturer \\\small Notes taken by \nauthor}
\date{\nterm\ \nyear}
\makeatother


\begin{document}
\maketitle

% Insert cool image here

\newpage
\tableofcontents
\newpage

\section{Introduction}

Before getting into the main part of the course, we can first look
at topology from the viewpoint it was considered hundreds of years ago.

\begin{definition}[Geodesic triangle]
  A geodesic triangle is the area enscribed inside $3$ points
  on a sphere.
\end{definition}

\begin{theorem}[Girard's theorem]
  \label{thm:girard}
  Let $T$ be a geodesic triangle.
  Denote its angles $\theta_1$, $\theta_2$, $\theta_3$.
  Then we have $\theta_1 + \theta_2 + \theta_3 > 180^{\circ}$ and
  \[
    \mathrm{Area}(T) = \theta_1 + \theta_2 + \theta_3 - \pi.
  \]
\end{theorem}
\begin{proof}
  Geometric.
\end{proof}

\begin{theorem}[Euler's theorem]
  Let $P$ be a convex polyhydron.
  Denote $E$ the number of edges in $P$, $V$ the number of vertices in $P$,
  and $F$ the number of faces of $P$.
  Then
  \[
    V - E + F = 2.
  \]
\end{theorem}
\begin{proof}
  Begin by ensphering the polyhydron in the unit sphere.
  Put a flashlight inside the polyhydron such that the faces of $P$
  cast shadows of geodesic polygons on the sphere.
  
  Now we trianglize all the geodesic polygons.
  We have
  \[
    \boxed{2 E = 3 F}
  \]
  Denote the trianlges $\sigma_1,\dots,\sigma_F$.
  From \Cref{thm:girard} we have that
  \[
    \sum_{i=1}^{F} \frac{\theta_1^i + \theta_2^i + \theta_3^i}{V} - \pi =
    4 \pi.
  \]
  From this we get
  \[
    2 \pi V - \pi F = 4 \pi \implies
    \boxed{2 V - F = 4}
  \]
  From these equations we can deduce the desired relation
  \[
    \boxed{V - E + F = 2}
  \]
  which completes the proof.
\end{proof}

\begin{example}
  box with a hold in the middle.
\end{example}

\newpage

\section{Quotient spaces and complexes}

\subsection{Quotient spaces}

Let $(X, \tau)$ be a topological space, and let $\sim$ be an equivalence
relation on $X$.
Denote:
\[
  [x] := \set{y \in X \mid y \sim x}.
\]
Also define the function
\[
  \fullfunction{\pi}{X}{\bigslant{X}{\sim}}{x}{[x]}.
\]

\begin{definition}[Quotient topology]
  We define on $\bigslant{X}{\sim}$ the quotient topology $\tau_{\sim}$
  by $U \in \tau_{\sim}$ if and only if $\pi^{-1}(U) \in \tau$.
\end{definition}

In other words, the quotient topolgoy is the topology generated by $\pi$.

\begin{proposition}[Universal property of quotient spaces]
  Let $X$, $Y$ be topological spaces, let $\sim$ be an equivalence relation
  on $X$, and $f \colon X \to Y$ be a function such that for every 
  $x \sim x' \in X$ we have $f(x) = f(x')$.
  Then there exists $f' \colon \bigslant{X}{\sim} \to Y$ such that
  $f' \circ \pi = f$ where $\pi$ is the quotient projection.
  Moreover, $f'$ is continuous if and only if $f$ is continuous.
  In a diagram it looks like this:
  \begin{center}
    \begin{tikzcd}
    X \arrow[rd, "f"] \arrow[d, "\pi"']         &   \\
    \bigslant{X}{\sim} \arrow[r, "f'"', dashed] & Y
    \end{tikzcd}
  \end{center}
\end{proposition}
\begin{proof}
  The first part of the proof is clear, so we will focus on the equivalence
  of continuity between $f$ and $f'$.
  It is clear that $\pi$ is continuous so when $f'$ is continuous
  we also have that $f = f' \circ \pi$ is continuous.

  Next suppose $f$ is continuous.
  To be added
\end{proof}

\begin{example}
  Let $X = [0,1]$ with the topology induced by the standard topology on $\R$.
  Let $\sim$ be the equivalence relation generated by $0 \sim 1$.
  Thus,
  \[
    \bigslant{X}{\sim} \cong S^1 = \set{z \in \C^2 \colon |z| = 1}.
  \]
  We define $f \colon X \to S^1$ by $f(t) = e^{2 \pi i t}$.
  Since $f$ is continuous and $f(0) = f(1)$, from the universal property
  of quotient spaces there exists a continuous function 
  $f' \colon \bigslant{X}{\sim} \to S^1$ defined by $f'([x]) = f(x)$.
  It is clear that $f$ is one to one and onto.
  To show that it is a homeomorphism we can use the following lemma
  \begin{lemma}
    Let $X$, $Y$ be topological spaces.
    Suppose that $X$ is compact, $Y$ is Hausdorff.
    Then every continuous function from $X$ to $Y$ is a closed transformation.
    In particular, if $f$ is one to one and onto, it is a homeomorphism.
  \end{lemma}
  \begin{proof}
    Let $C$ be a closed set in $X$.
    Since $X$ is compact $C$ is compact.
    Since $f$ is continuous $f(C)$ is compact in the Hausdorff space $Y$
    and thus closed which completes the proof.
  \end{proof}
  \begin{remark}
    Recall that the last part of the lemma is true because an open bijection
    is a homeomorphism.
  \end{remark}
\end{example}

\begin{example}
  We will show that
  \[
    X = \bigslant{\C \setminus \set{0}}{x \sim \lambda x} \quad
    \forall 0 \neq \lambda \in \R
  \]
  is homeomorphic to the unit sphere $S^1$.
  First define the function
  \[
    \fullfunction{f}{\C \setminus \set{0}}{S^1}{z}{\del{\frac{z}{|z|}}^2}
  \]
  The function $f$ is continuous and satisfies $f(x) = f(\lambda x)$ for all
  $0 \neq \lambda \in \R$.
  Thus, from the universal property of quotient spaces there exists
  a continuous function $f' \colon X \to S^1$.
  It is clear that $f$ is one to one and onto.
  We have that $X = \pi(S^1)$ because every element in $C \setminus \set{0}$
  is equivalent to some element in $S^1$, and since $\pi$ is continuous
  and $S^1$ is compact, it follows that $X$ is also compact.
  It is clear that $S^1$ is Hausdorff, and thus from the previous lemma
  we have that $f'$ is a homeomorphism between $X$ and $S^1$.
\end{example}

From now on we denote
\begin{align*}
  &D^n := \set{x \in \R^n \colon \norm{x} \le 1}; \\
  &S^n := \partial D^{n+1} = \set{x \in \R^{n+1} \colon \norm{x} = 1}; \\
  &T^n := (S^1)^n.
\end{align*}
We can see that
\[
  D^0 = \set{0} \stand \partial D^0 = \emptyset \stand S^0 = \set{-1, 1}.
\]

\begin{example}
  We have that $\bigslant{D^n}{S^{n-1}} \cong S^n$.
  The equivalence relation here is $x \sim y$ if and only if $x,y \in S^{n-1}$.
\end{example}
\begin{example}
  We have that $\bigslant{\R}{\Z} \cong S^1$ where $x \sim x + k$ for all
  $x \in \R$, $k \in \Z$.
\end{example}
\begin{example}
  For all $x \in \R^n \setminus \set{0}$ and $\lambda \in (0,\infty)$ we
  have:
  \[
    S^n \cong \bigslant{\R^n \setminus \set{0}}{x \sim \lambda x}.
  \]
\end{example}

\subsection{Pastings}

The following lemma will allow us to discuss more pastings of spaces.

\begin{lemma}
  Let $X$ be a topological space, $\sim$ an equivalence relation on $X$,
  let $\pi$ the quotient projection, and $f$ a continuous function such that
  \begin{enumerate}
    \item[(1)] $f$ is constant on the fibers of $\pi$.
    \item[(2)] $f'$ is one to one and onto.
    \item[(3)] $f$ is closed, open, or ($\bigslant{X}{\sim}$ is compact and
      $Y$ is Hausdorff).
  \end{enumerate}
  Then $f'$ is a homeomorphism.
\end{lemma}

\begin{example}
  Let $X = [0,1] \cup [2,3]$ and $\sim := 1 \sim 2$.
  Then $\bigslant{X}{\sim} = [0,2]$.
  We can see this by defining the function $f \colon X \to [0,2]$ as such
  \[
    x \mapsto
    \begin{cases}
      x, & 0 \le x \le 1 \\
      x - 1, & 2 \le x \le 3
    \end{cases}.
  \]
\end{example}

\begin{example}
  Define
  \begin{align*}
    \R_n^+ &= \set{(x_1,\dots,x_n) \mid x_i \geq 0 } \\
    \R_n^- &= \set{(x_1,\dots,x_n) \mid x_i \le 0 }
  \end{align*}
  Define $X = \R_n^+ \cup \R_n^+$ (note that we treat this as a dijoint union).
  We define the equivalence relation 
  $(x_1^+,\dots,x_{n-1}^+, 0) \sim (x_1^-,\dots,x_{n-1}^-,0)$
  and then we have $\R^n \cong \bigslant{X}{\sim}$.
\end{example}

\begin{example}
  Let $X = I \times I$ where $I = [0,1]$.
  We define the equivalence relation as such $(s,0) \sim (s,1)$ for all
  $0 \le s \le 1$.
  Then $\bigslant{X}{\sim}$ is homeomorphic to a cylinder.
  We can see this by defining the function
  \[
    \fullfunction{f}{I \times I}{S \times I}
    {(s,t)}{(\cos 2 \pi t, \sin 2 \pi t, s)}
  \]
\end{example}

\begin{example}
  Let $X = I \times I$.
  Define the equivalence relation $(s, 0) \sim (1 - s, 1)$.
  The space $\bigslant{X}{\sim}$ is homeomorphic to a Mobius strip.
\end{example}

\begin{example}
  Let $X = I \times I$.
  Define the equivalence relation $(s, 0) \sim (s, 1)$ and 
  $(0, t) \sim (1, t)$.
  The space $\bigslant{X}{\sim}$ is homeomorphic to a torus (specifically
  $T^2$).
\end{example}

\subsection{CW complexes}

\subsubsection{Disjoint union}

\begin{definition}[Disjoint union topology]
  Let $\set{X_{\alpha}}_{\alpha}$ be a collection of topological spaces.
  We define the dijoint union topology in the following manner.
  We say that $U \subset \coprod_{\alpha} X_{\alpha}$ is open if and only
  if $U \cap X_{\alpha}$ is open in $X_{\alpha}$.
\end{definition}

\begin{example}
  A disjoint union of points is an open set if every $X_{\alpha}$ is endowed
  with the discrete topology.
\end{example}

\begin{example}
  We have that $D^1 \coprod D^1 \cong [0,1] \cup [2,3]$ with the standard
  topology.
\end{example}

\begin{proposition}[Universal property of disjoint union]
  Let $\set{X_{\alpha}}_{\alpha}, Y$ be topological spaces,
  let $f_{\alpha} \colon X_{\alpha} \to Y$ be a collection of continuous
  functions.
  Then exists a unique continuous function 
  $f \colon \coprod_{\alpha} X_{\alpha} \to Y$ such that
  $f_{\alpha} = f \circ i_{\alpha}$ where $i_{\alpha} \colon X_{\alpha} \to
  \coprod_{\alpha} X_{\alpha}$ is the injection map.
  In a diagram, it looks like this:
  \begin{center}
    \begin{tikzcd}
      \coprod_{\alpha} X_{\alpha} \arrow[rd, "\exists !f", dashed] &   \\
      X_{\alpha} \arrow[u, "i_{\alpha}"] \arrow[r, "f_i"']         & Y
    \end{tikzcd}
  \end{center}
\end{proposition}

\subsubsection{CW complexes}

\begin{definition}[CW complex]
  A CW complex is a topological space $X$ with subspaces
  \[
    \emptyset= X^{-1} \subseteq X^{0} \subseteq X^{1} \subseteq \cdots
    \subseteq X^n = X = \bigcup_{n} X^{n}
  \]
  such that the spaces $X^n$ are constructed inductively in the following
  way. Define
  \[
    X^{-1} := \emptyset.
  \]
  For $n \geq 0$ assume that $X^{n-1}$ is defined.
  Let $\set{D^n_{\alpha}}_{\alpha}$ be a collection of $n$-dimensional discs,
  let $\set{f_{\alpha} \colon \partial D^n_{\alpha} \to X^{n-1}}_{\alpha}$
  be a collection of continuous functions. Define:
  \[
    X^{n} := 
    \bigslant{\del{
    X^{n-1} \rotatebox[origin=c]{180}{$\Pi$} \coprod_{\alpha} D^n_{\alpha}
    }}
    {\sim}
  \]
  where $\sim$ is generated by $x \sim f_{\alpha}(x)$ for all $\alpha$ and
  for all $x \in \partial D^n$.
  We endow the space $X = \cup_{n} X^n$ with the topology such that 
  $U \subset X$ open if and only if $U \cap X^n$ is open for every $n$.
  The space $X$ is called a CW complex. The $X^n$ subspace is called the
  $n$-skeleton of the complex.
  The pairs $(D^n_{\alpha}, f_{\alpha})$ are called the $n$-cells of
  the complex.
\end{definition}

\begin{remark}
  In most examples, the construction process of the complex is finite.
  In this case there exists $n$ such that $X^n = X$.
  We call $n$ the dimension of the complex.
  In particular, the $n$-skeleton $X^n$ is a CW complex of dimension $n$
  (at most).
\end{remark}

\begin{example}
  Note that since $\partail D^0 = \emptyset$ a $0$-dimensional complex $X$ 
  is always a set of points with the discrete topology.
  The $0$-cells are called the vertices of $X$.
\end{example}

\begin{example}
  A $1$-dimensional complex $X$ is a topological graph.
  The $1$-cells are called the edges of $X$.
  Each edge connects to vertices in its extremes.
  \begin{remark}
    Notice that a topological graph is not a simple graph --- each edge
    can connect a vertice to itself, and more than a single edge can
    connect the same vertices.
  \end{remark}
\end{example}

\begin{example}
  We describe a CW complex for the $n$-dimensional sphere $S^n$.
  Let $X^{0} = \set{*}$ be with a single vertice, and a single $n$-cell
  $D^n_{\alpha}$.
  Since there are no cells of dimensions $0 < i < n$ we get $X^{n-1} = \set{*}$.
  The pasting map of the cell $D^n_{\alpha}$ is the constant map
  $F_{\alpha} \colon \partial D^n_{\alpha} \to X^{n-1} = \set{*}$,
  and indeed we have $S^n = X^n = \bigslant{\set{*} \coprod D^n_{\alpha}}{\sim}$
  where $x \sim *$ for all $x \in \partial D^n_{\alpha}$.
\end{example}

In the above example it might be useful to imagine the cases for $n = 1$
and $n = 2$.
In $n = 1$ we paste the boundary $(-1,0)$ and $(0,1)$ to get $S^1$ just as
expected.
In $n = 2$ we paste the boundary again, closing the disc to form the sphere
$S^2$ as expected.

\begin{example}
  Using the construction from the previous example, we can construct a CW
  complex for the torus $T^2 = S^1 \times S^1$ in the following way.
  Let $v$ be a vertice, $a$, $b$ be edges, and $s$ be $2$-cell.
  We connect the $2$-cell by
  TO BE CONTINUED
\end{example}

In general, we only require the pasting maps of CW complexes to be continuous,
but in practice we will only consider very nice pasting maps.
For example pasting faces of polygons.

\begin{definition}[Polygonal complex]
  A polygonal complex is a CW complex of dimension $2$ such that the pasting
  maps of the $2$-cells identify the $2$-cell as a (regular) polygon,
  and paste each of its edges to an edge in $X^1$.
\end{definition}

\section{Topological manifolds and classification of surfaces}

\subsection{Closed topological spaces}
\begin{definition}[Topological manifold]
  A topological manifold of dimension $n$ is a Hausdorff
  topological space $M$ that is locally homeomorphic to $\R^n$.
\end{definition}

\begin{definition}[Closed topological manifold]
  A closed topological manifold of dimension $n$ is a compact, Hausdorff
  topological space $M$ that is locally homeomorphic to $\R^n$.
\end{definition}
\begin{remark}
  A topological space $X$ is locally homeomorphic to a topological space
  $Y$ if for all $x \in X$ there exists an neighbourhood $x \in U \subset X$
  that is homeomorphic to $Y$.
\end{remark}

\begin{example}
  A topological manifold of dimension $0$ is a finite set of points endowed
  with the discrete toplogy.
\end{example}

\begin{example}
  The space $S^1$ is a closed topological manifold of dimension $1$.
\end{example}
\begin{remark}
  In fact, $S^1$ is the only connected closed topological manifold
  of dimension $1$.
\end{remark}

\begin{remark}
  Manifolds of dimension $2$ are called surfaces.
\end{remark}

Some examples of compact surfaces are:
\begin{itemize}
  \item The sphere $S^2$.
  \item The torus $T^2 = S^1 \times S^1$.
  \item The projective plane
    \begin{align*}
      P^2 &= \bigslant{\R^3 \setminus \set{0}}{x \sim \lambda x},\ 
        \forall \lambda \in \R \setminus \set{0} \\
      &= \bigslant{S^2}{(x,y) \sim (-x,-y)} \\
      &= \bigslant{D^2}{(x,y) \sim (-x,-y)},\ \forall (x,y) \in \partial D^2
    \end{align*}
  we can see it's a surface very easily from the second equivalence.
  \item The Klein bottle $K^2$.
  \item The orientable surface of genus $g$.
\end{itemize}
\begin{remark}
  The map $(x,y) \mapsto (-x,-y)$ is called the antipodal map.
\end{remark}
\begin{remark}
    In general, any product of closed manifolds of dimensions
    $u$, $v$ is a closed manifold of dimension $u + v$.
\end{remark}

% Show what the last two mean

Some examples of $n$-dimensional topological manifolds
\begin{itemize}
  \item The spehre $S^n$.
  \item The torus $T^n = (S^1)^n$.
  \item The projective $n$-dimensional space
    \begin{align*}
      P^n &= \bigslant{\R^{n+1} \setminus \set{0}}{x \sim \lambda x},\ 
        \forall \lambda \in \R \setminus \set{0} \\
      &= \bigslant{S^n}{x \sim -x} \\
      &= \bigslant{D^n}{x \sim -x},\ \forall x \in \partial D^n
    \end{align*}
  we can see it's a surface very easily from the second equivalence.
\end{itemize}
We will see some more examples of topological manifolds of dimension
$3$, and see that classifying manifolds of dimension $4$ is impossible
in practice.

\subsection{Manifolds with boundary}
\begin{definition}[Compact manifold with boundary]
  A compact manifold with boundary is a compact, Haudorff topological 
  space $M$ that is locally homeomorphic to an open subset of
  $\overline {\mathbb H^n} = \set{(x_1,x_2,\dots,x_n) \in 
  \R^n \mid x_n \geq 0}$.
  The set of the points that correspond to the points in
  $\partial \mathbb H = \R^{n-1} \times \set{0}$ is called the boundary of
  the manifold, and we denote it $\partial M$.
\end{definition}
\begin{remark}
  The boundary $\partial M$ is well defined.
  That is, it is not dependent on the choice of the local homeomorphisms.
  We won't prove this fact in this course.
\end{remark}

\begin{proposition}
  Let $M$ be a compact $n$-dimensional manifold with boundary.
  Then $\partial M$ is a closed $n-1$-dimensional manifold.
\end{proposition}
Some examples of manifolds with boundary
\begin{itemize}
  \item The closed interval $[0,1]$ is a compact $1$-dimensional manifold
    with boundary.
  \item A surface from which we remove an open disc is a compact surface
    with boundary, and the boundary is homeomeomorphic to $S^1$.
  \item An orientable handlebody of genus $g$ is the \dots TO BE CONTINUTED
\end{itemize}

\subsection{Triangulation}
\begin{definition}[Triangulation of a surface]
  A triangulation of a surface is a polygonal complex (made of trianlges)
  that is homeomorphic to the surface.
\end{definition}
\begin{remark}
  That is like saying there exists a finite collection of triangles, and
  pastings of their edges (each edge is pasted to at most one other edge).
  If the surface is closed, each edge is pasted to exactly one other edge.
\end{remark}

\begin{theorem}[Rad\'o]
  Every surface is can be triangulated.
\end{theorem}
The proof of this theorem will be omitted in this course.
\begin{remark}
  We have not defined what is triangularization means in $3$ dimensions,
  but according to one definition, every manifold of dimension $3$ can be
  triangulated.
\end{remark}
\begin{remark}
  There is no similar theorem for dimensions $n > 3$.
\end{remark}
\begin{remark}
  The question whether for every $4$-dimensional there exists a CW complex
  homeomorphic to it is open.
\end{remark}

\subsection{Connected sum of surfaces}
\begin{exercise}
  Let $M$ be a compact manifold with boundary.
  Suppose there exist two homeomorphic connected components
  $N_1,N_2 \subseteq \partial M$.
  Then for any homeomorphism $f \colon N_1 \to N_2$ the space
  \[
    \bigslant{M}{x \sim f(x)}, \quad \forall x \in N_1
  \]
  is a compact manifold with boundary.
\end{exercise}

\begin{remark}
  Every closed $3$-dimensional manifold is the connected sum of two
  handlebodies on their boundary.
\end{remark}

\begin{definition}[Knot]

\end{definition}

\begin{definition}[Link]

\end{definition}

\begin{remark}
  Every closed $3$-dimensional manifold is CONTINUE LATER
\end{remark}

\begin{definition}[Connected sum]
  Let $S_1$, $S_2$ be two connected compact manifolds.
  The connected sum $S_1 \# S_2$ is the space we get by
  choosing two discs $D_1 \subset S_1$ and $D_2 \subset S_2$,
  and a homeomorphism $f \colon \partial D_1 \to \partial D_2$ and setting
  \[
    S_1 \# S_2 = 
    (S_1 \setminus D_1^{\circ}) \amalg
    (S_2 \setminus D_2^{\circ}) \slash x \sim f(x),
    \quad \forall x \in \partial D_1
  \]
\end{definition}
\begin{remark}
  The connected sum is not dependent on the choice of discs,
  nor the homeomorphism.
\end{remark}
% examples
\begin{remark}
  We can also define the connected of manifolds in $n$ dimensions.
  In the general case, the result can be dependent on the choice of the
  homeomorphism - there are two options to the choice of the homeomorphism
  ASK YALI
\end{remark}


\begin{example}
  The manifold $S^2$ is the identity element of connected addition.
  For any manifold $\Sigma$ we have $\Sigma \# S^2 = \Sigma$.
\end{example}
\begin{example}
  The sum $T^2 \# T^2$ is the orientable closed surface of genus $2$.
\end{example}
\begin{example}
  The sum $T^2 \# T^2$ is the closed orientable surface of genus $2$.
  We can get this by pasting edges of a haptagon as can be seen in ADD IMAGE
\end{example}
\begin{example}
  We have $P^2 \# P^2 = K^2$ as can be seen in ADD IMAGE.
\end{example}
\begin{example}
  %% P
\end{example}

\begin{remark}
  If $S_1$, $S_2$ are given by a pasting of a boundary of an $n_1$-gon and
  $n_2$-gon, then $S_1 \# S_2$ can be given as a pasting of the boundary of
  an $n_1 + n_2$-gon, with the pasting of its edges being the concatenation
  of the pastings of $S_1$, $S_2$.
\end{remark}

\begin{proposition}
  \label{prop:pp-kp-tp}
  $P^2 \# P^2 \# + P^2 \cong K^2 \# P^2 \cong T^2 \# P^2$
\end{proposition}
\begin{proof}
  TO BE ADDED.
\end{proof}

\newpage

\section{Classification of surfaces}
\subsection{Classification of compact surfaces}


\begin{theorem}[The classification theorem of compact surfaces]
  \label{thm:surfaces-classification}
  Every connected surface is homeomorphic to one of the following:
  \begin{enumerate}
    \item[(1)] $S^2$;
    \item[(2)] $T^2 \# \cdots \# T^2$;
    \item[(3)] $P^2 \# \cdots \# P^2$.
  \end{enumerate}
\end{theorem}
\begin{definition}[Genus]
  The number of summands in cases $(2)$ and $(3)$ is called the genus of
  the surface.
\end{definition}

\begin{theorem}
  \label{thm:classification-lemma}
  Every compact surface with triangulation is homeomorphic to
  \[
    \underbrace{T^2 \# \cdots \# T^2}_{n \text{ times}} \# 
    \underbrace{P^2 \# \cdots \# P^2}_{m \text{ times}}
  \]
\end{theorem}
The idea of the proof of \Cref{thm:surfaces-classification} is to show
we can assume $S$ is not homeomorphic to $S^2$ and then using 
\Cref{prop:pp-kp-tp} and \Cref{thm:classification-lemma} we would get the
desired result.

\subsection{Euler's characteristic}
\begin{definition}[Euler characteristic]
  Let $X$ be a CW complex with a finite amount of cells.
  The Euler characteristic of $X$ is defined as the following alternating
  sum:
  \[
    \chi(X)=
    \#\set{0\text-\Cells} -
    \#\set{1\text-\Cells} +
    \#\set{2\text-\Cells} -
    \#\set{3\text-\Cells} + \cdots
  \]
\end{definition}
\begin{example}
  A CW complex of the sphere $S^2$ with one $0$-cell and one $2$-cell gives
  \[
    \chi(S^2) = 1 - 0 + 1 = 2
  \]
\end{example}
\begin{example}
  A CW complex of the sphere $S^2$ with one $0$-cell and one $2$-cell gives
  \[
    \chi(S^2) = 1 - 0 + 1 = 2
  \]
\end{example}

%%%%%%%%


% Add example of pasting all sides of the triangle to one another.

% locally homeomorphic to R but not hausdorff




\end{document}
