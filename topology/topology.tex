\documentclass[11pt,a4paper]{article}
\usepackage{amssymb,amsfonts,amsmath,calc,tikz,pgfplots,geometry}
\usepackage{color}   %May be necessary if you want to color links
\usepackage{hyperref}
\usepackage{amsthm}
\usepackage{fancyhdr}
\pagestyle{fancy}
\usetikzlibrary{positioning}
\geometry{margin=1in}
\pgfplotsset{compat=1.18}
\setlength{\headheight}{14.6pt}
\addtolength{\topmargin}{-1.6pt}
\hypersetup{
    colorlinks=false, %set true if you want colored links
    linktoc=all,   %set to all if you want both sections and subsections linked
    linkcolor=black,  %choose some color if you want links to stand out
}
%%%%%%%%%%%%%%%%%%%%%%%%%%%%%%%%%%%%%%%%%%%%%%%%%%%%%%%%%%%%%%%%%%%%%%%%%%%%%%%
\theoremstyle{definition}
\newtheorem{definition}{Definition}[section]
\newtheorem{remark}{Remark}[section]
\newtheorem{example}{Example}[section]
\theoremstyle{plain}
\newtheorem{theorem}{Theorem}[section]
\newtheorem{proposition}[theorem]{Proposition}
\newtheorem{lemma}[theorem]{Lemma}
\newtheorem{corollary}[theorem]{Corollary}

\DeclareMathOperator{\lcm}{lcm}
\DeclareMathOperator{\idealin}{\triangleleft}
\DeclareMathOperator{\im}{im}
\DeclareMathOperator{\Aut}{Aut}
\DeclareMathOperator{\End}{End}
\DeclareMathOperator{\Inn}{Inn}
\DeclareMathOperator{\Out}{Out}
\DeclareMathOperator{\Mat}{Mat}
\DeclareMathOperator{\std}{std}
\DeclareMathOperator{\Int}{Int}
\DeclareMathOperator{\diam}{diam}
\newcommand{\N}{\mathbb{N}}
\newcommand{\st}{\text{ s.t. }}
\newcommand{\Z}{\mathbb{Z}}
\newcommand{\Q}{\mathbb{Q}}
\newcommand{\R}{\mathbb{R}}
\newcommand{\C}{\mathbb{C}}
\newcommand{\F}{\mathbb{F}}
\newcommand{\Omicron}{O}
\newcommand{\ip}[2]{\langle #1, #2 \rangle}
\newcommand{\set}[2]{ \left\{ #1 \mid #2 \right\} }
\newcommand{\abs}[1]{\left\lvert #1\right\rvert}
\newcommand{\norm}[1]{\left\lVert #1\right\rVert}
\renewcommand{\tt}[1]{\textnormal{\textbf{(#1).}}} %tt=theorem title
\newcommand{\bigslant}[2]
{{\raisebox{.2em}{$#1$}\left/\raisebox{-.2em}{$#2$}\right.}}
%%%%%%%%%%%%%%%%%%%%%%%%%%%%%%%%%%%%%%%%%%%%%%%%%%%%%%%%%%%%%%%%%%%%%%%%%%%%%%%
\title{\textbf{Introduction to Metric and Topological Spaces}}
\author{heavily based on notes by Ariel Rapaport}
\date{}
\begin{document}
  \maketitle
  \newpage
  \section{Metric Spaces}
  First we will begin with metric spaces.
  \begin{definition}
  Let $X$ be a non-empty set. A metric on $X$ is a function 
  $d \colon X \times X \to [0,\infty]$ such that for all $x,y,z \in X$,
  \end{definition}
  \begin{enumerate}
  \item $d(x,y) = 0$ if and only if $x = y$;
  \item $d(x,y) = d(y,x)$ (symmetry);
  \item $d(x,z) \le d(x,y) + d(y,z)$ (triangle inequality);
  \end{enumerate}
  \emph{The pair $(X,d)$ is said to be a \textbf{metric space}.}
  \begin{example}
  Let $X$ be a non-empty set. Let $d \colon X \times X \to [0,\infty)$ be
  the function such that for $x,y \in X$,
  \[
    d(x,y) := \begin{cases}
      0, & x=y \\
      1, & x \neq y
    \end{cases}
  \]
  The function $d$ is a metric and it is called \textbf{the discrete metric}
  on $X$.
  \end{example}
  \begin{example}
  Let $X = \R^n$ and define the function:
  \[
    d(x - y) := \abs{x - y}
  \]
  Where $|\cdot | \colon \R \to \R$ is the Eclidean norm function. 
  Then the pair $(X, d)$ forms a metric space.
  \end{example}
  \begin{example}
  Let $(X, N)$ be an arbitrary normed space and define the function:
  \[
    d(x-y) := N(x - y)
  \] 
  Then the pair $(X, d)$ forms a metric space.
  \end{example}
  \begin{example}
  The pair $(C([0,1]), d)$ such that $C([0,1])$ is the space of all
  continuous functions on $[0,1]$ paired with the metric:
  \[
    d(f,g) = \int_{0}^{1}{\abs{f(x) - g(x)}\,dx}
  \]
  Is also a metric space.
  \end{example}
  \begin{example}
  The pair $(C([0,1]), d)$ paired with the supremum metric:
  \[
    d(f,g) = \sup_{x \in [0,1]}{|f(x) - g(x)|}
  \]
  Is also also metric space.
  \end{example}
  \begin{example}
  Let $\Lambda$ be a nonempty set which will represent an alphabet.
  The set $\Lambda^{\N}$ represents all the sequences over that alphabet.
  The pair $(\Lambda^\N, d)$ with the metric $d$ defined on two sequences
  $\omega = (\omega_n)_{n=1}^{\infty}, \eta = (\eta_n)_{n=1}^{\infty}$
  as:
  \[
    d(\omega, \eta) = \begin{cases}
      2^{-\min\{n \geq 0 \mid \omega_n \neq \eta_n\}} & 
      \omega \neq \eta \\
      0 & \omega = \eta
    \end{cases}
  \]
  \end{example}

  \newpage

  \section{Topological Spaces}
  \begin{definition}
    Let $X$ be a nonempty set. A collection $\tau \subset P(X)$ is said to
    be a topology on $X$ if it satisfies the following properties,
    \begin{enumerate}
      \item $X$, $\emptyset \in \tau$;
      \item Any union of sets in $\tau$ is a set in $\tau$;
      \item Any finite intersection of sets in $\tau$ is a set in $\tau$;
    \end{enumerate}
    The pair $(X,\tau)$ is said to be a topological space and $U \in \tau$
    an open set of $(X,\tau)$. An element $x \in X$ is said to be a point
    of $(X,\tau)$.
  \end{definition}
  \begin{example}
    Every metric spaces can induce a topological space. Let $(X,d)$ be a
    metric space. It can be verified that:
    \[
      \tau := \set{U \subset X}{\forall x \in U \quad \exists \epsilon > 0
      \st B(x,\epsilon) \subset U}
    \]
    Is a topology on $X$.
  \end{example}
  \begin{definition}
    We say that a topological space $(X,\tau)$ is metrizable if exists a 
    metric $d$ on $X$, such that the topology that $d$ induces on $X$
    is equal to $\tau$.
  \end{definition}
  \begin{example}
    Let $X$ be a nonempty set and let $\tau := P(X)$. The topology $\tau$
    is called the discrete topology and the space $(X,\tau)$ is called the 
    discrete space. Is it metrizable?
  \end{example}
  \begin{example}
    Let $X$ be a nonempty set and let $\tau := \{\emptyset,X\}$. The topology 
    $\tau$ is called the trivial topology. Is it metrizable when $|X| = 1$?
    Is it metrizable when $|X| > 1$?
  \end{example}
  \begin{example}
    Let $X$ be any infinite set and let 
    $\tau := \set{A \subset X}{\abs{X \setminus A} < \infty} \cup
    \{\emptyset\}.$ 
    The topology $\tau$ is called the finite complement topology. Is it 
    metrizable?
  \end{example}
  \begin{definition}
    A mapping $f \colon X \to Y$ is said to be continuous if for every open
    set $U \subset Y$ the set $f^{-1}(U)$ is open.
  \end{definition}
  \begin{definition}
    Given $x \in X$, an open $U \subset X$ containing $x$ is said to be a
    neighbourhood of $x$.
  \end{definition}
  \begin{definition}
    A mapping $f \colon X \to Y$ is said to be continuous at $x$ if for 
    every neighbourhood $U$ of $f(x)$ there exists a neighbourhood $V$
    of $x$ such that $f(V) \subset U$.
  \end{definition}
  \begin{definition}
    A mapping $f \colon X \to Y$ is said to be open if $f(U)$ is open in $Y$
    for every open $U \subset X$.
  \end{definition}
  \begin{definition}
    A mapping $f \colon X \to Y$ is said to be a homeomorphism if it is
    injective, surjective, continuous and open. If there exists such an $f$, 
    then we say that $X$ and $Y$ are homeomorphic.
  \end{definition}
  \begin{remark}
    We say that a property $P$ is a \emph{topological property} if for every
    two homeomorphic spaces $X$ and $Y$, then $P$ holds for $X$ if and only
    if it holds for $P$. The branch that deals with topological properties
    is called topology.
  \end{remark}
  \begin{definition}
    Let $(X,\tau_X)$ be a topological space and let $\emptyset \neq Y \subset 
    X$. Define
    \[
      \tau_Y = \set{U \cap Y}{U \in \tau_X}
    \]
    We call $\tau_Y$ the subspace topology, induced by $\tau_X$ on $Y$.
  \end{definition}
  \begin{proposition}
    \tt{Characteristic property of the subspace topology}
    Let $(X, \tau_X)$
    be a topological space, let $\emptyset \neq Y \subset X$, and write 
    $\tau_Y$ for the subspace topology on $Y$. Then $\tau_Y$ is the unique 
    topology on $Y$ which satisfies the following property. Let $Z$
    be a topological space and let $f \colon Z \to X$ be with 
    $f(Z) \subset Y$. Then $f$ is continuous as a map into $(X, \tau_X)$ if 
    and only if it is continuous as a map into $(Y, \tau_Y)$.
  \end{proposition}
  Throughout this section let $X$ be a fixed topological space.
  \begin{definition}
    A subset $F$ of $X$ is said to be closed if $F^c = X \setminus F$ is open.
  \end{definition}
  \begin{proposition}
    The following properties are always satisfied:
    \begin{enumerate}
      \item $X$, $\emptyset$ are closed;
      \item Any intersection of closed sets is closed;
      \item Any finite union of closed sets is closed;
    \end{enumerate}
  \end{proposition}
  \begin{definition}
    Given $A \subset X$ we denote $\overline{A}$ to be the intersection of
    all $F \subset X$ such that $A \subset F$ and $F$ is closed. We call
    $\overline{A}$ the closure of $A$.
  \end{definition}
  We can also define the closure of $A$ in an alternate way:
  \[
    \overline{A} = \set{x \in X}{A \cap U \neq \emptyset \,\,\text{for each 
    neighbourhood $U$ of $x$}}.
  \]
  You may try to prove that both definitions are equivalent.
  \begin{definition}
    A subset $A$ of $X$ is said to be dense in $X$ if $\overline{A} = X$.
  \end{definition}
  Note that the second definition of closure we get that $A$ is dense in $X$
  if and only if $A \cap U \neq \emptyset$ for every nonempty $U \subset X$.
  \begin{definition}
    We say that $X$ is seperable if it has a countable dense subset.
  \end{definition}
  \begin{definition}
    Let $A \subset X$. We say that $x$ is an isolated point of $A$ if exists
    $U$ open in $X$ such that $X \cap U = \{x\}$.
  \end{definition}
  This is exactly the same as saying that $x$ is an isolated point if and
  only if the singleton $\{x\}$ is open in the subspace topology.
  \begin{definition}
    Let $A \subset X$. We say that $x$ is a limit point of $A$ if for every
    neighourhood $U$ of $x$ there exists $a \in U \cap A$ with $a \neq x$.
    The set of all limit points of $A$ is called the derived set of $A$
    and is denoted by $D(A)$.
  \end{definition}
  \begin{proposition}
    Let $A \subset X$ be given, then
    \begin{enumerate}
      \item $\overline{A} = A \cup D(A)$.
      \item $A$ is closed if and only if $D(A) \subset A$.
    \end{enumerate}
  \end{proposition}
  \begin{corollary}
    Let $A \subset X$ be closed and write $I(A)$ for the set of all isolated
    points of $A$. Then $A$ is the disjoint union of $D(A)$ and $I(A)$.
  \end{corollary}
  \begin{definition}
    Let $A$ be a subset of $X$. The interior of $A$ is denoted by $\Int(A)$
    and is defined to be the union of all open subsets $U$ of $X$ so that 
    $U \subset A$. A point $x \in \Int(A)$ is said to be an interior point 
    of $A$.
  \end{definition}
  \begin{proposition}
    Let $A \subset X$ be given, then
    \begin{enumerate}
      \item $\Int{A} = X \setminus \overline{(X \setminus A)}$.
      \item $\Int(A)$ is open and contained in $A$.
      \item $A$ is open if and only if $A = \Int(A)$.
    \end{enumerate}
  \end{proposition}
  \begin{definition}
    Let $A \subset X$ be given. A point $x \in X$ is said to be a boundary
    point of $A$ if for every neighbourhood $U$ of $x$ we have 
    $U \cap A \neq \emptyset$ and $U \cap (X \setminus A) \neq \emptyset$.
    The set of all boundary points of $A$ is called the boundary of $A$ and 
    is denoted by $\partial A$.
  \end{definition}
  \begin{proposition}
    For $A \subset X$ we have 
    $\partial A = \overline{A} \cap \overline{X \setminus A}$ and in 
    particular $\partial A$ is closed.
  \end{proposition}
  \begin{proposition}
    Let $A \subset X$ be given, then $A$ is the disjoint union of $\Int(A)$
    and $\partial A$.
  \end{proposition}
  \begin{definition}
    A family $\mathcal{B}$ of subsets of $X$ is said to be an open base
    for $X$ if for each open $U \subset X$ and $x \in U$ exists 
    $B \in \mathcal{B}$ such that $x \in B \subset U$.
  \end{definition}
  It is easy to see that a family $\mathcal{B}$ of open subsets of $X$ is an 
  open base for $X$ if and only if each open $U \subset X$ is a union of 
  elements of $\mathcal{B}$.
  \begin{definition}
    We say that $X$ is second countable, or that it satisfies the second
    axiom of countability, if it has a countable open base.
  \end{definition}
  \begin{proposition}
    Suppose that $X$ is second countable, then $X$ is separable.
  \end{proposition}
  \begin{proof}
    Let $\mathcal{B}$ be a countable open base for $X$. We have that 
    $\mathcal{B} \setminus \{\emptyset\}$ is also an open base. Choose
    an arbitrary $x_B \in B$ for each $B \in \mathcal{B}$. Since 
    $\mathcal{B}$ is countable $\{x_B\}_{B \in \mathcal{B}}$ is also 
    countable. Let $U$ be open in $X$. By defintion of an open base exists 
    $B \in \mathcal{B}$ such that $x_B \in B$ and $B \subset U$ so 
    $x_B \in U$. This shows that $\{x_B\}_{B \in \mathcal{B}}$ is dense
    in $X$ and thus $X$ is seperable.
  \end{proof}
  \begin{remark}
    In metric spaces, the property of being seperable and second countable
    is equivalent. If we denote $(X,d)$ the metric space and $A$ the 
    countable dense set, then
    \[
      \mathcal{B} = \set{B(a,q)}{a \in A \text{ and } q \in \Q \cap 
      (0,\infty)}
    \]
    Will form a countable open base for $X$.
  \end{remark}
  \begin{example}
    A classic example of a topological space that is seperable but not
    second countable is the Sorgenfrey line, also known as the lower
    limit topology, which we will discuss later.
  \end{example}
  \begin{theorem}
    \tt{Lindelöf’s Theorem} Suppose that $X$ is second countable. Let
    $\{U_i\}_{i \in I}$ be a family of open subsets of $X$. Then there
    exists a countable $I_0 \subset I$ so that 
    $\cup_{i \in I_{0}}{U_i} = \cup_{i \in I}{U_i}$
  \end{theorem}
  \begin{proof}
    Let $\mathcal{B}$ be a countable open base for $X$. Set,
    \[
      \mathcal{B}_0 = \set{B \in \mathcal{B}}{B \subset U_i
      \text{ for some } i \in I}
    \]
    For each $B \in \mathcal{B}_0$ choose an arbitrary $i_B \in I$ such
    that $B \subset U_{i_B}$. Set $I_0 = \set{i_B}{B \in \mathcal{B}_0}$.
    Since $\mathcal{B}$ is countable, $I_0$ is also countable. It remains
    to show that $\cup_{i \in I_{0}}{U_i} = \cup_{i \in I}{U_i}$. Let
    $x \in \cup_{i \in I}{U_i}$ then exists some $j$ such that $x \in U_j$
    since $\mathcal{B}$ is an open base exists $B \subset U_j$ such that
    $x \in B$. By definition we see that $B \in \mathcal{B}_0$, thus
    $i_B \in I_0$ and then:
    \[
      x \in B \subset U_{i_B} \subset \cup_{i \in I_{0}}{U_i}
    \]
    The other side of the inclusion is obvious which concludes the proof.
  \end{proof}
  \begin{corollary}
    Suppose that $X$ is second countable and that $\mathcal{B}$ is an open
    base for $X$. Then exists a countable $\mathcal{B}_0 \subset \mathcal{B}$ 
    which is also an open base for $X$.
  \end{corollary}
  \begin{proof}
    Let $\mathcal{E}$ be a countable open base for $X$. Since $\mathcal{B}$
    is an open base, for each $E \in \mathcal{E}$ exists 
    $\mathcal{B}_E \subset \mathcal{B}$ such that 
    $E = \cup_{B \in \mathcal{B}_E}$. From Lindel\"of's theorem
    we have that exists a countable $\mathcal{B}^0_E \subset \mathcal{B}_E$
    such that $U_{B \in \mathcal{B}^0_E} = \cup_{i \in \mathcal{B}_E}{U_i}$.
    Now set $\mathcal{B}_0 = \cup_{E \in \mathcal{E}}{\mathcal{B}^0_E}$.
    It is countable as a countable union of countable sets. Moreover,
    since $\mathcal{E}$ is an open set, and since each $E \in \mathcal{E}$
    is a union of elements from $\mathcal{B}_0$, it is clear that
    $\mathcal{B}_0$ is also an open base for $X$ which completes the proof.
  \end{proof}
  \begin{definition}
    Let $x \in X$. A class of neighbourhoods $B_x$ of $x$ is called an open
    base at $x$ if for every neighbourhood $U$ of $x$ exists $B \in B_x$
    such that $B_x \subset U$.
  \end{definition}
  \begin{definition}
    We say that $X$ is first countable, or that it satisfies the first
    axiom of countability, if for each $x \in X$ there exists a countable 
    open base at $x$.
  \end{definition}
  \begin{remark}
    It is clear that if $X$ is second countable, it is also first countable.
  \end{remark}
  \begin{definition}
    Let $X$ be a topological space. A family $\mathcal{S}$ of open subsets 
    of $X$ is said to be an open subbase for $X$ if the collection of all 
    finite intersections of elements of $\mathcal{S}$ forms an open base 
    for $X$.
  \end{definition}
  \begin{proposition}
    Let $X$ and $Y$ be topological spaces, let $\mathcal{S}$ be an open
    subbase for $Y$. Then if $f^{-1}(S)$ is open for each $S \in \mathcal{S}$
    then $f$ is continuous.
  \end{proposition}
  The above proposition shows how convenient working with subbases can be.
  The following will show how to easily generate topologies using the notion.
  \begin{proposition}
    Let $X$ be a set, and let $\mathcal{S} \subset P(X)$. Set,
    \[
      \mathcal{B} := \set{\cap_{i=1}^{n}{S_i}}{n \geq 0 \textnormal{ and }
      S_1,\dots,S_n \in \mathcal{S}}
    \]
    And,
    \[
      \tau := \set{U \subset X}{\forall x \in U \quad \exists B \in 
      \mathcal{B} \st x \in B \subset U}
    \]
    Then $\tau$ is a topology on $X$, $\mathcal{B}$ is an open base for 
    $\tau$ and $\mathcal{S}$ is an open subbase for it.
  \end{proposition}
  We may want to compare topologies now. Let $\mathcal{T}(X)$ be the set of
  all topologies on a set $X$.
  \begin{definition}
    Let $\tau_1,\tau_2 \in \mathcal{T}(X)$. We say that $\tau_1$ is weaker
    than $\tau_2$, or that $\tau_2$ is stronger than $\tau_1$ if 
    $\tau_1 \subset \tau_2$.
  \end{definition}
  For a simple reality check, notice that every topology is weaker than the
  discrete topology and stronger than the discrete topology. Also, the
  pair $(\mathcal{T}(X), \subset)$ form a partially ordered set.
  \begin{definition}
    Let $\{Y_i\}_{i \in I}$ be a family of topological spaces. For each
    $i \in I$ let $f_i \colon X \to Y_i$. Write $\mathcal{T}_0 \subset 
    \mathcal{T}_0$ for the set of all $\tau$ with respect to which all
    $\{f_i\}_{i \in I}$ are continuous. The greatest lower bound of 
    $\mathcal{T}_0$ is called the weak topology generated by 
    $\{f_i\}_{i \in I}$.
  \end{definition}
  \begin{remark}
  It is easy to verify that $\tau = \cap_{\tau_0 \in \mathcal{T}_0}{\tau_0}$
  and also that $\tau$ is generated by 
  \[
    \set{f_i^{-1}(U)}{i \in I \text{ and $U$ is open in $Y_i$}}
  \]
  \end{remark}
  \begin{definition}
    The product topology on a cartesian product of topological spaces
    $\prod_{i \in I} X_i$ is defined to be the weak topology generated by
    the projections $\{\pi_i\}_{i \in I}$. Equipped with the product
    topology, the space $X$ is called the product space of the spaces
    $\{X_i\}_{i \in I}$.
  \end{definition}
  This definition is a bit abstract, but we can give a more concrete 
  definition by setting,
  \[
    \mathcal{S} = \set{\prod_{i \in I}{U_i}}{\exists j \in I \st 
    \text{ $U_i = X_i$ for $i \in I \setminus \{j\}$ and $U_j$ is open in
    $X_j$}}
  \]
  Now the product topology on $X$ is equal to the topology on $X$ generated
  by $\mathcal{S}$ as a subbase. From this we can also deduce that
  {\small
  \[
    \mathcal{B} = \set{\prod_{i \in I}(U_i)}{\text{Exists a finite
    $I_0 \subset I$ \st $U_i = X_i$ for $i \in I \setminus I_0$ and
    $U_i$ is open in $X_i$ for $i \in I_0$}}
  \]
  }
  Is an open base for the product topology.
  \begin{proposition}
    \tt{Characteristic property of the product topology}
    The product topology is the unique topology on $X$ which satisfies the 
    following property. Let $Y$ be a topological space and let 
    $f \colon Y \to X$. Then $f$ is continuous if and only if 
    $\pi_i \circ f$ is continuous for each $i \in I$.
  \end{proposition}
  \begin{definition}
    Let $X$ be a topological space. We write $C(X)$ for the collection of 
    all continuous functions from $X$ to $\R$. We denote by $C_b(X)$ the set 
    of all bounded elements of $C(X)$. It has the a natural norm defined on
    it, the supremum norm.
  \end{definition}
  More about algebras in a later section.

  \newpage

  \section{Complete Metric Spaces}
  Let $(X,d)$ be a fixed metric space.
  \begin{definition}
    We say that a sequence $\{a_n\}_{n \geq 1} \subset X$ is a Cauchy
    sequence if for all $\epsilon > 0$ exists $N \geq 1$ such that 
    $d(x_n,x_m) < \epsilon$ for all $n,m > N$.
  \end{definition}
  It is easy to verify that all Cauhcy sequences converge, but the converse
  is not always true. This leads us to formulate the following notion.
  \begin{definition}
    We say that the metric space $(X,d)$ is complete if every Cauchy
    sequence $\{a_n\}_{n \geq 1} \subset X$ converges to some $x \in X$.
  \end{definition}
  \begin{proposition}
    Suppose that $X$ is complete and let $Y$ be a nonempty subset of $X$. 
    Then $Y$ is complete (with respect to the metric induced by $X$) if and 
    only if $Y$ is closed in $X$.
  \end{proposition}
  The following lemma demonstrates the usefulness of the completeness property.
  Given a nonempty subset $A$ of $X$ we write
  \[
    \diam(A) := \set{d(x_1,x_2)}{x_1,x_2 \in X}
  \]
  And we say that the number $\diam(A)$ is the diameter of $A$.
  \begin{lemma}
    \tt{Cantor’s intersection lemma for complete metric spaces}
    Let $F_1,F_2,\dots$ be nonempty closed subsets of $X$. Suppose that
    \begin{itemize}
      \item $X$ is complete;
      \item $F_{n+1} \subset F_{n}$ for all $n \geq 1$;
      \item $\diam(F_n) \to 0$ as $n \to \infty$.
    \end{itemize}
    Then $\cap_{n \geq 1}{F_n} = \{x\}$ for some $x \in X$.
  \end{lemma}
  \begin{proof}
    For each $n \geq 1$ choose $x_n \in F_n$. For each $n \geq m \geq 1$
    we have that $x_n,x_m \in F_n$ and thus $d(x_n,x_m) \le \diam(F_n)$.
    Since $\diam(F_n) \to 0$ we have that $\{x_n\}_{n \geq 1}$ is a Cauchy
    sequence in $X$. Since $X$ is complete exists $x \in X$ such that 
    $x_n \to x$ as $n \to \infty$. For each $n \geq 1$ we have that 
    $\{x_m\}_{m \geq n} \subset F_n$ and since each $F_n$ is closed, we
    have that $x \in F_n$. Thus $x \in \cap_{n \geq 1}{F_n}$ and in 
    particular $\cap_{n \geq 1}{F_n} \neq \emptyset$. Let 
    $x,y \in \cap_{n \geq 1}{F_n}$. We have that $d(x,y) \le \diam(F_n)$
    for each $n \geq 1$. Since $\diam(F_n) \to 0$ as $n \to \infty$ we
    have that $d(x,y) = 0$ and thus $x=y$. This implies that
    $\cap_{n \geq 1}{F_n} = \{x\}$ which completes the proof.
  \end{proof}
  \begin{definition}
    Let $(X,d_X)$ and $(Y,d_Y)$ be metric spaces. A map $f \colon X \to Y$ 
    is said to be an isometry if $d_X(x_1,x_2) = d_Y(f(x_1),f(x_2))$ for all 
    $x_1,x_2 \in X$. We say that $X$ and $Y$ are isometric if there exists 
    an isometry from $X$ onto $Y$.
  \end{definition}
  \begin{remark}
    Every isomorphism is continuous and injective. A surjective isomorphism
    is thus a homeomorphism.
  \end{remark}
  \begin{theorem}
    \tt{The completion theorem for metric spaces}
    Let $X$ be a metric space. Then there exists a complete metric space
    $\overline{X}$ and an isometry $f \colon X \to \overline{X}$ such that 
    $f(X)$ is dense in $\overline{X}$. Moreover, if $Y$ is another complete 
    metric space such that exists an isometry $g \colon X \to Y$ such that
    $g(X)$ is dense in $Y$ then there exists a surjective isometry
    $h \colon \overline{X} \to Y$ so that $g = h \circ f$.
  \end{theorem}
  \begin{remark}
    The space $\overline{X}$ is called the completion of $X$. As the theorem
    states it is unique up isometry.
  \end{remark}
  \begin{definition}
    A mapping $f \colon X \to Y$ is said to be uniformly continuous if for
    every $\epsilon > 0$ there exists $\delta > 0$ so that 
    $d_Y(f(x_1),f(x_2)) < \epsilon$ for all $x_1,x_2 \in X$ with 
    $d_X(x_1,x_2) < \delta$.
  \end{definition}
  \begin{proposition}
    Let $A \subset X$, let $f \colon X \to Y$ be uniformly continuous.
    Then there existsa unique $\overline{f} \colon \overline{X} \to Y$ 
    which extends $Y$ such that $\overline{f}$ is also continuous.
  \end{proposition}
  Preperation for Baire's theorem
  \begin{definition}
    A subset $A$ of $X$ is said to be nowhere dense if $\Int(A) = \emptyset$.
  \end{definition}
  \begin{remark}
    Note that a closed $A \subset X$ is nowhere dense if and only if 
    $\Int(A) = \emptyset$.
  \end{remark}
  \begin{example}
    Let $W$ be a linear subspace of $\R^d$ with $\dim W < d$. We will show
    that $W$ is nowhere dense. Let $\ip{\cdot}{\cdot}$ be the standard
    inner product on $\R^d$. We notice that for every $v \in \R^d$ the
    map $x \mapsto \ip{x}{v}$ is continuous. Thus the set 
    $\set{x \in \R^d}{f(x) = 0}$ is closed in $\R^d$ as the preimage of the
    closed set $\{0\}$. From this and from the fact that:
    \[
      W = (W^{\perp})^{\perp} = \cap_{u \in W^{\perp}}{\set{x \in \R^d}
      {\ip{x}{u} = 0}}
    \]
    It follows that $W$ is closed in $\R^d$. Moreover, for every $w \in W$,
    $\neq u \in W^{\perp}$ and $\epsilon > 0$ we have that 
    $w+u\epsilon \notin W$. Since $W^{\perp} \neq \emptyset$ we see that
    $\Int(W) = \emptyset$, which shows that $W$ is nowhere dense.
  \end{example}
  \begin{definition}
    A subset $E$ of $X$ is said to be of the first category if there exist
    nowhere dense subsets $A_1,A_2,\dots \subset X$ so that 
    $E = \cup_{n \geq 1}{A_n}$. A subset of $X$ which is not of the first 
    category is said to be of the second category.
  \end{definition}
  \begin{theorem}
    \tt{The Baire category theorem}
    Suppose that $X$ is complete, and let $E \subset X$ be of the first
    category. Then $\Int(E) = \emptyset$.
  \end{theorem}
  \begin{proof}
    It suffices to prove that exists $x_0 \in X$ and $\epsilon_0 > 0$ such
    that $B(x_0,\epsilon) \setminus E \neq \emptyset$. Since $E$ is of
    the first category, there exist closed subsets $F_1,F_2,\dots$ such that
    $E \subset \cup_{n \geq 1}{F_n}$ and $\Int(F_n) = \emptyset$ for each
    $n \geq 1$. We are going to construct sequences 
    $\{e_n\}_{n \geq 1} \subset (0,\infty)$ such that 
    $\epsilon_n < \frac{1}{n}$ and $\overline{B}(x_n,\epsilon_n) \subset
    B(x_{n-1},\epsilon_{n-1}) \setminus \cup_{k=1}^{n}{F_k}$. 
    
    Let $n \geq 1$ and assume that we already constructed the sequences
    for $1 \le k \le n-1$. From 
    $B(x_{n-1},\epsilon_{n-1}) \cap (\cup_{k=1}^{n-1}{F_n})$ it follows that 	
    $V:=B(x_{n-1},\epsilon_{n-1}) \setminus \cup_{k=1}^{n-1}{F_n} 
    \neq \emptyset$. From this, and since $V$ is open and 
    $\Int(F_n) = \emptyset$ we get that $V \setminus F_n \neq \emptyset$.
    Since $V \setminus F_n$ is nonempty and open we get that there exists
    $x_n \in X$ and $0 < \epsilon_n < \frac{1}{n}$ such that
    $\overline{B}(x_n,\epsilon_n) \subset V \setminus F_n = 
    B(x_{n-1},\epsilon_{n-1}) \setminus \cup_{k=1}^{n}{F_k}$.
    This completes the inductive construction.
    
    From Cantor's intersection lemma it now follows that
    $\cap_{n \geq 1}{\overline{B}(x_n,\epsilon_n)} = \{x\}$ for some
    $x \in X$. For every $n \geq 1$ we have 
    $x \in \overline{B}(x_n,\epsilon_n) \subset 
    B(x_{n-1},\epsilon_{n-1}) \setminus \cup_{k=1}^{n}{F_k}$. This
    shows that:
    \[
      x \in B(x_{0},\epsilon_{0}) \setminus \cup_{k=1}^{\infty}{F_k}
      \subset B(x_0,\epsilon_0) \setminus E,
    \]
    which completes the proof of the theorem.
  \end{proof}
  The following is an immediate corollary from Baire's theorem.
  \begin{corollary}
    Suppose that $X$ is complete. Then $X$ is of the second category as a
    subset of itself. Consequently, if $F_1,F_2,\dots$ are closed subsets 
    of $X$ with $X = \cup_{n \geq 1}{F_n}$, then $\Int(F_n) \neq \emptyset$ 
    for some $n \geq 1$.
  \end{corollary}
  Here's another useful corollary of Baire's theorem.
  \begin{corollary}
    Suppose that $X$ is complete. Let $U_1,U_2,\dots$ be open subsets of $X$.
    Suppose that $U_n$ is dense in $X$ for all $n \geq 1$. Then 
    $\cap_{n \geq 1}{U_n}$ is also dense in $X$.
  \end{corollary}
  \begin{proof}
    To be added
  \end{proof}
  \begin{definition}
    Let $Y$ be a topological space. A countable intersection of open
    subsets of $Y$ is called a $G_\delta$ set. A countable union of closed 		
    subsets of $Y$ is called an $F_\sigma$ set.
  \end{definition}
  \begin{definition}
    An irrational real number $x$ is said to be a Liouville number if
    for every integer $n \geq 1$ there exist integers $p$ and $q \geq 2$ 
    so that $\abs{x - \frac{p}{q}} < \frac{1}{q^n}$.
  \end{definition}
  \begin{example}
    The number $\sum_{k \geq 1}{\frac{1}{10^{k!}}}$ is called Liouville’s 
    constant. It is not difficult to show that it is a Liouville number.
  \end{example}
  \begin{proposition}
    Write $L$ for the set of Liouville numbers. Then $L$ is a dense
    $G_\delta$ subset of $\R$.
  \end{proposition}
  \begin{proof}
    For every $n \geq 1$ set
    \[
      V_n := \bigcup_{q=2}^{\infty} \bigcup_{p \in \Z}
      \left(\frac{p}{q} - \frac{1}{q^n},\frac{p}{q} + \frac{1}{q^n}\right).
    \]
    Note that $\Q \subset V_n$ which means $V_n$ is dense in $\R$. For each
    $r \in \Q$ denote $U_r = \R \setminus \{r\}$. It follows directly from
    the definition of Liouville numbers that:
    \[
      L = \left(\bigcap_{n=1}^{\infty}{V_n}\right) \cap
        \left(\bigcap_{r \in \Q}{U_r}\right)
    \]
    Now since that sets $\{V_n\}_{n=1}^{\infty}$ and $\{U_r\}_{r \in \Q}$
    are all open and dense, and since $\Q$ is countable, it follows from
    the previous corollary that $L$ is a dense $G_\delta$ subset of $\R$.
    This completes the proof.
  \end{proof}
  \begin{definition}
    A mapping $f \colon X \to X$ is called a contraction of $X$ if there 
    exists $c \in [0, 1)$ so that $d(f(x),f(y)) \le cd(x,y)$ for all 
    $x,y \in X$.
  \end{definition}
  \begin{theorem}
    \tt{The Banach fixed-point theorem}
    Suppose that $X$ is complete
    and let $f \colon X \to X$ be a contraction. Then $f$ has a unique fixed 
    point. That is, there exists a unique $x \in X$ so that $f(x) = x$.
  \end{theorem}
  \begin{proof}
    First we show that $f$ has a fixed point. Choose an arbirary $x_0 \in X$
    and define a sequence $\{x_n\}_{n \geq 0}$ by setting
    $x_n := f(x_{n-1})$ for $n \geq 1$. It is easy to show by induction
    that:
    \[
      d(x_{n+1},x_{n}) \le c^n d(x_1,x_0)
    \]
    Now we will show that $\{x_n\}_{n \geq 1}$ is a Cauchy sequence.
    Let $\epsilon > 0$. Choose $N \geq 1$ such that
    $c^N d(x_0,x_1) (1-c)^{-1} < \epsilon$. For $n \geq m \geq N$,
    \begin{align*}
      d(x_n,x_m) \le 
      \sum_{k=m}^{n-1}{d(x_{k},x_{k+1})} &\le 
      \sum_{k=m}^{n-1}{c^k d(x_{0},x_{1})} \\ &\le
      c^m d(x_0,x_1) \sum_{k=1}^{\infty}{c^k} =
      \frac{c^m d(x_0,x_1)}{1-c} < \epsilon
    \end{align*}
    which shows that $\{x_n\}_{n \geq 1}$ is Cauchy. Since $X$ is complete
    exists $x \in X$ such that $\{x_n\}_{n \geq 1} \to x$ as $n \to \infty$.
    Since $f$ is a contraction it is continuous. We get:
    \[
      x = \lim_{n\to\infty}{x_n} = \lim_{n \to \infty}{f(x_{n-1})} = f(x),
    \]
    which shows that $f$ has a fixed point.
    
    Next we show uniqueness. Suppose there were $y \in X$ another fixed
    point of $f$. Then
    \[
      d(x,y) = d(f(x),f(y)) \le cd(x,y)
    \]
    Thus we have $(1-c)d(x,y) \le 0$. This is only possible if $d(x,y)=0$
    thus $x=y$ which completes the proof.
  \end{proof}
  Notice that the proof of the theorem also gives an explicit way to 
  approximate the fixed point of $f$.

  The following is a simplified version of Picard's theorem regarding the 
  existence and uniqueness of solutions for ordinary differential equations. 	
  For $\epsilon > 0$ we set $I_\epsilon := [-\epsilon, \epsilon]$.
  \begin{theorem}
    \tt{Picard's theorem}
    Let $F \colon I_1 \times I_1 \to \R$ be continuous. Suppose that there 
    exists $K > 0$ so that $\abs{F(t,x) - F(t,y)} \le K\abs{x - y}$ for all 
    $t, x, y \in I_1$. Then there exists $\epsilon > 0$ for which there 
    exists a unique $f \colon I_{\epsilon} \to I_{1}$ so that,
    \begin{itemize}
      \item $f$ is differentiable on $I_{\epsilon}$;
      \item $f(0)=0$
      \item $f'(t) = F(t,f(t))$ for $t \in I_{\epsilon}$
    \end{itemize}
  \end{theorem}
  Is this the real life?? Is this just fantasy??	
  \newpage

  \section{Compactness}
  Let $X$ be a fixed topological space.
  \begin{definition}
    A class $\mathcal{U} := \{U_i\}_{i \in I}$ of open subsets of a
    $X$ is said to be an \emph{open cover of $X$} if 
    $X = \bigcup_{i \in I} U_i$. A subclass of $\mathcal{U}$ is said
    to be a subcover of $\mathcal{U}$ if it is in itself an open cover
    of $X$.
  \end{definition}
  \begin{definition}
    The space $X$ is said to be compact if every open cover of $X$
    has a finite subcover.
  \end{definition}
  \begin{definition}
    A subset $Y$ of $X$ is said to be compact if for every family of open
    sets $\{U_i\}_{i \in I}$ such that $Y \subset \bigcup_{i \in I} U_i$
    exists a finite index set $I_0 \subset I$ such that 
    $Y \subset \bigcup_{i \in I_0} U_i$.
  \end{definition}
  \begin{remark}
    It follows easily from the definition of the subspace topology that 
    a nonempty subset $Y$ of $X$ is compact if and only if $Y$ is a 
    compact space when equipped with the subspace topology.
  \end{remark}
  \begin{proposition}
    Suppose that $X$ is compact and let $F \subset X$ be closed. Then
    $F$ is compact.
  \end{proposition}
  \begin{proof}
    Let $\{U_i\}_{i \in I}$ be an open cover of $F$. Since $F$ is closed
    we know that $X \setminus F \cup \{U_i\}_{i \in I}$ is an open cover
    of $X$. Since $X$ is compact exists a finite index set $I_0 \subset I$
    such that $X \setminus F \cup \{U_i\}_{i \in I_0}$ is a finite
    open cover of $X$. It is clear that $F \subset \{U_i\}_{i \in I_0}$
    which completes the proof.
  \end{proof}
  \begin{proposition}
    Suppose $X$ is compact, let $Y$ be a topological space, and let
    $f \colon X \to Y$ be continuous. Then $f(X)$ is compact.
  \end{proposition}
  \begin{proof}
    Let $\{U_i\}_{i \in I}$ be an open cover of $f(X)$. Since $f$
    is continuous $\{f^{-1}(U_i)\}_{i \in I}$ is an open cover of $X$.
    Since $X$ is compact exists a finite index set $I_0 \subset I$
    such that $\{f^{-1}(U_i)\}_{i \in I_0}$ is an open cover of $X$.
    We now have:
    \[
      f(X) = f(\cup_{i \in I_0} f^{-1}(U_i)) = 
      \cup_{i \in I_0} f(f^{-1}(U_i)) = 
      \cup_{i \in I_0} U_i
    \]
    Which completes the proof.
  \end{proof}
  Here are some more equivalent forms of compactness that are ofter easier
  to apply.
  \begin{proposition}
    The space $X$ is compact if and only if for every class 
    $\{F_i\}_{i \in I}$ of closed subsets of $X$ with 
    $\cap_{i \in I}{F_i} = \emptyset$ there exists a finite 
    $I_0 \subset I$ with $\cap_{i \in I_0} {F_i} = \emptyset.$
  \end{proposition}
  \begin{proof}
    Assume $X$ is compact. Let $\{F_i\}_{i \in I}$ be a family of closed 
    subsets of $X$ with $\cap_{i \in I}{F_i} = \emptyset$ then 
    we have
    $\cap_{i \in I} {X \setminus F_i} = X$ which is a cover of $X$
    thus exists a finite $I_0 \subset I$ with 
    $\cap_{i \in I_0} {X \setminus F_i} = X$ being a finite subcover
    of $X$. This implies that $\cap_{i \in I_0} {F_i} = \emptyset$
    which completes the proof. The proof of the other direction is
    similar and thus omitted. 
  \end{proof}
  \begin{definition}
    Let $S$ be a nonempty set. A class of subsets $\{E_i\}_{i \in I}$ of 
    $S$ is said to have the \textbf{finite intersection property} if 
    $\cap_{i \in I_0}{E_i} \neq \emptyset$ for every finite 
    $I_0 \subset I$.
  \end{definition}
  \begin{proposition}
    The space $X$ is compact if and only if every class of closed
    subsets of $X$ with the finite intersection property has nonempty 
    intersection.
  \end{proposition}
  \begin{proposition}
    Let $B$ be an open base for $X$. Suppose that every open cover
    $\{B_i\}_{i \in I} \subset B$ of $X$ has a finite subcover. Then $X$ 
    is compact.
  \end{proposition}
  \begin{definition}
    A family $\mathcal{B}$ of closed subsets of $X$ is called a 
    \textbf{closed base} for $X$ if the collection
    \[
      \{X \setminus B \colon B \in \mathcal{B}\}
    \]
    is an open base for $X$. Similarly, a family $\mathcal{S}$ of
    closed subsets of $X$ is called a \textbf{closed subbase} for $X$ 
    if the collection $\{X \setminus S \colon S \in \mathcal{S}\}$ is 
    an open subbase for $X$.
  \end{definition}
  \begin{remark}
    Note that if $\mathcal{S}$ is a closed subbase for $X$ then the set 
    $\mathcal{B}$ of all finite unions of elements of $\mathcal{S}$ forms 
    a closed base for $X$. This is so since, by definition,
    the set of all finite intersections of an open subbase forms an open 
    base. We call $\mathcal{B}$ the closed base generated by 
    $\mathcal{S}$.
  \end{remark}
  \begin{proposition}
    Let $\mathcal{B}$ be a closed base for $X$. Suppose that for every 
    $\{B_i\}_{i \in I} \subset \mathcal{B}$ with the finite intersection 
    property we have $\cap_{i \in I}{B_i} = \emptyset$. Then $X$ is 
    compact.
  \end{proposition}
  In the following two theorems are let $X$ be a fixed topological space.
  \begin{theorem}
    \tt{The Alexander subbase theorem, first form}
    Let $\mathcal{S}$ be an open subbase for $X$. Suppose that every 
    open cover $\{S_i\}_{i \in I} \subset \mathcal{S}$ of $X$ has a 
    finite subcover. Then $X$ is compact.
  \end{theorem}
  \begin{theorem}
  \tt{The Alexander subbase theorem, second form}
  Let $\mathcal{S}$ be a closed subbase for $X$. Suppose that 
  $\cap_{i \in I}{S_i} = \emptyset$ for every 
  $\{S_i\}{i \in I} \subset \mathcal{S}$ with the finite intersection 
  property. Then $X$ is compact.
  \end{theorem}
  The proof of these theorems is concerned with Zorn's lemma and will
  be omitted for now.
  \begin{definition}
    Let $X$ be a metric space. We say that $A \subset X$ is 
    \textbf{bounded} if exists $r > 0$ and $x \in X$ such that
    $A \subset B(x,r)$.
  \end{definition}
  Note that it is easy to see that $A \subset X$ is bounded if and only
  if it has a finite diameter.
  \begin{lemma}
    Let $\mathcal{S}$ be an open subbase for a topological space $X$.
    If $Y \subset X$ is a subset of $X$ equiped with the subspace
    topology induced by $X$ then $\{S \cap Y \mid S \in \mathcal{S}\}$
    is an open subbase for $Y$.
  \end{lemma}
  \begin{proof}
    Let $U$ be a nonempty subset of $Y$ and let $y \in U$. There
    exists $W$ an open set in $X$ such that $W \cap Y = U$. Because
    $\mathcal{S}$ is a subbase for $X$ exists 
    $S_1,\dots,S_n \in \mathcal{S}$ such that 
    $y \in \cap_{i=1}^{n}{S_i} \subset W$ and thus because $y \in Y$:
    \[
      y \in \cap_{i=1}^{n}{S_i \cap Y} \subset W \cap Y = U
    \]
    Because $S_i \cap Y$ are all open in $Y$ we have that indeed
    $\{S \cap Y \mid S \in \mathcal{S}\}$ is an open subbase as wanted.
  \end{proof}
  \begin{theorem}
    \tt{Heine–Borel theorem in $\R$}
    Every closed and bounded set in $\R$ is compact.
  \end{theorem}
  \begin{proof}
    Let $A$ be a closed and bounded set in $\R$. Because $A$ is bounded
    we know that exist real numbers $a,b \in \R$ such that $a < b$ and
    also $A \subset [a,b]$. If we equip $[a,b]$ with the subspace
    topology induced on it by $\R$ it is not hard to see that $A$ is
    closed in $[a,b]$ and thus it suffices to verify that $[a,b]$
    is compact in $\R$. It's easy to check that the set:
    \[
      \{(-\infty, c) \mid c \in \R\} \cup 
      \{(d, \infty) \mid d \in \R\}
    \]
    Is an open subbase to $\R$. From the lemma we have that the set:
    \[
      S = \{[a, c) \mid a < c \le b\} \cup 
      \{(d, b] \mid a < d \le b\}
    \]
    Is an open subbase for $[a,b]$. Let $\mathcal{U} \subset S$ be an
    open cover of $[a,b]$, by Alexander's subbase theorem it suffices
    to show that $\mathcal{S}$ has a finite subcover. Since 
    $\mathcal{U} \subset \mathcal{S}$ there exist index sets $I,J$ such 
    that:
    \[
      \mathcal{U} = 
      \{[a,c_i) \mid i \in I\} \cup \{(d_j,b] \mid j \in J\}
    \]
    We have that $a \in [a,b]$ and $\mathcal{U}$ a cover of $[a,b]$
    which means that $I \neq \emptyset$. 
    Denote $s = \sup\{c_i\}_{i\in I}$,  if we have $s \le d_j$ for
    all $j \in J$ we have $s \notin \cup\mathcal{U}$ which is a 
    contradiction. Otherwise exists $j_0 \in J$ such that 
    $d_{j_0} < s$ and then by definition exists $i_0 \in I$ such that
    $d_{j_0} < c_{i_0} < s$ and then we have that 
    $\{[a,c_{i_0}), (d_{j_0},b]\}$ is a finite subcover of $[a,b]$ which
    completes the proof.
  \end{proof}
  \begin{theorem}
    \tt{Tychonoff’s theorem}
    Let $\{X_i\}_{i \in I}$ be a nonempty family of compact topological 
    spaces. Equip $\prod_{i \in I}{X_i}$ with the product topology. Then 
    $\prod_{i \in I}{X_i}$ is compact.
  \end{theorem}
  \begin{proof}
    Set:
    \[
      \mathcal{S} = 
      \left\{\prod_{i \in I}{F_i} \mid \exists i_0 \in I \st
      (\forall i \in I \setminus \{i_0\})(F_i = X_i) \text{ and }
      F_{i_0} \text{ is closed in } X_{i_0}
      \right\}
    \]
    This is the standard closed subbase for $\prod_{i \in I}{X_i}$.
    Let $\{S_j\}_{j \in J} \subset \mathcal{S}$ be with the finite
    intersection property. By Alexander's subbase theorem, second form, 
    it suffices to prove that $\cap_{j \in J}{S_j} \neq \emptyset$.
    For every $j \in J$ exists a family $\{F_{j,i}\}_{i \in I}$ so that
    $F_{j,i}$ is a closed of $X_i$ for each $i \in I$, and 
    $S_j = \prod_{i \in I}{F_{j,i}}$. Thus, for every $J_0 \subset J$
    \[
      (*) \quad \bigcap_{j \in J_0}{{S}_j} = 
      \set{\prod_{x \in I}{x_i} \in \prod_{i \in I}{X_i}}
      {x_i \in F_{j,i} \text{ for all $i \in I$ and $j \in J_0$}}
    \]
    From this, and since $\{S_j\}_{j \in J}$ has the finite intersection
    property, it follows that $\{F_{j,i}\}_{j \in J}$ has the finite
    intersection property for each $i \in I$. From this, and from 
    proposition $4.4$, and since the spaces $X_i$ are compact, we obtain
    that for each $i \in I$, there exists 
    $\tilde{x_i} \in \cap_{j \in J}{F_{j,i}}$. From $(*)$ it now follows
    that $\{\tilde{x_i}\}_{i \in I} \in \cap_{j \in J}{{S}_j}$, which
    completes the proof of the theorem.
  \end{proof}
  We can now prove the following classic result.
  \begin{theorem}
    \tt{Heine–Borel theorem}
    Let $d \geq 1$ be an integer, and equip $\R^d$ with its standard 
    Euclidean metric. Then every closed and bounded subset of $\R^d$ is
    compact.
  \end{theorem}
  First we need to prove a couple of lemmas.
  \begin{lemma}
    Let $\{X_i\}_{i \in I}$ be a nonempty family of topological spaces, 
    and equip $\prod_{i \in I}{X_i}$ with the product topology. Let $Y$ be 
    a nonempty subset of $\prod_{i \in I}{X_i}$. For each $i \in I$ let 
    $\pi_i$ be the coordinate projection from $\prod_{i \in I}{X_i}$ onto 
    $X_i$, and denote by $\pi_i\vert_Y$ the restriction of $\prod_i$ to $Y$.
    Then the subspace topology induced by $\prod_{i \in I}{X_i}$ on $Y$ is 
    equal to the weak topology generated by $\{\pi_i \vert_Y\}_{i \in I}$.
  \end{lemma}
  \begin{proof}
    By definition of the product topology, the collection
    \[
      \set{\pi_{i}^{-1}(U)}{\text{$i \in I$ and $U$ is open in $X_i$}}
    \]
    is an open subbase for the product space. By a previous lemma
    we have that
    \[
      \set{\pi_{i}^{-1}(U) \cap Y}
      {\text{$i \in I$ and $U$ is open in $X_i$}}
    \]
    is an open subbase for $Y$ with respect to the subspace topology.
    From this, and since $\pi_{i}^{-1}(E) \cap Y = \pi_{i}^{-1} \vert_Y(E)$
    for each $i \in I$ and $E \subset X_i$, and now by Remark $2.4$ we
    see that indeed the subspace topology induced by $\prod_{i \in I}{X_i}$ 
    on $Y$ is equal to the weak topology generated by 
    $\{\pi_i \vert_Y\}_{i \in I}$.
  \end{proof}
  \begin{lemma}
    Let $\{X_i\}_{i \in I}$ be a nonempty family of topological spaces, 
    and equip $\prod_{i \in I}{X_i}$ with the product topology. For each 
    $i \in I$ let $Y_i$ be a nonempty subset of $X_i$, and set 
    $Y := \prod_{i \in I}{Y_i}$. Let $\tau_1$ be subspace topology induced 
    by $\prod_{i \in I}{X_i}$ on $Y$. Let $\tau_2$ be the product topology 
    on $Y$, where each $Y_i$ is equipped with the subspace topology induced 
    by $X_i$. Then $\tau_1 = \tau_2$.
  \end{lemma}
  \begin{proof}
    For each $i \in I$ let $\pi_i$ be the coordinate projection from 
    $\prod_{i \in I}{X_i}$ onto $X_i$, and denote by $\pi_i\vert_Y$ the 
    restriction of $\prod_i$ to $Y$. From the previous lemma we have
    that $\tau_1$ is equal to the weak topology generated by 
    $\{\pi_i \vert_Y\}_{i \in I}$. 
    
    For each $i \in I$ let $\tilde{\pi}_i$ be the coordinate projection from 
    $Y$ onto $Y_i$. By the definition of the product topology, the collection
    \[
      \mathcal{S} := \set{\tilde{\pi}_{i}^{-1}(U)}
      {\text{$i \in I$ and $U$ is open in $Y_i$}}
    \]
    is an open subbase for $Y$ in respect to $\tau_2$. We see that:
    \[
      \mathcal{S} := \set{({\pi \vert_Y}_{i})^{-1}(V \cap Y_i)}
      {\text{$i \in I$ and $V$ is open in $X_i$}}
    \]
    Now since $({\pi \vert_Y}_{i})^{-1}(Y_i) = Y$ for all $i \in I$
    \[
      \mathcal{S} := \set{({\pi \vert_Y}_{i})^{-1}(V)}
      {\text{$i \in I$ and $V$ is open in $X_i$}}
    \]
    From this, and since $\mathcal{S}$ is an open subbase for $Y$ with 
    respect to $\tau_2$, it follows that $\tau_2$ is also equal to the weak 
    topology generated by $\{\pi_i \vert_Y\}_{i \in I}$. This completes the
    proof of the lemma.
  \end{proof}
  \begin{proof}
    
  \end{proof}

  \newpage

  \begin{definition}
    A topological space $X$ is called \textbf{locally compact} if for
    any $x \in X$ exists a neighbourhood $U \subset X$ of $x$ so that
    $\overline{U}$ is compact.
  \end{definition}
  As an immediate result we get that for each $d \geq 1$ that $\R^d$
  is locally compact.
  \begin{definition}
    The metric space $X$ is said to be \textbf{sequentially compact} 
    if every sequence in $X$ has a convergent subsequence.
  \end{definition}
  \begin{definition}
    The metric space $X$ is said to have the 
    \textbf{Bolzano–Weierstrass property} if every infinite subset of 
    $X$ has a limit point in $X$.
  \end{definition}
  It is important to note that in metric spaces, sequential compactness and 
  the Bolzano Weierstrass property are both
  equivalent to compactness. We will omit the proofs because there's not
  enough time. Here are some more definitions without
  motivation, and a lemma without a proof.
  \begin{definition}
    Let $\{U_i\}_{i \in I}$ be an open cover of $X$. A real number 
    $\delta > 0$ is said to be a \textbf{Lebesgue number} 
    for $\{U_i\}_{i \in I}$
    if for all nonempty $A \subset X$ with $\text{diam}(A) < \delta$
    there exists $i \in I$ so that $A \subset U_i$.
  \end{definition}
  \begin{lemma}
    (Lebesgue’s covering lemma). Suppose that $X$ is sequentially 
    compact. Let $\{U_i\}_{i \in I}$ be an open cover of $X$. 
    Then $\{U_i\}_{i \in I}$ has a Lebesgue number.
  \end{lemma}
  \begin{definition}
    Let $\epsilon > 0$ be given. A nonempty subset $A$ of $X$ is 
    said to be an \textbf{$\epsilon$-net} if $A$ is finite and 
    $X = \cup_{a \in A}{B(a, \epsilon)}$.
  \end{definition}
  \begin{definition}
    We say that $X$ is \textbf{totally bounded} if it has an 
    $\epsilon$-net for all $\epsilon > 0$.
  \end{definition}
  It is clear that a totally bounded space is also bounded. 
  Using Lebesgue's lemma we can also prove the following proposition:
  \begin{proposition}
    Suppose that a metric space $X$ is compact. Let $(Y, d_Y)$ be a 
    metric space, and let $f \colon X \to Y$ be continuous. Then $f$ is 
    uniformly continuous.	
  \end{proposition}
  \begin{proof}
    Let $\epsilon > 0$. Since $f$ is continuous the set 
    $f^{-1}(B(f(x),\epsilon/2))$ is open for any $x \in X$ and thus
    the set:
    \[
      \mathcal{U} := \{f^{-1}(B(f(x),\epsilon/2))\}_{x \in X}
    \]
    Is an open cover for $X$. Because $X$ is a compact metric space
    it is also sequencially compact, and thus from Lebesgue's lemma
    we have that exists a Lebesgue number $\rho > 0$ for $\mathcal{U}$.
    Now let $x_1,x_2 \in X$ such that $d(x_1,x_2) < \rho$, by definition
    exists $x \in X$ such that $x_1,x_2 \in f^{-1}(B(f(x),\epsilon/2))$,
    thus:
    \[
      d_Y(f(x_1),f(x_2)) \le d_Y(f(x_1),f(x)) + d_Y(f(x),f(x_2)) <
      \frac{\epsilon}{2} + \frac{\epsilon}{2} = \epsilon
    \]
  \end{proof}
  There is also a connection between compactness and total boundness as
  we see in the following proposition.
  \begin{proposition}
    The metric space $X$ is compact if and only if it is complete
    and totally bounded.
  \end{proposition}
  The proof will be omitted for now.
  \begin{corollary}
    Suppose that $X$ is complete and let $A$ be a nonempty closed subset
    of $X$. Then $A$ is compact if and only if it is totally bounded.
  \end{corollary}

  \newpage

  \section{The Arzelà–Ascoli theorem}
  First we define a new structure. Let $K$ be a field and $A$ a vector 
  space. Let $|\cdot| \colon A \times A \to A$ be a binary operation.
  Then $A$ is called an \textbf{algebra} if for each $x,y,z \in V$ 
  the following identites hold:
  \begin{itemize}
    \item Left distributiviy: $(x + y) \cdot z = x \cdot z + y \cdot z$.
    \item Right distributiviy: $z \cdot (x + y) = z \cdot x + z \cdot y$.
    \item Compatibility with scalars: 
    $(ax) \cdot (by) = (ab) (x \cdot y)$.
  \end{itemize}
  These identites actually just imply that the operation is bilinear.
  An algebra over $K$ is sometimes called a $K$-algebra and $K$ is
  called the base field of $A$. Notice that we didn't require the operation
  to be associative or commutative, although some authors use the term
  ``algebra'' to refer to an associative algebra.
  \begin{definition}
    Given $K$-algebras $A$, $B$ then a homomorphism of $K$-algebras
    is a $K$-linear map $f \colon A \to B$ such that $f(xy)=f(x)f(y)$
    for all $x,y \in A$. If $A$ and $B$ are unital then the morphism
    $f(1_A) = 1_B$ is called the unital homomorphism. The space
    of all $K$-algebra homomorphisms between $A$ and $B$ is usually 
    written as ${\mathrm{Hom}}_{K\text{-alg}}(A,B)$. A $K$-algebra
    isomorphism is a bijective $K$-algebra homomorphism.
  \end{definition}
  A subalgebra of a $K$-algebra $A$ is a linear subspace of $A$ such
  that all products and sums of the subspace are themselves elements
  of the subspace. For examples $\R$ with complex addition and 
  multiplication as a subspace of the $\R$-algebra $\C$ is an example
  of a subalgebra.

  Similarly to rings, algebras also have a concept of ideals. A left ideal
  $L$ of a $K$-algebra $A$, is a linear subspace of $A$ such that for
  any $x,y \in L$, $c \in K$, $z \in A$ the following three identities
  are satisfied:
  \begin{itemize}
    \item $L$ is closed under addition: $x + y \in L$
    \item $L$ is closed under scalar multiplication: $cx \in L$
    \item $L$ is closed under vector multiplication from the left
    by arbitrary elements: $z \cdot x \in L$
  \end{itemize}
  We can similarly define a right ideal. An ideal that is both a left and
  a right ideal is called a two-sided ideal or simply an ideal. Notice
  that every ideal is a subalgebra and that in a commutative algebra
  any ideal is a two-sided ideal. Also notice that in contrast to an
  ideal of rings, here we also have a the requirement for closure under
  scalar multiplication and not just being a subgroup of addition.
  If the algebra is unital then the third requirement implies the second
  one.

  You can also talk about extension of scalars but I don't know what
  that is yet.

  Let $(X,d)$ be a fixed compact metric space. Denote $C(X)$ the algebra
  of all continuous functions $f \colon X \to \R$ and $C_b(X)$ the
  subalgebra of all the boundded functions in $C(X)$. Because $X$
  is compact we know that the image $f(X)$ of any $f \in C(X)$ is compact
  and in particular bounded and thus $C_b(X) = C(X)$. This means we can set 
  the norm $|\cdot|_\infty$ on $C(X)$. We can thus consider $C(X)$ as
  a metric space with the metric induced on it by $|\cdot|_\infty$.
  We will soon establish a useful characterisation of the compact sets
  in $C(X)$.
  \begin{definition}
    A subset $F \subset C(X)$ is called \textbf{equicontinuous} if
    for any $\varepsilon > 0$ exists a $\delta > 0$ such that
    $\abs{f(x) - f(y)} < \varepsilon$ for any $f \in F$ and $x,y \in X$ with
    $d(x,y) < \delta$.
  \end{definition}
  \begin{theorem}
    \tt{Arzelà–Ascoli theorem} Let $F$ be a nonempty closed subset of 
    $C(X)$. Then $F$ is compact if and only if it is bounded and 
    equicontinuous.
  \end{theorem}
  \begin{remark}
    It is easy to see that $F$ is bounded if and only if there exists 
    $M > 1$ so that $|f (x)| \le M$ for all $f \in F$ and $x \in X$.
  \end{remark}
   
  \newpage

  \section{Seperation}
  Let $X$ be a fixed topological space.
  \begin{definition}
    We say that $X$ is a $T_1$\textbf{-space} if and only if for every
    $x_1, x_2 \in X$ exist neighbourhoods $U_1$ of $x_1$ and $U_2$ of
    $x_2$ such that $x_1 \notin U_2$ and $x_2 \notin U_1$.
  \end{definition}
  We can also verify that if $X$ is a $T_1$-space then every topological
  subspace of $X$ is also a $T_1$-space.
  \begin{proposition}
    The space $X$ is a $T_1$\textbf{-space} if and only if $\{x\}$ is 
    closed in $X$ for every $x \in X$.
  \end{proposition}
  \begin{proof}
    Suppose that $X$ is a $T_1$-space. Let $x \in X$. For every 
    $y \in X \setminus \{x\}$ exists a neighbourhood 
    $U_y \subset X \setminus \{x\}$ the union of which gives 
    $X \setminus \{x\}$ and then $\{x\}$ is closed as wanted. Now
    assume that $\{x\}$ is closed for every $x \in X$. For two points
    $x_1,x_2 \in X$ the sets $\{x_1\},\{x_2\}$ are closed and thus
    we have $U_1 := X \setminus \{x_1\}$ neighbourhood of $x_1$ and
    $U_2 := X \setminus \{x_2\}$ neighbourhood of $x_2$ such that
    $x_1 \notin U_2$ and $x_2 \notin U_1$.
  \end{proof}
  \begin{definition}
    We say that $X$ is a \textbf{Hausdorff space} if for all distinct 
    $x_1, x_2 \in X$ there exist open sets $U1, U2 \subset X$ with 
    $x_1 \in U_1, x_2 \in U_2$ and $U_1 \cap U_2 = \emptyset$.
  \end{definition}
  We can verify that every Hausdorff space is a $T_1$-space and that if
  $X$ if a Hausdorff is a topological space then every subspace of $X$
  is also a Hausdorff space.
  \begin{proposition}
    Let $\{X_i\}_{i \in I}$ be a nonempty family of Hausdorff spaces. 
    Then the product space $\prod_{i \in I}{X_i}$ is also a Hausdorff 
    space.
  \end{proposition}
  \begin{proof}
    Let $\{x_i\}_{i \in I}, \{y_i\}_{i \in I}$ be distinct points
    in $\prod_{i \in I}{X_i}$. Therefore exists $i_0 \in I$ such that
    $x_{i_0} \neq y_{i_0}$. Because $X_{i_0}$ is a Hausdorff space 
    there exist open sets $U_x, U_y \subset X_{i_0}$ with 
    $x \in U_x, y \in U_y$ and $U_x \cap U_y = \emptyset$. We know that
    the projection $\pi_{i_0} \colon \prod_{i \in I}{X_i} \to X_{i_0}$
    is continuous and thus $\pi_{i_0}^{-1}(U_x)$ and $\pi_{i_0}^{-1}(U_y)$
    are two open and disjoint sets of $\prod_{i \in I}{X_i}$ such
    that $\{x_i\}_{i \in I} \in \pi_{i_0}^{-1}(U_x)$ and 
    $\{y_i\}_{i \in I} \in \pi_{i_0}^{-1}(U_y)$ as wanted. This shows
    that $\prod_{i \in I}{X_i}$ is a Hausdorff space which completes
    the proof.
  \end{proof}
  The following proposition is one of the most important properties of
  Hausdorff spaces.
  \begin{proposition}
    Suppose that $X$ is a Hausdorff space. Let $K$ be a compact subset
    of $X$ with $K \neq X$, and let $x \in X \setminus K$. Then there 	
    exist open sets $U, V \subset X$ so that $x \in U , K \subset V$ 
    and $U \cap V = \emptyset$.
  \end{proposition}
  \begin{proof}
    First we may suppose that $K \neq \emptyset$ otherwise we could
    choose $U = X$ and $V = \emptyset$. Since $X$ is Hausdorff for
    every $y \in K$ exist $U_y, V_y \subset X$ disjoint open sets 
    such that $x \in U_y$ and $y \in V_y$. We have 
    $K \subset \cup_{y \in Y}{V_y}$ but since $K$ is compact exist
    $y_1,\dots,y_n$ such that $K \subset \cup_{i=1}^{n}{V_{y_i}}$.
    We now define:
    \begin{align*}
      &V := \cup_{i=1}^{n}{V_{y_i}} \\
      &U := \cap_{i=1}^{n}{U_{y_i}}
    \end{align*}
    It is clear that both sets are open, and that $x \in U$ and
    $K \subset V$ and for every $i \in [n]$ we also see that:
    \[
      Y_{y_i} \cap U \subset V_{y_i} \cap U_{y_i} = \emptyset
    \]
    Which means that $U \cap V = \emptyset$ as wanted which completes
    the proof.
  \end{proof}
  \begin{corollary}
    Suppose that $X$ is a Hausdorff space. Then every compact subset
    of $X$ is closed.
  \end{corollary}
  \begin{proof}
    Let $K \subset X$ be compact. We may clearly assume that 
    $K \neq X$. Given $x \in X \subset K$, it follows from the previous
    proposition that there exists a neighbourhood $U$ of $x$ which is 
    contained in $X \setminus K$. This shows that $X \setminus K$ is a 
    union of open sets, and so it is itself open. Thus $K$ is 
    closed, which completes the proof. 
  \end{proof}
  One particularly useful result of this corollary is the following
  proposition:
  \begin{proposition}
    Suppose that $X$ is a Hausdorff space, let $Y$ be a compact 
    topological space, and let $f \colon Y \to X$ be a continuous 
    bijection. Then $f$ is a homeomorphism.
  \end{proposition}
  \begin{proof}
    All that's left to show is that $f$ is an open map. Let $U \subset Y$
    be open. It follows that $Y \setminus U$ is closed in a compact space
    and thus compact. Since $f$ is continuous $f(Y \setminus U)$ is
    compact. From the previous corollary $f(Y \setminus U)$ is closed.
    Since $f$ is a bijection we also have 
    $f(Y \setminus U) = X \setminus f(U)$. This implies that $U$ is open,
    so $f$ is an open map and the proof is complete.
  \end{proof}

  \newpage

  \section{Completely regular spaces and normal spaces.}
  \begin{definition}
    We say that $C_b(X)$ separates points if for every distinct 
    $x, y \in X$ there exists $f \in C_b(X)$ with $f(x) \neq f(y)$.
  \end{definition}






\end{document}
