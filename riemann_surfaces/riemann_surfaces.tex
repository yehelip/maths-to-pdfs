\documentclass[11pt,a4paper]{article}

\def\nyear {2025}
\def\nterm {Winter}
\def\nlecturer {}
\def\ncourse {Riemann Surfaces}

\usepackage{blochsphere}
\makeatletter

% packages
\usepackage{amssymb,amsfonts,amsmath,calc,tikz,pgfplots,geometry,mathtools}
\usepackage{color}   % May be necessary if you want to color links
\usepackage[hidelinks]{hyperref}
\usepackage{forest}
\usepackage{commath}
\usepackage{amsthm}
\usepackage{fancyhdr}
\usepackage{bm}
\usepackage{witharrows}
\usepackage{bookmark}
\usepackage{tikz-cd}
\usepackage{bbm}
\usepackage{textcomp}
\usepackage{gensymb}
\usepackage{cleveref}

% tikz libraries
\usetikzlibrary{positioning}
\usetikzlibrary{matrix}
\usetikzlibrary{arrows}
\usetikzlibrary{arrows.meta}
\usetikzlibrary{decorations.markings}

% Page style setup
\pagestyle{fancy}
\geometry{margin=1in}
\pgfplotsset{compat=1.18}
\setlength{\headheight}{14.6pt}
\addtolength{\topmargin}{-1.6pt}
\hypersetup{
    colorlinks=false,
    linktoc=section,
    linkcolor=black,
}

%% maketitle setup
\ifx \nauthor\undefined
  \def\nauthor{yehelip}
\else
\fi

\ifx \ncoursehead \undefined
\def\ncoursehead{\ncourse}
\fi

\lhead{\emph{\nouppercase{\leftmark}}}
\ifx \nextra \undefined
  \rhead{
    \ifnum\thepage=1
    \else
      \ncoursehead
    \fi}
\else
  \rhead{
    \ifnum\thepage=1
    \else
      \ncoursehead \ (\nextra)
    \fi}
\fi

\let\@real@maketitle\maketitle
\renewcommand{\maketitle}{\@real@maketitle\begin{center}
\begin{minipage}[c]{0.9\textwidth}\centering\footnotesize
These notes are not endorsed by the lecturers.
I have revised them outside lectures to incorporate supplementary explanations,
clarifications, and material for fun.
While I have strived for accuracy, any errors or misinterpretations 
are most likely mine.
\end{minipage}\end{center}}

% theorem environments
\theoremstyle{definition}
\newtheorem{definition}{Definition}[section]
\newtheorem{remark}{Remark}[section]
\newtheorem{example}{Example}[section]
\newtheorem{exercise}{Exercise}[section]
\newtheorem{paradox}{Paradox}[section]
\newtheorem*{solution}{Solution}
\theoremstyle{plain}
\newtheorem{theorem}{Theorem}[section]
\newtheorem{proposition}[theorem]{Proposition}
\newtheorem{lemma}[theorem]{Lemma}
\newtheorem{corollary}[theorem]{Corollary}

% tikz customization
\pgfarrowsdeclarecombine{twolatex'}{twolatex'}{latex'}{latex'}{latex'}{latex'}
\tikzset{->/.style = {decoration={markings,
                                  mark=at position 1
                                  with {\arrow[scale=2]{latex'}}},
                      postaction={decorate}}}
\tikzset{<-/.style = {decoration={markings,
                                  mark=at position 0 with {\arrowreversed[scale=2]{latex'}}},
                      postaction={decorate}}}
\tikzset{<->/.style = {decoration={markings,
                                   mark=at position 0 with {\arrowreversed[scale=2]{latex'}},
                                   mark=at position 1 with {\arrow[scale=2]{latex'}}},
                       postaction={decorate}}}
\tikzset{->-/.style = {decoration={markings,
                                   mark=at position #1 with {\arrow[scale=2]{latex'}}},
                       postaction={decorate}}}
\tikzset{-<-/.style = {decoration={markings,
                                   mark=at position #1 with {\arrowreversed[scale=2]{latex'}}},
                       postaction={decorate}}}
\tikzset{->>/.style = {decoration={markings,
                                  mark=at position 1 with {\arrow[scale=2]{latex'}}},
                      postaction={decorate}}}
\tikzset{<<-/.style = {decoration={markings,
                                  mark=at position 0 with {\arrowreversed[scale=2]{twolatex'}}},
                      postaction={decorate}}}
\tikzset{<<->>/.style = {decoration={markings,
                                   mark=at position 0 with {\arrowreversed[scale=2]{twolatex'}},
                                   mark=at position 1 with {\arrow[scale=2]{twolatex'}}},
                       postaction={decorate}}}
\tikzset{->>-/.style = {decoration={markings,
                                   mark=at position #1 with {\arrow[scale=2]{twolatex'}}},
                       postaction={decorate}}}
\tikzset{-<<-/.style = {decoration={markings,
                                   mark=at position #1 with {\arrowreversed[scale=2]{twolatex'}}},
                       postaction={decorate}}}

\pgfarrowsdeclare{biggertip}{biggertip}{%
  \setlength{\arrowsize}{1pt}
  \addtolength{\arrowsize}{.1\pgflinewidth}
  \pgfarrowsrightextend{0}
  \pgfarrowsleftextend{-5\arrowsize}
}{%
  \setlength{\arrowsize}{1pt}
  \addtolength{\arrowsize}{.1\pgflinewidth}
  \pgfpathmoveto{\pgfpoint{-5\arrowsize}{4\arrowsize}}
  \pgfpathlineto{\pgfpointorigin}
  \pgfpathlineto{\pgfpoint{-5\arrowsize}{-4\arrowsize}}
  \pgfusepathqstroke
}
\tikzset{
	EdgeStyle/.style = {>=biggertip}
}

\tikzset{circ/.style = {fill, circle, inner sep = 0, minimum size = 3}}
\tikzset{scirc/.style = {fill, circle, inner sep = 0, minimum size = 1.5}}
\tikzset{mstate/.style={circle, draw, black, text=black, minimum width=0.7cm}}

\tikzset{eqpic/.style={baseline={([yshift=-.5ex]current bounding box.center)}}}

\definecolor{mblue}{rgb}{0.2, 0.3, 0.8}
\definecolor{morange}{rgb}{1, 0.5, 0}
\definecolor{mgreen}{rgb}{0, 0.4, 0.2}
\definecolor{mred}{rgb}{0.5, 0, 0}

% algebra
\DeclareMathOperator{\lcm}{lcm}
\DeclareMathOperator{\Out}{Out}
\DeclareMathOperator{\Aut}{Aut}
\DeclareMathOperator{\End}{End}
\DeclareMathOperator{\Inn}{Inn}
\DeclareMathOperator{\Mat}{Mat}
\DeclareMathOperator{\std}{std}
\DeclareMathOperator{\sgn}{sgn}
\DeclareMathOperator{\id}{id}
\newcommand{\idealin}{\triangleleft}
\newcommand{\ip}[2]{\langle #1, #2 \rangle}
\newcommand{\bigslant}[2]
{{\raisebox{.2em}{$#1$}\left/\raisebox{-.2em}{$#2$}\right.}}

% analysis
\newcommand{\dx}{\dif x}
\newcommand{\dt}{\dif t}
\newcommand{\du}{\dif u}
\newcommand{\dv}{\dif v}
\DeclareMathOperator{\im}{im}
\DeclareMathOperator{\cis}{cis}
\DeclareMathOperator{\Int}{Int}
\DeclareMathOperator{\diam}{diam}
\DeclareMathOperator{\supp}{supp}
\DeclareMathOperator{\Vol}{Vol} % Volume

% logic
\DeclareMathOperator{\MOD}{MOD}
\DeclareMathOperator{\Theory}{Theory}


% nice
\newcommand{\half}{\frac{1}{2}}
\newcommand{\pair}{\del}
\newcommand{\taking}[1]{\xrightarrow{#1}}
\newcommand{\inv}{^{-1}}
\newcommand{\ot}{\leftarrow}
\newcommand{\ninfty}{-\infty}
\newcommand{\floor}[1]{\left\lfloor #1 \right\rfloor}
\newcommand{\ceil}[1]{\left\lceil #1 \right\rceil}

% probability
\newcommand{\Prob}{\mathbf{P}}
\renewcommand{\vec}[1]{\boldsymbol{\mathbf{#1}}}
\DeclareMathOperator{\Bin}{Bin}
\DeclareMathOperator{\Geo}{Geo}
\DeclareMathOperator{\Poi}{Poi}
\DeclareMathOperator{\Exp}{Exp}
\DeclareMathOperator{\Var}{Var} % Variance
\DeclareMathOperator{\Cov}{Cov}

% special letters
\newcommand{\N}{\mathbb{N}}
\newcommand{\Z}{\mathbb{Z}}
\newcommand{\Q}{\mathbb{Q}}
\newcommand{\R}{\mathbb{R}}
\newcommand{\C}{\mathbb{C}}
\newcommand{\F}{\mathbb{F}}
\newcommand{\E}{\mathbb{E}}
\newcommand{\ps}{\mathcal{P}}
\newcommand{\M}{\mathcal{M}}
\renewcommand{\L}{\mathcal{L}}
\newcommand{\Omicron}{O}
\newcommand{\powerset}{\mathcal{P}}

% text
\newcommand{\st}{\text{ s.t. }}
\newcommand{\tand}{\quad \text{and} \quad}
\newcommand{\tor}{\quad \text{or} \quad}
\newcommand{\stand}{\text{ and }}
\newcommand{\stor}{\text{ or }}
\renewcommand{\tt}[1]{\textnormal{\textbf{(#1).}}} %tt=theorem title GET RID OF

% title format
\title{\textbf{\ncourse}}
\author{Based on lectures by \nlecturer \\\small Notes taken by \nauthor}
\date{\nterm\ \nyear}
\makeatother


\DeclareMathOperator{\Bih}{Bih}
\renewcommand{\H}{\mathbb H}
\DeclareMathOperator{\Homeo}{Homeo}
\DeclareMathOperator{\Diff}{Diff}
\DeclareMathOperator{\MCG}{MCG}
\DeclareMathOperator{\Teich}{Teich}
\DeclareMathOperator{\SL}{SL}


\begin{document}
\maketitle

% Insert cool image here

\newpage
\tableofcontents
\newpage

\section{Introduction}
\begin{definition}[Riemann surface]
    A Riemann surface is a $1$-dimensional complex manifold.
\end{definition}

\begin{definition}[Riemann surface]
    A Riemann surface is a topological space $X$ together with
    open subsets $\set{U_k}_{k \in I}$ of $X$ with
    $\cup_{k \in I} U_k = X$ together with maps $f_i \colon U_i \to \C$
    such that
    \begin{enumerate}
        \item[(1)] Each $f_i$ is a homeomorphism onto its image.
        \item[(2)] If $U_i \cap U_j \neq \emptyset$ then 
            $f_i \circ f_j^{-1} \colon 
            f_j(U_i \cap U_j) \to f_i(U_i \cap U_j)$ are \emph{biholomorphic}.
    \end{enumerate}
\end{definition}

\begin{remark}
    A function $f \colon \C \to \C$ is holomorphic at $p$ if 
    $f'(p) = \lim_{z \to p} \frac{f(z) - f(p)}{z - p}$ exists.
\end{remark}

\begin{definition}[Biholomorphism]
    A function $f \colon \C \to \C$ is called biholomorphic if it has an inverse
    and both $f$ and $f'$ are holomorphic.
\end{definition}

\begin{definition}[Atlas]
    The $\set{(U_i, f_i)}_{i \in I}$ are called an atlas of the Riemann surface.
\end{definition}

\begin{definition}[Chart]
    Each individual $(U_i, f_i)$ is called a chart of the Riemann surface.
\end{definition}

\begin{example}
    Let $U \subset \C$.
    Then $U$ can take an atlas with one chart which is the identity map.
\end{example}

\begin{example}[Riemann sphere]
    Let $X = \set{(z,t) \in \C \times \R \colon |z|^2 + t^2 = \R}$.
    We identify $\C$ with the $xy$ plane.
    Denote $N$ and $S$ the north and south poles of the sphere accordingly.
    We define $\pi_N \colon \C \to S$ such that $\pi_N$ sends each point
    $(z,t)$ on the sphere to its stereographic projection 
    from $N$ onto the plane (point $\xi$) as can be seen in the figure
    below:
    \begin{center}
      \includegraphics[scale=0.5]{RiemannSphere.png}
    \end{center}
    We can similarly define $\pi_S$ and verify that the images of the
    projections are $X \setminus \set{N}$ and $X \setminus \set{S}$
    accordingly.
    
    Now $X$ is a Riemann surface with an atlas consisting of 
    $\pi_S \colon X \setminus \set{S} \to \C$ and
    $\pi_N \colon X \setminus \set{N} \to \C$.
    We denote the Riemann sphere as $\hat{\C}$.
\end{example}

\begin{definition}[Biholomorphism of Riemann surfaces]
    Let $(X, (U_i, f_i))$, $(Y, (W_i,g_i))$ be two Riemann surfaces.
    A biholomorphism between them is a homeomorphism
    $X \xrightarrow{\phi} Y$ such that $g_i \circ \phi \circ f_i^{-1}$ 
    are biholomorphisms on their domains of definition.
\end{definition}

A main problem in Riemann surfaces was classifying certain types of 
Riemann surfaces up to biholomorphisms.

\begin{theorem}\tt{Riemann mapping theorem}
    Any two proper open simply connected subsets of $\C$ are biholomorphic.
\end{theorem}

A generalization of the Riemann mapping theorem is the uniformization theorem
proved by Kobe in 1907.

\begin{theorem}\tt{Uniformization theorem}
    Any simply connected Riemann surface is biholomorphic to one of the
    following:
    \begin{enumerate}
        \item[(1)] $\C$
        \item[(2)] $\hat{\C}$
        \item[(3)] $\H = \set{z \in \C \colon \mathrm{Im}(z) > 0}$
    \end{enumerate}
\end{theorem}

We will give a proof for this theorem in the end of the class.

A natural question that arises is what about non simply connected Riemann
surfaces?

\begin{theorem}\tt{Uniformization theorem, part II}
    Any connected Riemann surface is biholomorphic either to $\hat{\C}$
    or to a quotient of $\C$ or $\H$ by a properly discontinuous
    torsion-free subgroup of biholomorphisms.
\end{theorem}

\begin{remark}
    Biholomorphisms of $U = \C$ or $\H$ (or any subset of $\C$)
    forms a group under composition.
    We denote that group by $\Bih(U)$.
\end{remark}

\begin{definition}[Properly discontinuous group]
    A countable subgroup of $\Bih(U)$ is said to be properly discontinuous
    if for all compact $K \subseteq U$, the set 
    $\set{g \in G \colon gK \cap K \neq \emptyset}$ is finite.
\end{definition}

\begin{definition}[torsion-free action]
    $G \subseteq \Bih(U)$ is torsion-free if $gp = p$ for some $p \in U$
    implies $g$ is the identity.
\end{definition}

\begin{remark}
  Notice that multiplication in $gp$ is the group action of $g$ on the 
  set $U$. That us $gp = g(p)$.
\end{remark}

We can know define the quotient space $\bigslant{U}{G}$ where
$p \sim q$ if there exists $g \in G$ such that $gp = q$.

Introduce a topology on $\bigslant{U}{G}$ which is the coarsest topology
such that the canonical projections $U \to \bigslant{U}{G}$ are continuous.

Under the assumptions that $G$ is properly discontinuous and torsion-free,
$\bigslant{U}{G}$ is a Riemann surface with the following charts.
By assumptions on $G$, we can find for any $p \in U$ a neighbourhood $W$ of
$p \in U$ such that $\pi \colon U \to \bigslant{U}{G}$ is a homeomorphism
onto its image when restricted to $W$.

So, restrictions of $\pi$ to these neighbourhoods $W$ provide an atlas.

\begin{definition}[Free action]
  Let $G$
\end{definition}

\newpage

\section{Introduction to Teichmuller spaces}
Let $S$ be a topological space.
Then
\[
  \mathrm{Modulispace}(S) = 
  \set{\text{Riemann surfaces homeomorphic to $S$ up to biholomorphisms}}
\]
\begin{definition}[Teichmuller space]
  We consider pairs $(X,f)$ where $X$ is a Riemann space and
  $f \colonS \to X$ is a homeomorphism.
  Then
  \[
    \mathrm{Teich}(S) = 
    \bigslant{\set{(X,f) \colon f \colon S \to X}}{\sim}
  \]
  where $(X_1,f_1) \sim (X_2,f_2)$ if there exists a biholomorphism 
  $X_1 \taking{\phi} X_2$ such that $f_2 \circ f_1^{-1}$ is homotopic to
  $\phi$.
  In other words
  % Triangle
  commutes up to homotopy.
\end{definition}

\begin{remark}
  The pair $(X,f)$ is called a \emph{marking} for $S$.
\end{remark}

\begin{definition}[Homotopy]
  We say that two continuous functions $f$, $g$ from a topological space $X$
  to $Y$ are homotopic if there exists a continuous function 
  $H \colon X \times [0,1] \to Y$ such that $H(x,0) = f(x)$ and
  $H(x,1) = g(x)$ for all $x \in X$.
\end{definition}

\begin{definition}[Mapping class group]
  First we define:
  \begin{align*}
    &\Homeo^+(S) = \set{\text{Orientation preserving homeomorphisms
    $S \to S$}} \\
    &\Homeo^0 \triangleleft \Homeo^+(S) = \set{
      \text{the homeo. homotopic to the identity.}}
  \end{align*}
  And now we define
  \[
    \MCG(S) = \bigslant{\Homeo^+(S)}{\Homeo^0(S)} =
    \frac{\Diff^+(S)}{\Diff^0(S)}.
  \]
\end{definition}

We have that $\MCS$ acts on $\Teich(S)$ by $\varphi \in \Homeo^+(S),
[\varphi] \in \MCG(S)$
as such
\[
  [\varphi] [(X,f)] = [(X,f \circ e^{-1})].
\]

Now $\M(S) = \bigslant{\Teich(S)}{\MCG(S)}$.
Our goal is to determine $\M(S)$ and $\MCG(S)$ when $S$ is a torus.

\begin{theorem}
  Assume $S$ is a torus $T \cong \bigslant{\R^2}{\Z^2}$.
  Then $\MCG(S) = \SL_2(\Z)$ acting linearly on the torus.
  $\M(S)$ can be identified with $\bigslant{\H}{\SL_2(\Z)}$ where the
  action is by Mobius transformations.
\end{theorem}
\begin{proof}
  Any Riemann surfaces homeomorphic to $S$ has to form 
  $\bigslant{\C}{\Lambda}$ where $\Lambda \subseteq \Bih(\C)$ is
  properly disc, free, and $\Lambda\ip{z \to z _ \tau_1, z \to z + \tau_2}$
  where $\tau_1,\tau_2 \notin \R$.
  We have to determine when different $\bigslant{\C}{\Lambda}$ are
  biholomorphic.
  We can also write $\biglsnat{\C}{\Lambda}$ where 
  $\Lambda = \ip{\tau_1}{\tau_2} \subseteq \C$.
  If $\Lambda_1 = \ip{\tau_1}{\tau_2}$, $\Lambda_2 = \ip{c \tau_1}{c \tau_2}$
  for $c \in \C^&$.
  Then $\bigslant{\C}{\Lambda_1} \to \bigslant{\C}{\Lambda_2}$
  is given by $[z] \to [cz]$.

  For any $\tau_1, \tau_2 \in \C$ there exists $c \in \C^{\times}$
  such that (up to change in order)

  % ADD STUFFFF

  This tells us that any Riemann torus is biholomorphic to 
  $\bigslant{\C}{\ip{XX}{XX}}$ where $\tau \in \H$.

  So we have a surjection $\H \to \M(S)$.
\end{proof}

\begin{theorem}
  $\bigslant{\C}{\ip{1}{\tau_1}}$ is biholomorphic to 
  $\bigslant{\C}{\ip{1}{\tau_2}}$ iff exists $A \in \SL_2(\Z)$
  such that writing $\tau_1, \tau_2$ as elements of $\R^2$, we have
  $A\tau_1 = \tau_2$.
\end{theorem}
\begin{proof}
  Suppose there exists a biholomorphism 
  $f \colon \bigslant{\C}{\ip{1}{\tau_1}} \to \bigslant{\C}{\ip{1}{\tau_2}}$.
  Let $\bar f \colon \C \to \C$ be a lift of $f$.
  This means $f(g + x) - f(x) \in \ip{1,\tau_1}$ whenever}
  $g \in \ip{1}{\tau_1}$.
  \begin{align*}
    &f \colon \bigslant{\C}{\ip{1}{\tau_1}} \to \bigslant{\C}{\ip{1}{\tau_2}}
    bih. \\
    &\bar f \colon \C \to \C lift.
  \end{align*}
  By post composing with a biholomorphism of $\C$ we can assume $\bar f(0) = 0$.
  We know that $\bar f(\tau_2)$ and $\bar f(1)$ are equivalent mod
  $\ip{1}{\tau_1}$.
\end{proof}

%%%% LECTURE 4

\begin{remark}
  Let $S$ be a Riemann surface.
  Recall that $\M(S)$ is the moduli space of $S$, which is the space of Riemann
  surfaces homeo to $S$ up to biholomorphism.
\end{remark}
Recall that we define
\[
  \MCG(S) = \bigslant{\Homeo^+(S)}{\Homeo^0(S)}
\]
And the Teichmuller space of $S$
\[
  \mathrm{Teich}(S) = 
  \bigslant{\set{(X,f) \colon f \colon S \to X}}{\sim}
\]
where $(X_1,f_1) \sim (X_2,f_2)$ if there exists a biholomorphism 
$X_1 \taking{\phi} X_2$ such that $f_2 \circ f_1^{-1}$ is homotopic to $\phi$.

Our goal today is to show that for $T = \text{torus} = \bigslant{\R^2}{\Z^2}$.
\begin{itemize}
  \item $\MCG(S) \cong \SL_2(\Z)$.
  \item $\M(S) \cong \bigslant{\H}{\SL_2(\Z)}$.
  \item $\Teich(S) \cong \H$.
\end{itemize}

Notice that because of the relation
\[
  \bigslant{\Teich(S)}{\MCG(S)} = \M(S)
\]
we only need to prove two of these propositions.

First let's prove that $\M(S) \cong \bigslant{\H}{\SL_2(\Z)}$.
Any Riemann surface is homeomorphic to $T$ (a complex torus) is biholomorphic
to $\bigslant{\C}{\ip{\tau_1}{\tau_2}}$.

Recall that if $c \in \C^{\times}$ then $\bigslant{\C}{\ip{\tau_1}{\tau_2}}$
is biholomorphic to $\bigslant{\C}{\ip{c \tau_1}{c \tau_2}}$ via the map
\[
  f(z + \ip{\tau_1}{\tau_2}) = cz + \ip{c \tau_1}{c \tau_2}.
\]
Up to changing the order of $\tau_1$, $\tau_2$ we can find $c \in \C{\times}$
such that $c \tau_1 = 1$ and $c_2 \in \H$.

Our goal now is to determine when there exists a biholomorphism
\[
  \bigslant{\C}{\ip{1}{\tau_1}} \mapsto
  \bigslant{\C}{\ip{1}{\tau_2}}, \quad \tau_1,\tau_2 \in \H
\]

Denote $\Lambda_{\tau} = \ip{1}{\tau}$, when $\tau \in \H$.

\begin{theorem}
There exists a biholomorphism
\[
  \bigslant{\C}{\ip{1}{\tau_1}} \mapsto
  \bigslant{\C}{\ip{1}{\tau_2}}, \quad \tau_1,\tau_2 \in \H
\]
iff there exists 
\[
  \begin{pmatrix}
    a & b \\
    c & d
  \end{pmatrix} \in \SL_2(\Z)
\]
with $\tau_2= \frac{a \tau_1 + b}{c \tau_2 + d}$.
This will imply that \item $\M(S) \cong \bigslant{\H}{\SL_2(\Z)}$.
\end{theorem}
\begin{proof}
  Suppose there exists a biholomorphism 
  \[
    f \colon \bigslant{\C}{\ip{1}{\tau_1}} \to
    \bigslant{\C}{\ip{1}{\tau_2}}.
  \]
  Lift it up to $\bar f \colon \C \to \C$.
  Then $\bar f(g + x) - \bar f(x) \in \Lambda_{\tau_1}$
  whenever $x \in \C$, $g \in \Lambda_{\tau_2}$.
  We can replace $\bar f$ with $\bar f - \bar f(0)$.
  We can assume that $\bar f(0) = 0$ (and is still a lift off).
  \begin{remark}
    Since $\bar f$ is a lift off of $f$ we have 
    $\bar f(\Lambda_{\tau_2}) = \Lambda_{\tau_1}$.
  \end{remark}
  We know that $\bar f$ has form $\bar f(z) = a z + b$ such that
  $a \in \C^{\times}$, $b \in \C$.
  As $\bar f(0) = 0$ we have $\bar f(z) = a z$.
  Also, $\set{0,1,\tau_2} \subseteq \Lambda_{\tau_2}$ so 
  $0 = \bar f(0), \bar f(1) = a, \bar f(\tau_2) = a \tau_2$ are in
  $\Lambda_{\tau_1}$ so we can write $a = \bar f(1) = $
  %% MOR$E STUFF?
  We now have
  \[
    \tau_2 =
    \begin{pmatrix}
      a & b \\
      c & d
    \end{pmatrix}
  \]
  and we still need to show it is an element of $\SL_2(\Z)$.
  %% THE PART
\end{proof}


\newpage

\section{Isotopy and Homotopy}
Let $X$ be a metric space and $F_0, F_1 \colon X \to X$ homeomorphisms.
Then we say $F_0$, $F_1$ are homotopic if there exists a family $F_t$
for $t \in [0,1]$ of continuous maps such that $t \mapstp F_t(x)$ is
continuous for all $x \in X$ (we don't require $F_0$, $F_1$ to be 
homeomorphisms).

The maps $F_0$, $F_1$ are said to be isotopic if $F_t$ are required to
be homeomorphisms.

For a surface $S$, we defined $\MCS(S)$

\begin{theorem}[Baire, Epstein]
  If $S$ is a finite type surface (e.g.\ closed surface of XXX)
  then two homeomorphisms $F_0, F_1 \colon X \to X$ are homotopic
  iff they are isotopic.
\end{theorem}

Last time we have shown that if $T = \R^2 / \Z^2$ is a torus,
then $\MCG(S) = \SL_2(\Z)$.
For any $A \in \SL_2(\Z)$ we obtained a homeomorphism
\[
  \fullfunction{\psi_A}{T}{T}{[x]}{[Ax]}
\]
We showed that any orientation preserving homeomorphism $\phi \colon T \to T$
is homotopic to $\psi_A$ for some $A \in \SL_2(\Z)$.

This gives a map
\[
  \fullfunction{\Phi}{\SL_2(\Z)}{\MCG(T)}{[A]}{[\psi_A]}
\]
which we know is surjective.
Why is $\Phi$ injective?


We want to show that for $A \in \SL_2(\Z)$ and $A \neq I$ that $\psi_A$
is not homotopic to the identity map.

We will see this by showing that $\psi_A$ acts nontrivially on the fundamental
group $\Pi_1(T)$.

\begin{definition}[Loop]
  A loop based at $p$ is a continuous function $\gamma \colon [0,1] \to X$ with 
  $f(0) = f(1) = 0$.
\end{definition}

\begin{definition}[Homotopy]
  We say that two continuous functions $f$, $g$ from a topological space $X$
  to $Y$ are homotopic if there exists a continuous function 
  $H \colon X \times [0,1] \to Y$ such that $H(x,0) = f(x)$ and
  $H(x,1) = g(x)$ for all $x \in X$.
\end{definition}

\begin{definition}[Fundamental group]
  The fundamental group of a topological space $X$ is the space of all loops
  in $X$ based at a point $p \in X$ up to homotopy.
\end{definition}
 
A homeomorphism $\psi \colon X \to X$ acts on $\Pi_1(X)$ by
\[
  \psi[\gamma] = [\psi \circ \gamma].
\]
The group operation is concatenation of loops.

Invesrion is changing the direction of the loop.

\begin{exercise}[Fundamental group of the torus]
  The fundamental group of $T$ is $\Pi_1(T) \cong \Z^2$ is generated
  by \[\left[\begin{pmatrix} 1 \\ 0 \end{pmatrix}\right] \tand
  \left[\begin{pmatrix} 0 \\ 1 \end{pmatrix}\right].\]
  This means any loop is homotopic to
  $[\begin{pmatrix} a \\ b \end{pmatrix}]$
  by doing $[\begin{pmatrix} 1 \\ 0 \end{pmatrix}]$ $a$ times
  and $[\begin{pmatrix} 0 \\ 1 \end{pmatrix}]$ $b$ times.
\end{exercise}

For $A \in \SL_2(\Z)$ and $\psi_A \colon T \to T$ action on
$\Pi_1(T, (0,0))$ (fundamental group of the torus based in $(0,0)$)
defined by
\[
  (x,y) \mapsto A \begin{pmatrix} x \\ y \end{pmatrix}.
\]
If $\psi_A$ was homotopic to the identity it owuld act trivially on
$\Pi_1(T, (0,0))$ which means that
\[
  [0,1] = [A (1,0)] = [(b,d)] \stand [(1,0)] = [A(1,0)] = [(a,c)]
\]
so $(a,c) = (1,0)$ and $(b,d) = (0,1)$ so $A$ is the identity matrix.

\newpage

\section{Hyperbolic geometry}
Recall that $\Nih(\H)$ is the set of Mobius transformations with a representing
matrix in $\SL_2(\Z)$.

But $z \mapsto \frac 1z$ does not present Euclidean metric on 
$\H = \set{z \colon \Im(z) > 0}$.

We will introduce a metric $\rho = \rho_{hyp}$ on $\H$ s.t.\
\[
  \Bih(\H) = \text{Orientation preserving isometrics of } (\H,\rho).
\]

\begin{definition}[Length]
  Let $\gamma \colon [0,1] \to \H$ be a pointwise continuous smooth path
  between $a$ and $b$.

  The hyperbolic length of $\gamma$ is
  \[
    L_{\rho}(\gamma) :=
    \int_0^1 
  \]
\end{definition}





\end{document}
