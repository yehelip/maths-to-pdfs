\documentclass[11pt,a4paper]{article}
\usepackage{amssymb,amsfonts,amsmath,calc,tikz,pgfplots,geometry}
\usepackage{color}   %May be necessary if you want to color links
\usepackage{hyperref}
\usepackage{amsthm}
\usepackage{fancyhdr}
\usepackage{MnSymbol}
\pagestyle{fancy}
\usetikzlibrary{positioning}
\geometry{margin=1in}
\pgfplotsset{compat=1.18}
\setlength{\headheight}{14.6pt}
\addtolength{\topmargin}{-1.6pt}
\hypersetup{
    colorlinks=false, %set true if you want colored links
    linktoc=all,   %set to all if you want both sections and subsections linked
    linkcolor=black,  %choose some color if you want links to stand out
}
%%%%%%%%%%%%%%%%%%%%%%%%%%%%%%%%%%%%%%%%%%%%%%%%%%%%%%%%%%%%%%%%%%%%%%%%%%%%%%%
\theoremstyle{definition}
\newtheorem{definition}{Definition}[section]
\newtheorem{remark}{Remark}[section]
\newtheorem{example}{Example}[section]
\theoremstyle{plain}
\newtheorem{theorem}{Theorem}[section]
\newtheorem{proposition}[theorem]{Proposition}
\newtheorem{lemma}[theorem]{Lemma}
\newtheorem{corollary}[theorem]{Corollary}

\DeclareMathOperator{\lcm}{lcm}
\DeclareMathOperator{\idealin}{\triangleleft}
\DeclareMathOperator{\im}{im}
\DeclareMathOperator{\Aut}{Aut}
\DeclareMathOperator{\End}{End}
\DeclareMathOperator{\Inn}{Inn}
\DeclareMathOperator{\Out}{Out}
\DeclareMathOperator{\Mat}{Mat}
\DeclareMathOperator{\std}{std}
\DeclareMathOperator{\Int}{Int}
\DeclareMathOperator{\diam}{diam}

\newcommand{\N}{\mathbb{N}}
\newcommand{\Z}{\mathbb{Z}}
\newcommand{\Q}{\mathbb{Q}}
\newcommand{\R}{\mathbb{R}}
\newcommand{\C}{\mathbb{C}}
\newcommand{\F}{\mathbb{F}}
\newcommand{\Omicron}{O}
\newcommand{\st}{\text{ s.t. }}
\newcommand{\tand}{\quad \text{and} \quad}
\newcommand{\tor}{\quad \text{or} \quad}
\newcommand{\ip}[2]{\langle #1, #2 \rangle}
\newcommand{\set}[2]{ \left\{ #1 \mid #2 \right\} }
\newcommand{\abs}[1]{\left\lvert #1\right\rvert}
\newcommand{\norm}[1]{\left\lVert #1\right\rVert}
\renewcommand{\tt}[1]{\textnormal{\textbf{(#1).}}} %tt=theorem title
\renewcommand{\implies}{\rightarrow}
\newcommand{\bigslant}[2]
{{\raisebox{.2em}{$#1$}\left/\raisebox{-.2em}{$#2$}\right.}}
%%%%%%%%%%%%%%%%%%%%%%%%%%%%%%%%%%%%%%%%%%%%%%%%%%%%%%%%%%%%%%%%%%%%%%%%%%%%%%%
\title{\textbf{Mathematical Logic}}
\author{}
\date{}
\begin{document}
	\maketitle
	\newpage
  \section{Propositional Logic}
  Before talking about propositional logic, let us review the basic
  connectives.

  \begin{center}
    \begin{tabular}{|l|l|l|}
      \hline
      Name        & Notation   & Meaning        \\ \hline
      Negation    & $\neg$     & not            \\ \hline
      Conjunction & $\land$    & and            \\ \hline
      Disjunction & $\lor$     & or             \\ \hline
      Implication & $\implies$ & if, ..., then  \\ \hline
      Equivalence & $\equiv$   & if and only if \\ \hline
    \end{tabular}
  \end{center}

  \begin{definition}[Set of atomic propositions]
    Given an arbitrary finite or countable set $AP = \{p_1,p_2,\dots\}$
    which we call the set of atomic propositions.
    The elements in the set describe our propositions.
  \end{definition}
  
  We will later see why this set is defined to be finite or countable.

  \begin{definition}[Propositions of depth $n$]
    Given a set $AP$ of atomic propositions, write $P_0 = AP$, and call
    the set $P_0$ the set of propositions of depth $0$.
    For any $n \geq 1$ define recursivly the set $P_n$ as the
    set of any element $A$ of one of the following forms:
    \begin{enumerate}
      \item[(1)] $A = \neg B$ for some $B \in P_{n-1}$.
      \item[(2)] $A = B * C$ for some propositions $B$ and $C$ of depth
        less than $n$, with at least one of them of depth $n-1$ and
        some connective $* \in \{\land, \lor, \implies, \equiv\}$.
    \end{enumerate}
    The set $P_n$ is called the set of propositions of depth $n$.
  \end{definition}

  \begin{definition}[Set of propositions]
    Given a set $AP$ of atomic propositions, the set of all propositions
    is denoted $P$ as defined as the unions of all propositions of all
    depths:
    \[
      P := \bigcup_{n=0}^{\infty} P_n.
    \]
  \end{definition}

  \begin{remark}
    Notice that by our construction, there does not exist a proposition
    with an infinite number of atomic propositions. Thus, the set of
    atomic proposition needs to contain at most a countable amount
    of atomic propositions, and that is why we defined it to be
    countable in the first place.
  \end{remark}

  \begin{definition}[Assignment]
    An assignment is a function $M \colon AP \to \{T,F\}$.
  \end{definition}

  Let $A$ be a proposition in $P$.
  We say that a certain assignment $M$ models $P$ if the truth
  value of $P$ under that assignment is $T$, and we denote $M \models P$.
  A rigorous definition for when a truth value of a proposition is $T$ 
  can be defined by using structural induction.

  \begin{definition}[Tautology]
    A proposition $A \in P$ is called a tautology if for any assignment
    $M$ we have $M \models A$.
  \end{definition}

  \begin{definition}[Contradiction]
    A proposition $A \in P$ is called a contradiction if for any assignment
    $M$ we have $M \nmodels A$.
  \end{definition}
  
  \begin{definition}[Logical equivalence]
    We say that two propositions $A,B \in P$ are logically equivalent
    if for any assignment $M$ we either have $M \models A$ and
    $M \models B$ or $M \nmodels A$ and $M \nmodels B$.
  \end{definition}

  \begin{remark}
    Logical equivalence is an equivalence relation.
  \end{remark}

  \begin{proposition}
    For every proposition $A \in P$, the propositions $A$ and
    $\neg(\neg A)$ are logically equivalent.
  \end{proposition}
  \begin{proof}
    Let $M$ be an assinment such that $M \models A$.
    Then by definition we have that $M \nmodels \neg A$.
    Finally, by definition we have that $M \models \neg(\neg A)$.

    Otherwise we have $M \nmodels A$.
    Then by definition we have that $M \models \neg A$.
    Finally, by definition we have that $M \nmodels \neg(\neg A)$.
  \end{proof}

  \begin{definition}[Opposite assignment]
    For each assignment $M \colon Ap \to \{T, F\}$ 
    define $\overline M \colon AP \to \{T, F\}$ as:
    \[
      \overline M(p) =
      \begin{cases}
        T, &M(p) = T \\
        F, &M(p) = F
      \end{cases}.
    \]
    The assignment $\overline M$ is called the opposite assignment
    of $M$.
  \end{definition}
  
  Informally, for any proposition $P$ we define $\overline P$ the
  propoisition such that any appearance of $p \in AP$ in $P$ is
  replaced by $\neg p$.

  \begin{proposition}
    For any assignment $M$ and proposition $P$ we have
    \[
      M \models P \iff \overline M \models \overline P.
    \]
  \end{proposition}

  \begin{definition}[$n$-ary truth function]
    An $n$-ary truth function is a function
    $f \colon \{T, F\}^n \to \{T, F\}$.
  \end{definition}

  \begin{definition}[Truth function representation]
    Let $A$ be a proposition comprised of $n$ atomic variables
    $\{p_i\}_{i=1}^{n}$, and let $f$ be an $n$-ary truth function.
    We say that $A$ represents $f$ if for every assignment
    $M$ we have
    \[
      M \models A \iff
      f(M(p_1),\dots,M(p_n)) = T.
    \]
  \end{definition}

  \begin{example}
    The proposition $A = p_1 \land \neg p_2$ represents the following
    truth function:
    \[
      f(x,y) = T \iff x = F \tand y = F.
    \]
  \end{example}

  \begin{definition}[Complete set of connectives]
    A set of connectives $K$ is called complete if any $n$-ary truth
    function can be represented using propositions that use connectives from 
    $K$ alone.
  \end{definition}

  \begin{proposition}
    The following sets of connectives are complete:
    \begin{enumerate}
      \item[(1)] $\{\neg, \land, \lor\}$.
      \item[(2)] $\{\neg, \lor\}$.
      \item[(3)] $\{\neg, \land\}$.
      \item[(4)] $\{\neg, \implies\}$.
    \end{enumerate}
  \end{proposition}

  \begin{definition}[Literal]
    A literal is an atomic formula or its negation.
  \end{definition}

  \begin{definition}[Disjunctive normal form]
    A logical formula is considered to be in DNF if it is a disjunction of 
    one or more conjunctions of one or more literals.
  \end{definition}

  \begin{remark}
    A DNF formula is in full disjunctive normal form if each of its 
    variables appears exactly once in every conjunction and each 
    conjunction appears at most once (up to the order of variables).
  \end{remark}

  \begin{proposition}
    For every natural $n$ and for any $n$-ary truth function $f$,
    there exists a proposition $A$ in DNF form representing $f$.
  \end{proposition}

  \begin{proposition}
    The set $\{\neg\}$ is not complete.
  \end{proposition}

  \begin{proposition}
    The set $\{\land, \lor\}$ is not complete.
  \end{proposition}

  \begin{definition}[Monotonic truth function]
    An $n$-ary truth function is called monotonic if there exists
    $\alpha \in \{T, F\}^n$ such that $f(\alpha) = T$ and $f(\beta) = T$
    if and only if for every $1 \le i \le n$ such that $\alpha_i = T$
    we have $\beta_i = T$.
  \end{definition}

  \begin{remark}
    The last definition is intuitive because we can change the order of
    coordinates of $\alpha$ such that all the $T$ coordinates are an
    initial segment of $\alpha$, and thus $f(\beta) = T$ if in that binary
    sense $\beta > \alpha$.
  \end{remark}

  \begin{proposition}
    A $n$-ary truth function can be represented using propositions that use
    connectives from $\{\land, \lor\}$ alone if and only if it is monotonic.
  \end{proposition}

  From now on we will assume all propositions are written using only 
  the connectives $\{\neg, \implies\}$.

  \begin{definition}[Propositional proof system]
    A propositional proof system is comprised of
    \begin{enumerate}
      \item[(1)] A set of propositions called axioms.
      \item[(2)] A set of rules of inference that allow us to deduce
        propositions from previous propositions.
    \end{enumerate}
  \end{definition}

  \begin{definition}[Proof]
    Given a set of propositions $\Sigma$ which we call the hypotheses,
    and a proposition $A$, we say that $A$ is provable from $\Sigma$,
    and denote $\Sigma \vdash A$
    if there exists a finite sequence of propoisitions $B_1,\dots,B_n$
    such that $B_n = A$ and for each $1 \le i \le n$, one of the
    following holds:

    \begin{itemize}
      \item $B_i$ is an axiom.
      \item $B_i \in \Sigma$ ($B_i$ is a hypothesis).
      \item $B_i$ is deducible from the propositions preceding it in the 
        sequence according to the rules of inference.
    \end{itemize}

    We call the sequence $\{B_i\}_{i=1}^{n}$ a proof for $A$.
  \end{definition}
  \begin{remark}
    In the case $\Sigma = \emptyset$ we denote $\vdash A$.
  \end{remark}

  The following standard proof system is the standard proof system we
  use, and is sometimes called a Hilbert system.

  \begin{definition}[Standard propositional proof system]
    The standard propositional proof system is comprised of the
    following three axiom schemes:
    \begin{enumerate}
      \item[(1)] $p\to (q\to p)$
      \item[(2)] $(p\to (q\to r))\to ((p\to q)\to (p\to r))$
      \item[(3)] $(\neg p\to \neg q)\to (q\to p)$
    \end{enumerate}
    Where $p,q,r$ are propositions,
    and the following rule of inference
    \[
      \begin{array}{c@{\,}l@{}}
                  & p         \\
                  & p \to q   \\ \cline{2-2}
      \therefore  & q
      \end{array}
    \]
    which is called Modus Ponens (MP).
  \end{definition}
  \begin{remark}
    Notice that we wrote three axioms schemes in the system, but we
    actually have infinitely many axioms.
  \end{remark}
  \begin{remark}
    All axioms are tautologies.
  \end{remark}

  \begin{proposition}
    For every proposition $A$ we have $\vdash A \implies A$.
  \end{proposition}

  \begin{definition}[Weak soundness]
    A propositoinal system is called sound if every provable proposition
    is a tautology. This can be written as
    \[
      \vdash A \quad\Rightarrow\quad \models A.
    \]
  \end{definition}

  \begin{definition}[Weak completeness]
    A propositoinal system is called complete if every tautology is
    provable. This can be written as
    \[
      \models A \quad\Rightarrow\quad \vdash A.
    \]
  \end{definition}

  \begin{theorem}\label{thm:weak-soundness}\tt{Weak Soundness Theorem}
    The standrad proof system is sound.
  \end{theorem}
  \begin{proof}
    To be added.
  \end{proof}

  \begin{theorem}\label{thm:weak-completeness}\tt{Weak Completeness Theorem}
    The standrad proof system is complete.
  \end{theorem}
  
  In order to prove the completeness theorem we first have to prove
  some other theorems.

  \begin{theorem}\tt{Propositional Deduction Theorem}
    Let $\Sigma$ be a set of propositions, let $A$ and $B$ be propositions.
    Then the following are equivalent
    \begin{itemize}
      \item[(1)] $\Sigma \cup \{A\} \vdash B$.
      \item[(2)] $\Sigma \vdash A \implies B$.
    \end{itemize}
  \end{theorem}
  \begin{proof}
    To be added.
  \end{proof}
  
  \begin{remark}
    By the deduction theorem, it suffices to prove that $A \vdash A$
    (which is trivial) to prove that $\vdash A \to A$ which was less
    trivial.
  \end{remark}

  \begin{proposition}
    Let $A$ and $B$ be propositions.
    \begin{enumerate}
      \item[(1)] $\vdash \neg A \implies (A \implies B)$.
      \item[(2)] $\vdash (\neg A \implies A) \implies A$.
      \item[(3)] $\vdash A \implies \neg \neg A$.
      \item[(4)] $\vdash (A \implies B) \implies 
        ((\neg A \implies B) \implies B)$.
    \end{enumerate}
  \end{proposition}

  \begin{proposition}[Principle of explosion]
    Let $A$ and $B$ be propositions. Then
    \[ A \implies (\neg A \implies B). \]
  \end{proposition}

  \begin{definition}[Consistency]
    A set of propositions $\Sigma$ is called consistent if there does
    not exist a proposition $A$ such that $\Sigma \vdash A$ and 
    $\Sigma \vdash \neg A$. If $\Sigma$ is not consistent it is said
    to be inconsistent.
  \end{definition}
  \begin{proposition}
    Let $\Sigma$ be a set of propositions. Then the following are equivalent:
    \begin{enumerate}
      \item[(1)] $\Sigma$ is inconsistent.
      \item[(2)] every proposition $A$ is provable from $\Sigma$.
    \end{enumerate}
  \end{proposition}

  \begin{definition}[Completeness of a set of propostions]
    A set of propositions $\Sigma$ is said to be complete for for every
    proposition $A$ we have
    \[
      \Sigma \vdash A \tor \Sigma \vdash \neg A.
    \]
  \end{definition}

  We can conclude that if $\Sigma$ is consistent and complete then
  for every proposition $A$ exactly one of the following is true:
  \begin{enumerate}
    \item[(1)] $\Sigma \vdash A$.
    \item[(2)] $\Sigma \vdash \neg A$.
  \end{enumerate}

  \begin{theorem}\tt{The expansion theorem}
    Let $\Sigma$ be a consistent set of propositions.
    Then there exists a consistent set of propositions $\Sigma'$ such
    that $\Sigma \subset \Sigma'$.
  \end{theorem}
  \begin{lemma}
    Let $A$ be a proposition, let $\Sigma$ be a set of propositions such
    that $\Sigma \nvdash A$. Then $\Sigma \cup \{\neg A\}$ is consisnent.
  \end{lemma}
  \begin{remark}
    Notice that if $\Sigma$ were inconsistent, then from the principle
    if inclusion and the deduction theorem we get that $\Sigma \vdash A$.
    That is the reason why $\Sigma$ is not required to be consistent.
  \end{remark}
  
  \begin{proof}
    Two proofs needed here.
  \end{proof}

  \begin{lemma}
    Let $\Sigma$ be a complete and consistent set of propositions.
    Then there exists an assignment $V$ such that for any proposition
    $A$ we have
    \[
      \Sigma \vdash A \iff
      V \models A
    \]
  \end{lemma}
  \begin{proof}
    Define the assignment $A \colon AP \to \{T,F\}$ as such:
    \[
      V(p) :=
      \begin{cases}
        T, &\Sigma \vdash p \\
        F, &\Sigma \vdash \neg p
      \end{cases}.
    \]
    It is well defined since $\Sigma$ is complete and consistent.
    The proof this is the correct assignment follows from structural
    induction.
  \end{proof}

  We can now proceed to prove \autoref{thm:weak-completeness}.
  \begin{proof}
    To be added.
  \end{proof}

  \begin{definition}[Semantic consequence]
    Let $\Sigma$ be a set of propositions, let $A$ be a proposition.
    We denote $\Sigma \models A$ if for each assignment $V$ that satisfies
    each $P \in \Sigma$ we have $V \models A$. We also say that $A$ is
    a semantic consequence of $\Sigma$.
  \end{definition}
  \begin{remark}
    Notice that in particular $\emptyset \models A$ if and only if $A$
    is a tautology.
  \end{remark}
  
  \begin{theorem}\label{thm:strong-soundness}\tt{Strong Soundness Theorem}
    Let $\Sigma$ be a set of propositions, let $A$ be a proposition.
    Then
    \[
      \Sigma \vdash A \quad\Rightarrow\quad \Sigma \models A.
    \]
  \end{theorem}

  \begin{theorem}\label{thm:strong-completeness}
    \tt{Strong Completeness Theorem}
    Let $\Sigma$ be a set of propositions, let $A$ be a proposition.
    Then
    \[
      \Sigma \vdash A \quad\Rightarrow\quad \Sigma \models A.
    \]
  \end{theorem}
  
  \begin{definition}[Model]
    Let $\Sigma$ be a set of propositions, let $M$ be an assignment.
    Then $M$ is said to be a model of $\Sigma$ if $M \models A$ 
    for all $A \in \Sigma$.
  \end{definition}

  The following is a corollary from \autoref{thm:strong-soundness}
  and \autoref{thm:strong-completeness}.
  \begin{corollary}
    Let $\Sigma$ be a set of propositions.
    Then $\Sigma$ is consistent if and only if there exists a model to it.
  \end{corollary}
  \begin{proof}
    $(\Rightarrow)$ Assume by contradiction that $\Sigma$ has no model.
    Then by definition we have vacuously that $\Sigma \models B$ for
    any proposition $B$. Then by the strong completeness theorem we have
    that $\Sigma \vdash B$ for any $B$. In particular for an arbitraty
    proposition $A$ we have $\Sigma \vdash A$ and $\Sigma \vdash \neg A$
    which shows that $\Sigma$ is inconsistent. This contradiction completes
    the proof of this direction.

    $(\Leftarrow)$ Assume by contradiction that $\Sigma$ is inconsistent.
    Then there exists some proposition $A$ such that $\Sigma \vdash A$
    and $\Sigma \vdash \neg A$. Then by the strong soundness theorem
    we have that $\Sigma \models A$ and $\Sigma \models \neg A$.
    Since there does not exist an assignment $V$ such that $V \models A$
    and $V \models \neg A$ we have that there does not exist a model
    for $\Sigma$.
  \end{proof}

  \begin{theorem}\tt{Compactness theorem}
    Let $\Sigma$ be a set of propositions.
    Then there exists a model for $\Sigma$ if and only if
    there exists a model for any finite subset of $\Sigma$.
  \end{theorem}
  \begin{proof}
    $(\Rightarrow)$ Let $M$ be a model for $\Sigma$.
    Then in particular $M$ is a model for any finite subset of $\Sigma$.

    $(\Leftarrow)$ Now suppose that there does not exist a model for $\Sigma$.
    From the above corollary that means that $\Sigma$ is inconsistent.
    Thus exists a proposition $A$ such that
    $\Sigma \vdash A$ and $\Sigma \vdash \neg A$.
    Let $\Sigma' \subset \Sigma$ be the set of all propositions from
    the proofs of $A$ and $\neg A$.
    Then $\Sigma'$ is finite since all proofs are finite by definition,
    and $\Sigma'$ is inconsistent because it can prove both $A$ and $\neg A$.
    Thus, from the corollary above, there exists no model to $\Sigma'$.
  \end{proof}

  \begin{theorem}\tt{De Bruijn–Erdős theorem}
    Let $G = (V, E)$ be a countable graph.
    Then the following are equivalent:
    \begin{enumerate}
      \item[(1)] $G$ is $k$-colorable.
      \item[(2)] every finite subgraph of $G$ is $k$-colorable.
    \end{enumerate}
  \end{theorem}

  \begin{corollary}
    Every planar graph is $4$-colorable.
  \end{corollary}

  Talk about Ramsey + Find the Ramsey theorem tex.

  \begin{definition}
    Let $\Sigma$ be a set of propositions. Define
    \[
      \mu(\Sigma) := \set{M}{M \models \Sigma}.
    \]
  \end{definition}

  \begin{definition}[Definability]
    A set of assignments $\mathcal M$ is said to be definable if there
    exists a proposition set $\Sigma$ such that $\mu(\Sigma) = \mathcal M$.
    If $\mathcal M$ is definable by a finite proposition set $\Sigma$ it is
    said to be finitely definable.
  \end{definition}

  \begin{example}
    The set of all assignments $\mathcal M = \{F, T\}^{n}$ is definable
    by any set of tautologies since all assignments satisfy all tautologies.
  \end{example}
  \begin{example}
    The set $\mathcal M = \emptyset$ is definable by the proposition
    $A = p \implies \neg p$ since there does not exist an assignment
    that satisfies $A$.
  \end{example}

  \begin{proposition}
    Must sets of assignments are undefinable.
  \end{proposition}
  \begin{proof}
    It is clear that the cardinality of the set of all definable sets of 
    assignments is smaller than the cardinality of the set of all
    sets of proposition. Therefore there are at most $2^{\aleph_0}$
    propositions.

    Since the cardinality of the set of sets of assignments is
    $2^{2^{\alpha_0}}$ the proposition follows.
  \end{proof}

  \begin{proposition}
    For every assignment $V$ the set $\{V\}$ is definable.
  \end{proposition}
  \begin{proof}
    Define
    \[
      \Sigma :=
      \set{p \in AP}{V(p) = T} \cup
      \set{\neg p \in AP}{V(p) = F}.
    \]
    We have that $\mu(\Sigma) = \{V\}$ which completes the proof.
  \end{proof}

  \begin{proposition}
    The set $\mathcal M = \set{M}{M \text{ is monotonic}}$ is definable.
    In this question $M$ is monotonic if for all $i < j$ we have
    $M(p_i) \Rightarrow M(p_j)$.
  \end{proposition}

  \begin{proposition}[Properties of definable sets]
    \begin{enumerate}
      \item[(1)] if $\Sigma_1 \subset \Sigma_2$ then
        $\mu(\Sigma_1) > \mu(\Sigma_2)$.
      \item[(2)] let $\{\mathcal M_i\}_{i \in I}$ be a set of definable sets.
        Then $\cap_{i \in I} \mathcal M_i$ is definable.
      \item[(3)] a finite union of definable sets is definable.
      \item[(4)] any finitely definable set is definable by a single
        proposition.
    \end{enumerate}
  \end{proposition}
  \begin{proof}
    \begin{enumerate}
      \item[(1)] trivial.
      \item[(2)] for any $\mathcal M_i$ there exists a set of propositions
        $\Sigma_i$ such that $\mu(\Sigma_i) = \mathcal M_i$.
        We will show that
        \[
          \bigcap_{i \in I} \mathcal M_i =
          \mu\left(\bigcup_{i \in I} \Sigma_i\right).
        \]
        We have that
        \[
          V \in \mu\left(\bigcup_{i \in I} \Sigma_i\right) \iff
          \forall i \in I \colon V \models \Sigma_i \iff
          \forall i \in I \colon V \in \mu(\Sigma_i) = \mathcal M_i \iff
          V \in \bigcap_{i \in I} \mathcal M_i.
        \]
      \item[(3)] it suffices to prove that if $\mathcal M_1$ and
        $\mathcal M_2$ are definable then $\mathcal M_1 \cup \mathcal M_2$
        is definable. Define
        \[
          \Sigma := \set{A_1 \lor A_2}{A_1 \in \Sigma_1 \land A_2 \in \Sigma_2}.
        \]
        We will show that $\mu(\Sigma) = \mathcal M_1 \cup \mathcal M_2$.
        It is clear that $\mathcal M_1 \cup \mathcal M_2 \subset \mu(\Sigma)$.
        Let $V \in \mu(\Sigma)$, and assume by contradiction that
        $V \notin \mathcal M_1 \cup \mathcal M_2$.
        Then there exist $A \in \Sigma_1$ and $B \in \Sigma_2$ such that
        $V \nmodels A$ and $V \nmodels B$.
        Since we have $A \lor B \in \Sigma$ we get that $V \notin \mu(\Sigma)$.
        This contradiction shows that
        $\mu(\Sigma) \subset \mathcal M_1 \cup \mathcal M_2$
        which completes the proof.
      \item[(4)] let $\mathcal M$ be finitely definable, let 
        $\mathcal M = \mu(\Sigma)$ such that $\Sigma$ is finite.
        Define
        \[
          \Sigma' := \set{\bigwedge_{i=1}^{n} A_i}{A_i \in \Sigma}.
        \]
        It is clear that $\mu(\Sigma') = \mathcal M$ which completes the
        proof.
    \end{enumerate}
  \end{proof}

  Consider the space of all assignment $S := \{F, T\}^{AP}$.
  \begin{definition}[Convergence of assignments]
    Let $\{M_i\}_{i=1}^{\infty} \subset S$.
    We say that $\{M_i\}_{i=1}^{\infty}$ converges if for every $p \in AP$
    the sequence $\{M_n(p)\}_{n=1}^{\infty}$ stablizes eventually.
    If $\{M_i\}_{i=1}^{\infty}$ converges, we say that its limit is
    the assignment $M$ such that $M(p)$ is the value to which
    $\{M_n(p)\}_{n=1}^{\infty}$ stablizes.
  \end{definition}
  
  \begin{definition}[Closed set of assignments]
    Let $\mathcal M$ be a set of assignments.
    We say that $\mathcal M$ is closed if the limit of any convergent sequence
    $\{M_n\}_{n=1}^{\infty} \subset \mathcal M$ is in $\mathcal M$.
  \end{definition}
  
  \begin{example}
    The set $\mathcal M = \set{M}{M \text{ is monotonic}} \setminus \{m_0\}$
    such that $m_0(p) = T$ for any $p \in AP$ is not closed.
    Choose the sequence $\{M_n\}_{n=1}^{\infty}$ such that
    \[
      M_n :=
      \begin{cases}
        T, &i \le n \\
        F, &i > n
      \end{cases}.
    \]
    The sequence $\{M_n\}_{n=1}^{\infty}$ converges to $m_0$ but 
    $m \notin \mathcal M$.
  \end{example}

  \begin{proposition}
    Let $\mathcal M$ be a set of assignment. Then
    \[
      \mathcal M \text{ is closed} \iff
      \mathcal M \text{ is definable}.
    \]
  \end{proposition}
  \begin{proof}
    To be added.
  \end{proof}

  \begin{proposition}
    Let $\mathcal M$ be a set of assignments. Then the following are equivalent:
    \begin{enumerate}
      \item[(1)] $\mathcal M$ is finitely definable.
      \item[(2)] $\mathcal M$ is definable by a single proposition.
      \item[(3)] $\mathcal M$ is definable and $\mathcal M^c$ is definable.
    \end{enumerate}
  \end{proposition}
  \begin{proof}
    We proved direction $(1) \to (2)$ before.
    \begin{enumerate}
      \item[$(2) \to (3)$] let $\mathcal M = \mu(\{A\})$.
        Then $\mathcal M = \mu(\{\neg A\})$.
      \item[$(3) \to (1)$] to be added.
    \end{enumerate}
  \end{proof}

  \section{Structures}
  \begin{definition}[Strcture]
    A structure $\mathcal M$ is composed of
    \begin{itemize}
      \item A nonempty set $U^M$ called the universe of $M$;
      \item Constants, which are distinguished elemenets of $U^M$;
      \item Functions of the form $f \colon (U^M)^n \to U^M$;
      \item Relations of the form $R \subseteq (U^M)^n$.
    \end{itemize}
  \end{definition}

  \begin{example}
    The structure of the integers as an additive group would be defined
    as such:
    \begin{itemize}
      \item $U^M = \Z$;
      \item $e^M = 0$ as the zero element (a constant);
      \item $f^M$ a binary function defined as $f(z_1,z_2) = z_1 + z_2$ 
        (addition function);
      \item $g^N$ a unary function defined as $g(z) = -z$ (inverse function);
      \item No relations except the equality relation.
    \end{itemize}
  \end{example}

  \begin{definition}[Language]
    A Language $\mathcal L$ is composed of
    \begin{itemize}
      \item Constant symbols ($c$, $d$, \dots);
      \item Function symbols ($f$, $g$, \dots), each one with a given arity;
      \item Relation symbols ($R$, $P$, \dots), each one with a given arity;
      \item Binary equality symbol $=$ (optional);
      \item Connectives ($\land$, $\lor$, $\neg$, $\implies$, $\equiv$);
      \item Variables ($x$, $y$, $z$);
      \item Quantifiers ($\exists$, $\forall$).
    \end{itemize}
  \end{definition}





\end{document}
