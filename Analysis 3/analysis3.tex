\documentclass[11pt,a4paper]{article}
\usepackage{amssymb,amsfonts,amsmath,calc,tikz,pgfplots,geometry}
\usepackage{color}   %May be necessary if you want to color links
\usepackage{hyperref}
\usepackage{amsthm}
\usetikzlibrary{positioning}
\geometry{margin=1in}
\hypersetup{
    colorlinks=false, %set true if you want colored links
    linktoc=all,   %set to all if you want both sections and subsections linked
    linkcolor=black,  %choose some color if you want links to stand out
}

%%%%%%%%%%%%%%%%%%%%%%%%%%%%%%%%%%%%%%%%%%%%%%%%%%%%%%%%%%%%%%%%%%%%%%%%%%%%%%%
\theoremstyle{plain}
\newtheorem{theorem}{Theorem}[section]
\newtheorem{lemma}[theorem]{Lemma}
\newtheorem{proposition}[theorem]{Proposition}
\newtheorem{corollary}[theorem]{Corollary}
\newtheorem{definition}{Definition}[section]
\newtheorem{remark}{Remark}[section]
\DeclareMathOperator{\lcm}{lcm}
\DeclareMathOperator{\idealin}{\triangleleft}
\DeclareMathOperator{\im}{im}
\DeclareMathOperator{\Aut}{Aut}
\DeclareMathOperator{\End}{End}
\DeclareMathOperator{\Inn}{Inn}
\DeclareMathOperator{\Out}{Out}
\DeclareMathOperator{\Mat}{Mat}
\DeclareMathOperator{\std}{std}
\newcommand{\N}{\mathbb{N}}
\newcommand{\Z}{\mathbb{Z}}
\newcommand{\Q}{\mathbb{Q}}
\newcommand{\R}{\mathbb{R}}
\newcommand{\C}{\mathbb{C}}
\newcommand{\F}{\mathbb{F}}
\newcommand{\Omicron}{O}
\newcommand{\ip}[2]{\langle #1, #2 \rangle}
\newcommand{\set}[2]{ \left\{ #1 \mid #2 \right\} }
\newcommand{\bigslant}[2]
{{\raisebox{.2em}{$#1$}\left/\raisebox{-.2em}{$#2$}\right.}}
%%%%%%%%%%%%%%%%%%%%%%%%%%%%%%%%%%%%%%%%%%%%%%%%%%%%%%%%%%%%%%%%%%%%%%%%%%%%%%%
\title{\textbf{Analysis 3}}
\author{Yeheli Fomberg}
\date{326269651}
\begin{document}
	\maketitle
	\newpage
	\section{Introduction to Topology}
	\subsection{Norms and Metrics}
	In this course we are going to analyze functions of the form 
	$f\colon \R^n\to\R^m$ which requires of course knowing analysis $2$ but
	since will also be using linear analysis tools this course is in
	some way a continuation of linear analysis as well. We will
	start by giving basic topological definitions in the eclidean
	space $\R^d$, first we define:
	\[
		\R^d = \set{(x_1,x_2,\dots,x_d)}{\substack{1 \le i \le d \\ x_i\in\R}}
	\]
	And now we can continue to define some more topological terms:
	\begin{definition}
	The \textbf{Eulidean norm} is defined as:
	\[
		\|x\| = \|x\|_2 = \sqrt{\sum_{i=1}^{d}{x_i^2}}
	\]
	We can similarly define the $L_p$ norm as:
	\[
		\|x\| = \|x\|_p = \sqrt[p]{\sum_{i=1}^{d}{x_i^p}}
	\]
	\end{definition}
	\noindent Which satisfies all properties of the norm
	\begin{definition}
	The \textbf{Euclidean metric} is defined as:
	\[
		d_2(P_1,P_2) = \|P_1-P_2\|_2 = \sqrt{(x_1-x_2)^2 - (y_1-y_2)^2}
	\]
	\end{definition}
	\noindent Notice that it is induced by the Eculidean norm and similarly
	we can induce $L_p$ metric using $L_p$ norms.
	\subsection{Sequences}
	Up until now we didn't have many problems using subscript for indexes
	of sequences, but now since we have coordinates we must denote a sequence
	in another similar way $^{\text{superscript}}$ as such 
	$(x^n)_{n=1}^{\infty}$. To define convergence:
	\[
		\lim_{n\to\infty}{x^n} = x 
		\quad \iff \quad
		\forall i\left(\lim_{n\to\infty}{x_i^n} = x_i\right)
	\]
	\begin{definition}
	A sequence $(x^n)_{n=1}^{\infty}$ is called a Cauchy sequence if and only 
	if:
	\[
		\lim_{n,m\to\infty}{\|x^n-x^m\|} = 0
	\]
	\end{definition}
	
	\newpage
	
	\subsection{Topology}
	\begin{definition}
	A \textbf{complete metric space} is a metric space $M$ such that every
	Cauchy sequence in $M$ converges to some limit in $M$.
	\end{definition}
	\begin{definition}
	An \textbf{open set} in a Euclidean space is a subset $U$ such that
	for any $x \in U$ exists $\varepsilon > 0$ such that any $y$ such that
	any $y \in B_\varepsilon(x)$ satisfies $y\in U$
	\end{definition}
	\begin{definition}
	In a topological space $X$ a space a \textbf{neighborhood} of a point 
	$x\in X$ is a subset such that exists an open set $U$ such that 
	$p\in U\subset V$
	\end{definition}
	\begin{definition}
	A \textbf{close set} $E$ in a Eculidean space $X$ is a subset of $X$ such
	that:
	\[
		(x^n)_{n=1}^{\infty}\subseteq E \quad x^n \xrightarrow{n\to\infty} x
		\implies x \in E
	\]
	\end{definition}
	\begin{definition}
	A topological space is called \textbf{compact} if every open cover of 
	$X$ has a finite subcover. An equivalent definition if the space is
	Eculidean is that for every sequence $(x^n)_{n=1}^{\infty}$ has
	a subsequence $(x^{n_k})_{k=1}^{\infty}$ that converges to a point
	$x$ in the space.
	\end{definition}
	\begin{definition}
	The \textbf{closure} of a topological space $X$ is defined as:
	\[
		\mathrm{Cl}(X) = \set{x}{\exists (x^n) \colon 
		x^n \xrightarrow{n\to\infty} x}
	\]
	\end{definition}
	\begin{definition}
	The \textbf{interior} of a topological space $X$ is defined as:
	\[
		\mathrm{Int}(X) = \set{x}{\exists r > 0 \colon B_r(x) \subseteq X}
	\]
	\end{definition}
	\begin{definition}
	The \textbf{boundary} of a topological space $X$ is defined as:
	\[
		\partial X = \set{x\in\R^d}{\forall r > 0\
		\exists x \in X \land \exists y \in X^c\ : y,z\in B_r(x)}
	\]
	\end{definition}
	\begin{definition}
	A function is continuous at $x\in X$ if for every $\varepsilon > 0$
	exists $\delta > 0$ such that for every $\|x-y\| < \delta$
	\[
		\|f(y) - f(x)\| < \varepsilon
	\]
	And we say that a function is continuous on $X$ if it is continuous
	for every $x\in X$
	\end{definition}
	\begin{definition}
	A function is uniformly continuous on $X$ if for every $\varepsilon > 0$
	exists $\delta > 0$ such that for every $\|x-y\| < \delta$
	\[
		\|f(x) - f(y)\| < \varepsilon
	\]
	\end{definition}
	\begin{remark}
	An important equivalent, more general definition for continuity is that
	if $f$ is a function from $A$ to $B$ then if $U$ is an open set in $B$
	implies $f^{-1}(U)$ is an open set we say that $f$ is continuous on $A$.
	\end{remark}
	\begin{definition}
	A \textbf{connected} space is a topological space that cannot be 
	represented as the union of two or more disjoint non-empty open subsets. 
	\end{definition}
	\begin{definition}
	A \textbf{path} from $x\in X$ to $y\in X$ is a continuous function from 
	the unit interval $[0,1]$ to $X$ such that $f(0) = 0$ and $f(1) = y$
	\end{definition}
	\begin{definition}
	A \textbf{path connected} space $X$ is a topological space such that 
	exists a path between any two points in $X$
	\end{definition}
	
	\newpage
	
	\section{Practice}
	Phew! These were a lot of definitions... Now it's time for some practice!
	\\ \textbf{Prove that a continuous function $f\colon A\to B$ has 
	a maximum in a compact space} \\
	By the completeness axiom for the real numbers we know that the set
	$f(A)$ has a supremum which we will denote $S$. By the definition
	of the supremum it is possible to construct a sequence that converges
	to it which we shall denote $f(x^n)$. We don't know whether $x^n$
	converges or not but we know it has a subsequence that converges so:
	\[
		\lim_{k\to\infty}{x^{n_k}} = x \quad \text{and} \quad 
		\lim_{k\to\infty}{f(x^{n_k})} = S
	\]
	Since $f$ is close we get $x\in A$ and since it is continuous $f(x) = S$
	which shows that it is continuous and also has a maximum. \\
	\textbf{Prove that a set $E$ is closed if and only if it's complement
	$E^c$ is open} \\
	\underline{$(\Rightarrow)$} \\
	Suppose that $E$ is closed, and $E^c$ is not open. Then exists
	$x\notin E$ such that for all $r > 0$ we get $B_r(x) \cap E \neq 
	\emptyset$ which means we can construct a sequence in $E$ that converges
	to $x$ but $x\notin E$ in contradiction to the assumption that $E$ is 
	close. \\
	\underline{$(\Leftarrow)$} \\
	Suppose that $E^c$ is open and $E$ is not closed, then exists a sequence
	$(x^n)_{n=1}^{\infty}$ that converges to some $x \in E^c$ which means
	that for every $r > 0$ that $B_r(x) \cap E \neq \emptyset$ in 
	contradiction to $E_c$ being open.
	
	\newpage
	
	\section{Differentiability}
	Let $A \in \R^{m \times n}$. We define the linear map 
	$T_A \colon \R^N \to \R^M$ by:
	\[
		T(x) = Ax \quad x\in \R^n
	\]
	Let $T \colon V \to M$ be a linear transformation between inner product
	spaces, we define the operator norm to be:
	\[
		\|T\|_{\mathrm{op}} = \|T\| = \sup_{v \neq 0}{\frac{\|Tv\|_W}{\|v\|_V}}
	\]
	and $T$ is said to be bounded if $\|T\| < \infty$. An important result
	to prove is that if $T$:
	\[
		\|T\|_{\mathrm{op}} \le 
		\left(\sum_{i,j}{}a_{ij}^{2}\right)^{\frac{1}{2}} < \infty
	\]
	since this gives that:
	\[
		\|T(x) - T(y)\| \le \|T\|_{\mathrm{op}}\|x-y\|
	\]
	which means that $T$ is continuous and even Lipschitz continuous.
	\begin{definition}
	An \textbf{affine function} is a function fo the form:
	\[
		T_{A,b}(x) = Ax + b
	\]
	such that $A\in\R^{m \times n}$ and $b\in x\in \R^{m}$ 
	\end{definition}
	
	
	
	
\end{document}