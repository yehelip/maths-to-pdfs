\documentclass[11pt,a4paper]{article}

\def\nyear {2025}
\def\nterm {Winter}
\def\nlecturer {Emanuel Milman}
\def\ncourse {Real Functions}

\makeatletter

% packages
\usepackage{amssymb,amsfonts,amsmath,calc,tikz,pgfplots,geometry,mathtools}
\usepackage{color}   % May be necessary if you want to color links
\usepackage[hidelinks]{hyperref}
\usepackage{forest}
\usepackage{commath}
\usepackage{amsthm}
\usepackage{fancyhdr}
\usepackage{bm}
\usepackage{witharrows}
\usepackage{bookmark}
\usepackage{tikz-cd}
\usepackage{bbm}
\usepackage{textcomp}
\usepackage{gensymb}
\usepackage{cleveref}

% tikz libraries
\usetikzlibrary{positioning}
\usetikzlibrary{matrix}
\usetikzlibrary{arrows}
\usetikzlibrary{arrows.meta}
\usetikzlibrary{decorations.markings}

% Page style setup
\pagestyle{fancy}
\geometry{margin=1in}
\pgfplotsset{compat=1.18}
\setlength{\headheight}{14.6pt}
\addtolength{\topmargin}{-1.6pt}
\hypersetup{
    colorlinks=false,
    linktoc=section,
    linkcolor=black,
}

%% maketitle setup
\ifx \nauthor\undefined
  \def\nauthor{yehelip}
\else
\fi

\ifx \ncoursehead \undefined
\def\ncoursehead{\ncourse}
\fi

\lhead{\emph{\nouppercase{\leftmark}}}
\ifx \nextra \undefined
  \rhead{
    \ifnum\thepage=1
    \else
      \ncoursehead
    \fi}
\else
  \rhead{
    \ifnum\thepage=1
    \else
      \ncoursehead \ (\nextra)
    \fi}
\fi

\let\@real@maketitle\maketitle
\renewcommand{\maketitle}{\@real@maketitle\begin{center}
\begin{minipage}[c]{0.9\textwidth}\centering\footnotesize
These notes are not endorsed by the lecturers.
I have revised them outside lectures to incorporate supplementary explanations,
clarifications, and material for fun.
While I have strived for accuracy, any errors or misinterpretations 
are most likely mine.
\end{minipage}\end{center}}

% theorem environments
\theoremstyle{definition}
\newtheorem{definition}{Definition}[section]
\newtheorem{remark}{Remark}[section]
\newtheorem{example}{Example}[section]
\newtheorem{exercise}{Exercise}[section]
\newtheorem{paradox}{Paradox}[section]
\newtheorem*{solution}{Solution}
\theoremstyle{plain}
\newtheorem{theorem}{Theorem}[section]
\newtheorem{proposition}[theorem]{Proposition}
\newtheorem{lemma}[theorem]{Lemma}
\newtheorem{corollary}[theorem]{Corollary}

% tikz customization
\pgfarrowsdeclarecombine{twolatex'}{twolatex'}{latex'}{latex'}{latex'}{latex'}
\tikzset{->/.style = {decoration={markings,
                                  mark=at position 1
                                  with {\arrow[scale=2]{latex'}}},
                      postaction={decorate}}}
\tikzset{<-/.style = {decoration={markings,
                                  mark=at position 0 with {\arrowreversed[scale=2]{latex'}}},
                      postaction={decorate}}}
\tikzset{<->/.style = {decoration={markings,
                                   mark=at position 0 with {\arrowreversed[scale=2]{latex'}},
                                   mark=at position 1 with {\arrow[scale=2]{latex'}}},
                       postaction={decorate}}}
\tikzset{->-/.style = {decoration={markings,
                                   mark=at position #1 with {\arrow[scale=2]{latex'}}},
                       postaction={decorate}}}
\tikzset{-<-/.style = {decoration={markings,
                                   mark=at position #1 with {\arrowreversed[scale=2]{latex'}}},
                       postaction={decorate}}}
\tikzset{->>/.style = {decoration={markings,
                                  mark=at position 1 with {\arrow[scale=2]{latex'}}},
                      postaction={decorate}}}
\tikzset{<<-/.style = {decoration={markings,
                                  mark=at position 0 with {\arrowreversed[scale=2]{twolatex'}}},
                      postaction={decorate}}}
\tikzset{<<->>/.style = {decoration={markings,
                                   mark=at position 0 with {\arrowreversed[scale=2]{twolatex'}},
                                   mark=at position 1 with {\arrow[scale=2]{twolatex'}}},
                       postaction={decorate}}}
\tikzset{->>-/.style = {decoration={markings,
                                   mark=at position #1 with {\arrow[scale=2]{twolatex'}}},
                       postaction={decorate}}}
\tikzset{-<<-/.style = {decoration={markings,
                                   mark=at position #1 with {\arrowreversed[scale=2]{twolatex'}}},
                       postaction={decorate}}}

\pgfarrowsdeclare{biggertip}{biggertip}{%
  \setlength{\arrowsize}{1pt}
  \addtolength{\arrowsize}{.1\pgflinewidth}
  \pgfarrowsrightextend{0}
  \pgfarrowsleftextend{-5\arrowsize}
}{%
  \setlength{\arrowsize}{1pt}
  \addtolength{\arrowsize}{.1\pgflinewidth}
  \pgfpathmoveto{\pgfpoint{-5\arrowsize}{4\arrowsize}}
  \pgfpathlineto{\pgfpointorigin}
  \pgfpathlineto{\pgfpoint{-5\arrowsize}{-4\arrowsize}}
  \pgfusepathqstroke
}
\tikzset{
	EdgeStyle/.style = {>=biggertip}
}

\tikzset{circ/.style = {fill, circle, inner sep = 0, minimum size = 3}}
\tikzset{scirc/.style = {fill, circle, inner sep = 0, minimum size = 1.5}}
\tikzset{mstate/.style={circle, draw, black, text=black, minimum width=0.7cm}}

\tikzset{eqpic/.style={baseline={([yshift=-.5ex]current bounding box.center)}}}

\definecolor{mblue}{rgb}{0.2, 0.3, 0.8}
\definecolor{morange}{rgb}{1, 0.5, 0}
\definecolor{mgreen}{rgb}{0, 0.4, 0.2}
\definecolor{mred}{rgb}{0.5, 0, 0}

% algebra
\DeclareMathOperator{\lcm}{lcm}
\DeclareMathOperator{\Out}{Out}
\DeclareMathOperator{\Aut}{Aut}
\DeclareMathOperator{\End}{End}
\DeclareMathOperator{\Inn}{Inn}
\DeclareMathOperator{\Mat}{Mat}
\DeclareMathOperator{\std}{std}
\DeclareMathOperator{\sgn}{sgn}
\DeclareMathOperator{\id}{id}
\newcommand{\idealin}{\triangleleft}
\newcommand{\ip}[2]{\langle #1, #2 \rangle}
\newcommand{\bigslant}[2]
{{\raisebox{.2em}{$#1$}\left/\raisebox{-.2em}{$#2$}\right.}}

% analysis
\newcommand{\dx}{\dif x}
\newcommand{\dt}{\dif t}
\newcommand{\du}{\dif u}
\newcommand{\dv}{\dif v}
\DeclareMathOperator{\im}{im}
\DeclareMathOperator{\cis}{cis}
\DeclareMathOperator{\Int}{Int}
\DeclareMathOperator{\diam}{diam}
\DeclareMathOperator{\supp}{supp}
\DeclareMathOperator{\Vol}{Vol} % Volume

% logic
\DeclareMathOperator{\MOD}{MOD}
\DeclareMathOperator{\Theory}{Theory}


% nice
\newcommand{\half}{\frac{1}{2}}
\newcommand{\pair}{\del}
\newcommand{\taking}[1]{\xrightarrow{#1}}
\newcommand{\inv}{^{-1}}
\newcommand{\ot}{\leftarrow}
\newcommand{\ninfty}{-\infty}
\newcommand{\floor}[1]{\left\lfloor #1 \right\rfloor}
\newcommand{\ceil}[1]{\left\lceil #1 \right\rceil}

% probability
\newcommand{\Prob}{\mathbf{P}}
\renewcommand{\vec}[1]{\boldsymbol{\mathbf{#1}}}
\DeclareMathOperator{\Bin}{Bin}
\DeclareMathOperator{\Geo}{Geo}
\DeclareMathOperator{\Poi}{Poi}
\DeclareMathOperator{\Exp}{Exp}
\DeclareMathOperator{\Var}{Var} % Variance
\DeclareMathOperator{\Cov}{Cov}

% special letters
\newcommand{\N}{\mathbb{N}}
\newcommand{\Z}{\mathbb{Z}}
\newcommand{\Q}{\mathbb{Q}}
\newcommand{\R}{\mathbb{R}}
\newcommand{\C}{\mathbb{C}}
\newcommand{\F}{\mathbb{F}}
\newcommand{\E}{\mathbb{E}}
\newcommand{\ps}{\mathcal{P}}
\newcommand{\M}{\mathcal{M}}
\renewcommand{\L}{\mathcal{L}}
\newcommand{\Omicron}{O}
\newcommand{\powerset}{\mathcal{P}}

% text
\newcommand{\st}{\text{ s.t. }}
\newcommand{\tand}{\quad \text{and} \quad}
\newcommand{\tor}{\quad \text{or} \quad}
\newcommand{\stand}{\text{ and }}
\newcommand{\stor}{\text{ or }}
\renewcommand{\tt}[1]{\textnormal{\textbf{(#1).}}} %tt=theorem title GET RID OF

% title format
\title{\textbf{\ncourse}}
\author{Based on lectures by \nlecturer \\\small Notes taken by \nauthor}
\date{\nterm\ \nyear}
\makeatother


\begin{document}
\maketitle

% Insert cool image here

\newpage
\tableofcontents
\newpage

\section{Introduction}

\subsection{Motivation}
The Riemann integral we have known so far is fairly limited.
For example it doesn't allow us to compute the Riemann integral
of Dirichlet's function $f \colon [0,1] \to \R$ defined as
\[
  f(x) = \mathbbm{1}_{\Q \cap [0,1]} =
  \begin{cases}
    1, &x \in \Q \cap [0,1] \\
    0, &\text{otherwise}
  \end{cases}.
\]
In his thesis Lebesgue introduced a new type of integral called
a Lebesgue integral that allows us to compute integrals for functions
like Dirichelt's function, and he continued to develop more concepts
like measure, and almost everywhere.

\subsection{Lebesgue integral}
Let $f(x) = \mathbbm{1}_A$ be the function that 
we want $\int \mathbbm{1}_A$ to be the volume of the set $A$.

First we would like we define what is a volume of a set.
We would want to require a couple of things
\begin{enumerate}
  \item[(1)] $\mu(A)$ is defined for all $A \subseteq \R^n$;
  \item[(2)] $\mu([0,1]^n) = 1^n = 1$;
  \item[(3)] $\mu$ to be invariant to congruations (isometries).
  \item[(4)] If $\set{A_i}_{i=1}^{\infty}$ is a countable sequence
    of pairwise disjoint sets then
    \[
      \mu\del{\bigcup_{i=1}^{\infty} = \sum_{i=1}^{\infty} \mu(A_i)}.
    \]
\end{enumerate}

\begin{remark}
  Property $(4)$ is called $\sigma$-additivity.
\end{remark}

\begin{theorem}[Hausdorrf, 1914]
  There is no function that satisfies $(1) - (4)$ at the same time.
\end{theorem}

We will prove this theorem later.
For now we can only try to weaken the requirements.
For example instead of $\sigma$-additivity we might require finite
additivity.

\begin{theorem}
  There exists a function that satisfies the wanted requirements in
  dimenstions $1$ and $2$ but not in dimension $n \geq 3$.
\end{theorem}

For example in $n = 3$ we have the Banach-Tarski paradox

\begin{paradox}[Banach--Tarski, 1924]
  For every $n \geq 1$ we can divide $S^2$ in $\R^n$ to a finite amount 
  of parts such that when they are rotated and rearranged,
  can form a new sphere of any desired size.
\end{paradox}
\begin{remark}
  The bigger sphere we would want to form, the greater is the minimal
  pieces we need to divide the unit sphere in order to form it.
\end{remark}

This paradox is based on the use of the axiom of choice, but since we
assume the axiom of choice in this course we still need to modify the
requirements.

Instead of $(4)$ we require subadditivity:
\[
  \mu\del{\bigcup_{i=1}^{\infty} \le \sum_{i=1}^{\infty} \mu(A_i)}.
\]
This is possible but instead we would like to keep $\sigma$-additivity
but instead give up on requirement $(0)$.
We will only define volume only for ``nice'' sets which include
all of the sets we work with in analysis or geometry etc.\ these
sets form a $\sigma$-algebra on $\R^n$.

\begin{definition}[Algebra]
  An algebra on a nonempty set $X$ is a collection $\mathcal A$ of
  subsets of $X$ such that
  \begin{enumerate}
    \item[(1)] $\emptyset, X \in \mathcal A$;
    \item[(2)] $\mathcal A$ is closed under complement;
    \item[(3)] $\mathcal A$ is closed under finite unions and intersections.
  \end{enumerate}
\end{definition}

\begin{definition}[$\sigma$-algebra]
  A $\sigma$-algebra on a nonemptyset $X$ is an algebra $\mathcal A$ on
  $X$ that is also closed under countable unions and intersections.
\end{definition}

\begin{remark}
  A $\sigma$-algebra is sometimes called a $\sigma$-field.
  That is why it is sometimes denoted $\mathcal F$.
\end{remark}

\begin{remark}
  From De-Morgan laws we know that
  \[
    \del{\bigcup_{i \in I} A_i}^c =
    \bigcap_{i \in I} A_i^c
  \]
  so it is only necessary to require closure under countable unions
  or countable intersections.
\end{remark}

\begin{remark}
  It is also possible to require $\mathcal A$ to be nonempty.
  Then for $A \in \mathcal A$ we have
  \[
    A \in \mathcal \implies A^c \in \mathcal A \implies
    A \cup A^c = X \in \mathcal A \stand
    A \cap A^c = \emptyset \in \mathcal A.
  \]
\end{remark}

\begin{example}
  $\mathcal A = 2^X$ and $\mathcal = \set{\emptyset, X}$ are the smallest
  and biggest $\sigma$-algebras.
\end{example}

\begin{example}
  If $X$ is not countable.
  Then
  \[
    \mathcal A =
    \set{E \subseteq X \colon E \stor E^c \text{ are countable}} \neq
    2^X
  \]
  is a $\sigma$-algebra.
\end{example}

\begin{definition}[Cocountablility]
  Let $X$ be a set. Then $A$ is called cocountable if $A^c$ is countable.
\end{definition}

\begin{example}[Borel $\sigma$-algebra]
  Let $F \subseteq 2^X$ be a family of subsets of $X$.
  The $\sigma$-algebra generated by $F$ is defined as
  \[
    \sigma(F) = \cap \set{\mathcal A \subseteq 2^X \colon 
  \mathcal A \text{ is a $\sigma$-algebra} \stand F \subset \mathcal A}
  \]
\end{example}

% Add intersection of sigma algebra is a sigma algebra.













\end{document}
